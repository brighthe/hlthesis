% !Mode:: "TeX:UTF-8"
%---------------------------------------------------------------
% 文档类设置
%---------------------------------------------------------------
\documentclass{beamer}  % 使用beamer文档类创建演示文稿

%---------------------------------------------------------------
% 主题与样式设置
%---------------------------------------------------------------
\usetheme{Darmstadt}      % 使用 Darmstadt 主题
\useinnertheme{rounded}   % 使用圆角内部主题
\usecolortheme{beaver}    % 使用 beaver 颜色主题

% 其他可选颜色主题(取消注释以启用)
%\usecolortheme{albatross}
%\usecolortheme{beetle}
%\usecolortheme{crane}
%\usecolortheme{dolphin}
%\usecolortheme{dove}
%\usecolortheme{fly}
%\usecolortheme{lily}
%\usecolortheme{orchid}
%\usecolortheme{rose}
%\usecolortheme{seagull}
%\usecolortheme{seahorse}
%\usecolortheme{whale}
%\usecolortheme{wolverine}
%\usecolortheme{default}

%---------------------------------------------------------------
% beamer设置与调整
%---------------------------------------------------------------
\numberwithin{subsection}{section}             % 子节按节编号
\usefonttheme[onlylarge]{structurebold}        % 仅对大标题使用粗体
%\usefonttheme[onlymath]{serif}                % 可选:仅数学内容使用衬线字体
%\setbeamercovered{transparent}                % 可选:设置覆盖项为透明

%---------------------------------------------------------------
% 文献引用设置
%---------------------------------------------------------------
\usepackage[numbers, sort&compress]{natbib}     % 使用natbib包管理参考文献,采用数字引用样式
% sort - 引用按编号排序;compress - 连续引用压缩为范围(如[1-3])
\setcitestyle{numbers}                         % 强制使用数字引用样式,覆盖可能的其他样式设置


% 设置帧标题格式
\setbeamerfont*{frametitle}{size=\normalsize,series=\bfseries}
\setbeamertemplate{navigation symbols}{}         % 隐藏导航符号
%\setbeamertemplate{bibliography item}[text]      % 参考文献项使用文本形式
\setbeamertemplate{bibliography item}[numbered]  % 参考文献项使用编号样式

%---------------------------------------------------------------
% 导入自定义样式文件
%---------------------------------------------------------------
%%%%%%%%------------------------------------------------------------------------
%%%% 必要宏包

%% 数学宏包
\usepackage{amsmath,amssymb}

%% 颜色支持
\usepackage{color,xcolor}

%% 超链接
\usepackage{hyperref}

%% 图形支持
\usepackage{graphicx}
\usepackage{subfig}

%% 基本绘图功能
\usepackage{pgf,tikz}
\usetikzlibrary{shapes,arrows}

%% 演示文稿中使用的基本数学命令
\newcommand{\mbR}{{\mathbb R}}
\newcommand{\mcZ}{{\mathcal Z}}
\newcommand{\mcV}{{\mathcal V}}
\newcommand{\bfx}{{\boldsymbol x}}
\newcommand{\bfz}{{\boldsymbol z}}
\newcommand{\rmd}{\,\mathrm d}

%% 基本颜色命令
\newcommand{\red}{\color{red}}
\newcommand{\blue}{\color{blue}}

%% 为beamer设置参考文献样式
\makeatletter
\@ifclassloaded{beamer}{
	\bibliographystyle{apalike}
}{
	\bibliographystyle{abbrv}
}
\makeatother
%%%%%%%%------------------------------------------------------------------------   % 导入英文环境设置
%%%%%%%%------------------------------------------------------------------------
%%%% XeCJK必要设置

\usepackage{xltxtra, fontenc, xunicode}
\usepackage[slantfont, boldfont]{xeCJK}

\setlength{\parindent}{1.5em} % 中文段落缩进

%% 中文断行设置
\XeTeXlinebreaklocale "zh"
\XeTeXlinebreakskip = 0pt plus 1pt minus 0.1pt

%% 中文标点样式
\punctstyle{kaiming}

%% 基本中文字体
\setCJKmainfont[BoldFont={Adobe Heiti Std}, ItalicFont={Adobe Kaiti Std}]{Adobe Song Std}

%% 英文字体
\setmainfont{DejaVu Serif}
\setmonofont{DejaVu Sans Mono}
\setsansfont{DejaVu Sans}

%% 常用中文字体系列
\setCJKfamilyfont{song}{Adobe Song Std}
\setCJKfamilyfont{kai}{Adobe Kaiti Std}
\setCJKfamilyfont{hei}{Adobe Heiti Std}

%% 字体命令
\newcommand{\song}{\CJKfamily{song}}
\newcommand{\kai}{\CJKfamily{kai}}
\newcommand{\hei}{\CJKfamily{hei}}

%% 环境的中文名称
\renewcommand{\contentsname}{目录}
\renewcommand{\figurename}{图}
\renewcommand{\tablename}{表}
\renewcommand{\today}{\number\year 年 \number\month 月 \number\day 日}

\makeatletter
\@ifclassloaded{book}{
	\renewcommand{\bibname}{参考文献}
}{
	\renewcommand{\refname}{参考文献}
}
\makeatother
%%%%%%%%------------------------------------------------------------------------  % 导入中文环境设置

%---------------------------------------------------------------
% 标题、作者和机构信息
%---------------------------------------------------------------
\title{{\small
湘潭大学博士研究生开题报告} \\
~~~~\\
开源拓扑优化软件包 SOPTX:数值算法实现、应用及性能研究}

\author{
	\begin{tabular}{rl}
		报告人: & 何~亮 \\
		专~~~~业: & 数~学 \\
		导~~~~师: & 魏华祎~教授 \\
	\end{tabular}
}

\institute[XTU]{
\vspace{5pt}
湘潭大学$\bullet$数学与计算科学学院\\
}
 
\date[XTU]{\today}  % 使用当前日期

%---------------------------------------------------------------
% 节和子节的目录自动生成设置
%---------------------------------------------------------------
% 在每节开始添加目录幻灯片
\AtBeginSection[]
{
	\frame<beamer>{ 
		\frametitle{目录}   
		\tableofcontents[currentsection] 
	}
}

% 在每子节开始添加目录幻灯片
\AtBeginSubsection[]
{
	\frame<beamer>{ 
		\frametitle{目录}   
		\tableofcontents[currentsubsection] 
	}
}

\begin{document}

\begin{frame}
  \titlepage
\end{frame}

\begin{frame}{目录}
  \tableofcontents
\end{frame}

\section{选题背景}
\begin{frame}
\frametitle{结构拓扑优化简介}
\small{
\begin{itemize}
	\item \textbf{定义与优势}: 拓扑优化是一种先进的结构设计方法,旨在给定载荷和约束条件下,在设计域内自动寻找最优的材料分布,以实现结构性能的最优。与传统优化方法相比,拓扑优化具有更高的智能化和设计自由度。
	
	\item \textbf{计算挑战}:
	\begin{itemize}
		\item \textbf{大规模计算}: 结构分析与优化耦合,计算量大。
		
		\item \textbf{迭代求解}: 需反复求解边值问题和灵敏度分析。
		
		\item \textbf{资源消耗}: 处理大型线性系统,需求高。
		
		\item \textbf{核心难题}: 提升效率与可扩展性,兼顾精度。
	\end{itemize}
\end{itemize}
}
\end{frame}

\begin{frame}
\frametitle{结构拓扑优化发展概述}
\small{
\begin{itemize}
	\item \textbf{早期解析阶段}:
	\begin{itemize}
		\item 1904 年,Michell~\cite{michell1904lviii} 提出最优材料分布理论(桁架结构),奠定理论基础并确立“最优布局”核心思想。
		
		\item Prager~\cite{prager1977optimization, rozvany1972optimal} 等人将理论扩展至板和梁结构(解析法),但仅限于简单案例。
	\end{itemize}
\end{itemize}

\begin{itemize}
	\item \textbf{数值方法突破}:
	\begin{itemize}
		\item 1984 年,Cheng 与 Olhoff~\cite{cheng1981investigation} 提出首个通用数值拓扑优化方法(固体弹性板),标志着数值方法在该领域的兴起。
	\end{itemize}
\end{itemize}

\begin{itemize}
	\item \textbf{现代发展}:
	\begin{itemize}
		\item 1988 年,Bendsøe~\cite{bendsoe1988generating} 引入均匀化方法,为变密度方法奠定基础。
		
		\item 如今,拓展出多种数值方法,例如变密度方法、边界演化方法、进化方法等,为不同设计提供技术支持。
	\end{itemize}
\end{itemize}
}
\end{frame}

\begin{frame}
\frametitle{结构拓扑优化的应用}
\small{
\begin{itemize}
	\item \textbf{应用领域}: 随着理论和方法的不断完善,拓扑优化作为一种先进的结构设计技术,已在汽车工业、船舶、机器人、生物工程、航空航天和土木工程等多个领域得到广泛应用~\cite{jihong2021review, zhu2016topology, xiao2020design, koper2021topology, mallek2024topological, park2019topology}。
\end{itemize}

\begin{itemize}
	\item \textbf{应用场景扩展}: 拓扑优化的应用场景日益多样化,不仅局限于传统的线弹性问题,还逐步扩展至非线性理论框架下的复杂工程问题,例如材料非线性约束~\cite{wang2023improved},应力约束~\cite{han2021topology},和几何非线性约束~\cite{zhang2020topology,zhu2020design} 等。
\end{itemize}
}
\end{frame}

\begin{frame}
\frametitle{结构拓扑优化代码的发展概述}
\small{
拓扑优化作为一个跨学科的复杂领域,其理论和方法的多样性为初学者带来挑战。为降低门槛并推动普及,研究人员通过教学论文和开源代码注入活力。
\begin{itemize}
	\item 2001 年,Sigmund 发布了首个 99 行 MATLAB 教学代码(top99),简洁且具教学性,标志开源代码发展的开端,极大促进研究与应用。
	\item 此后,开源代码的数量和质量不断提升,涵盖多种优化方法和技术,为大规模问题求解和实际工程应用提供了重要支持。
\end{itemize}
}
\end{frame}

\section{研究内容}
\begin{frame}
\frametitle{研究内容}
\small{
本文提出了基于 FEALPy 平台开发的高性能拓扑优化框架 SOPTX。FEALPy 是一款开源的智能 CAE 计算引擎,为 SOPTX 提供了高效且灵活的底层支持。SOPTX 通过模块化设计显著提升了框架的可扩展性和可复用性,为复杂工程应用提供了有力支持。
}
\end{frame}

\begin{frame}
\frametitle{SOPTX 框架设计}
\small{
\begin{itemize}
	\item \textbf{实现层面}: SOPTX 基于 FEALPy 开发,支持 NumPy、PyTorch、Jax 等多种后端,确保框架的灵活性和广泛兼容性。同时,适配 CPU 和 GPU 等设备,利用 FEALPy  高效计算引擎,显著提升大规模拓扑优化问题的求解效率。
		
	\item \textbf{算法层面}: SOPTX 支持多种拓扑优化算法,包括变密度方法和水平集方法。其模块化设计便于集成其他先进算法,例如 ESO/BESO、相场方法以及 MMC,从而增强框架的灵活性和可扩展性。
	 
	\item \textbf{应用层面}: SOPTX 支持求解多种拓扑优化应用,包括经典的静力学问题、热传导问题以及柔顺机械设计。同时,框架能够处理复杂的工程应用,例如应力约束优化、制造约束、3D 打印优化以及多物理场耦合,为工程设计提供全面支持。
\end{itemize}
}
\end{frame}

\begin{frame}
\frametitle{SOPTX 的变密度拓扑优化}
\small{
\begin{itemize}
	\item \textbf{任意维数与网格}: SOPTX 支持任意维数的拓扑优化问题,兼容结构化与非结构化网格,提升适应性和精度。
		
	\item \textbf{分析与优化解耦}: SOPTX 将结构分析和优化过程分离,提升框架的灵活性和可维护性。
	
	\item \textbf{多种优化方法}: SOPTX 支持灵活切换优化算法,例如 OC、MMA、SQP 等,提升优化效率与结果质量。
	
	\item \textbf{任意过滤方法}: SOPTX 支持灵活切换过滤方法,例如灵敏度过滤、密度过滤等,提升设计质量和计算稳定性。
\end{itemize}
}
\end{frame}

\begin{frame}
\frametitle{SOPTX 的水平集拓扑优化}
\small{
	\begin{itemize}
		\item \textbf{分析与演化解耦}: SOPTX 将结构分析和求解演化方程更新水平集的过程分离,提升框架的灵活性和可维护性。
		
		\item \textbf{灵活的方程配置}: SOPTX 提供多样化的 Hamilton-Jacobi 方程选择,支持拓扑导数项的包含或排除,以适应不同的拓扑变化需求。
		
		\item \textbf{多种水平集方法}: 
		\begin{itemize}
			\item 参数化水平集方法:通过参数化表示提升效率,减少重初始化。
			
			\item 基于反应扩散方程的水平集方法:引入扩散机制,增强结构演化的平滑性与稳定性。
		\end{itemize}
		
		\item \textbf{结合变密度方法}: SOPTX 支持水平集方法与变密度方法的融合,扩展优化设计的适用性。
	\end{itemize}
}
\end{frame}

\begin{frame}
\frametitle{SOPTX 中的新兴技术}
\small{
\begin{itemize}
	\item \textbf{矩阵快速组装}: 通过分离刚度矩阵计算中的单元相关部分和无关部分,实现高效的计算与缓存。
	
	\item \textbf{GPU 加速}: 通过 GPU 并行处理实现计算密集型任务的显著加速,提升大规模问题处理效率。
	
	\item \textbf{自动微分}: 精确高效地计算梯度,支持基于梯度的优化方法,确保设计迭代精度。
	
	\item \textbf{自适应网格加密}: 动态调整网格分辨率,聚焦关键区域的计算资源,兼顾精度与效率。
\end{itemize}
}
\end{frame}

\begin{frame}
\frametitle{SOPTX 在拓扑优化中的应用}
\small{
	\begin{itemize}
		\item \textbf{柔顺度最小化}: 优化结构刚度,减少变形,广泛应用于航空航天和汽车设计。
		
		\item \textbf{热传导优化}: 提升散热效率,适用于电子设备和能源系统的高性能设计。
		
		\item \textbf{柔顺机械设计}: 实现复杂运动与变形控制,用于微机电系统和机器人技术。
		
		\item \textbf{屈曲约束优化}: 增强结构稳定性,防止失稳,适用于桥梁和高层建筑。
		
		\item \textbf{应力约束优化}: 控制局部应力,确保结构安全性与耐久性。
	\end{itemize}
}
\end{frame}



\section{工作进展与论文安排}
\begin{frame}
\frametitle{工作进展}
\small{
\begin{itemize}
	\item 系统调研了拓扑优化的常用方法、数值不稳定现象及其解决方案,以及拓扑优化问题的基本过程和定义。
	
	\item 基于调研成果,形成了变密度拓扑优化的编程数学文档,详细推导了柔顺度最小化的目标函数、密度惩罚模型和灵敏度分析公式。
	
	\item 开发了拓扑优化软件包 SOPTX,实现了变密度拓扑优化,支持任意维数、任意网格的柔顺度最小化问题的求解,数值结果达到预期。
	
	\item SOPTX 中集成了快速矩阵组装、 GPU 加速、自动微分技术,验证了其在计算精度和效率上的显著提升。
\end{itemize}
}
\end{frame}

\begin{frame}
\frametitle{未来计划}
\small{
\begin{itemize}
	\item 在 SOPTX 中集成水平集拓扑优化方法,丰富软件的优化算法库,进一步提升其在拓扑优化领域的适用性。
	
	\item 在 SOPTX 中引入自适应网格加密技术,提升 SOPTX 在复杂结构优化中的计算精度和效率。
	
	\item 扩展 SOPTX 的应用范围,包括热传导优化、柔顺机械设计、以及更复杂的工程应用(如应力约束优化、屈曲约束优化等)。
\end{itemize}
}
\end{frame}


\begin{frame}
\frametitle{论文安排}
\small{
\begin{itemize}
	\item 2025.04-2025.10:调研相关文献资料,解决理论上的难点、实现相关程序,完成论文框架;
	
	\item 2025.10-2025.12:撰写论文,完成论文初稿;
	
	\item 2026.01-2026.04:修改论文,和老师讨论,完成论文终稿。
\end{itemize}
}
\end{frame}

\section*{参考文献}
\begin{frame}[allowframebreaks]
\frametitle{参考文献}
\tiny
\bibliographystyle{abbrv}
\bibliography{cvtxinjiang}
\end{frame}

\frame{
\begin{center}\color{black} \Huge
\textsl{感谢聆听}
\end{center}
}

%\begin{frame}
%\frametitle{参考文献}
%\tiny
%\bibliographystyle{abbrv}
%\bibliography{cvtxinjiang}
%\end{frame}

\end{document}

