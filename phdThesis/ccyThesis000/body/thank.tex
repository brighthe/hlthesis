\chapter*{致~~~~谢}
\addcontentsline{toc}{chapter}{致谢}
在论文的最后,我想借此机会感谢所有帮助和支持过我的人。

首先要感谢我的导师{\CJKfontspec{SimSun}魏华祎}教授。这篇论文能顺利完成,离不开魏老师一直以来的指导和鼓励。
魏老师非常耐心,在我刚入门时经常手把手教我编程,帮我调试
BUG。在遇到困难时,魏老师总是耐心地帮我分析问题,鼓励我,让我对自己有信心。
希望以后能成为像魏老师一样优秀的科研工作者,能像魏老师一样耐心地帮助他人。

非常感谢黄学海教授和郭汝驰教授。本文的主要内容也是在两位老师的指导下完成的。在和他们合作的过程中,我不仅学到了很多理论知识,
也提高了对科研的理解,这些都让我受益很多。

感谢陈龙教授,以及陈龙教授团队的同学们:徐泽一博士、于雅新博士。在 UCI
期间,不仅在学术上得到了
很多指导,在生活中也得到了很多关心和照顾,使我能在异国他乡安心学习。

感谢师母熊婷在生活中对我的照顾和关心。师母对待任何人,任何事都简单、阳光、真诚,
总能用温和而坚定的态度影响身边的人。
在师母身上,我学到了很多生活的智慧,
比如要立志,要学会复盘反思,还要重视时间管理,要有长远眼光,
这都是我在生活中需要不断学习和改进的地方。
%求真务实,独立自主,艰苦奋斗,长期主义,自我反省,精诚团结,统合综效。

感谢我的爷爷,我的父母,我所有的家人们,感谢他们默默地支持我,给我提供了良好的学习环境和条件,
让我能够安心地完成学业。
感谢尚若彤同学七年来的陪伴和支持,她不喜欢来虚的,我就不多说了。

感谢我的朋友们:周铁军博士、曾港博士、杨骏博士,
在学习和生活中大家互相帮忙,一起学习、一起吐槽,
我会永远怀念贰幺贰讨论班的日子。
感谢我的同门田甜博士以及彭娟老师,
我平时处理班级事务时经常马虎出问题,感谢他们的提醒和包容。

还要感谢一下王冬岭教授、岳孝强教授,读博这些年也多次得到他们的指导和帮助,对我来说都很重要。 

最后,感谢数学。数学的美吸引我去思考,去探索,去实践,数学使这个世界变得有趣。
在未来的日子里,
希望数学能让我探索到更多的美,冲!






