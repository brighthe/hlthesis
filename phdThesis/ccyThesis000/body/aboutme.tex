%---个人简历、在学期间发表的学术论文及研究成果-----------------------------------------------------------------------------------------

%%%%%%%%%%%%%%%%%%%%%%%%%%% 盲审时使用 %%%%%%%%%%%%%%%%%%%%%%%%%%%%%
%\chapter*{博士期间发表的学术论文及研究成果}
%\addcontentsline{toc}{chapter}{在校期间发表的学术论文及研究成果}
%\begin{itemize}
%\item[1.] {\heiti{攻读博士学位期间已发表学术论文}}
%    \begin{itemize}
%    \setlength{\leftmargin}{1.5em}  % 控制整体左侧缩进
%    \setlength{\itemindent}{-1em}   % 让每行文本向左突出(减少缩进
%    \item [[1]]
%        Virtual element methods
%        without extrinsic stabilization. SIAM Journal on Numerical Analysis,
%        62(1):567–591, 2024. (SCI 收录,第一作者)
%
%    \item [[2]]
%        Geometric Decomposition
%        and Efficient Implementation of High Order Face and Edge Elements.
%        Communications in Computational Physics, 35(4):1045-1072, 2024.(SCI 收录,第一作者)
%
%    \item [[3]]
%        An adaptive virtual
%        element method for the polymeric self-consistent field theory, Computers
%        \& Mathematics with Applications, 141:242-254, 2023.(SCI 收录,第三作者)
%
%    \item [[4]]
%        $H^{m}$-Conforming Virtual
%        Elements in Arbitrary Dimension[J]. SIAM Journal on Numerical Analysis,
%        60(6): 3099-3123, 2022.(SCI 收录,第一作者)
%    \item [[5]]
%        The VEM for time-harmonic
%        Maxwell equations in inhomogeneous media with Lipschitz interface.
%        Mathematical Models and Methods in Applied Sciences.(拟接收,第一作者)
%    \item [[6]]
%        Adaptive Finite
%        Element Method for Phase Field Fracture Models Based on Recovery Error
%        Estimates; Journal of Computational and Applied Mathematics;
%        (已提交返修,第二作者)
%    \item [[7]]
%        High-Order Interior Penalty
%        Finite Element Methods for Fourth-Order Phase-Field Models in Fracture
%        Analysis; Applied Mathematics Letters; (已投稿,第二作者)
%    \end{itemize}
%\end{itemize}





%%%%%%%%%%%%%%%%%%%%%%%%%%% 非盲审时使用 %%%%%%%%%%%%%%%%%%%%%%%%%%%%%

\chapter*{个人简历、在学期间发表的学术论文及研究成果}
\addcontentsline{toc}{chapter}{个人简历、在学期间发表的学术论文及研究成果}
\begin{itemize}
    \item[1.] {\heiti{个人简历}}\\
        \textbf{姓名:陈春雨}     
        \hspace{3.0cm}
        \textbf{性别:男} \\
        \textbf{籍贯:河南许昌}  
        \hspace{2.6cm}
        \textbf{出生年月:1997年3月}
    \begin{itemize}
        \setlength{\leftmargin}{1.5em}  % 控制整体左侧缩进
        \setlength{\itemindent}{-1em}   % 让每行文本向左突出(减少缩进
        \item[\textbullet]
            2015.09-2019.06,在河南师范大学数学与信息科学学院获得学士学位。
        \item[\textbullet]
            2019.09-2021.06,在湘潭大学数学与计算科学学院攻读硕士学位。
        \item[\textbullet]
            2021.09-2025.06,在湘潭大学数学与计算科学学院攻读博士学位。
    \end{itemize}
\item[2.] {\heiti{攻读博士学位期间已发表学术论文}}
    \begin{itemize}
    \setlength{\leftmargin}{1.5em}  % 控制整体左侧缩进
    \setlength{\itemindent}{-1em}   % 让每行文本向左突出(减少缩进
    \item [[1]]
        \textbf{C. Chen}, R. Guo, and H. Wei. The VEM for time-harmonic
        Maxwell equations in inhomogeneous media with Lipschitz interface.
        Mathematical Models and Methods in Applied Sciences, 2025.

    \item [[2]]
        \textbf{C. Chen}, X. Huang, and H. Wei. Virtual element methods
        without extrinsic stabilization. SIAM Journal on Numerical Analysis,
        62(1):567–591, 2024.

    \item [[3]]
        \textbf{C. Chen}, L. Chen, X. Huang and H. Wei. Geometric Decomposition
        and Efficient Implementation of High Order Face and Edge Elements.
        Communications in Computational Physics, 35(4):1045-1072, 2024.

    \item [[4]]
        H. Wei, X. Wang, \textbf{C. Chen} and K. Jiang.  An adaptive virtual
        element method for the polymeric self-consistent field theory, Computers
        \& Mathematics with Applications, 141:242-254, 2023.

    \item [[5]]
        \textbf{C. Chen}, X. Huang, H. Wei. $H^{m}$-Conforming Virtual
        Elements in Arbitrary Dimension[J]. SIAM Journal on Numerical Analysis,
        60(6): 3099-3123, 2022.
     \item [[6]] T. Tian, \textbf{C. Chen}, L. He, and H. Wei. Adaptive Finite
         Element Method for Phase Field Fracture Models Based on Recovery Error
         Estimates; Journal of Computational and Applied Mathematics; 已接收.
     \item [[7]] T. Tian, \textbf{C. Chen}, and H. Wei. High-Order Interior Penalty
         Finite Element Methods for Fourth-Order Phase-Field Models in Fracture
         Analysis; Applied Mathematics Letters; 已投稿.
    \end{itemize}
\end{itemize}
