\chapter{虚单元方法的实现}
在本章中,我们会详细介绍虚单元方法的实现细节,从网格数据结构到缩放单项式空间的构造,
再到常见的虚单元空间的构造,以及 $H^m$
协调虚单元,无稳定化条件的虚单元方法等。
%\section{引言}
%首先考虑一个模型问题,以这个问题为例介绍虚单元方法的实现需要注意的问题。
%\begin{equation}
%\left\{
%    \begin{aligned}
%        -\Delta u + u&= f, \quad \text{in} \quad \Omega,\\
%        u &= 0, \quad \text{on} \quad \partial \Omega,
%    \end{aligned}
%\right.
%\end{equation}
%其中 $\Omega$ 是一个多边形或多面体,$f \in L^2(\Omega)$。 
%其对应的变分问题为:找到 $u \in H^1_0(\Omega)$,使得
%$$
%a(u, v) = f(v), \quad \forall v \in H^1_0(\Omega),
%$$
%其中 $a(u, v) = (\nabla u, \nabla v) + (u, v)$, $f(v) = (f, v)$,$(\cdot,
%\cdot)$ 表示 $L^2$ 内积。
%
%将 $\Omega$ 离散化为一个多面体网格 $\mathcal{T}_h$,
%在构造一个虚单元空间 $V_h$,使得 $V_h \subset H^1_0(\Omega)$,

在本章节中会定义大量的矩阵,有的会在后面章节用到,有的仅在某个小章节中用到,
为了避免混淆,
本文使用代码风格的矩阵定义,对于后面章节会用到的矩阵,我会给出矩阵算法的伪代码,
用于后面章节``调用'',
在每个小章节的开头会给出矩阵的定义,类似于程序中调用其他函数生成矩阵,
这些矩阵只在小章节中有定义,与其他章节无关,即每个变量有自己的``作用域'',

\section{网格数据结构}
%虚单元方法需要先将计算区域离散化为网格,
%在网格上构造函数空间。从程序实现的角度来看,网格的作用有两个:
%1. 在网格上进行数值积分。2. 将

虚单元方法的一个特点是可以适应任意形状的多边形网格(二维)或多面体网格
(三维),在虚单元计算的过程中,需要使用网格中各实体(点、边、面、体)
之间的邻接关系,因此我们需要一个方便的网格数据结构来多边形多面体存储网格的各种信息。

在有限元方法中,常用的网格是三角形,四边形,四面体和六面体等简单几何体组成的网格,
这些网格的结构比较简单,一般使用以单元为中心的网格数据结构来存储网格信息。
即由顶点数组和单元数组组成,其中顶点数组存储网格的顶点坐标,
单元数组存储单元的顶点编号,因为单元形状规则,
自然的可以根据单元数组构造出所有的边和面,
构造的过程中也可以计算出各实体之间的邻接关系。

但是对于多边形网格或多面体网格,由于网格的形状比较复杂。
使用上述的网格数据结构来存储网格信息会比较困难,比如六面体单元每个面都是一个四边形,
六个面的顶点在单元中的编号是固定的,而对于一个多面体,其面数和面的顶点数都是不固定的,
因此使用以单元为中心的网格数据结构方表示多面体网格比较困难。

在本文中,我们使用组合映射网格数据结构表示一个网格,
这是一种以边为中心的网格数据结构,由 Larson 等人提出,
通常在计算几何建模和计算机图形学中使用,这种数据结构与以单元为中心网格数据结构
的不同在于,以单元为中心的数据结构需要把所有的邻接信息存储起来,
而组合映射网格数据结构引入了一种 “半边”
的结构,存储的是半边与实体的邻接信息,
通过这些信息可以方便的遍历每个实体的邻接实体。
在本文中我们使用组合映射网格的二维三维版本,分别是 Halfedge 网格和 Dart 网格。

\subsection{Halfedge 网格}
Halfedge 网格添加了给原本的二维网格实体(点、边、面)添加了半边的概念,
将网格中每条边分成两个半边,每个半边有一个方向,每个半边都有一个对应的另一半边,
每个半边有其顶点,所在的边,其左边的面,以及其下一条半边,上一条半边,
对边这些几个属性,图一中展示了一个 Halfedge 各个属性的例子。

Halfedge 网格数据结构在存储网格信息时,使用了一个顶点数组,一个半边数组,
其中顶点列表 \lstinline{node} 用来存储网格的顶点坐标(几何信息),
半边列表 \lstinline{halgedge}
用来存储网格的拓扑信息,其中每个半边存储了指向的顶点,所在的边,左边的面,
以及下一条半边,上一条半边,对边等信息。
然后每个单元存储一个半边 \lstinline{hcell},每个边存储一个半边 \lstinline{hedge},
每个点存储一个半边 \lstinline{hnode}, 
图二展示了一个 Halfedge 网格的例子。

半边网格数据结构的一个优点是可以方便的遍历每个实体的邻接实体。
对于一条半边定义两种迭代操作:
\begin{itemize}
    \item \lstinline{nextIter}: 返回所有的 \lstinline{next} 半边
    \item \lstinline{nextoppoIter}: 返回所有的 \lstinline{next oppo} 半边 
\end{itemize}
图三展示了 Halfedge 网格中半边的迭代操作。
根据 \lstinline{nextIter} 操作,我们可以方便的获取每个单元相邻的实体。
\begin{lstlisting}
HalfEdge * h = hcell[k];
for (auto & hiter : h->nextIter()) 
{
  Cell * c = hiter->oppo()->cell();
  Edge * e = hiter->edge();
  Node * n = hiter->node();
}
\end{lstlisting}
根据 \lstinline{nextoppoIter} 操作,我们可以方便的获取每个顶点相邻的实体。
\begin{lstlisting}
HalfEdge * h = hnode[k];
for (auto & hiter : h->nextoppoIter()) 
{
  Cell * c = hiter->cell();
  Edge * e = hiter->edge();
  Node * n = hiter->node();
}
\end{lstlisting}
\subsection{Dart 网格}
Dart 网格可以用来表示三维多面体网格,非流行网格等,但是本文中我们只使用 Dart
网格来表示多面体网格,在这种情况下,Dart 网格可以看做是 Halfedge
网格的扩展。
首先我们回到 Halfedge 网格,理解 Halfedge 网格是如何表示多边形网格的,
如果我们只知道 Halfedge 网格中每个半边的 next 
信息,等价于我们不知道一个多边形和谁相邻,我们只知道多边形的顶点,
而加上半边的对边这个信息以后,我们相当于把这些多边形集合按照对边给粘起来了。

类似的,每个多边形的表面其实是一个多边形网格,
我们可以使用 Halfedge 网格来表示,每个多边形的 Halfedge
网格集合在一起我们只知道有一堆多面体,但是我们不知道多面体之间的邻接关系,
这时候我们给每个半边加上一个面对边信息,如图四所示,
我们就可以把这些多面体粘起来了,从而可以表达多面体网格。

在 Dart 网格中,Halfedge 被称为 Dart(因为每条边不再只被分成两份),每个 Dart
存储自己的顶点,所在的边,面,体,以及三个半边,
\begin{itemize}
    \item \lstinline{next}: 相同体和面的下一个 Dart。
    \item \lstinline{oppo}: 相同体并相对的 Dart。
    \item \lstinline{prev}: 相同面不同体且相对的 Dart。
\end{itemize}
Dart 网格数据结构在存储网格信息时,使用了一个顶点数组 \lstinline{node},
一个 Dart 数组 \lstinline{dart},和 Halfedge 网格一样,每个实体都会存储一个
Dart。图五展示了一个 Dart 网格的例子。

\section{缩放单项式空间}
在虚单元方法中,由于使用的函数是 ``虚''
函数,其具体形式是未知的,所以需要将函数投影到一个已知的函数空间中,
最常见的是将函数投影到多项式空间中,因此需要在多边形,多面体上定义多项式空间。

考虑一个 $n$ 维多面体 $K$,其中心为 $\bx^K$, 直径为 $h_K$,
$\mathbb{P}_k(K)$ 
最简单的一组基就是形如 $x_0^{\alpha_0}x_1^{\alpha_1}\cdots x_{n-1}^{\alpha_{n-1}}$ 
的单项式,但是这样的函数在不同单元上的函数值量级完全不同,比如 $x^3y^3$ 在
$K = (0, 1)\times(0, 1)$ 上函数值量级是 $1$,而在 $K = (99, 100)\times(99, 100)$ 上的
函数值量级是 $10^{12}$,这样的函数在计算机中会有很大的误差。

因此我们将原点移动到单元的中心 $\bx_K$,再把单项式函数除以 $h^k_K$, 
定义如下的缩放单项式
$$
m_{\boldsymbol{\alpha}}(\bx) = 
\frac{\Pi_{i=0}^{n-1} (x_i-x_i^K)^{\alpha_i}}{h^{|\boldsymbol{\alpha}|}_K}
= \frac{(\bx-\bx_K)
^{\boldsymbol{\alpha}}}{h^{|\boldsymbol{\alpha}|}_K}
$$
这样 $m_{\boldsymbol{\alpha}}$ 的函数值就会和 1 一个量级。
而且小于等于 $k$ 次的缩放单项式同样组成了一组 $\mathbb{P}_k(K)$ 的基:
$$
\mathcal{M}_k(K) = \{m_{\boldsymbol{\alpha}}: |\boldsymbol{\alpha}| \le k,
    \alpha_i \ge 0\}
$$
缩放单项式结构简单,计算方便,有两个良好的性质:1.
求导后仍然是缩放单项式,且不同单元上的缩放单项式求导后仅相差一个常数;
2.
缩放单项式相乘后仍然是缩放单项式,且其在单元上的积分可以化简到单元边界上进行积分,
大大减少了计算量。

\subsection{缩放单项式的导数}
缩放单项式的偏导数如下:
$$
\partial_{x_i} m_{\boldsymbol{\alpha}} = 
\partial_{x_i}
\frac{(\bx-\bx_K)^{\boldsymbol{\alpha}}}{h^{|\boldsymbol{\alpha}|}_K}
= \frac{\alpha_i}{h_K}
\frac{(\bx-\bx_K)^{\boldsymbol{\alpha-e}_i}}{h^{|\boldsymbol{\alpha}|-1}_K}
= \frac{\alpha_i}{h_K} m_{\boldsymbol{\alpha-e}_i}
$$
所以缩放单项式的偏导数还是缩放单项式,
且偏导 $\partial_{x_i}$ 作为 $\mathbb{P}_{k}(K)$
上的线性变换,其矩阵形式在缩放单项式基下很容易写出来: $\bP^{K, i} =
(P^{K, i}_{\boldsymbol{\alpha\beta}})_{n_k\times n_k}$,其中
$$
P_{\boldsymbol{\alpha\beta}}^{K, i} = \frac{\alpha_i}{h_K}
\delta_{\boldsymbol{\alpha-e}_i, \boldsymbol{\beta}}
$$
令 $\bP^{i} = (\alpha_i \delta_{\boldsymbol{\alpha-e}_i,
\boldsymbol{\beta}})_{n_k\times n_k}$,那么 $\bP^{K, i} =
\frac{1}{h_K}\bP^{i}$。
这是一个非常好的性质,因为我们只需要计算一个矩阵 $\bP^{i}$,
那么每个单元上的偏导数矩阵就可以通过简单地缩放得到。

根据偏导矩阵,我们可以构造出所有与导数有关的矩阵,比如拉普拉斯算子在缩放单项式基下的矩阵形式为
$$
\bL^{K} = \sum_{i=0}^{n-1} \bP^{K, i}\bP^{K, i}
= \frac{1}{h_K^2} \sum_{i=0}^{n-1} \bP^{i}\bP^{i}
$$
这个性质在 $H^m$
协调协调虚单元空间中非常重要,因为我们需要计算多项式的任意阶导数。

\subsection{缩放单项式的积分}
缩放单项式另一个优势是其在单元上的积分计算简单,用到的原理是
齐次函数在单元 $K$ 上的积分可以化简到单元边界上的积分。对于一个齐次函数 
$f$,其满足:
$$
f(k \bx) = k^q f(\bx)
$$
令等式两边同是对 $k$ 求导,可得
$$
\nabla f(k \bx)\cdot \bx = q k^{q-1} f(\bx)
$$
令 $k = 1$,可得
$$
f(\bx) = \frac{1}{q} \nabla f(\bx)\cdot \bx
$$
那么:
$$
\int_K f(\bx)d\bx = \frac{1}{q} \int_{K}
\nabla f(\bx)\cdot \bx d\bx
= -\frac{1}{q} \int_{K} f(\bx)\mathrm{div}{\bx}
\mathrm{d}\bx
+ \frac{1}{q} \int_{\partial K} f(\bx)\bx\cdot
\bn\mathrm{d}s
$$
注意 $\mathrm{div}{\bx} = n$,
所以我们可以把积分化简到单元边界上的积分:
$$
\int_K f(\bx)\mathrm{d}\bx = \frac{1}{q+n} \int_{\partial K}
f(\bx)\bx\cdot \bn\mathrm{d}s
$$
上式对于任意 $n$ 维多面体 $K$,以及任意齐次函数 $f$
都成立,对于边界是平面的多面体,
边界上的积分还可以进一步化简,
令 $\mathcal{F}$ 为 $K$ 的边界面集合,对于 $F\in \mathcal{F}$,找到一个点
$\bx_F$ 在 $F$ 所在平面上,那么 
$(\boldsymbol{x-x}_F)\cdot \bn_F = 0$,
所以
$$
\int_{\partial K} f(\bx)\bx\cdot \bn
\mathrm{d}s = \sum_{F\in \mathcal{F}}
\int_{F} f(\bx)(\boldsymbol{x-x_F+x_F})\cdot \bn_F
\mathrm{d}s = \sum_{F\in \mathcal{F}}
\int_{F} f(\bx)\boldsymbol{x_F}\cdot \bn_F
\mathrm{d}s
$$

而缩放单项式 $m_{\boldsymbol{\alpha}}$
是一个齐次函数:
$$
m_{\boldsymbol{\alpha}}(k\bx) = k^{|\boldsymbol{\alpha}|}
m_{\boldsymbol{\alpha}}(\bx)
$$
所以其在单元上的积分可以化简到单元边界上的积分:
$$
\int_K m_{\boldsymbol{\alpha}}(\bx)\mathrm{d}\bx =
\frac{1}{|\boldsymbol{\alpha}|+n} \int_{\partial K}
m_{\boldsymbol{\alpha}}(\bx)\bx\cdot \bn
\mathrm{d}s
$$
且 $m_{\boldsymbol{\alpha}}m_{\boldsymbol{\beta}} =
m_{\boldsymbol{\alpha+\beta}}$,所以缩放单项式的质量矩阵为
$$
\bM^{K}_{\boldsymbol{\alpha\beta}} = \int_K
m_{\boldsymbol{\alpha}}m_{\boldsymbol{\beta}}\mathrm{d}\bx =
\frac{1}{|\boldsymbol{\alpha}+\boldsymbol{\beta}|+n} \int_{\partial K}
m_{\boldsymbol{\alpha}}m_{\boldsymbol{\beta}}\bx\cdot \bn
\mathrm{d}s
$$
刚度矩阵为:
$$
\begin{aligned}
\bA^{K}_{\boldsymbol{\alpha\beta}} & = \int_K
\nabla m_{\boldsymbol{\alpha}}\cdot \nabla
m_{\boldsymbol{\beta}}\mathrm{d}\bx\\
& = 
\sum_{i=0}^{n-1} \int_K \partial_{x_i} m_{\boldsymbol{\alpha}}
\partial_{x_i} m_{\boldsymbol{\beta}}\mathrm{d}\bx
& = \sum_{i=0}^{n-1} P^{K, i}_{\boldsymbol{\alpha\gamma}}
\int_K m_{\boldsymbol{\gamma}}m_{\boldsymbol{\delta}}\mathrm{d}\bx
P^{K, i}_{\boldsymbol{\beta\delta}}
\end{aligned}
$$
即: $A^{K} = \sum_{i=0}^{n-1} \bP^{K, i}
\bM^{K} (\bP^{K, i})^T = \frac{1}{h_K^2}
\sum_{i=0}^{n-1} \bP^{i} \bM (\bP^{i})^T$。
同理,对于任意 $m>0$, 定义:
$$
A^{K, m}_{\boldsymbol{\alpha\beta}} = \int_K \nabla^m m_{\boldsymbol{\alpha}}
\cdot \nabla^m m_{\boldsymbol{\beta}}\mathrm{d}\bx
$$
那么 $A^{K, m} =$

\subsection{向量型缩放单项式和多项式分解}
定义向量型多项式空间 $\mathbb{P}_k(K, \mathbb{R}^n): \mathbb{P}_k(K) \otimes 
\mathbb{R}^n$,记 $\be_0, \be_1, \cdots, \be_{n-1}$ 为 $\mathbb{R}^n$ 的标准正交基,
那么可以自然的得到 $\mathbb{P}_k(K, \mathbb{R}^n)$ 的一组由向量缩放单项式
组成的基:
$$
\mathcal{M}_k^n(K) = \{\mathcal{M}_k(K)\otimes \be_i: i = 0, 1, \cdots, n-1\}
$$
现在我们考虑二维的多项式空间分解:
$$
\mathbb{P}_k(K, \mathbb{R}^2) = \nabla \mathbb{P}_{k+1}(K) + \bx^{\perp}
\mathbb{P}_{k-1}(K)
= \bcurl \mathbb{P}_{k+1}(K) + \bx\mathbb{P}_{k-1}(K)
$$
其中 $\bx = (x- x_K, y - y_K)$,$\bx^{\perp} = (-y + y_K, x -
x_K)$。多项式分解定理在缩放单项式基下有一个非常简单的证明:
以 $m_{\balpha}\otimes \be_0$ 为例,存在 $a, b, \bbeta, \bgamma$ 使得
\begin{equation}
\label{eq:polydecomp}
m_{\balpha}\otimes \be_0 = a\nabla m_{\bbeta} + b\bx^{\perp}m_{\bgamma}
\end{equation}
现在考察 $a, b, \bbeta, \bgamma$ 的取值。首先我们有:
$$
\nabla m_{\bbeta} = 
\begin{pmatrix}
    \frac{\beta_0}{h_K} m_{\bbeta-\be_0}\\
    \frac{\beta_1}{h_K} m_{\bbeta-\be_1}
\end{pmatrix}, \quad
\bx^{\perp} m_{\bgamma} =
\begin{pmatrix}
    -\frac{\gamma_1}{h_K} m_{\bgamma+\be_0}\\
    \frac{\gamma_0}{h_K} m_{\bgamma+\be_1}
\end{pmatrix}
$$
所以等式当 $\bbeta = \balpha + \be_0$,$\bgamma = \balpha - \be_1$
时才可能成立,代入等式 \eqref{eq:polydecomp},可得
$$
\begin{cases}
    a \frac{\beta_0}{h_K} - b\frac{\gamma_1}{h_K} = 1\\
    a \frac{\beta_1}{h_K} + b\frac{\gamma_0}{h_K} = 0
\end{cases}
$$
解得 $a = \frac{h_K\gamma_0}{\bgamma\cdot\bbeta} =
\frac{h_K\alpha_0}{(\balpha+ \be_0)\cdot(\balpha-\be_1)},
b =-
\frac{h_K\beta_1}{\bgamma\cdot\bbeta} = 
-\frac{h_K\alpha_1}{(\balpha+ \be_0)\cdot(\balpha-\be_1)}$,
可得以下性质:
\begin{property}
    \label{prop:polydecomp}
    第一个多项式分解定理在缩放单项式基下的表达式如下:
    \begin{itemize}
        \item 令 $c_{\balpha} = (\balpha + \be_0)\cdot(\balpha-\be_1), 
            a_{\balpha} = \frac{1}{c_{\balpha}}h_K\alpha_0, 
            b_{\balpha} = -\frac{1}{c_{\balpha}}h_K\alpha_1$,
            $\bbeta = \balpha + \be_0$,$\bgamma = \balpha - \be_1$,
            则有
            $$
            m_{\balpha}\otimes \be_0 = a\nabla m_{\bbeta} +
            b\bx^{\perp}m_{\bgamma}
            $$
            令 $C^{\nabla, k, 0}_{\bbeta\balpha} = a_{\balpha}\delta_{\balpha, \bbeta-\be_0}$,
            $C^{X\perp, k, 0}_{\bgamma\balpha} =
            b_{\balpha}\delta_{\balpha, \bgamma+\be_1}$,那么有:
            $$
            m_{\balpha}\otimes \be_0 =
            C^{\nabla, k, 0}_{\bbeta\balpha}
            \nabla m_{\bbeta} + C^{X\perp, k, 0}_{\bgamma\balpha}
            \bx^{\perp}m_{\bgamma}
            $$
        \item 令 $c_{\balpha} = (\balpha + \be_1)\cdot(\balpha-\be_0),
            a_{\balpha} = \frac{1}{c_{\balpha}}h_K\alpha_1,
            b_{\balpha} = -\frac{1}{c_{\balpha}}h_K\alpha_0$,
            $\bbeta = \balpha + \be_1$,$\bgamma = \balpha - \be_0$,
            则有
            $$
            m_{\balpha}\otimes \be_1 = a\nabla m_{\bbeta} +
            b\bx^{\perp}m_{\bgamma}
            $$
            令 $C^{\nabla, k, 1}_{\bbeta\balpha} = a_{\balpha}\delta_{\balpha,
            \bbeta-\be_1}$,
            $C^{X\perp, k, 1}_{\bgamma\balpha} =
            b_{\balpha}\delta_{\balpha, \bgamma+\be_0}$,那么有:
            $$
            m_{\balpha}\otimes \be_1 =
            C^{\nabla, k, 1}_{\bbeta\balpha}
            \nabla m_{\bbeta} + C^{X\perp, k, 1}_{\bgamma\balpha}
            \bx^{\perp}m_{\bgamma}
            $$
        \item 令 $c_{\balpha} = (\balpha + \be_0)\cdot(\balpha-\be_1),
            a_{\balpha} = \frac{1}{c_{\balpha}}h_K\alpha_0,
            b_{\balpha} = -\frac{1}{c_{\balpha}}h_K\alpha_1$,
            $\bbeta = \balpha + \be_0$,$\bgamma = \balpha - \be_1$,
            则有
            $$
            m_{\balpha}\otimes \be_0 = a \bcurl m_{\bbeta} +
            b\bx m_{\bgamma}
            $$
            令 $C^{\bcurl, k, 0}_{\bbeta\balpha} = a_{\balpha}\delta_{\balpha,
            \bbeta-\be_0}$,
            $C^{X, k, 0}_{\bgamma\balpha} =
            b_{\balpha}\delta_{\balpha, \bgamma+\be_1}$,那么有:
            $$
            m_{\balpha}\otimes \be_0 =
            C^{\bcurl, k, 0}_{\bbeta\balpha}
            \bcurl m_{\bbeta} + C^{X, k, 0}_{\bgamma\balpha}
            \bx m_{\bgamma}
            $$
        \item 令 $c_{\balpha} = (\balpha + \be_1)\cdot(\balpha-\be_0),
            a_{\balpha} = \frac{1}{c_{\balpha}}h_K\alpha_1,
            b_{\balpha} = -\frac{1}{c_{\balpha}}h_K\alpha_0$,
            $\bbeta = \balpha + \be_1$,$\bgamma = \balpha - \be_0$,
            则有
            $$
            m_{\balpha}\otimes \be_1 = a \bcurl m_{\bbeta} +
            b\bx m_{\bgamma}
            $$
            令 $C^{\bcurl, k, 1}_{\bbeta\balpha} = a_{\balpha}\delta_{\balpha,
            \bbeta-\be_1}$,
            $C^{X, k, 1}_{\bgamma\balpha} =
            b_{\balpha}\delta_{\balpha, \bgamma+\be_0}$,那么有:
            $$
            m_{\balpha}\otimes \be_1 =
            C^{\bcurl, k, 1}_{\bbeta\balpha}
            \bcurl m_{\bbeta} + C^{X, k, 1}_{\bgamma\balpha}
            \bx m_{\bgamma}
            $$
    \end{itemize}
\end{property}

 
\subsection{低维面上的缩放单项式空间} 
在虚单元方法中,我们不仅需要将虚单元函数投影到多项式空间中,还要考虑虚单元函数
限制在低维面上时,投影到超平面上的多项式空间中。

对于一个
$d$ 维多面体 $K$,$F$ 是 $K$ 的一个 $\tilde{d}$ 维面,其单位正交切向量为 
$$
\{\bt_{F, 0}, \bt_{F, 1}, \cdots, \bt_{F, \tilde{d}-1}\}
$$
定义 $F$ 上的多项式空间
$$
\mathbb{P}_k(F) = \{q|_F : q \in \mathbb{P}_k(K)\}
$$
即 $F$ 上的多项式是 $K$ 上的多项式在 $F$ 上的限制。对于任意 $\boldsymbol{\alpha}
\in \mathbb{T}_{\tilde{d}}^k$ 定义 $F$ 上的缩放单项式 $m^F_{\boldsymbol{\alpha}}$ 
$$
m^F_{\boldsymbol{\alpha}}(\bx) = \frac{
\prod_{i=0}^{d-1}((\bx-\bx_F)\cdot
\bt_{F, i})^{\alpha_i}}
{h^{|\boldsymbol{\alpha}|}_F}
$$
其中 $\bx_F$ 是 $F$ 的中心,$h_F$ 是 $F$ 的直径。
显然 $m^F_{\boldsymbol{\alpha}} \in \mathbb{P}_k(F)$,且 $F$
上所有小于等于 $k$ 次的缩放单项式组成了 $\mathbb{P}_k(F)$ 的一组基:
$\mathcal{M}_k(F) = \{m^F_{\boldsymbol{\alpha}}: |\boldsymbol{\alpha}|\le k,
\alpha_i \ge 0\}$。

面上的缩放单项式具有与一般的缩放单项式类似的求导和积分性质,
一方面其沿切线方向求导后仍然是缩放单项式,另一方面,它也是齐次函数,
积分也可以化简到面上边界上。

$\mathcal{M}_k(K)$ 中的函数限制 $F$ 上属于 $\mathbb{P}_k(F)$,可以被 $\mathcal{M}_k(F)$
线性表出,即存在 $\bR^{K, F}$ 使得
$$
m_{\boldsymbol{\alpha}}^K = 
m^F_{\boldsymbol{\beta}}R^{K, F}_{\boldsymbol{\beta\alpha}}
$$
$\bR^{K, F}$ 的计算有多种方式,这里给出一个 $L^2$ 投影的计算方式:
令上式两边乘以 $m^F_{\boldsymbol{\gamma}}$ 并在 $F$ 上积分,可得
$$
\int_F m_{\boldsymbol{\alpha}}^Km^F_{\boldsymbol{\gamma}}\mathrm{d}s =
R^{K, F}_{\boldsymbol{\beta\alpha}}
\int_F m^F_{\boldsymbol{\beta}}m^F_{\boldsymbol{\gamma}}\mathrm{d}s
$$
所以 $\bR^{K, F} = (\bM^F)^{-T}\bG^T$,其中
$$
G_{\balpha\bbeta} = \int_F m^K_{\balpha}m^F_{\bbeta}\mathrm{d}s
$$
注意 $G$ 的计算不能化简到单元边界上。

 
\section{二维虚单元空间}
在本节中,我们将介绍常用的虚单元空间的构建,包括 $H^1$ 协调,非协调虚单元空间,
$H^2$,$H^3$ 协调虚单元空间,$H(\mathrm{curl})$ 协调虚单元空间,
$H(\mathrm{div})$ 协调虚单元空间,以及无稳定化条件的 $H^1$ 协调虚单元空间。

本节会涉及到在一维边上的数值积分,我们使用如下积分公式:
$$
\int_e f(x)\mathrm{d}x \approx |e|\sum_{i=0}^{n-1} w_i f(x_i)
$$
其中 $x_i$ 是 $e$ 上的高斯积分点,$w_i$ 是对应的高斯权重,$|e|$ 是 $e$ 的长度。
我们使用两种积分公式,一种是高斯-勒让德积分,
另一种是高斯-拉盖尔积分, 他们的区别如下:
\begin{itemize}
    \item 高斯-拉盖尔积分的积分点包含两个端点,$k$ 个积分点的积分公式代数精度为
        $2k-3$.
    \item 高斯-勒让德积分的积分点都在内部,$k$ 个积分点的积分公式代数精度为
        $2k-1$.
\end{itemize}

\subsection{$H^1$ 协调虚单元空间}
根据文献 [] 中的定义,首先定义多边形 $K$ 上的空间:
$$
\tilde{V}_k(K) = \{v \in H^1(K): v|_{\partial K} \in B_k(\partial K), \Delta v \in
    \mathbb{P}_k(K)\}
$$
其中 $B_k(\partial K) = \{v \in C(\partial K), v|_e \in \mathbb{P}_k(e), e \in
\partial K\}$.
定义一个 $\tilde{V}_k(K)$ 到 $\mathbb{P}_k(K)$ 的投影算子 $\Pi_k^{1, K}$ 满足:
\begin{align}
\label{eq:proj1}
(\nabla \Pi_k v, \nabla q) & = (\nabla v, \nabla q), \quad \forall v \in \tilde{V}_k(K), q \in
\mathbb{P}_k(K)\\
Q_0(\Pi_k v - v) & = 0, \quad \forall v \in \tilde{V}_k(K)
\end{align}
其中 $Q_0$ 定义如下:
$$
\begin{cases}
    Q_0(v) = \int_{\partial K} v \mathrm{d}s \quad \text{for } k = 1\\
    Q_0(v) = \int_{K} v \mathrm{d}\bx \quad \text{for other}
\end{cases}
$$
现在定义虚单元空间 $V_k$ 为:
$$
V_k = \{v \in \tilde{V}_k: (v - \Pi_k^{1, K}v, q) = 0, \forall q \in
\mathbb{P}_{k-2}(K)^{\perp}\cap \mathbb{P}_k(K)\}
$$
注意当 $k<0$,$\mathbb{P}_k(K) = \{0\}$。
定义 $V_k$ 上的一组线性泛函 $\mathrm{dof}_i$:
$$
\mathrm{dof}_i(v) = \begin{cases}
    v(\delta_i) & \quad \delta_i \in \mathcal{N}_K\\
    v(\bx_i) & \quad x_i \text{ 属于 $e$ 内部的 $k-1$ 个 Gauss-Lobatto 点}\\
    \int_K v q_i\mathrm{d}\bx & \quad q_i \in \mathcal{M}_{k-2}(K)
\end{cases}
$$
对于函数 $v \in V_k$,可以被 $\{\mathrm{dof}_i(v)\}$ 唯一确定 $\mathrm{dof}_i$
称为 $V_k$ 上的自由度。
$V_k$ 上的基函数 $\phi_i$ 满足与自由度对偶:
$$
\mathrm{dof}_i(\phi_j) = \delta_{ij}
$$
记 $e$ 上的自由度为 $\bd^e = \{d^e_0, d^e_1, \cdots, d^e_{k-1}\}$,
单元内部的自由度为 $\bd^K = \{N_1, N_1+1, \cdots, N_K\}$,
其中 $N_1 = NV + NE \times (k-1)$,是边界上的自由度个数。

网格 $\mathcal{T}_h$ 上的虚单元空间 $V_h$ 定义为:
$$
V_h = \{v \in H^1(\Omega): v|_K \in V_k, \forall K \in \mathcal{T}_h\}
$$
\subsubsection{$H^1$ 投影矩阵}
$V_k$ 是 $\tilde{V}_k$ 的一个子空间,所以 $\Pi_k^{1, K}$ 也可以看作是 $V_k$ 到
$\mathbb{P}_k(K)$ 的投影算子,称为 $H^1$ 投影算子。
下面给出其在 $\{\phi_i\}$ 和 $\{m_{i}\}$ 下的矩阵形式 $\boldsymbol{\Pi}^{1, K}$:
$$
\Pi_k^{1, K} \phi_i = m_j \Pi^{1, K}_{ji}
$$
带入其定义 \eqref{eq:proj1},可以得到:
$$
\begin{aligned}
\Pi^{1, K}_{ki}Q_0(m_k) & = Q_0(\phi_i)\\
\Pi^{1, K}_{ki} (\nabla m_k, \nabla m_j) & = (\nabla \phi_i, \nabla m_j)\\
& = -(\phi_i, \Delta m_j) + \sum_{e \in \mathcal{E}_K}\mathrm{sign}_{e, K}\langle \phi_i,
\partial_{\bn_e} m_j \rangle_e\\
& = -L_{kj}(\phi_i, m_k) + \sum_{e \in \mathcal{E}_K}\mathrm{sign}_{e,
K}P^{K, \bn}_{kj}\langle \phi_i, m_k \rangle_e\\
& = -L_{kj}\delta_{i, k+N_1} + \sum_{e \in \mathcal{E}_K}\mathrm{sign}_{e,
K}P^{K, \bn}_{kj} \phi_i(\bx_q^e) m_k(\bx_q^e)w_q\\
\end{aligned}
$$
最后一个式子中 $\bx_q^e$ 表示 $e$ 上第 $q$ 个
$k$ 次 Gauss-Lobatto 积分点,$w_q$ 表示对应的权重。
最后一个等式成立是因为 $\phi_i$ 在 $e$ 上是一个 $k$ 次多项式,那么 $\phi_i
\partial_n m_j$ 是一个 $2k-1$ 次多项式,所以可以 $k$ 次 Gauss-Lobatto
积分公式精确积分。
右端出现的 $L_{kj}, \delta_{i, k+N_1}, 
P^{K, \bn}_{kj}, \phi_i(\bx_q^e), m_k(\bx_q^e), w_q$ 都是已知的量。
记 $X_{ij}^e = \phi_i(\bx_j^w)w_j$,$W_{ij}^e = m_i(\bx_j)$,记上式右端为
$\tilde{\bG}$,那么 
$$
\tilde{\bG} = \bI_{N_1}\bL +
\sum_{e\in\mathcal{E}_K} \mathrm{sign}_{e, K} \bX^e (\bW^{e})^T\bP^{K, \bn} 
$$
定义矩阵 $\bB, \bG$:
$$
B_{ij} = \begin{cases}
    Q_0(m_j) & \quad \text{if $i = 0$}\\
    A_{ij}^K & \quad \text{if $i > 0$}
\end{cases},\quad 
G_{ij} = \begin{cases}
    Q_0(\phi_j) & \quad \text{if $j = 0$}\\
    \tilde{G}_{ij} & \quad \text{if $j > 0$}
\end{cases}
$$
那么有 $\boldsymbol{\Pi}^{1, K} = \bB^{-1}\bG^{T}$。 
下面三个注释给出了 $\bB, \bG$ 的计算过程。
\begin{remark}
矩阵 $\bB, \bG$ 实际上就是修改了 $\bA^K$ 和 $\tilde{\bG}$ 的第 $0$
行,分别使用了 $Q_0(m_j)$ 和 $Q_0(\phi_i)$ 替代。
实际上,原本 $\bA^K$ 和 $\tilde{\bG}$ 的第 $0$ 行都是 0。
\end{remark}
\begin{remark}
在矩阵 $\tilde{\bG}$ 的组装公式中,第一部分 $\bI_{N_1}\bL$ 前
$N_1$ 行都是 0,而第二部分 $\sum_{e\in\mathcal{E}_K} \mathrm{sign}_{e, K}
\bX^e (\bW^{e})^T\bP^{K, \bn}$ 在大于等于 $N_1$ 行的部分是 0。
进一步的,$\bX^e$ 仅在 $d^e$ 行上非零,取出 $\bX^e$ 的第 $\bd^e$ 行,记为
$\bX^e_{\bd^e}$,实际上 $\bX^e_{\bd^e}$ 是一个单位矩阵,但是在 $H^2$, $H^3$
虚单元空间中并不是,为了一致性仍然保留这个部分。
最终 $\tilde{\bG}$ 的组装过程见算法 \ref{alg:assembleGtilde}。

\begin{algorithm}
\caption{组装 $\tilde{\bG}$}\label{alg:assembleGtilde}
\begin{algorithmic}[1]
\Require $K$ \Comment{输入参数 $K$}
\Ensure $\tilde{\bG}$ \Comment{输出结果 $\tilde{\bG}$}

\State Compute $\bL_{\bd^K}, \bP^{K, \bn}_{\bd^K}, \bX^e_{\bd^e}, \bW^e_{\bd^e}$
\State $\tilde{\bG} \gets \boldsymbol{0}_{N_K \times N_K}$
\State $\tilde{\bG}_{\bd^K} \gets \bL_{\bd^K}$

\For{$e \in \mathcal{E}_K$}
    \State $\tilde{\bG}_{\bd^e} \gets \tilde{\bG}_{\bd^e} + \mathrm{sign}_{e, K} \bX^e_{\bd^e} (\bW^{e})^T \bP^{K, \bn}$
\EndFor

\State \Return $\tilde{\bG}$
\end{algorithmic}
\end{algorithm}
\end{remark}

\begin{remark}
$Q_0(\phi_i)$ 的计算结果可以直接给出:
$$
Q_0(\phi_i) = 
\begin{cases}
    \frac{1}{2}(|e_0| + |e_1|) & \quad \text{if $\delta_i = e_0\cap e_1$ and $k = 1$}\\
    \delta_{i, N_1} & \quad \text{if $k>1$}\\
\end{cases}
$$
\end{remark}

\subsubsection{自由度矩阵}
因为 $\mathbb{P}_k(K)\subseteq V_k(K)$,所以 $\mathcal{M}_k(K)$ 中的缩放单项式
可以由 $\{\phi_i\}$ 线性表出,即存在 $\bD^{K}$ 使得:
$$
m_{i} = \phi_j D^{K}_{ji} 
$$
对上式取自由度可得:
$$
\dof_k(m_i) = \dof_k(\phi_j) D^{K}_{ji} = \delta_{kj}D^{K}_{ji} = D^{K}_{ki}
$$
自由度矩阵会用于下面 $L^2$ 投影矩阵和后面椭圆方程求解时稳定化项的计算。

\subsubsection{$L^2$ 投影矩阵}
定义 $V_k(K)$ 到 $\mathbb{P}_{k}(K)$ 的 $L^2$ 投影算子为 $Q_k^{K}$ 满足:
$$
(Q_k^{K}v, q) = (v, q), \quad \forall v \in V_k, q \in \mathbb{P}_k(K)
$$
假设其在 $\{\phi_i\}$ 和 $\mathcal{M}_k$ 下的矩阵为 $\bQ^k$ 那么就有:
\begin{align}
\label{eq:proj2}
Q_{li}^{k}(m_l, m_j) = (\phi_i, m_j)
\end{align}
那么 $\bQ^{k-2} = \bM^{-1}_{k-2}\bI_{N_1, 0}$,$Q^{k-2}$ 也可以看做为
$V_{k}$ 到 $\mathbb{P}_{k}(K)$ 的投影算子,其矩阵形式 
$\tilde{\bQ}^{k-2}$ 实际上就是在 $\bQ^{k-2}$ 的后面加上了 $N =
\frac{(k+1)(k+2)-k(k-1)}{2}$ 列 0。特别的,对于 $m_i \in \mathcal{M}_k(K)$,其到
$\mathbb{P}_{k-2}(K)$ 的投影可以如下计算:
\begin{align}
\label{eq:proj3} 
Q_{k-2} m_i = Q_{k-2} \phi_j D_{ji} = m_j \tilde{Q}^{k-2}_{kj}D_{ji}
\end{align}
现在考察 $Q^{K}$ 的矩阵形式: 任取 $v \in V_k, q \in \mathbb{P}_k(K)$,有:
$$
\begin{aligned}
(Q_k v, q) & = (Q_k v, q-Q_{k-2} q) + (Q_k v, Q_{k-2} q)\\
& = (\Pi_k v, q-Q_{k-2} q) + (Q_{k-2} v, q)\\
& = (\Pi_k v, q) - (Q_{k-2} \Pi_k v, q) + (Q_{k-2} v, q)\\
\end{aligned}
$$
所以 $Q^{K}_k = \Pi^{1, K}_k - Q^{K-2}_k\Pi^{1, K}_k + Q^{K-2}_k$。
其矩阵形式为 
$$
\bQ^K = \bPi^{1, K} - \tilde{\bQ}^{K-2}\bD\bPi^{1, K} + \tilde{\bQ}^{K-2}
$$

\subsection{$H^1$ 非协调虚单元空间} 
非协调虚单元空间的定义类似与 $H^1$ 协调虚单元,这里仍然沿用
文献 [] 中的定义,首先定义多边形 $K$ 上的空间: 
$$
\tilde{V}_k(K) = \{v \in H^1(K): \frac{\partial v}{\partial \bn}|_{e} \in
    \mathbb{P}_{k-1}(e)\ \forall e \in \partial K, \Delta v \in
    \mathbb{P}_k(K)\}
$$
同样根据 \eqref{eq:proj1} 定义投影算子 $\Pi_k^{1, K}$,并定义虚单元空间 $V_k$ 为:
$$
V_k = \{v \in \tilde{V}_k: (v - \Pi_k^{1, K}v, q) = 0, \forall q \in
\mathbb{P}_{k-2}(K)^{\perp}\cap \mathbb{P}_k(K)\}
$$
注意当 $k<0$,$\mathbb{P}_k(K) = \emptyset$。
定义 $V_k$ 上的一组线性泛函 $\mathrm{dof}_i$:
$$
\mathrm{dof}_i(v) = \begin{cases}
    v(\bx_i^e) & \quad x_i^e \text{ 属于 $e$ 内部的 $k-1$ 个 Gauss-Legrande 点}\\
    \int_K v q_i\mathrm{d}\bx & \quad q_i \in \mathcal{M}_{k-2}(K)
\end{cases}
$$
对于函数 $v \in V_k$,可以被 $\{\mathrm{dof}_i(v)\}$ 唯一确定,定义
$V_k$ 的基函数 $\phi_i$ 满足与自由度对偶:
$$
\mathrm{dof}_i(\phi_j) = \delta_{ij}
$$
非协调虚单元函数不像协调虚单元函数一样在边上是多项式,但是其同样可以在边上和一个
$k-1$ 次多项式进行积分:
$$
\int_e v q\mathrm{d}s = \int_e Q_{k-1}v q\mathrm{d}s = \int_e
v(\bx_i^e) \psi_i q\mathrm{d}s = v(\bx_i^e) q(\bx_i^e)w_i
$$
其中 $\{\psi_i\}$ 是 $e$ 上由 $\{\bx_i^e\}$ 构成的 
$k-1$ 次 Lagrange 插值基函数。

定义网格 $\mathcal{T}_h$ 上的 $H^1$ 非协调虚单元空间 $V_h$ 为:
$$
V_h = \{v \in L^2(\Omega): v|_K \in V_h, \forall K \in \mathcal{T}_h, 
\int_{e} [\![v ]\!]q \dd s = 0 \ \forall e \in \mathcal{E}^i_h\}
$$

非协调虚单元的 $H^1$ 投影矩阵, $L^2$ 投影矩阵和自由度矩阵的计算与协调虚单元
类似,唯一区别是 $x_q^e$ 是 $e$ 上的 $k-1$ 个 Gauss-Legendre 积分点,
其他计算过程完全一致,这里不再赘述。
\subsection{$H^1$ 协调的张量虚单元空间}
在二维和三维问题中,我们通常会遇到未知量为向量甚至张量的情况,如线弹性问题中位移场
$\bu$,其属于 $H^1(\Omega, \mathbb{R}^d)$,我们可以将其分解为 $d$ 个标量场
$\bu = (u_1, u_2, \cdots, u_d)$,其中 $u_i \in H^1(\Omega)$。
这样的空间可以通过将标量的 $H^1$ 协调或非协调虚单元空间 $V_h$
进行张量积来进行逼近。

严格来说定义张量空间 $\mathbb{E}$ 的基为 $\{\bE_i\}$,定义 $\bV_h = V_h\otimes
\mathbb{E}$,
那么 $\bV_h$ 的基为 $\{\phi_i \otimes \bE_j\}$,其中 $\phi_i$ 是 $V_h$ 的基。
对于单元 $K$ 上的局部虚单元空间 $\bV_h^K$,同样可以定义 $H^1$ 投影算子和 $L^2$


\subsection{$H(\curl)$ 协调虚单元空间}
本节主要介绍 $H(\curl)$ 协调虚单元空间的构建,该空间在文献 [] 中被首次定义,
其定义如下:
$$
V_k(K) = \{\bv \in H(\curl, K)\cap H(\diver, K): \bv\cdot \bt|_{e} \in
\mathbb{P}_k(e), \curl \bv \in \mathbb{P}_{k-1}(K), 
\diver \bv \in \mathbb{P}_{k-1}(K)\}
$$
其自由度为:
\begin{align}
    \bv(\bx_i^e)\cdot \bt & \quad \forall e \in \mathcal{E}_K, \bx_i^e\in G^{0, e}_k\\
    \int_K \bv \cdot \bq \dd \bx & \quad \forall \bq \in
    \bcurl(\mathcal{M}_{k-1}(K)\setminus m_{\boldsymbol{0}})\\
    \int_K \bv \cdot \bq \dd \bx & \quad \forall \bq \in
    \bx\mathcal{M}_{k-1}(K)
\end{align}
定义 $\{\dof_i\}$ 为上述自由度按顺序排列,$\{\bphi_i\}$ 为 $V_k(K)$
的基函数,满足与自由度对偶:
$$
\dof_i(\bphi_j) = \delta_{ij}
$$
定义网格 $\mathcal{T}_h$ 上的 $H(\curl)$ 协调虚单元空间 $V_h$ 为:
$$
V_h = \{\bv \in H(\curl, \Omega): \bv|_K \in V_k(K), \forall K \in \mathcal{T}_h\}
$$
\subsubsection{自由度矩阵}
定义 $\bD^d$ 为 $\mathcal{M}_k(K) \otimes \be_d$ 的自由度矩阵:
$$
D_{ij}^d = \dof_i(m_j\otimes \be_d)
$$
那么有:$m_i\otimes \be_d = \bphi_j D_{ji}^d$
\subsubsection{$\curl$ 的矩阵表示}
$V_k(K)$ 中的函数的旋度属于 $\mathbb{P}_{k-1}(K)$,所以 $\curl$ 算子可以看做
是从 $V_k(K)$ 到 $\mathbb{P}_{k-1}(K)$ 的线性算子,记其在 $\{\bphi_i\}$ 和 
$\mathcal{M}_{k-1}(K)$ 下的矩阵为 $\bPi^{\bcurl, K}$,简记为 $\bPi^{\bcurl}$, 
那么有:
$$
\curl \phi_i = m_j \Pi^{\bcurl}_{ji} 
$$
根据 Stokes 公式:
$$
\int_K \curl \bv \ q \dd \bx = \int_K \bv \cdot \bcurl q \dd \bx
+ \int_{\partial K} \bv \cdot \bt\ q \dd s
$$
和自由度的定义,$C_{ij}$ 可以通过下面的公式计算:
$$
\begin{aligned}
\int_K \curl \bphi_i \ m_j \dd \bx & = \int_K \bphi_i\cdot \bcurl m_j \dd \bx 
+ \int_{\partial K} \bphi_i\cdot \bt\ m_j \dd s\\
& = \bI_{N_1} + \sum_{e\in \mathcal{E}_K}\mathrm{sign}_{e, K} 
\bphi_i(\bx_q^e)\cdot \bt \ m_j(\bx_q^e)w_q 
\end{aligned}
$$
左段为 $\bM_{k-1}\bPi^{\bcurl}$,右端记为 $\bG$。
所以 
$$
\bPi^{\bcurl}
= \bM_{k-1}^{-1}\bG
$$
在 $\bG$ 的计算公式中,记
$X^e_{iq} = \bphi_i(\bx_q^e)\cdot \bt w_q$,$\bW^e_{jq} = m_j(\bx_q^e)$,
与 $H^1$ 协调虚单元空间的计算过程类似,
$\bX^e$ 仅在 $d^e$ 行上非零,$\bX^e_{\bd^e} = \diag\{w_i\}$ 是一个对角矩阵,与
$K, e$ 无关,$\bG$ 的计算过程在算法 \ref{alg:assembleG} 中给出。

\begin{algorithm}
\caption{组装 $\bG$}\label{alg:assembleG}
\begin{algorithmic}[1] % [1] 表示带行号
\Require $K$ 
\Ensure $\bG$ 

\State Compute $\bX^e, \bW^e$
\State $\bG \gets (\bI_{N_1})_{N_K \times N_K}$

\For{$e \in \mathcal{E}_K$}
    \State $\bG_{\bd^e} \gets \bG_{\bd^e} + \mathrm{sign}_{e, K} \bX^e (\bW^{e})^T$
\EndFor

\State \Return $\bG$
\end{algorithmic}
\end{algorithm}

\subsubsection{$L^2$ 投影矩阵}
定义 $V_k(K)$ 到 $\mathbb{P}_{k-1}(K)$ 的 $L^2$ 投影算子为 $\Pi^{K, 0, k}$
简写为 $\Pi$ 满足:
$$
(\bq, \Pi\bv)_K = (\bq, \bv), \quad \forall \bv \in V_k, \bq \in 
\mathcal{M}_{k}^2(K)
$$
假设 $\Pi$
在 $\{\bphi_i\}$ 和 $\mathcal{M}_{k-1}(K)$ 下的矩阵表示如下:
$$
\Pi^{0, k} \bphi_i = m_j\otimes \be_d \Pi^d_{ji}, \quad 
$$
根据性质 \eqref{prop:polydecomp} 中的缩放单项式的分解,可得:
\begin{equation}
\label{eq:hcurll2proeq0}
\begin{aligned}
    (m_i\otimes \be_d, Q_k^{K}\bphi_j) & = (m_i\otimes \be_d, \bphi_j)\\
    & = (C^{\bcurl, d}_{li}\bcurl m_l + C^{X, d}_{li}\bx m_l, \bphi_j)\\
    & = C^{\bcurl, d}_{li}(\bcurl m_l, \bphi_j) + C^{X, d}_{li}(\bx m_l, \bphi_j)\\
\end{aligned}
\end{equation}
其中 $(\bx m_l, \bphi_j) = \delta_{l+N_1, j}$,$N_1 = NE \times k + N_0$,
$(\bcurl m_l, \bphi_j)$ 需要使用分部积分公式计算:
$$
\begin{aligned}
(\bcurl m_l, \bphi_j) & = (m_l, \curl \bphi_j) + \sum_{e\in \mathcal{E}_K}
\mathrm{sign}_{e, K}\langle m_l, \bphi_j\cdot\bt_e \rangle_e\\
& = (m_l, m_i)\Pi_{ij}^{\bcurl} +
\sum_{e\in \mathcal{E}_K}\mathrm{sign}_{e, K}m_l(\bx_q^e)\bphi_j(\bx_q^e)\cdot \bt_e
w_q
\end{aligned}
$$
令 $X^e_{qj} = \bphi_j(\bx_q^e)\cdot \bt_e w_q$,$W^e_{lq} = m_l(\bx_q^e)$,
代入到 \eqref{eq:hcurll2proeq0} 中,得到:
$$
\bM\bPi^d = (\bC^{\bcurl, d})^T\left(\bM_{k}\bPi^{\bcurl} + \sum_{e\in \mathcal{E}_K} 
\sign_{e,K}\bW^e\bX^e\right) + (\bC^{X, d})^T\bI_{N_1, 0}
$$
其中 $\bM$ 是 $\mathcal{M}_k(K)$ 的质量矩阵。
组装算法
\begin{algorithm}
\caption{组装 $\bQ^d$}\label{alg:assembleQd}
\begin{algorithmic}[1]
\Require $K$
\Ensure $\bQ^d$
\State Compute $\bPi^{\bcurl}, \bX^e, \bW^e, \bC^{\bcurl, d}, \bC^{X, d}$
\State $\bQ^d_{\bd^K} \gets (\bPi^{\bcurl})^T\bM_{k}$
\For{$e \in \mathcal{E}_K$}
    \State $\bQ^d_{\bd^e} \gets \bQ^d_{\bd^e} + \bX^e (\bW^e)^T\bC^{\bcurl, d}\bM_k^{-1}$
\EndFor
\State $\bQ^d \gets \bM_k^{-1}\bQ^d + \bI_{0, N_1}\bC^{X, d} \bM_k^{-1}$
\end{algorithmic}
\end{algorithm}

\subsection{$H(\diver)$ 协调虚单元空间}
定义 $\boldsymbol{IP}^{k}_e$ 为边 $e$ 上的 $k+1$-高斯-勒让德积分的积分点。
$H(\diver)$ 协调虚单元空间的定义如下:
$$
\begin{aligned}
V_{k}^{face}(K) := \{\bv & \in H(\diver, K)\cap H(\curl, K): \bv\cdot\bn|_e \in 
    \mathbb{P}_k(e) \ \forall e \in \mathcal{E}(K), \\
    & \grad\ \diver\ \bv \in \nabla \mathbb{P}_{k-1}(K), \curl \bv \in 
    \mathbb{P}_{k-1}(K)\}
\end{aligned}
$$
其自由度为:
$$
\begin{aligned}
    \bv(\bx_i^e)\cdot \bn & \quad \forall e \in \mathcal{E}_K, \bx_i^e\in
    \boldsymbol{IP}^{k}_e\\
    \int_K \bv \cdot \bq \dd \bx & \quad \forall \bq \in
    \grad(\mathcal{M}_{k-1}(K)\setminus m_0)\\
    \int_K \bv \cdot \bq \dd \bx & \quad \forall \bq \in
    \bx\mathcal{M}_{k-1}(K)
\end{aligned}
$$
定义网格 $\mathcal{T}_h$ 上的 $H(\diver)$ 协调虚单元空间 $V_h$ 为:
$$
V_h = \{\bv \in H(\diver, \Omega): \bv|_K \in V_k(K), \forall K \in
\mathcal{T}_h\}
$$

\subsubsection{散度的 $L^2$ 投影的矩阵表示}
根据自由度可以计算
$V_k(K)$ 中函数的散度
到 $\mathbb{P}_{k-1}(K)$ 的 $L^2$ 投影,定义 $\Pi^{\diver}$: $V_k^{face}(K) \to
\mathbb{P}_{k-1}(K)$ 满足:
$$
(\bq, \Pi^{\diver} \bv) = (\bq, \diver\bv), \quad \forall \bv \in V_k, \bq \in
\mathcal{M}_{k-1}(K)
$$
即其在 $\{\bphi_i\}$ 和 $\mathcal{M}_{k-1}(K)$ 下的矩阵为 $\bPi^{\diver}$:
$$
\Pi^\diver \bphi_i = m_j \Pi^{\diver}_{ji}
$$
那么有:
$$
\begin{aligned}
    (m_i, \Pi^{\diver}\bphi_j) & = (m_i, \diver \bphi_j)\\
    & = (\grad m_i, \bphi_j) + \langle m_i, \bphi_j\cdot \bn \rangle_e\\ 
    & = \delta_{i+1, j+N1+1} + \sum_{e\in \mathcal{E}_K} \sign_{e, K}
    m_i(\bx_q^e)\bphi_j(\bx_q^e)\cdot \bn_e w_q
\end{aligned}
$$
记 $X^e_{qj} = \bphi_j(\bx_q^e)\cdot \bn_e w_q$,$W^e_{iq} = m_i(\bx_q^e)$,
那么有:
$$
\bM_k\bPi^{\diver} = \bI_{N_1}\bI_{N_1} + \sum_{e\in \mathcal{E}_K} \sign_{e, K}
\bW^e\bX^e
$$
注意 $X^e_{qj}$ 仅在 $d^e$ 行上非零,$\bX^e_{\bd^e} = \diag\{w_i\}$
是一个对角矩阵。
$\bPi^{\diver}$ 的计算过程在算法 \ref{alg:assemblePidiver} 中给出。
\begin{algorithm}
    \caption{组装 $\bPi^{\diver}$}\label{alg:assemblePidiver}
    \begin{algorithmic}[1]
    \Require $K$
    \Ensure $\bPi^{\diver}$
    \State Compute $\bX^e, \bW^e$
    \State $\bPi^{\diver}_{\bd^K} \gets \bI_{N_1}$
    \For{$e \in \mathcal{E}_K$}
        \State $\bPi^{\diver}_{\bd^e} \gets \bPi^{\diver}_{\bd^e} + \sign_{e, K}\bW^e\bX^e$
    \EndFor
    \State \Return $\bPi^{\diver}$
\end{algorithmic}
\end{algorithm}

\subsubsection{$L^2$ 投影矩阵}
定义 $V_k(K)$ 到 $\mathbb{P}_{k}(K)$ 的 $L^2$ 投影算子为 $\Pi^{K, k}$ 满足:
$$
(\bq, \Pi^{K, k}\bv) = (\bq, \bv), \quad \forall \bv \in V_k, \bq \in
\mathcal{M}_{k}(K)
$$
根据自由度可以计算出其在 $\{\bphi_i\}$ 和 $\mathcal{M}_{k-1}(K)$ 下的矩阵
$\bPi$:
$$
\Pi^{K, k}\bphi_j = m_l \otimes \be_d \Pi_{lj}^d
$$
根据性质 \eqref{prop:polydecomp} 中的缩放单项式的分解,得到其分解系数矩阵
$\bC^{\nabla, d}$ 和 $\bC^{X, d}$,那么有:
$$
\begin{aligned}
    (m_i\otimes \be_d, \Pi_k^{K}\bphi_j) & = (m_i\otimes \be_d, \bphi_j)\\
    & = (C^{\nabla, d}_{li}\nabla m_l + C_{li}^{X, d}\bx m_l, \bphi_j)\\
    & = C_{li}^{\nabla, d}(\nabla m_l, \bphi_j) + C_{li}^{X, d}(\bx m_l, \bphi_j)
\end{aligned}
$$
其中 $(\bx m_l, \bphi_j) = \delta_{l, j-N_1}$, $(\nabla m_l, \bphi_j)$,
使用分布积分进行计算:
$$
\begin{aligned}
(\nabla m_l, \bphi_j) & = (m_l, \diver\bphi_j) + \langle m_l, \bphi_j\cdot \bn
\rangle_e \\
& = (m_l, m_i)\Pi_{ij}^{\diver} + \sum_{e\in \mathcal{E}_K}\sign_{e, K}
m_l(\bx_q^e)\bphi_j(\bx_q^e)\cdot \bn_e w_q
\end{aligned}
$$
记 $X^e_{qj} = \bphi_j(\bx_q^e)\cdot \bn_e w_q$,$W^e_{lq} = m_l(\bx_q^e)$,
那么有:
$$
\bM_k\bPi = (\bC^{\nabla, d})^T\left(\bM_k\bPi^{\diver} + \sum_{e\in
    \mathcal{E}_K}
\sign_{e, K}\bW^e\bX^e\right) + (\bC^{X, d})^T\bI_{N_1, 0}
$$
的组装过程见算法 \ref{alg:assemblePi}。
\begin{algorithm}
    \caption{组装 $\bPi$}\label{alg:assemblePi}
    \begin{algorithmic}[1]
    \Require $K$
    \Ensure $\bPi$
    \State Compute $\bPi^{\diver}, \bX^e, \bW^e, \bC^{\nabla, d}, \bC^{X, d}$
    \State $\bPi_{\bd^K} \gets (\bPi^{\diver})^T\bM_{k}$
    \For{$e \in \mathcal{E}_K$}
        \State $\bPi_{\bd^e} \gets \bPi_{\bd^e} + \bX^e (\bW^e)^T\bC^{\nabla, d}\bM_k^{-1}$
    \EndFor
    \State $\bPi \gets \bM_k^{-1}\bPi + \bI_{0, N_1}\bC^{X, d} \bM_k^{-1}$
\end{algorithmic}
\end{algorithm}

\subsection{$H^m$ 协调虚单元空间}
重申一下 $H^m$ 协调虚单元空间的定义,首先是定义一个大空间 $\tilde{K}$:
$$
\begin{aligned}
    \tilde{V}_k^m(K) := \{v & \in H^m(K): (-\Delta^m)v \in \mathbb{P}_k(K), \\
        & \frac{\partial^j v}{\partial \bn^j}|_e \in V_{k-j}^{m-j}(e)
    \ \forall e \in \mathcal{E}_K, 0\leq j \leq m-1\}
\end{aligned}
$$
其中 $V_k^{m}(e) = \mathbb{P}_{\max\{k, 2m-1\}}$。
定义一个从 $\tilde{V}_k^m(K)$ 到 $\mathbb{P}_k(K)$ 的 $H^m$ 投影算子 $\Pi^{m,
K}$ 满足:
\begin{equation}
\label{eq:hmproj}
\begin{aligned}
(\nabla^m q, \nabla^m\Pi^{m, K}v) & = (\nabla^m q, \nabla^m v)  \quad q \in
\mathbb{P}_k(K)\\
Q^{\balpha}(\Pi^{m, K}v) & = Q^{\balpha}(v) \quad  \balpha \in
\mathbb{T}_{1}^j, 0\leq j \leq m-1
\end{aligned}
\end{equation}
其中 $Q^{\balpha}$ 定义如下:
$$
Q^{\balpha}(v) = \sum_{\delta} h_{\delta}^{|\balpha|} \frac{\partial^{|\balpha|}
v}{\partial \bx^{\balpha}}(\delta)
$$
局部虚单元空间的自由度定义如下:
$$
\begin{aligned}
h_\delta^j (\frac{\partial^j v}{\partial \bx^{\balpha}})(\delta) &
    \quad \forall \delta \in \mathcal{V}(K), \balpha \in \mathbb{T}_{1}^j, 0\leq
    j \leq m-1\\
    h_e^{j-1}(\frac{\partial^{j} v}{\partial \bn_e^j}, q) & \quad \forall e \in
    \mathcal{E}_K, q \in \mathcal{M}_{k-2m+j}(e), 0\leq j \leq m-1\\
    \frac{1}{|K|}(v, q) & \quad \forall q \in \mathcal{M}_{k-2m}(K)
\end{aligned}
$$
使用 $\bd_{\delta}^i$ 表示所有 $\delta$ 上与$i$ 阶梯度有关的自由度,
$\bd_{\delta} = \{\bd_{\delta}^0, \ldots, \bd_{\delta}^{m-1}\}$ 表示所有
$\delta$ 上的自由度,
$\bd_e^j$ 表示 $e$ 上与 $j$ 阶法向导数有关的自由度,
$\bd_e = \{\bd_e^0, \ldots, \bd_e^{m-1}\}$ 表示 $e$ 上的所有自由度。
$\bd_K$ 表示 $K$ 内部自由度。
晶格顺序自然的给出了 $\bd_{\delta}^i$,$\bd_e^j$
的排序,从而给出了所有自由度排序 $\{\bd_{\delta}, \bd_e, \bd_K\}$。
另外定义其对偶基函数为 $\bphi_j$.

定义 $V_k^m(K)$ 为 $\tilde{V}_k^m(K)$ 的子空间:
$$
V_k^m(K) = \{v \in \tilde{V}_k^m(K): (v-\Pi^{K}_kv, q) = 0,\ \forall q \in
\mathbb{P}_{k-2m}(K)^{\perp}\cap \mathbb{P}_k(K)\}
$$
\subsubsection{$H^m$ 协调虚单元函数在边界上的表示}
对于 $v \in V_k^m(K)$,其 $j\leq m-1$
阶法向导数在边界上都是多项式,与前面的自由度不同,
由于顶点上有 $m-1$ 阶导数作为自由度,
自由度都分配在了顶点上,这样边内部用高斯积分点上的点值做自由度的话积分精度不够,
所以我们需要根据自由度将法向导数算出来。

考虑一条边 $e$, 定义如下自由度:
$$
\begin{aligned}
    \partial_{\bt_e}^{j}v(e_i) & \quad i = 0, 1; 0\leq j \leq l-1\\
b^{\balpha}(v) & \quad \balpha \in \mathbb{T}_{1}^{k-2l}, \balpha > l-1 
\end{aligned}
$$
实际上这些自由度就是 $H^m$ 协调虚单元空间的自由度限制在边上的情况。
记顶点上的自由度为 $\tilde{\bd}_{e_i} : \{v(e_i), \partial_{\bt_e}v(e_i),
\ldots, \partial_{\bt_e}^{l-1}v(e_i)\}$,边上的自由度为 $\tilde{\bd}_e$
按照晶格顺序排序,所有自由度为 $\tilde{\bd}^{e, l} := \{\tilde{\bd}_{e_0}, \tilde{\bd}_e,
\tilde{\bd}_{e_1}\}$ 即: 
$$
\{v(e_0), \partial_{\bt_e}v(e_0), \ldots, \partial_{\bt_e}^{l-1}v(e_0),
    b^{(k-l, l)}(v), \ldots, b^{(l, k-l)}(v), v(e_1), \partial_{\bt_e}v(e_1),
\ldots, \partial_{\bt_e}^{l-1}v(e_1)\}
$$
现在我们根据 Bernstein 多项式找 $\tilde{\bd}^{e, l}$ 的对偶基函数。
将 Bernstein 多项式做如下排序:
$$
\{B^{(k, 0)}, B^{(k-1, 1)}, \ldots, B^{(0, k)}\}
$$
第 $i$ 个 Bernstein 多项式记为 $B_i$。那么有:
$$
(\tilde{d}^{e, l}_i(B_j))_{(k+1)\times (k+1)} = 
\begin{pmatrix}
    \boldsymbol{D}_0^{e, l} & 0 & \cdots & 0\\
    0 & \boldsymbol{I} & \cdots & 0\\
    0 & 0 & \cdots & \boldsymbol{D}_1^{e, l}
\end{pmatrix}
$$
其中 $\boldsymbol{D}_0^{e, l}$ 和 $\boldsymbol{D}_1^{e, l}$ 分别是
$$
\boldsymbol{D}^{e, l}_0 = \begin{pmatrix}
    1 & 0 & \cdots & 0\\
    \frac{1}{|e|} & \frac{1}{|e|} & \cdots & 0\\
    \vdots & \vdots & \ddots & \vdots\\
    \frac{1}{|e|^{l-1}} & \frac{1}{|e|^{l-1}} & \cdots & \frac{1}{|e|^{l-1}}
\end{pmatrix}
= 
\diag{\{1, \frac{1}{|e|}, \ldots, \frac{1}{|e|^{l-1}}\}}
\begin{pmatrix}
    1 & 0 & \cdots & 0\\
    1 & 1 & \cdots & 0\\
    \vdots & \vdots & \ddots & \vdots\\
    1 & 1 & \cdots & 1
\end{pmatrix}
$$
$$
\boldsymbol{D}^{e, l}_1 = \begin{pmatrix}
    0 & \cdots & 0 & 1\\ 
    0 & \cdots & \frac{1}{|e|} & \frac{1}{|e|}\\
    \vdots & \iddots & \vdots & \vdots\\
    \frac{1}{|e|^{l-1}} & \cdots & \frac{1}{|e|^{l-1}} & \frac{1}{|e|^{l-1}}
\end{pmatrix}
= 
\bD^{e, l}_0 
\begin{pmatrix}
    0 & \cdots & 0 & 1\\ 
    0 & \cdots & 1 & 0\\
    \vdots & \iddots & \vdots & \vdots\\
    1 & \cdots & 0 & 0
\end{pmatrix}
$$
其逆矩阵如下:
$$
(\boldsymbol{D}^{e, l}_0)^{-1} = \begin{pmatrix}
    1 & 0 & \cdots & 0\\
    -1 & 1 & \cdots & 0\\
    \vdots & \vdots & \ddots & \vdots\\
    0 & 0 & \cdots & 1
\end{pmatrix}
\diag\{1, |e|, \ldots, |e|^{l-1}\}
, \quad
(\boldsymbol{D}^{e, l}_1)^{-1} = \begin{pmatrix}
    0 & \cdots & 0 & 1\\
    0 & \cdots & 1 & 0\\
    \vdots & \iddots & \vdots & \vdots\\
    1 & \cdots & 0 & 0
\end{pmatrix}
(\bD^{e, l}_0)^{-1}
$$
定义 
$$
\bC^{e, l} = \begin{pmatrix}
    (\boldsymbol{D}^{e, l}_0)^{-1} & 0 & \cdots & 0\\
    0 & \boldsymbol{I} & \cdots & 0\\
    0 & 0 & \cdots & (\boldsymbol{D}^{e, l}_1)^{-1} 
\end{pmatrix}
$$
那么有:
$$
\tilde{d}^{e, l}_i(B_jC_{jk}^{e, l}) = \delta_{ik}
$$
所以 $\{B_jC_{jk}^{e, l}\}$ 是 $\tilde{\bd}^{e, l}$ 的对偶基函数:
那么对于虚单元基函数 
$\phi_i$:
\begin{equation}
\label{eq:computeplnphi}
\partial^l_{\bn_e}\phi_i|_e(\bx_q) = B_j^{k-l}(\bx_q)C_{ja}^{e, m-l} 
\tilde{d}^{e, m-l}_{a}(\partial^{l}_{\bn_e}\phi_i)
\end{equation}
因为 $\tilde{\bd}^{e, m-l}\circ \partial_{\bn_e}^l$ 是虚单元空间的自由度的一部分,
可以由 $\bd^l := \{\bd_{e_0}, \bd_{e}^l, \bd_{e_1}\}$
线性表出,
根据高阶导数的计算公式:
$$
\partial_{\bt}^a \partial_{\bn}^b \phi_i = \nabla^{a+b} \phi_i : \bt^a\otimes\bn^b
= \frac{(a+b)!}{\balpha!}\partial_{\bx^{\balpha}}^{a+b}\phi_i 
\sym(\be^{\balpha}): \bt^a\otimes\bn^b
= \frac{(a+b)!}{\balpha!}\sym(\bt^a\otimes\bn^b)_{\balpha}
\partial_{\bx^{\balpha}}^{a+b}\phi_i 
$$
定义 $\{\frac{(a+b)!}{\balpha!}\sym(\bt^a\otimes\bn^b)_{\balpha}\}_{\balpha \in
T_1^{a+b}}$ 
组成向量 $\bT^{a, b}$,那么其可以看做 $\bd^{e,
m-l}$ 在 $\bd^l$ 上的系数表示:
定义如下矩阵:
$$
\bT^{l}_{\delta} = 
\begin{pmatrix}
    \bT^{0, l} & 0 & \cdots & 0\\
    0 & \bT^{1, l} & \cdots & 0\\
    \vdots & \vdots & \ddots & \vdots\\
    0 & 0 & \cdots & \bT^{m-l, l} 
\end{pmatrix}
,\quad
\bT^{l} = \diag\{\bT^{l}_{\delta},  \bI, \bT^{l}_{\delta},\}
$$
那么我们有:
$$
\tilde{\bd}^{e, m-l}\circ \partial_{\bn_e}^l = \bT^{l}\bd^{l}
$$
所以根据 \eqref{eq:computeplnphi} 可以得到:
\begin{equation}
\label{eq:computepltplnphi}
\begin{aligned}
\partial_{\bn_e}^l\phi_{\bd^l}|_e(\bx_q) & = B_j^{k-l}(\bx_q)\bC_{j}^{e, m-l}
\bT^l\\
\partial_{\bt_e}^a\partial_{\bn_e}^l\phi_{\bd^l}|_e(\bx_q) & =
\partial_{\bt_e}^aB_j^{k-l}(\bx_q)\bC_{j}^{e, m-l}\bT^l
\end{aligned}
\end{equation}
这样就得到了虚单元函数在边界上小于 $m$ 阶的法向导数及其切向导数的计算方式。
\subsubsection{二维格林公式}
二维情况下,有以下格林公式:
\begin{equation}
\label{eq:greem2d}
\begin{aligned}
    (\nabla^m v, \nabla^m q) & = (v, (-\Delta)^m q) + \sum_{i=0}^{m-1}
    (\nabla^i v, \nabla^{i}((-\Delta)^{m-i-1}\partial_{\bn} q))_{\partial K}\\
\end{aligned}
\end{equation}
对边 $e$ 上的函数 $\nabla^i v$ 进行分解:
$$
\nabla^i v = \sum_{j=0}^{i} C_i^j \partial^j_{\bt} \partial^{i-j}_{\bn} v
\ \sym({\bt^j\otimes\bn^{i-j}})
$$
其中 $C_i^j = \frac{i!}{j!(i-j)!}$, 
\textbf{
$\bt, \bn$ 分别是 $e$ 上的单位切向量和单位法向量,$\bt$ 的方向是单元的逆时针方向,
$\bn$ 的方向是指向单元外部。
} 
代入到 \eqref{eq:greem2d} 中,得到:
\begin{equation}
\begin{aligned}
(\nabla^i v, \nabla^{i} q)_e & = 
\sum_{j=0}^{i} (C_i^j)^2 
(\partial^j_{\bt} \partial^{i-j}_{\bn} v, \partial^{j}_{\bt}
\partial^{i-j}_{\bn} q)_e\\
\end{aligned}
\end{equation}
记 $\tilde{q}^{i, e} := (-\Delta)^{m-i-1} \partial_{\bn_e} q|_e$,将上式代入
\eqref{eq:greem2d} 中,得到:
\begin{equation}
\label{eq:greem2d2}
\begin{aligned}
    (\nabla^m v, \nabla^m q) & = (v, (-\Delta)^m q) + \sum_{i=0}^{m-1}
    \sum_{e\in \mathcal{E}(K)}
    (\nabla^i v, \nabla^{i}\tilde{q}^{i,e})_{e}\\
    & = (v, (-\Delta)^m q) + \sum_{e\in \mathcal{E}(K)} \sum_{i=0}^{m-1}
    \sum_{j=0}^{i} (C_i^j)^2 (\partial^j_{\bt} \partial^{i-j}_{\bn} v,
    \partial^{j}_{\bt} \partial^{i-j}_{\bn} \tilde{q}^{i,e})_e
\end{aligned}
\end{equation}
可以看到公式中包括了非常多各式各样的导数,似乎很难计算,但实际上 
$\partial_{\bt}^j\partial_{\bn}^{i-j} v$ 上一节已经讨论过其计算方式
\eqref{eq:computepltplnphi},
另一方面当 $q$ 是一个缩放单项式时,$\partial_{\bt}^j\partial_{\bn}^{i-j}
\tilde{q}^{i,e}$ 可以根据缩放单项式的性质进行计算。

\subsubsection{$H^m$ 投影矩阵}
根据 \eqref{eq:hmproj} 定义 $H^m$ 投影算子 $\Pi^{m, K}$,其
也是 $V_k^m(K)$ 到 $\mathbb{P}_k(K)$ 的线性算子,记其在 $\{\phi_i\}$ 和
$\mathcal{M}_{k}(K)$ 下的矩阵为 $\bPi^m$,即:
$$
\Pi^{m, K}\phi_j = m_l\Pi_{lj}^m
$$
代入到 \eqref{eq:hmproj} 中,得到:
\begin{equation}
\begin{aligned}
    Q^{\balpha}(m_l)\Pi^{m}_{li} & = Q^{\balpha}(\phi_i) \quad \balpha \in
    \mathbb{T}_{1}^j, 0\leq j \leq m-1\\
    (\nabla^m m_i, \nabla^m m_l) \Pi^m_{lj}& = (\nabla^m m_i, \nabla^m \phi_j)
\end{aligned}
\end{equation}
第一个等式代表了 $\#(\cup_{j=0}^{m-1} \mathbb{T}_{1}^j) =: N_{\ker}$ 个方程,
将 $\cup_{j=0}^{m-1} \mathbb{T}_{1}^j$ 进行字典排序,用 $Q^i$ 表示第 $i$ 个
$Q^{\balpha}$。
类似于 $H^1$ 协调虚单元空间中 $H^1$ 投影矩阵的计算,定义矩阵 $\bB, \bG$:
$$
B_{ij} = \left\{ 
\begin{aligned}
    & Q^i(m_j) \quad i < N_{\ker}\\
    & (\nabla^m m_i, \nabla^m m_{j}) \quad \text{otherwise}
\end{aligned}
\right., 
\quad
G_{ij} = \left\{
\begin{aligned}
    & Q^i(\phi_j) \quad i < N_{\ker}\\ 
    & (\nabla^m \phi_j, \nabla^m m_{i}) \quad \text{otherwise}
\end{aligned}
\right.
$$
那么有 $\bPi^m = \bB^{-1}\bG^T$。
两个矩阵都有两部分:前 $N_{\ker}$ 行和大于 $N_{\ker}$ 行。
现在考虑两个矩阵的组装,对于矩阵
$\bB$,第一部分的计算可以根据缩放单项式的性质进行计算:
$$
Q^{\balpha}(m_l) = \sum_{\delta} h_{\delta}^{|\balpha|}
\frac{\partial^{|\balpha|} m_l}{\partial \bx^{\balpha}}(\delta)
= \sum_{\delta} h_{\delta}^{|\balpha|} m_j(\delta)
P^{\balpha}_{jl} 
$$
第二部分实际上就是缩放单项式的 $m$ 阶刚度矩阵:
$$
B_{ij} = A_{ij}^m \quad i \geq N_{\ker}
$$
$\bG$ 第一部分 $\bG^0$ 是 $N_v$ 个单位矩阵组成
$$
Q^{\balpha}(\phi_j) = \sum_{i=0}^{N_v-1} h_{\delta_i}^{|\balpha|}
\frac{\partial^{|\balpha|} \phi_j}{\partial \bx^{\balpha}}(\delta_i)
= \sum_{i=0}^{N_v-1} \delta_{\balpha, i*N_{\mathrm{ker}} + j}
$$
即:
\begin{equation}
\bG^{0} = 
\begin{pmatrix}
    \bI_{N_{\ker}} & \bI_{N_{\ker}} & \cdots & \bI_{N_{\ker}} & \boldsymbol{0}\\
\end{pmatrix}
\end{equation}
第二部分的计算可以根据格林公式 \eqref{eq:greem2d2} 进行计算,有:
\begin{equation}
\begin{aligned} 
    (\nabla^m \phi_j, \nabla^m m_i) & = 
    (\phi_j, (-\Delta)^m m_i) + \sum_{e\in \mathcal{E}_K} \sum_{l=0}^{m-1}
    \sum_{j=0}^{l} (C_i^j)^2 (\partial^j_{\bt} \partial^{l-j}_{\bn} \phi_j,
    \partial^{j}_{\bt} \partial^{l-j}_{\bn} \tilde{m}^{l, e}_i)_e\\ 
\end{aligned}
\end{equation}
令 $X^{e, a, b}_{qj} = \partial_{\bt^e}^a\partial^b_{\bn_e}\phi_j(\bx_q^e) w_q$,
$\bP^{e, a, b, c} = 
(\bP^{\bt_e})^a(\bP^{\bn_e})^{b+1}
(-\bL)^{m-c-1}$,
$W^{e}_{iq} = m_i(\bx_q^e)$,
那么:
$$
\begin{aligned}
    (\partial_{\bt_e}^j\partial^{l-j}_{\bn} \phi_j,
    \partial^{j}_{\bt}\partial^{l-j}_{\bn} \tilde{m}^{l, e}_i)_e & = 
    P^{e, j, i-j,  l}_{ai} W^{e}_{aq} X^{e, j, l-j}_{qj}
\end{aligned}
$$
那么有:
$$
\begin{aligned}
    (\nabla^m \phi_j, \nabla^m m_i) = (-L_{ki})^m & \delta_{j, k+N_{\ker}} +
    \sum_{e\in \mathcal{E}_K} \sum_{l=0}^{m-1} \sum_{j=0}^{l} (C_i^j)^2
    P^{e, j, l-j, l}_{ai} W^{e}_{aq}
    X^{e, j, l-j}_{qj}
\end{aligned}
$$

%\begin{algorithm}
%    \caption{组装 $\bPi^m$}\label{alg:assemblePim}
%    \begin{algorithmic}[1]
%    \Require $K$
%    \Ensure $\bPi^m$
%    \State Compute $\bX^{e, l}, \bW^e, \bP^{e, a, b, c}$
%    \State $\bPi^m_{\bd^K} \gets \bI_{N_{\ker}}$
%    \For{$e \in \mathcal{E}_K$}
%        \For{$l=0$ to $m-1$}
%            \For{$i=0$ to $N_{\ker}$}
%                \For{$j=0$ to $N_{\ker}$}
%                    \State $\bPi^m_{\bd^e} \gets \bPi^m_{\bd^e} + \bP^{e, 2j, l, l}_{ai}
%                    \bW^e \bX^{e, l}_{\bd^e}$
%                \EndFor
%            \EndFor
%        \EndFor
%    \EndFor
%    \State \Return $\bPi^m$
%\end{algorithmic}
%\end{algorithm}

\subsubsection{$L^2$ 投影矩阵}
与 $H^1$ 协调虚单元类似,
定义 $V_k(K)$ 到 $\mathbb{P}_{k}(K)$ 的 $L^2$ 投影算子为 $\Pi^{K, 0, k}$ (简写为
$\Pi^{0, k}$) 满足:
$$
(\Pi^{0, k}v, q) = (v, q), \quad \forall v \in V_k, q \in \mathbb{P}_k(K)
$$
假设其在 $\{\phi_i\}$ 和 $\mathcal{M}_k$ 下的矩阵为 $\bPi^{0, k}$ 那么就有:
\begin{align}
\label{eq:proj2}
\Pi_{li}^{0, k}(m_l, m_j) = (\phi_i, m_j)
\end{align}
那么 $\bPi^{0, k-2m} = \bM^{-1}_{k-2m}\bI_{N_1, 0}$,$\Pi^{0, k-2m}$ 也可以看做为
$V_{k}$ 到 $\mathbb{P}_{k}(K)$ 的投影算子,其矩阵形式 
$\tilde{\bPi}^{k-2m}$ 实际上就是在 $\bPi^{0, k-2m}$ 的后面加上了几列 0 使之扩充为
$\frac{(k+1)(k+2)}{2}$ 列。特别的,对于 $m_i \in \mathcal{M}_k(K)$,其到
$\mathbb{P}_{k-2}(K)$ 的投影可以如下计算:
\begin{align}
\label{eq:proj3} 
\Pi^{0, k-2m} m_i = \Pi^{0, k-2m} \phi_j D_{ji} = m_j \tilde{\Pi}^{k-2m}_{kj}D_{ji}
\end{align}
其中 $\bD$ 是自由度矩阵,$D_{ij} = \dof_i(m_j)$。
现在考察 $Pi^{K}$ 的矩阵形式: 任取 $v \in V_k, q \in \mathbb{P}_k(K)$,有:
$$
\begin{aligned}
    (\Pi^{0, k} v, q) & = (\Pi^{0, k} v, q-\Pi^{0, k-2m} q) + 
    (\Pi^{0, k} v, \Pi^{0, k-2m} q)\\
    & = (\Pi^{0, k} v, q-\Pi^{0, k-2m} q) + (\Pi^{0, k-2m} v, q)\\
    & = (\Pi^{0, k} v, q) - (\Pi^{0, k-2m} \Pi^m v, q) + (\Pi^{0, k-2m} v, q)\\
\end{aligned}
$$
所以 $\Pi^{0, k} = \Pi^{m} - \Pi^{0, k-2m}\Pi^{m} + \Pi^{0,
k-2m}$。
其矩阵形式为 
$$
\bPi^{0, k} = \bPi^{m} - \tilde{\bPi}^{0, k-2m}\bD\bPi^{m} + \tilde{\bPi}^{0, k-2m}
$$

\section{常见方程的求解}
前面章节讨论了虚单元空间的构造,包括 $H^1$, $H(\div)$ 和 $H(\curl)$
协调虚单元空间,以及 $H^m$ 协调虚单元空间,这已经满足了大多数方程的求解,如
Poisson 方程,弹性方程,Maxwell 方程等,本节将讨论这些方程的求解。
\subsection{Poisson 方程}
考虑多边形区域 $\Omega$ 上带有低阶项的 Poisson 方程:
\begin{equation}
    \label{eq:poisson}
    \left\{
    \begin{aligned}
        -\Delta u +\beta u& = f \quad \text{in} \ \Omega\\
        u & = 0 \quad \text{on} \ \partial \Omega
    \end{aligned}
\right.
\end{equation}
其中 $\beta$ 是一个常数,$f \in L^2(\Omega)$,$g \in H^{1/2}(\partial \Omega)$。
首先将方程 \eqref{eq:poisson} 乘以一个测试函数 $v \in H^1_0(\Omega)$,并且使用
Green 公式,得到:
$$
\begin{aligned}
    (\nabla u, \nabla v) + \beta(u, v) & = (f, v)\\
\end{aligned}
$$
定义双线性形式 $a(u, v) = (\nabla u, \nabla v) + \beta(u, v)$,线性形式
$L(v) = (f, v)$,那么方程 \eqref{eq:poisson} 对应的变分问题为:找到 $u \in
H^1_0(\Omega)$ 使得:
$$
a(u, v) = L(v), \quad \forall v \in H^1_0(\Omega)
$$
\subsubsection{虚单元方法}
现在使用虚单元方法求解这个问题,首先将区域 $\Omega$ 分解为多边形网格
$\mathcal{T}_h$,定义虚单元空间 $V_h$ 为 $H^1$
协调虚单元空间或非协调虚单元空间,定义 $V_h$ 上的双线性形式 $a_h(u_h, v_h)$ 和
线性形式 $F_h(v_h)$ 为:
$$
\begin{aligned}
    a_h(u_h, v_h) & = \sum_{K \in \mathcal{T}_h}a_h^K(u_h, v_h)\\
    a_h^K(u_h, v_h) & = (\nabla \Pi^{1}u_h, \nabla \Pi^1 v_h)_K +
    \beta(\Pi^{0,k}u_h,
    \Pi^{0, k}v_h)_K + S^K((I-\Pi^1)u_h, (I-\Pi^1)v_h)\\
    F_h(v_h) & = \sum_{K\in \mathcal{T}_h}(f, \Pi^{0, k}v_h)_K
\end{aligned}
$$
其中 $\Pi^1$ 是 $H^1$ 投影算子,$\Pi^{0, k}$ 是 $L^2$ 投影算子,$S^K$ 是
虚单元空间的稳定项,其选取有很多种方式,这里选取为:
$$
S^K(u_h, v_h) = \bd(u_h)\cdot \bd(v_h) = d_i(u_h)d_i(v_h)
$$
其中 $\bd(u_h)$ 是 $u_h$ 在虚单元空间的自由度数组。
那么虚单元方法的离散问题为:找到 $u_h \in V_h$ 使得:
$$
a_h(u_h, v_h) = F_h(v_h), \quad \forall v_h \in V_h
$$
记 $V_h$ 的基函数为 $\{\phi_i\}_{i=1}^{N}$,$N_{dof}$ 是自由度的个数,
设 $u_h = U_i \phi_i$, $A_{ij} = a_h(\phi_i, \phi_j)$,$F_i = F_h(\phi_i)$,那么有:
$$
\bA \bU = \bF
$$
\subsubsection{矩阵组装}
现在考虑 $a_h^K(u_h, v_h)$ 的矩阵形式 $\bA^K$,
$$
\begin{aligned}
A_{ij}^K & = a_h^K(\phi_i, \phi_j)\\
& = (\nabla \Pi^1 \phi_i, \nabla \Pi^1 \phi_j)_K + \beta(\Pi^{0, k}\phi_i,
\Pi^{0, k}\phi_j)_K + S^K((I-\Pi^1)\phi_i, (I-\Pi^1)\phi_j)\\
& = (\nabla \Pi^1_{li} m_l, \nabla \Pi^1_{qj} m_q)_K + \beta(\Pi^{0,
    k}_{li}m_l,
\Pi^{0, k}_{qj}m_q)_K + d_a(\phi_i-\Pi^1_{li}m_l)d_a(\phi_j-\Pi^1_{qj}m_q)\\
\end{aligned}
$$
所以:
$$
\bA^K = (\bPi^1)^T\bM^{1, k}\bPi^1 + \beta(\bPi^{0, k})^T\bM^{0, k} \bPi^{0, k} +
(\bI - \bD\bPi^1)^T(\bI - \bD\bPi^1)
$$
右端项 $\bF^K$ 的计算为:
$$
F_{i}^K = (f, \Pi^{0, k} \phi_i)_K = (f, m_l)_K \Pi^{0, k}_{li}
$$
记 $\bd^K$ 单元 $K$ 的自由度,$\bA$ 和 $\bF$ 的组装算法如下:
\begin{algorithm}[H]
\caption{Poisson 方程虚单元方法:组装 $\bA$ 和 $\bF$}\label{alg:assembleAF}
\begin{algorithmic}[1]
    \Require VEM 空间 $V_h$
    \Ensure $\bA, \bF$
    \State Get $\mathcal{T}_h, \bPi^1, \bPi^{0, k}, \bM^{1, k}, \bM^{0, k}, \bD$ 
    \State $\bA \gets \boldsymbol{0}, \bF \gets \boldsymbol{0}$
    \For {each element $K \in \mathcal{T}_h$}
        \State $\bA^K \gets \bPi^1\bM^{1, k}\bPi^1 + \beta\bPi^{0, k}\bM^{0, k}\bPi^{0, k} +
        (\bI - \bD\bPi^1)^T(\bI - \bD\bPi^1)$
        \State $\tilde{\bF}^K \gets ((f, m_l)_K)$
        \State $\bF^K \gets \bPi^{0, k}\tilde{\bF}^K$ 
        \State $\bA_{\bd^K, \bd^K} \gets \bA_{\bd^K, \bd^K} + \bA^K$
        \State $\bF_{\bd^K} \gets \bF_{\bd^K} + \bF^K$
    \EndFor
    \State \Return $\bA, \bF$
\end{algorithmic}
\end{algorithm}
组装完成后,求解代数方程 $\bA \bU = \bF$ 即可得到数值解 $u_h = \sum_{i=1}^{N}
U_i \phi_i$。

\subsection{时谐 Maxwell 方程}
考虑时谐 Maxwell 方程:
\begin{equation}
    \label{eq:maxwell}
    \left\{
    \begin{aligned}
        \curl (\alpha\curl \bu) - \beta \bu & = \boldsymbol{f} \quad \text{in} \ \Omega\\
        \bu \times \bn & = \boldsymbol{0} \quad \text{on} \ \partial \Omega
    \end{aligned}
\right.
\end{equation}
其中 $\alpha, \beta$ 是常数,$\boldsymbol{f} \in L^2(\Omega)$。
首先将方程 \eqref{eq:maxwell} 乘以一个测试函数 $\bv \in H_0(\curl,
\Omega)$,并且使用 Green 公式,得到:
$$
\begin{aligned}
    (\alpha\curl \bu, \curl \bv) - \beta(\bu, \bv) & = (\boldsymbol{f}, \bv)\\
\end{aligned}
$$
定义双线性形式 $a(\bu, \bv) = (\alpha\curl \bu, \curl \bv) -
\beta(\bu, \bv)$,线性形式 $L(\bv) = (\boldsymbol{f}, \bv)$,那么方程
\eqref{eq:maxwell} 对应的变分问题为:找到 $\bu \in H(\curl, \Omega)$ 使得:
$$
a(\bu, \bv) = L(\bv), \quad \forall \bv \in H_0(\curl, \Omega)
$$
\subsubsection{虚单元方法}
现在使用虚单元方法求解这个问题,和 Poisson 方程类似,首先将区域 $\Omega$
分解为多边形网格,定义虚单元空间 $V_h$ 为 $H(\curl)$
协调虚单元空间或非协调虚单元空间,定义 $V_h$ 上的双线性形式 $a_h(\bu_h, \bv_h)$
和线性形式 $F_h(\bv_h)$ 为:
$$
\begin{aligned}
    a_h(\bu_h, \bv_h) & = \sum_{K \in \mathcal{T}_h}a_h^K(\bu_h, \bv_h)\\
    a_h^K(\bu_h, \bv_h) & = (\alpha\curl \bu_h, \curl
    \bv_h)_K - \beta(\Pi^{0, k}\bu_h, \Pi^{0, k}\bv_h)_K 
    + S^K((I - \Pi^{0, k})\bu_h, (I - \Pi^{0, k})\bv_h)\\
    F_h(\bv_h) & = \sum_{K\in \mathcal{T}_h}(\boldsymbol{f}, \Pi^{0, k}\bv_h)_K
\end{aligned}
$$
其中 $\Pi^{0, k}$ 是 $L^2$ 投影算子,$S^K$
是虚单元空间的稳定项,其定义与 Poisson 方程中的一样。
那么虚单元方法的离散问题为:找到 $\bu_h \in V_h$ 使得:
$$
a_h(\bu_h, \bv_h) = F_h(\bv_h), \quad \forall \bv_h \in V_h
$$
记 $V_h$ 的基函数为 $\{\bphi_i\}_{i=1}^{N}$,$N$ 是自由度的个数,
设 $\bu_h = U_i \bphi_i$, $A_{ij} = a_h(\bphi_i, \bphi_j)$,$F_i =
F_h(\bphi_i)$,那么有:
$$
\bA \bU = \bF
$$
\subsubsection{矩阵组装}
本节给出 $a_h^K(\phi_i, \phi_j)$ 和 $F_h^K(\phi_j)$ 的矩阵向量
形式 $\bA^K$ 和 $\bF^K$ 的组装算法。
$$
\begin{aligned}
    A_{ij}^K & = a_h^K(\bphi_i, \bphi_j)\\
    & = (\alpha\curl \bphi_i, \curl \bphi_j)_K -
    \beta(\Pi\bphi_i, \Pi\bphi_j)_K + S^K((I - \Pi)\bphi_i,
    (I - \Pi)\bphi_j)\\
    & = (\alpha m_l \Pi^{\curl}_{li}, m_q \Pi^{\curl}_{qj})_K -
    \beta(\Pi_{li}^dm_l, \Pi_{qj}^dm_q)_K +
    d_a(\bphi_i-\Pi_{li}^dm_l\otimes\be_d)
    d_a(\bphi_j-\Pi_{qj}^dm_q\otimes\be_d)\\
\end{aligned}
$$
所以:
$$
\bA^K = \alpha(\bPi^{\curl})^T\bM\bPi^{\curl} - 
\sum_{d=0}^1\beta(\bPi^d)^T\bM\bPi^d +
\sum_{d=0}^1(\bI - \bD^d\bPi^d)^T(\bI - \bD^d\bPi^d)
$$
其中 $\bM$ 是 $\mathcal{M}_k$ 的质量矩阵,$\bD^d$ 是 $\mathcal{M}_k\otimes \be_d$
的自由度矩阵。
右端项 $\bF^K$ 的计算为:
$$
F_{i}^K = (\boldsymbol{f}, \Pi \bphi_i)_K = (\boldsymbol{f}, m_l)_K \Pi_{li}
$$
记 $\bd^K$ 单元 $K$ 的自由度,$\bA$ 和 $\bF$ 的组装算法如下:
\begin{algorithm}[H]
    \caption{Maxwell 方程虚单元方法:组装 $\bA$ 和 $\bF$}\label{alg:assembleAF}
\begin{algorithmic}[1]
    \Require VEM 空间 $V_h$
    \Ensure $\bA, \bF$
    \State Get $\mathcal{T}_h, \bPi^{\curl}, \bPi, \bM, \bD$ 
    \State $\bA \gets \boldsymbol{0}, \bF \gets \boldsymbol{0}$
    \For {each element $K \in \mathcal{T}_h$}
        \State $\bA^K \gets \alpha(\bPi^{\curl})^T\bM\bPi^{\curl} - 
        \sum_{d=0}^1\beta(\bPi^d)^T\bM\bPi^d +
        \sum_{d=0}^1(\bI - \bD^d\bPi^d)^T(\bI - \bD^d\bPi^d)$
        \State $\tilde{\bF}^K \gets ((\boldsymbol{f}, m_l)_K)$
        \State $\bF^K \gets \bPi\tilde{\bF}^K$ 
        \State $\bA_{\bd^K, \bd^K} \gets \bA_{\bd^K, \bd^K} + \bA^K$
        \State $\bF_{\bd^K} \gets \bF_{\bd^K} + \bF^K$
    \EndFor
    \State \Return $\bA, \bF$
\end{algorithmic}
\end{algorithm}
求解 $\bA \bU = \bF$ 即可得到数值解 $\bu_h = \sum_{i=1}^{N} U_i \bphi_i$。

\subsection{线弹性方程}
考虑如下平衡方程:
\begin{equation}
    \label{eq:elastic}
    \left\{
    \begin{aligned}
        -\nabla \cdot \bsigma & = \boldsymbol{f} \quad \text{in} \ \Omega\\
        \bu & = \boldsymbol{0} \quad \text{on} \ \partial \Omega
    \end{aligned}
\right.
\end{equation}
其中 $\bsigma$ 是应力张量,$\bu$ 是位移向量,$\boldsymbol{f} \in L^2(\Omega)$ 是体力密度。
对于小位移小应变的情况,即线弹性问题,应力张量 $\bsigma$ 和位移向量 $\bu$ 之间
满足线性关系:
$$
\bsigma(\bu) = 2\mu\varepsilon(\bu) + \lambda \diver \bu \boldsymbol{I}
$$
其中 $\varepsilon(\bu) = \dfrac{1}{2}(\nabla \bu + \nabla \bu^T)$, 张量形式为:
$$
\varepsilon(u\be_i) = \frac{1}{2}
\partial_{\bx_l}u
(\be_i\otimes \be_l + \be_l\otimes \be_i)
$$
特别的对于  
首先将方程 \eqref{eq:elastic} 乘以一个测试函数 $\bv \in H_0^1(\Omega,
\mathbb{R}^2)$,
使用 Green 公式得到:
$$
\begin{aligned}
    2(\mu\varepsilon(\bu), \varepsilon(\bv)) +  \lambda(\diver \bu, \diver \bv)
    & = (\bsf, \bv)\\
\end{aligned}
$$
定义双线性形式 $a(\bu, \bv) = (\varepsilon(\bu), \varepsilon(\bv))$,
线性形式 $L(\bv) = (\bsf, \bv)$,那么方程 \eqref{eq:elastic} 对应的变分问题为:
找到 $\bu \in H_0^1(\Omega, \mathbb{R}^2)$ 使得:
$$
a(\bu, \bv) = L(\bv), \quad \forall \bv \in H_0^1(\Omega, \mathbb{R}^2)
$$
\subsubsection{虚单元方法}
现在使用虚单元方法求解这个问题,首先将区域 $\Omega$ 分解为多边形网格,
令 $V_h$ 为 $H^1$ 协调虚单元空间或非协调虚单元空间,
定义 $\bV_h := (V_h, V_h)$ 作为 $H^1(\Omega, \mathbb{R}^2)$ 协调虚单元空间,
其中的函数 $\bu_h \in \bV_h$ 可以表示为 $\bu_h = u_h^0\be_0 + u_h^1\be_1$。 
定义 $\bV_h$ 上的双线性形式 $a_h(\bu_h, \bv_h)$ 和线性形式 $F_h(\bv_h)$ 为:
$$
\begin{aligned}
    a_h(\bu_h, \bv_h) & = \sum_{K \in \mathcal{T}_h}a_h^K(\bu_h, \bv_h)\\
    a_h^K(\bu_h, \bv_h) & = 2(\mu\varepsilon(\Pi^1 u_h^0 \be_0 + \Pi^1 u_h^1 \be_1), 
    \varepsilon(\Pi^1 v_h^0 \be_0 + \Pi^1 v_h^1 \be_1))_K\\
    & \qquad\qquad + (\lambda\diver (\Pi^1 u_h^0 \be_0 + \Pi^1 u_h^1 \be_1), \diver(\Pi^1 v_h^0
    \be_0 + \Pi^1 v_h^1 \be_1))_K\\ 
    & \qquad\qquad + S^K((I - \Pi^1)u_h^0, (I - \Pi^1)v_h^0) + S^K((I -
    \Pi^1)u_h^1, (I - \Pi^1)v_h^1)\\
    F_h(\bv_h) & = \sum_{K\in \mathcal{T}_h}(\bsf, \Pi v_h^d \be_d)_K
\end{aligned}
$$
其中 $\Pi^1$ 和 $\Pi$ 是 $V_h^K$ 上的 $H^1$ 投影算子和 $L^2$ 投影算子,
$S^K$ 是虚单元空间的稳定项,其定义与
Poisson 方程中的一样。
那么虚单元方法的离散问题为:找到 $\bu_h \in \bV_h$ 使得:
$$
a_h(\bu_h, \bv_h) = F_h(\bv_h), \quad \forall \bv_h \in \bV_h
$$
记 $\bV_h$ 的基函数为 $\{\phi_i\otimes \be_0, \phi_i\otimes \be_1\}_{i=1}^{N}$, 
$N$ 是 $V_h$ 的基函数个数。
设 $\bu_h = U_i^0 \phi_i\otimes \be_0 + U_j^1 \phi_j\otimes \be_1$,
定义:
\begin{equation}
\label{eq:elasticmatrixij}
\begin{aligned}
    A_{ij}^{00} & = a_h(\phi_i\otimes \be_0, \phi_j\otimes \be_0), \quad 
    A_{ij}^{01} & = a_h(\phi_i\otimes \be_0, \phi_j\otimes \be_1)\\
    A_{ij}^{10} & = a_h(\phi_i\otimes \be_1, \phi_j\otimes \be_0), \quad
    A_{ij}^{11} & = a_h(\phi_i\otimes \be_1, \phi_j\otimes \be_1)\\
\end{aligned}
\end{equation}
$$
F_i^0 = F_h(\phi_i\otimes \be_0), \quad F_i^1 = F_h(\phi_i\otimes \be_1)
$$
其中可以证明 $\bA^{01} = (\bA^{10})^T$。
进一步定义矩阵 $\bA$ 和向量 $\bF$:
$$
\bA =
\begin{pmatrix}
    \bA^{00} & \bA^{01}\\
    \bA^{10} & \bA^{11}
\end{pmatrix}, \quad
\bF =
\begin{pmatrix}
    \bF^{0}\\
    \bF^{1}
\end{pmatrix}, \quad
\bU =
\begin{pmatrix}
    \bU^{0}\\
    \bU^{1}
\end{pmatrix}
$$
那么有:
$$
\bA \bU = \bF
$$
\subsubsection{矩阵组装}
本节给出 $\bA$ 和 $\bF$ 的组装。
首先给出单元矩阵的组装过程,类似的我们仍然用 $\bA$ 和 $\bF$
表示单元的矩阵和向量。
定义 $\tilde{A}_{ij}^{ab}$:
$$
\begin{aligned}
\tilde{A}_{ij}^{ab} 
& = 2(\mu\varepsilon(\Pi^1 \phi_i\be_a), \varepsilon(\Pi^1 \phi_j\be_b))
  + \lambda(\diver(\Pi^1 \phi_i\be_a), \diver(\Pi^1 \phi_j\be_b))\\
& = 2(\mu\varepsilon(m_l\Pi^1_{li}\be_a), \varepsilon(m_q\Pi^1_{qj}\be_b))_K
+ \lambda(\diver(m_l\Pi^1_{li}\be_a), \diver(m_q\Pi^1_{qj}\be_b))_K\\ 
& = (\mu\partial_{\bx_r}m_l\Pi^1_{li}(\be_a\otimes \be_r + \be_r\otimes \be_a), 
\partial_{\bx_s}m_q\Pi^1_{qj}(\be_b\otimes \be_s + \be_s\otimes \be_b))_K\\
& \qquad\qquad + \lambda(\partial_{\bx_a}m_l\Pi^1_{li}, \partial_{\bx_b}m_q\Pi^1_{qj})_K\\
& = (\mu\partial_{\bx_r}m_l\Pi^1_{li}, \partial_{\bx_s}m_q\Pi^1_{qj})_K(\be_a\otimes
\be_r + \be_r\otimes \be_a):(\be_b\otimes \be_s + \be_s\otimes \be_b)\\
& \qquad\qquad + \lambda(\partial_{\bx_a}m_l\Pi^1_{li}, \partial_{\bx_b}m_q\Pi^1_{qj})_K\\
& = (\mu\partial_{\bx_r}m_l\Pi^1_{li}, \partial_{\bx_s}m_q\Pi^1_{qj})_K
(2\delta_{ab}\delta_{rs} + 2\delta_{as}\delta_{rb})\\
& \qquad\qquad + \lambda(\partial_{\bx_a}m_l\Pi^1_{li}, \partial_{\bx_b}m_q\Pi^1_{qj})_K\\
& = 2\mu\delta_{ab}(\partial_{\bx_r}m_l\Pi^1_{li}, \partial_{\bx_r}m_q\Pi^1_{qj})_K
+ (2\mu+\lambda)(\partial_{\bx_a}m_l\Pi^1_{li}, \partial_{\bx_b}m_q\Pi^1_{qj})_K\\
\end{aligned}
$$
所以:
$$
\tilde{\bA}^{ab} = 2\mu\delta_{ab}(\bPi^1)^T\bM^1\bPi^1 + 
(2\mu + \lambda)(\bP^a\bPi^1)^T\bM\bP^{b}\bPi^1
$$
对于稳定项矩阵 $S_{ij} = S^K((I - \Pi^1)\phi_i, (I - \Pi^1)\phi_j)$,其与
Poisson 方程中的稳定项矩阵 $S^K$ 的计算方式一样:
$$
\bS = (\bI - \bD\bPi^1)^T(\bI - \bD\bPi^1)
$$
根据 \eqref{eq:elasticmatrixij}:
$$
A_{ij}^{ab} = \tilde{A}_{ij}^{ab} + \delta_{ab}S_{ij}
$$
具体来说:
$$
\begin{aligned}
    \bA^{00} & = \tilde{\bA}^{00} + \bS, \quad \bA^{01} = \tilde{\bA}^{01}\\
    \bA^{10} & = \tilde{\bA}^{10}, \quad \bA^{11} = \tilde{\bA}^{11} + \bS
\end{aligned}
$$
右端项 $\bF$ 的计算为:
$$
F_{i}^{0} = (\bsf, \Pi^1\phi_i\be_0)_K = (\bsf, m_l\be_0)_K\Pi^1_{li}
$$
$$
F_{i}^{1} = (\bsf, \Pi^1\phi_i\be_1)_K = (\bsf, m_l\be_1)_K\Pi^1_{li}
$$
记 $\bd^K$ 单元 $K$ 上 $V_h^K$ 的自由度,$\bA$ 和 $\bF$ 的组装算法如下:
\begin{algorithm}[H]
    \caption{线弹性方程虚单元方法:组装 $\bA$ 和 $\bF$}\label{alg:assembleAF}
    \begin{algorithmic}[1]
    \Require VEM 空间 $V_h$
    \Ensure $\bA, \bF$
    \State Get $\mathcal{T}_h, \bPi^1, \bM^1, \bD$
    \State $\bA \gets \boldsymbol{0}, \bF \gets \boldsymbol{0}$
    \For {each element $K \in \mathcal{T}_h$}
        \State $\bS \gets (\bI - \bD\bPi^1)^T(\bI - \bD\bPi^1)$
        \State $\tilde{\bA}^{00} \gets 2\mu(\bPi^1)^T\bM^1\bPi^1 +
            (2\mu + \lambda)(\bP^0\bPi^1)^T\bM\bP^{0}\bPi^1 + \bS$ 
        \State $\tilde{\bA}^{11} \gets 2\mu(\bPi^1)^T\bM^1\bPi^1 +
            (2\mu + \lambda)(\bP^1\bPi^1)^T\bM\bP^{1}\bPi^1 + \bS$ 
        \State $\tilde{\bA}^{01} \gets (2\mu+\lambda)(\bP^0\bPi^1)^T\bM^1\bP^{1}\bPi^1$
        \State $\tilde{\bA}^{10} \gets (\tilde{\bA}^{01})^T$ 
        \State $\tilde{\bF}^0 \gets ((\bsf, m_l\be_0)_K)$
        \State $\tilde{\bF}^1 \gets ((\bsf, m_l\be_1)_K)$
        \For {each $a, b \in \{0, 1\}$}
            \State $\bA^{ab}_{\bd^K, \bd^K} \gets \bA^{ab}_{\bd^K, \bd^K} + \tilde{\bA}^{ab}$
        \EndFor
        \State $\bF_{\bd^K}^0 \gets \bF_{\bd^K}^0 + \bF^0$
        \State $\bF_{\bd^K}^1 \gets \bF_{\bd^K}^1 + \bF^1$
    \EndFor
    \State \Return $\bA, \bF$
\end{algorithmic}
\end{algorithm}























 
\section{软件设计}

\section{数值算例}



















 

