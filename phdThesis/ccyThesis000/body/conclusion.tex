\chapter{总结及下一步工作}
\label{chap:conclusion}

虚单元方法具有高度的灵活性,对网格单元的形状要求低,可以处理复杂的几何形状,
易于构造协调的虚单元空间。基于这些特点,本文在高正则性空间的构造,
时谐 Maxwell 方程界面问题的求解方面进行了研究,同时针对虚单元方法的稳定化项
提出了无稳定化项的虚单元方法。

对于 $H^m$ 协调的离散空间,本文通过递归的方法,将低维面上的协调虚单元
空间粘贴到高维多面体上,构造了任意维 $H^m$
协调的虚单元。我们证明了虚单元的逆不等式以及范数等价性,给出了多重调和方程的
协调虚单元方法求解。
$H^m$ 协调虚单元方法对多项式次数要求仅为 $k\geq m$,远小于 $H^m$ 协调有限元方法
的要求。这使得虚单元方法在高阶偏微分方程问题的求解中具有优势。
下一步我们将考虑将二维三维 $H^m$ 协调虚单元方法使用统一的方法分别实现,
并应用于到实际问题中,如四阶相场模型,退化的非线性抛物方程等高阶问题。

对于虚单元方法的稳定化项的问题,
我们构造了多面体上的宏有限元空间,
通过将虚单元函数投影到宏有限元空间上,提出了无稳定化项的虚单元方法,在该方法中,
仅刚度矩阵的组装与标准方法有所不同,其他部分均与标准虚单元方法相同。
我们证明了到宏有限元空间上的范数等价性,给出的到该空间 $L^2$
投影在 $L^2$ 范数下有下界,分析了无稳定化项的虚单元方法的稳定性,收敛性。
该去除稳定化项的方法可用于任意维的多面体网格,且易于扩展到其他类型的问题中,
如对于 $m$ 重调和方程的虚单元方法,可以将函数的 $m$ 
阶梯度投影到一个 $H(\mathrm{div}^m, K)$ 协调的宏元空间上,以去除稳定化项,
这也是我们下一步的研究方向。此外,在实际问题中,
无稳定化项的虚单元方法具有一定的应用价值,稳定化项的去除使得
其与有限元方法使用方式更加接近,而又保留了虚单元方法的优势,因此下一步我们也会
将无稳定化项的虚单元方法应用到一些实际的非线性问题中,如超弹性问题,
断裂力学问题等。

对于时谐 Maxwell 方程界面问题,我们对该问题的虚单元方法进行了误差分析。
分析过程考虑了界面存在几何奇异性、且解可能具有较低的光滑性的情况,使用的分析技术
是“虚拟网格”技术,通过该技术,我们可以将虚单元空间重新定义为传统有限元空间,
将分析的困难转为考察虚拟网格的几何特征。
最终得到的误差界对于高度各向异性单元仍然有效。
未来的研究方向有几个方面:(i)针对更高正则性解的高阶方法;(ii)允许
$\bH^{\theta}(\rot;\Omega)$ 正则性的解的分析,其中 $\theta \in (0,
1/2]$;(iii)三维情形。此外在移动界面问题方面,我们会考虑研究更加实际的应用问题,
将虚单元方法应用到实际的电磁问题中,如电磁成型问题的高效数值模拟,永磁体电机及
发电机的电磁仿真等。

作为一种仅有 12 年历史的方法,虚单元方法仍有很多问题有待解决,
从实现的角度来看,
其基函数隐式定义,需要计算其到多项式空间(或其他可计算空间)的投影,相比于有限
元方法基函数可显式计算,虚单元方法实现更加复杂,因此我们下一步也会在
虚单元方法的实现方面进行研究,基于开源 CAX 共性基础算法库 FEALPy 开发高效的虚单元方法
模块,为虚单元方法的应用,以及科研人员的研究提供便利。


%传统的多边形单元分析方法通常依赖于将物理单元映射到规则的参考单元上,并要求该映射保持均匀有界的导数。然而,这一要求对于各向异性单元无法满足。与此不同,我们当前的分析方法使用了“虚拟网格”,在该网格上,VEM空间可以重新定义为传统有限元空间,而无需改变计算方法或妥协“虚拟”概念的本质。通过这种方法,我们能够有效地将分析的挑战转向考察虚拟网格的几何特征。值得注意的是,我们揭示了网格形状规则性与解的规则性之间的关系。
%
%在当前的研究中,我们故意将研究限制在各向异性单元上的最低阶空间,因为这些单元通常出现在界面附近,而该区域的解的规则性通常较低,采用高阶方法可能无法达到预期的效率。尽管如此,未来的研究方向仍然有几个:(i)针对更规则解的高阶方法;(ii)更多允许
%$\bH^{\theta}(\rot;\Omega)$ 规则性的奇异解,其中 $\theta \in (0,
%1/2]$;(iii)三维情形。
%
%对于高阶方法,一个关键问题是将经典虚单元空间的内部自由度(DoF)与虚拟网格上经典Nédélec空间的某些自由度联系起来。对于更加奇异的解,主要困难不仅来自于推导合适的拟插值方法
%\cite{2017ErnGuermond,2008SCHOBERL},还来自于各向异性单元的形状。具体来说,在估计式 \eqref{verify_lem_Pi_est_polygon_eq7} 时,应用引理 \ref{lem_geometry_patch},当 $s < 1/2$ 时,$h_K^2l_K^{2\theta-1}$ 可能会发散,因为式 \eqref{verify_lem_Pi_est_polygon_eq7} 中的高度 $l_K$ 可能会极度收缩,而 $l_K^{2\theta-1}$ 无法被 $h_K$ 所抵消。此外,在三维情形中,关键引理 \ref{lem_L2stab} 中的估计式 \eqref{lem_L2stab_eq01} 需要扩展到那些边缘不收缩到两个方向的多面体。然而,存在一种特殊类型的四面体,其边缘可能会收缩到两个方向,但仍满足三维最大角度条件。因此,这三个方向的进一步研究仍然是必要的,而且肯定不是简单的任务。
%




