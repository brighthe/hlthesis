%!TEX program = xelatex
% 完整编译: xelatex -> biber/bibtex -> xelatex -> xelatex
\documentclass[lang=cn,a4paper,newtx]{elegantpaper}

\title{引力波学习笔记}
\author{陈春雨 \\ 湘潭大学\ 数学与计算科学学院}

\date{\zhdate{2024/9/18}}

\begin{document}

\maketitle

广义相对论是天文学中的一个重要理论,它将宇宙描述为一个四维的空间时间流形,
具有弯曲的度量,其曲率描述了引力场的本质。
\section{宇宙的几何描述}
宇宙是一个四维的空间时间流形 $M$,配备了一个 Lorentz 度量 $g_{ab}$,
$$
g_{ab} = g_{\mu\nu}(dx^{\mu})_a\otimes(dx^{\nu})_b
$$
其中 $g_{\mu\nu}$ 是一个对称非退化矩阵,$\{(dx^{\mu})_a\}_{\mu=0}^3$ 
是 $M$ 上的余切向量基。最简单的 Lorentz 度量是 Minkowski 度量 $\eta_{ab}$:
$$
(\eta_{\mu\nu}) = \mathrm{diag}(-1, 1, 1, 1)
$$
对于向量 $v^a \in T_pM$, 若 
\begin{itemize}
    \item $g_{ab}v^av^b = 0$,则称 $v^a$ 是类光向量。
    \item $g_{ab}v^av^b > 0$,则称 $v^a$ 是类空向量。
    \item $g_{ab}v^av^b < 0$,则称 $v^a$ 是类时向量。
\end{itemize}
如果一个超曲面 $\Sigma$ 的法向量 $n^a$ 是类时向量,那么称 $\Sigma$ 是类空超曲面。
此时度量 $g_{ab}$ 在 $\Sigma$ 上的限制是一个 Riemann 度量。

\section{Einstein 方程}
Einstein 方程是广义相对论的核心, 
其描述了空间中质量和能量的分布如何影响时空的几何形状:
$$
G_{ab} = \frac{8\pi G}{c^4}T_{ab} 
$$
其中 $G_{ab} = R_{ab} - \frac{1}{2}g_{ab}R$ 是 Einstein 张量,
$T_{ab}$ 是能量动量张量,$R_{ab}$ 是 Ricci 张量,$R$ 是标量曲率。
在真空中 $T_{ab} = 0$,Einstein 方程变为:
$$
G_{ab} = 0
$$
显然 $\mathrm{tr}(G) = 0$,即 
$g^{ab}G_{ab} = g^{ab}R_{ab} - \frac{1}{2}g^{ab}g_{ab}R = 0$,
所以 $R = 0, R_{ab} = 0$。反过来,如果 Ricci
张量为零,那么 Einstein 张量也为零。所以真空中的 Einstein 方程等价于:
$$
R_{ab} = 0
$$
\section{线性化 Einstein 方程}
假设 $g_{ab}$ 是 Minkowski 度量的微小扰动:
$$
g_{ab} = \eta_{ab} + \epsilon h_{ab} + O(\epsilon^2)
$$
其中 $\eta_{ab}$ 是 Minkowski 度量,$0<\epsilon\ll 1$,未知量变成了 $h_{ab}$。
将上面的度量代入 Einstein 方程,令 $\nabla$ 是 $\eta_{ab}$ 的联络,
那么与 $g_{ab}$ 适配的联络与 $\nabla$ 之间的克氏符为:
$$
\Gamma_{abc} = \frac{1}{2}(\nabla_ag_{bc} + \nabla_bg_{ac} - \nabla_cg_{ab})
= \epsilon\frac{1}{2}(\nabla_ah_{bc} + \nabla_bh_{ac} - \nabla_ch_{ab}) + O(\epsilon^2)
$$
$\Gamma_{abc} = O(\epsilon)$,所以 Riemann 张量为:
$$
\begin{aligned}
R_{abcd} & = \nabla_c\Gamma_{bda} - \nabla_d\Gamma_{bca} + O(\epsilon^2)\\
& = \frac{1}{2} \epsilon(\nabla_c\nabla_bh_{da} - \nabla_c\nabla_ah_{bd} -
\nabla_d\nabla_bh_{ca} + \nabla_d\nabla_ah_{bc}) + O(\epsilon^2)\\
\end{aligned}
$$
所以 Ricci 张量为:
$$
\begin{aligned}
R_{bd} = R^{c}_{\ bcd} & = \frac{1}{2}\epsilon(\nabla_c\nabla_bh^{c}_{\ d} 
- \nabla_c\nabla^ch_{bd} - \nabla_d\nabla_bh^{c}_{\ c} + \nabla_d\nabla^ch_{bc}) + O(\epsilon^2)\\
& = \epsilon\nabla_c\nabla_{(b}h^{\ c}_{d)} - 
\frac{1}{2}\epsilon(\nabla_c\nabla^ch_{bd} + \nabla_d\nabla_b h^c_{\ c}) + 
O(\epsilon^2)
\end{aligned}
$$
$$
R = R^{a}_{\ a} = \epsilon(\nabla_c\nabla^ah^{\ c}_{a} - \nabla_c\nabla^ch^{a}_{\ a}) + O(\epsilon^2)
$$
$R = O(\epsilon)$ 那么 Einstein 张量为:
$$
\begin{aligned}
G_{ab} & = R_{ab} - \frac{1}{2}g_{ab}R\\
& = \epsilon\nabla_c\nabla_{(a}h^{\ c}_{b)} 
- \frac{1}{2}\epsilon(\nabla_c\nabla^ch_{ab} 
+ \nabla_b\nabla_a h^c_{\ c}) - \frac{1}{2}\epsilon\eta_{ab}( \nabla_c\nabla^dh^{c}_{\ d} - \nabla_c\nabla^ch^{d}_{\ d})
+ O(\epsilon^2)\\ 
\end{aligned}
$$
令 $h = h^a_a, \bar{h}_{ab} = h_{ab} - \eta_{ab} h/2$ 那么 $\bar{h} = -h, 
h_{ab} = \bar{h}_{ab} + \eta_{ab}\bar{h}/2$,所以:
$$
\nabla_{a}\nabla_{b} h_{cd} = \nabla_{a}\nabla_{b} \bar{h}_{cd} + \frac{1}{2}\eta_{ab}\nabla_{c}\nabla_{d}\bar{h}
$$
所以有:
$$
\frac{\mathrm{d}}{\mathrm{d}\epsilon}G_{ab} = \nabla^c\nabla_{(a}\bar{h}_{b)c} - 
\frac{1}{2}(\eta_{ab}\nabla^c\nabla^d\bar{h}_{cd} + \nabla^c\nabla_c\bar{h}_{ab})
=: H(\bar{h}_{ab})
$$
显然 $H$ 是一个线性算子,所以 Einstein 方程的线性化形式为:
$$
H(\bar{h}_{ab}) = \frac{8\pi G}{c^4}\mathcal{T}_{ab}
$$
其中 $\mathcal{T}_{ab} = \frac{\mathrm{d}}{\mathrm{d}\epsilon}T_{ab}$。
需要注意的是,上式保证了 $\nabla_a\mathcal{T}^{ab} = 0$,所以线性化的 Einstein 方程
可以保证能量,动量,角动量守恒。

\section{规范自由性}
假设 $\phi$ 是两个微分流形 $M, N$ 之间的微分同胚,
那么 $M$ 上的度量 $g$ 可以推前到 $N$,$(N, \phi_*g)$ 与 $(M, g)$ 
等距同构。那么如果 $g$ 是 Einstein
方程的解,那么由于:
$$
R(\phi_*g) = \phi_*R(g)
$$
所以 $\phi_*g$ 也是 Einstein 方程的解。也是 Einstein 方程的解,他们互相等价,称为规范相关。
对于线性化的 Einstein 方程,任取向量场 $v^a$,定义 $h'_{ab} = \mathcal{L}_v\beta_{ab} = \nabla_{(a}v_{b)}$,
那么有:
$$
H(h'_{ab}) = 0 
$$
即 $h'_{ab}\in \ker(H)$,所以 Einstein 方程的解不是唯一的,
这个问题不能由边界条件来解决。我们可以施加一个规范条件,在等价的解中选择一个。
类似的情况在电磁学中也有,当我们求解磁矢势 $A$ 时:
$$
\boldsymbol{B} = \nabla \times \boldsymbol{A}
$$
这个方程的解不是唯一的,当 $\boldsymbol{A}'\in\ker(\nabla\times)$, 
$\boldsymbol{A+A}'$ 也是方程的解。
这时可以加上一个条件:
$$
\nabla\cdot\boldsymbol{A} = 0
$$
这个条件就是规范条件。在线性 Einstein 方程中,常用的规范条件是:
$$
\nabla^b\bar{h}_{ab} = 0
$$
在这个规范条件下,线性化的 Einstein 方程的形式更加简单:
$$
H(\bar{h}_{ab}) = \square\bar{h}_{ab} = \frac{8\pi G}{c^4}\mathcal{T}_{ab} 
$$
其中 $\square = \nabla^a\nabla_a$ 是 d'Alembert 算子。
\begin{note}
对于任意 $h_{ab}$,选择 $v^a$ 满足 $\nabla^b\nabla_b v^a = -\nabla^b h_{ab}$,
令 $h'_{ab} = h_{ab}+\nabla_{(a}v_{b)}$ 那么有:
$$
\begin{aligned}
\bar{h'}_{ab} 
& = h'_{ab} - \frac{1}{2}\eta_{ab}h'\\
& = h_{ab} + \nabla_{(a}v_{b)} - \frac{1}{2}\eta_{ab}(h_{c}^{\ c} + 
\eta^{cd}\nabla_{c}v_{d})\\
& = \bar{h}_{ab} + \nabla_{(a}v_{b)} - \frac{1}{2}\eta_{ab}\nabla^{c}v_{c}\\
\end{aligned}
$$
所以:
$$
\begin{aligned}
\nabla^b \bar{h'}_{ab} & = \nabla^b\bar{h}_{ab} + \nabla^b\nabla_{(a}v_{b)} - \frac{1}{2}\nabla_a\nabla^bv_b\\
& = \nabla^b\bar{h}_{ab} + \frac{1}{2}\nabla_b\nabla^bv_a\\
& = 0
\end{aligned}
$$
所以,对于任意 $h_{ab}$,总可以找到一个等价的 $h'_{ab}$ 满足规范条件。
\end{note}

\begin{note}
关于规范相关有一个等价的解释:选择一个合适的坐标系,
 Einstein 方程可以写成分量形式:
$$
R_{\mu\nu}(x) - \frac{1}{2}g_{\mu\nu}(x)R(x) = \frac{8\pi G}{c^4}T_{\mu\nu}(x)
$$
其中每个分量都是 $x$ 的函数。这个方程的未知量是 $g_{\mu\nu}(x)$,因为 $g$ 对称,
所以有 $10$ 个未知量。选择适当的边界条件,上式同时也是 10 个偏微分方程,
原本非常合理,10 个方程求解 10 个未知量,但是因为有 Bianchi 恒等式:
$$
\nabla_{[a}R_{bc]d}^{\quad\ e} = 0
$$
这可以推出 $\nabla_aG^{ab} = 0$, 其分量形式是四个方程:
$$
\nabla_{\mu}G^{\mu\nu} = 0
$$
这表明 10 个方程中只有 6 个是独立的,所以实际上只有 6 个方程求解 10 个未知量。
实际上,如果 $g_{\mu\nu}(x)$ 是 Einstein 方程的解,那么 $g_{\mu\nu}(x)$ 的任意坐标变换
$g'_{\mu\nu}(x)$ 也是 Einstein 方程的解,那么 $g_{\mu\nu}(x)$ 和 $g'_{\mu\nu}(x)$ 是
规范相关。
最后添加的规范条件就是为了消除这种规范相关性,使得解是唯一的。规范条件的分量形式是:
$$
\nabla^{\mu}\bar{h}_{\mu\nu} = 0
$$
这是四个方程,所以加上规范条件后,10 个方程求解 10 个未知量。
\end{note}
加上 Lorenz 规范条件后,方程的解仍然不是唯一的, 令 $h'_{ab} = \partial_a v_b +
\partial_b v_a$,
其中 $v_a$ 满足 $\partial^b\partial_bv^a = 0$,那么 
$$
\bar{h'}_{ab} = \partial_a v_b + \partial_b v_a - \eta_{ab}\partial^c v_c
$$
那么:
$$
\partial^b\bar{h'}_{ab} = \partial^b\partial_a v_b + \partial^b\partial_b v_a - \partial_a\partial^b v_b
= 0
$$
且:
$$
H(\bar{h'}_{ab}) = \partial^c\partial_{c}\bar{h'}_{ab} = 
\partial^c\partial_{c}\partial_a v_b + \partial^c\partial_{c}\partial_b v_a -
\partial^c\partial_{c}\eta_{ab}\partial^d v_d = 0 
$$
所以要给 $\bar{h}_{ab}$ 再加一些条件使得满足这些条件的 $h_{ab}' = \partial_a v_b
+ \partial_b v_a$ 当 $\partial^b\partial_b v^a = 0$ 时要能推出 $v^a = 0$。
其实就是边界条件:
$$
h = 0, \quad h_{0i} = 0
$$

\section{线性 Einstein 方程的平面波解}
现在考虑一个平面波解,
令 $k^a$ 是一个类光矢量,即 $k^ak_a = 0$,$A_{ab}$ 是一个对称张量,满足:
\begin{align}
\label{proposition1}
\nabla_cA_{ab} = 0, \quad A_{ab}k^b = 0
\end{align}
对于任意的 $C^2$ 函数 $f(x)$,定义:
$$
\bar{h}_{ab} = A_{ab}f(k_cx^c)
$$
那么:
$$
\square \bar{h}_{ab} = A_{ab}\square f(k_cx^c) = A_{ab}k^ck_c f''(k_cx^c) = 0
$$
而且:
$$
\nabla^b\bar{h}_{ab} = A_{ab}\nabla^bf(k_cx^c) = A_{ab}k^bf'(k_cx^c) = 0
$$
所以 $\bar{h}_{ab}$ 是线性化 Einstein 方程的解。这种解称为平面波解。
在平面波解中,我们可以转化原始的规范条件为:
\begin{itemize}
    \item[1.] $A^a_{\ a} = 0$
    \item[2.] 对于某个类时矢量 $u^b$, $A_{ab}u^b = 0$.
\end{itemize}
这个条件被称为 TT 规范条件。
现在选择 $u^b = (1, 0, 0, 0)$,$k^a = k(1, 0, 0, 1)$,那么 
可以根据 TT 规范条件以及前述的 $A_{ab}$ 的性质,得到:
$$
A_{ii} = 0, A_{0i} = 0, A_{i0} = 0, A_{3i} = 0, A_{i3} = 0
$$
所以 $A_{ij}$ 如下:
$$
\begin{pmatrix}
0 & 0 & 0 & 0\\
0 & A^{+} & A^{\times} & 0\\
0 & A^{\times} & -A^{+} & 0\\
0 & 0 & 0 & 0
\end{pmatrix}
$$

\begin{proposition}
    若有一个平面波解 $\bar{h}_{ab} = A_{ab}f(k_cx^c)$,其中 $k^a = k(1, 0, 0,
    1)$, $A_{ab}$ 满足条件
    \eqref{proposition1} ,那么存在 $\bar{h'}_{ab} = A'_{ab}f(k_cx^c)$ 满足 TT
    规范条件, 且 $\bar{h}_{ab}$ 和 $\bar{h'}_{ab}$ 是规范相关的。
\end{proposition}

\section{Wely 张量和 Bel 分解}
黎曼张量 $R_{abcd}$ 可以做一个分解:
$$
R_{abcd} = M_{abcd} + W_{abcd}
$$
其中:
$$
M_{abcd} = \frac{2}{n-2}(g_{a[c}R_{b]d} - g_{b[c}R_{d]a} -
\frac{R}{n-1}g_{a[c}R_{d]b})
$$
分解的两部分都满足与 $R_{abcd}$ 相同的对称性,特别的满足如下反对称性:
\begin{align}
\label{asymmetry}
M_{abcd} = -M_{abdc}, \quad W_{abcd} = -W_{abdc}
\end{align}
和如下对称性:
\begin{align}
\label{symmetry}
M_{abcd} = M_{cdab}, \quad W_{abcd} = W_{cdab}
\end{align}
$M_{abcd}$ 与 Ricci 张量有关,
$W_{abcd}$ 称为 Wely 张量,是完全无迹的 : $W^a_{\ bac} = 0$,且满足 Bianchi 恒等式:
$$
\nabla_{[a}W_{bc]de} = 0
$$
在真空中,$R_{ab} = 0$,那么 $M_{abcd} = 0$,所以 $R_{abcd} = W_{abcd}$。
现在我们仅考虑 Wely 张量,在 4 维时空中,Wely 张量有 10 个独立分量。

由于 $W_{abcd}$ 满足 \eqref{asymmetry} 和 \eqref{symmetry},那么固定 $a, b$,那么 $W_{abcd}$ 是一个
2阶反对称张量,固定 $c, d$,那么 $W_{abcd}$ 是一个 2 阶对称张量,
因为 2 阶反对称张量等价于一个 2 形式,加之 $W_{abcd}$ 的对称性,
所以存在两个 2 形式 $\Lambda^1, \Lambda_0$ 使得:
$$
W_{abcd} = \Lambda^0_{ab}\otimes\Lambda^1_{cd} +
\Lambda^1_{ab}\otimes\Lambda^0_{cd}
$$
那么可以定义两个 $W_{abcd}$ 的对偶(分别是 Hodge star 算子作用在不同的部分):
$$
\begin{aligned}
    W_{abcd}^* & := (1\otimes\star)W_{abcd} 
    = \frac{1}{2}W_{abef}\epsilon^{ef}_{\ \ \ cd}\\
    ^*W_{abcd} & := (\star\otimes 1)W_{abcd}
    = \frac{1}{2}\epsilon^{\ \ \ ef}_{ab}W_{efcd}
\end{aligned}
$$
其中 $\epsilon_{abcd}$ 是空间的体元。两个对偶满足:
$$
W_{abcd}^{**} = -W_{abcd}, \quad W_{abcd}^* = ^*W_{cdab}
$$
根据 Wely 张量定义如下电张量和磁张量:
$$
\begin{aligned}
    \bar{E}_{ab} & = n^an^bW_{acbd}\\
    \bar{B}_{ab} & = n^an^bW_{abcd}^*
\end{aligned}
$$
根据 $W_{abcd}$ 的对称性,这两个张量是对称迹零的,且根据 Wely 张量的反对称性,
这两个张量是类空的,即:
$$
\bar{E}_{ab} n^b = 0, \quad \bar{B}_{ab}n^b = 0
$$
反过来,Wely
张量可以由电磁张量重构:
$$
W_{abcd} = 2(l_{a[c}\bar{E}_{d]b} - l_{b[c}\bar{E}_{d]a} - 
n_{[c}\bar{B}_{d]e}\epsilon^{e}_{\ \ \ abf}n^f - 
n_{[a}\bar{B}_{b]e}\epsilon^{e}_{\ \ \ cdf}n^f)
$$
其中 $l_{ab}$ 是 $g_{ab}$ 在类空超曲面上的诱导度量:
$$
l_{ab} = g_{ab} + 2n_an_b
$$
所以 $W_{abcd}$ 可以由 $\bar{E}_{ab}$ 和 $\bar{B}_{ab}$ 决定,这个过程称为 Bel
分解。

\section{线性化的 Bel 分解}
在上述带有微小扰动的度量 $g_{ab}$ 的情况下,令 $C_{cbcd}$ 是 Wely 张量的微分:
$$
W_{abcd} = C_{abcd}^0 + \epsilon C_{cbcd} + O(\epsilon^2)
$$
在真空情况下: $C_{abcd}^0 = 0$,分别定义电磁张量的微分:
$$
\begin{aligned}
    \bar{E}_{ab} & = E_{ab}^0 + \epsilon E_{ab} + O(\epsilon^2)\\
    \bar{B}_{ab} & = B_{ab}^0 + \epsilon B_{ab} + O(\epsilon^2)
\end{aligned}
$$
那么有:
$$
\begin{aligned}
    E_{ab} & = n^an^bC_{acbd}^0 = C_{0a0b}\\
    B_{ab} & = n^an^b(C_{abcd}^0)^* = (C_{0a0b})^* =
    C_{0aef}\epsilon^{ef}_{\ \ \ 0b}
\end{aligned}
$$
类似于 Maxwell 方程,可以定义磁场 $H_{ab}$ 和电场 $D_{ab}$:
$$
\begin{aligned}
    H_{ab} & = *C_{0a0b} = \frac{1}{2}\epsilon_{0a}^{\ \ \
    ef}C_{ef0b}\\
    D_{ab} & = *C_{0aef}^* = \frac{1}{2}\epsilon_{0a}^{\ \ \
    ef}C_{ef0b}^*
\end{aligned}
$$
根据上节的讨论,$C_{abcd}$ 具有和 $W_{abcd}$ 相同的对称性与反对称性,所以
$C_{abcd}$ 也可以看做是两个 2 形式的张量积,在 Minkowski 空间中,2 形式的基为:
\begin{align}
\label{2formbasis}
\boldsymbol{t}\wedge\boldsymbol{x},\  \boldsymbol{t}\wedge\boldsymbol{y},\ 
\boldsymbol{t}\wedge\boldsymbol{z},\  \boldsymbol{y}\wedge\boldsymbol{z},\ 
\boldsymbol{z}\wedge\boldsymbol{x},\  \boldsymbol{x}\wedge\boldsymbol{y}
\end{align}
那么双 2 形式的基就是上面的基的张量积,在这组基下,$C_{abcd}$ 可以写成矩阵形式:
$$
\begin{pmatrix}
    \boldsymbol{E} & \boldsymbol{B}\\
    \boldsymbol{H} & \boldsymbol{D}
\end{pmatrix}
$$
其中 $\boldsymbol{E}$ 是 $E_{ab}$ 的矩阵形式,
由于 
$$
*(\boldsymbol{t}\wedge\boldsymbol{x}) = \boldsymbol{y}\wedge\boldsymbol{z},
\quad 
*(\boldsymbol{t}\wedge\boldsymbol{y}) = \boldsymbol{z}\wedge\boldsymbol{x},
\quad 
*(\boldsymbol{t}\wedge\boldsymbol{z}) = \boldsymbol{x}\wedge\boldsymbol{y},
$$
$$
*(\boldsymbol{y}\wedge\boldsymbol{z}) = -\boldsymbol{t}\wedge\boldsymbol{x},
\quad 
*(\boldsymbol{z}\wedge\boldsymbol{x}) = -\boldsymbol{t}\wedge\boldsymbol{y},
\quad 
*(\boldsymbol{x}\wedge\boldsymbol{y}) = -\boldsymbol{t}\wedge\boldsymbol{z}
$$
所以 $C^*_{abcd}$ 的矩阵形式为:
$$
\begin{pmatrix}
    \boldsymbol{B} & -\boldsymbol{E}\\
    \boldsymbol{D} & -\boldsymbol{H}
\end{pmatrix}
$$
$^*C_{abcd}$ 的矩阵形式为:
$$
\begin{pmatrix}
    \boldsymbol{H} & \boldsymbol{D}\\
    -\boldsymbol{E} & -\boldsymbol{B}
\end{pmatrix}
$$
$^*C_{abcd}^*$ 的矩阵形式为:
$$
\begin{pmatrix}
    \boldsymbol{D} & -\boldsymbol{H}\\
    -\boldsymbol{B} & \boldsymbol{E}
\end{pmatrix}
$$
根据 $C_{abcd}$ 的对称性,反对称性,迹零性质可以有一下结论:
\begin{itemize}
    \item $\boldsymbol{E}, \boldsymbol{D}$ 是对称的,$\boldsymbol{H} =
        \boldsymbol{B}^T$. (由于 $C_{abcd}$ 的对称性)
    \item $\mathrm{tr}(\boldsymbol{E}) + \mathrm{tr}(\boldsymbol{H}) = 0$.
    \item $\boldsymbol{E}$ 是迹零的,$\boldsymbol{H}, \boldsymbol{D}$ 是对称的。
        $\boldsymbol{D} = -\boldsymbol{E}^T$.
\end{itemize}
所以 $C_{abcd}$ 的矩阵形式可以简化为:
$$
\begin{pmatrix}
    \boldsymbol{E} & \boldsymbol{B}\\
    \boldsymbol{B} & -\boldsymbol{E}
\end{pmatrix}
$$
其中 $\boldsymbol{E}$ 和 $\boldsymbol{B}$ 是对称且迹零的。

\section{线性 Bel 分解}
在真空中 $R_{ab} = 0$,$R_{abcd} = W_{abcd}$,而 $C_{abcd}$ 是 Wely 张量的微分,
$R_{abcd}$ 满足 Bianchi 恒等式, $C_{abcd}$ 也满足 Bianchi 恒等式:
$$
\nabla_{[a}C_{bc]de} = 0
$$
根据前面的讨论,$C_{abcd}$ 可以写成矩阵形式:
$$
\begin{pmatrix}
    \boldsymbol{E} & \boldsymbol{B}\\
    \boldsymbol{B} & -\boldsymbol{E}
\end{pmatrix}
$$
那么在矩阵形式下,Bianchi 恒等式可以写成:
\begin{align}
    \boldsymbol{\dot{E}} + \mathrm{curl}\boldsymbol{B} = 0\\
    \boldsymbol{\dot{B}} - \mathrm{curl}\boldsymbol{E} = 0
\end{align}
满足:
$$
\mathrm{div}\boldsymbol{E} = 0, \quad \mathrm{div}\boldsymbol{B} = 0
$$
上述方程称为线性 Einstein-Bianchi 方程。
称对称,迹零且散度为零的张量为 TSD 张量。
显然上述方程的解 $\boldsymbol{E}, \boldsymbol{B}$ 是 TSD 张量且属于 $H(curl,
\mathbb{M})$,其中 $\mathbb{M}$ 是 $3\times 3$ 实矩阵空间。
对于 $\boldsymbol{A} \in H(curl, \mathbb{M})$,
\begin{itemize}
    \item $\mathrm{curl}\boldsymbol{A}$ 散度为 0。
    \item $\boldsymbol{A}$ 对称那么 $\mathrm{curl}\boldsymbol{A}$ 迹为 0。
    \item $\boldsymbol{A}$ 是 TSD 张量,那么 $\mathrm{curl}\boldsymbol{A}$ 也是
        TSD 张量。
\end{itemize}

\section{平面引力波与线性 Einstein-Bianchi 方程} 
在前述的平面波情况下,度量的微小扰动 $h_{ab}$ 可以写为:
$$
h_{ab} = A_{ab}f(k_cx^c)
$$
其分量形式为:
$$
h_{\mu\nu} = 
\begin{pmatrix}
  0 & 0 & 0 & 0\\
  0 & A^{+} & A^{\times} & 0\\
  0 & A^{\times} & -A^{+} & 0\\
  0 & 0 & 0 & 0
\end{pmatrix}
f(\boldsymbol{k}\cdot\boldsymbol{x})
$$
其中 $\boldsymbol{k} = (k, 0, 0, k)$,$\boldsymbol{x} = (t, x, y, z)$。
那么 $C_{abcd}$ 的分量可以如下计算:
$$
\begin{aligned}
C_{\alpha1\beta1}=-\frac12\partial_{\alpha}\partial_{\beta}h_{11}=-\frac12k_{\alpha}k_{\beta}A^{+}f^{\prime\prime}(k_{c}x^{c}),\\C_{\alpha2\beta2}=-\frac12\partial_{\alpha}\partial_{\beta}h_{22}=\frac12k_{\alpha}k_{\beta}A^{+}f^{\prime\prime}(k_{c}x^{c}),\\C_{\alpha1\beta2}=-\frac12\partial_{\alpha}\partial_{\beta}h_{12}=-\frac12k_{\alpha}k_{\beta}A^{\times}f^{\prime\prime}(k_{c}x^{c})
\end{aligned}
$$
其中 $\alpha, \beta = 0, 3$, 其余分量为零。
对应的 $\boldsymbol{E}, \boldsymbol{B}$ 为:
$$
E_{\mu\nu} = C_{0\mu0\nu} = 
\begin{pmatrix}
    -A^{+} & -A^{\times} & 0\\
    -A^{\times} & A^{+} & 0\\
    0 & 0 & 0
\end{pmatrix}
\frac{k^2}{2}f^{\prime\prime}(\boldsymbol{k}\cdot\boldsymbol{x})
$$
$$
B_{\mu\nu} = C_{0\mu0\nu}^* =
\begin{pmatrix}
    -A^{\times} & A^{+} & 0\\
    A^{+} & A^{\times} & 0\\
    0 & 0 & 0
\end{pmatrix}
\frac{k^2}{2}f^{\prime\prime}(\boldsymbol{k}\cdot\boldsymbol{x})
$$
显然这样定义的 $\boldsymbol{E}, \boldsymbol{B}$ 是 TSD 张量,且满足线性
Einstein-Bianchi 方程。


\end{document}
