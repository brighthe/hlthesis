%!TEX program = xelatex
% 完整编译: xelatex -> biber/bibtex -> xelatex -> xelatex
\documentclass[lang=cn,a4paper,newtx]{elegantpaper}

\title{张量有限元研究计划书}
\author{陈春雨 \\ 湘潭大学\ 数学与计算科学学院}

\date{\zhdate{2024/9/18}}

% 本文档命令
\usepackage{array}
\usepackage{fontspec}
\newcommand{\ccr}[1]{\makecell{{\color{#1}\rule{1cm}{1cm}}}}

\begin{document}

\maketitle

\section{研究背景}

有限元方法\cite{brenner2008mathematical}是一种求解偏微分方程的数值方法,
于20世纪50年代由Courant等人提出,
我国的冯康院士独立于国外给出了有限元的严格数学理论基础。
有限元方法的基本思想是将复杂求解区域分解为简单几何单元
(如三角形、四边形、四面体、六面体等)组成的网格,在网格上建立具备合适连续性的
分片多项式函数组成的有限元空间, 然后基于 Galerkin
方法,将偏微分方程离散为代数方程组,通过求解代数方程组得到数值解。
在应用方面因为有限元方法可以在复杂的物理区域上建立数值模型,适用于各种复杂的边界条件,
所以在结构力学、传热学、流体力学、电磁场计算等领域都有广泛应用。

有限元方法因为其完备且优雅的数学基础,在理论方面也得到广泛研究,特别是最近关于
有限元外微分\cite{arnold2006finite, arnold2018finite}的研究,通过 Hilbert
复形将一类的问题联系起来,使用上同调,Hodge 分解等代数拓扑中的工具研究 Hilbert 
复形,给出一类问题的高效数值方法。
有限元外微分的主要思想是构造与 Hilbert 复形对应的有限元复形,
根据它们之间的有界上链投影算子,
建立离散有限元复形的稳定性,收敛性理论\cite{arnold2010finite},
且有限元外微分技术易于推广到高次元情况,为构造稳定且精确的有限元方法提供了理论框架。
基于这样的框架,很多悬而未决的问题被解决,如 divergence free 的 Stokes 元
\cite{neilan2015discrete},弹性复形中稳定的 $H(\mathbf{div}, \mathbb{S})$ 
协调对称张量有限元\cite{hu2015finite, hu2015family},可用于离散线性 
Einstein-Bianchi 方程的 divdiv 有限元复形和 Hessian 有限元复形
\cite{huliangma2022conforming, chen2024new, huliang2021conforming} 等。
最近 Arnold, Kaibo Hu \cite{arnold2021complexes} 使用 
Bernstein-Galfeld-Galfeld 序列
给出了一种基于已有的有界 Hilbert 复形,重新构造新复形的方法,
极大的扩展了有限元外微分的应用范围。

%\subsection{$L^2$ Hilbert 复形}
%Hessian 复形
%$$
%H^{q}\otimes \mathbb{R} \xrightarrow{\mathrm{hess}} H^{q-2}\otimes \mathbb{S}
%\xrightarrow{\mathrm{curl}} H^{q-3}\otimes \mathbb{T} \xrightarrow{\mathrm{div}}
%H^{q-4}\otimes \mathbb{R}^3 \rightarrow 0
%$$
%
%$$
%L^2\otimes \mathbb{R} \xrightarrow{\mathrm{hess}} L^2 \otimes \mathbb{S}
%\xrightarrow{\mathrm{curl}} L^2 \otimes \mathbb{T} \xrightarrow{\mathrm{div}}
%L^2 \otimes \mathbb{R}^3 \rightarrow 0
%$$
%
%
%
%Hilbert 复形形式如下: 
%$$
%\cdots\to H^{q}\otimes\mathbb{W}^k\xrightarrow{D^k}H^{q}\otimes\mathbb{W}^{k+1}\to\cdots,
%$$
%其中 $\mathbb{W}^k$ 是一个有限维的内积空间,$H^{q}$ 是一个通常的 Sobolev 空间。
%定义 domain 空间:
%$$
%H^{q}\mathbb{W}^k=\{u\in H^{q}\otimes\mathbb{W}^k, D^k u \in H^{q}\otimes\mathbb{W}^{k+1}\},
%$$
%那么可以得到一个以上 Hilbert 复形的 domain 复形:
%$$
%\cdots\to H^{q}\mathbb{W}^k\xrightarrow{D^k}H^{q}\mathbb{W}^{k+1}\to\cdots,
%$$
%$q = 0$ 时对应的是 $L^2$ Hilbert 复形,这是研究最广泛的 Hilbert 复形。

有限元外微分中主要研究的是有界 Hilbert 复形,以 3 维 $L^2$ de Rham 复形为例:
$$
\mathbb{R} \xrightarrow{\subset} H^1(\Omega) \xrightarrow{\mathrm{grad}} H(\text{curl}, \Omega)
\xrightarrow{\mathbf{curl}} H(\text{div}, \Omega) \xrightarrow{\mathbf{div}}
L^2(\Omega) \rightarrow 0 
$$ 
其中 $\Omega$ 是 $\mathbb{R}^3$ 中的多面体区域. 将 $\Omega$ 剖分为单纯形网格
$\mathcal{T}$,构造 $\mathcal{T}$ 上的
有限元复形的关键之一是构造具有对应协调性的有限元,根据如下格林公式, 
$$
\begin{aligned}
    (\mathrm{grad} u, \boldsymbol{v}) & = - (u, \mathbf{div} \boldsymbol{v}) +
    (u\boldsymbol{v}\cdot \boldsymbol{n})_{\partial \Omega}\\
    (\mathbf{curl} \boldsymbol{u}, \boldsymbol{v}) & = 
    (\boldsymbol{u}, \mathbf{curl} \boldsymbol{v}) +
    (\boldsymbol{n}\times \boldsymbol{u}, \boldsymbol{v})_{\partial \Omega}\\
    (\mathbf{div} \boldsymbol{u}, v) & = - (\boldsymbol{u}, \mathrm{grad} v) +
    (\boldsymbol{u}\cdot \boldsymbol{n}, v)_{\partial \Omega}
\end{aligned}
$$
可知对于分片多项式函数 $u$ (向量函数 $\boldsymbol{u}$), $H^1$
协调性要求函数在整个网格上连续, $H(\text{curl})$ 协调性要求函数
在两个相邻单元的面上切向分量连续, $H(\text{div})$
协调性要求函数在两个相邻单元的面上法向分量连续。 常用的 $H^1$
协调的有限元有 Lagrange 有限元、Hermite 有限元等, $H(\text{curl})$
协调的有限元有第一类 Nedelec 有限元、第二类 Nedelec 有限元,
$H(\text{div})$ 协调的有限元有 RT 有限元、BDM 有限元。

从实现的角度来看,标量的 Lagrange 有限元基函数简单容易构造,
而 $H(\text{curl})$ 和 $H(\text{div})$ 协调的有限元属于向量型有限元,
其自由度相比于标量的 Lagrange 有限元复杂,基函数的构造也相对困难。
传统的向量型有限元的基函数构造方式是在参考单元上构造基函数,然后 对于
$H(\text{curl})$ 协调有限元使用协变 Piola 变换, 对于
$H(\text{div})$ 协调有限元使用逆变 Piola 变换,
将参考单元上的基函数映射到实际单元上,从而得到实际单元上的基函数。
在 \cite{Chen2024GeometricDA} 中本人及合作者基于对 $k$
次多项式空间的几何分解,给出了一种不需要映射的基函数构造方法,使用
Lagrange 有限元基函数乘以不同的向量,对函数施加不同的连续性,可以得到 BDM
元和第二类 Nedelec 元的基函数,与\cite{christiansen2018nodal}
中 Hu-Zhang 元基函数构造方法类似。

Stokes 复形是一个和 $L^2$ de Rham 复形类似的复形,三维的 Stokes 元复形如下:
$$
\mathbb{P}_1(\Omega) \xrightarrow{\subset} H^2(\Omega) \xrightarrow{\mathrm{grad}} 
H^1(\text{curl}, \Omega)
\xrightarrow{\mathbf{curl}} H^1(\Omega, \mathbb{R}^3) \xrightarrow{\mathbf{div}}
L^2(\Omega) \rightarrow 0
$$
其中涉及的微分算子与 de Rham 复形一致,不同点在于 Stokes
复形的空间正则性更高。对于不可压流体的数值求解 Stokes 复形有重要应用,
其关键在于构造 Stokes 复形中的后两个空间对应的有限元,由于三维 $H^2$ 
协调的有限元在顶点上要求 4 阶连续,所以对应的 $H^1$ 协调向量元要在顶点上 2 阶连续,
$L^2$ 协调有限元在顶点上 1 阶连续。Neilan 在 \cite{neilan2015discrete} 
中给出了对应的构造。最近 Hu, Lin, Wu 等人在\cite{hu2023construction} 
构造了具有 $C^m$ 光滑性的有限元,
为其他类型具有更高光滑性的有限元构造提供了理论基础。
基于这样的光滑元,Chen, Huang\cite{chenhuangstokes2024} 构造了具有不同光滑性的 
de Rham 有限元复形。需要说明的是,由于多项式的光滑性,
更高光滑性的有限元需要把自由度更多的集中到顶点上,这导致了有限元的的次数升高,
例如三维 $H^2$ 协调的有限元需要至少 9 次多项式,$H^3$ 协调的有限元需要至少 17 
次多项式,这对显式基函数构造来说是一个挑战,本人在博士期间对此进行了研究,
给出了一种\cite{hu2023construction}
中光滑有限元基函数的构造方法,构造过程仅需要求解一个下三角矩阵的逆,
这样的基函数结合\cite{Chen2024GeometricDA} 中构造 BDM 元,第二类 Nedelec 
元基函数的方法,将光滑元基函数乘上不同的向量,
对函数在网格单元的边界上施加不同的连续性,可以得到具有不同光滑性的 de Rham 
有限元复形的基函数。

%\subsection{弹性复形}
除了 de Rham 复形外,三维空间中有三种基本的复形:弹性复形,Hessian 复形,divdiv
复形。
三维空间的弹性复形: 
$$
RM \xrightarrow{\subset} H^1(\Omega, \mathbb{R}^3) 
\xrightarrow{\mathrm{def}} H(\mathbf{inc}, \Omega, \mathbb{S})
\xrightarrow{\mathbf{inc}} H(\mathbf{div}, \Omega, \mathbb{S})
\xrightarrow{\mathbf{div}} L^2(\Omega, \mathbb{R}^3) \rightarrow 0
$$ 
弹性复形在固体力学的数值模拟中有重要应用,
在固体力学问题的数值求解中,常用方式是位移方法,以位移为未知量,
通过位移的梯度计算应力与应变,但是对于粘弹性问题,塑性问题,位移法会出现一些问题,
将应力和位移一起作为未知量的混合方法是一种合适的选择,其中位移属于 $L^2$ 空间,
应力属于 $H(\text{div}, \mathbb{S})$ 空间,
在1970年就提出了混合方法\cite{fraeijs1965displacement},但是其对应的稳定的有限元方法一直没有解决,
直至\cite{arnold2002mixed}中,Arnold 等人根据有限元外微分技术给出了一种稳定的有限元离散方法。
根据格林公式,
$$
\begin{aligned}
  (\mathbf{div} \boldsymbol{\sigma}, v) & = -(\boldsymbol{\sigma}, \mathrm{grad} v) +
  (\boldsymbol{\sigma}\cdot \boldsymbol{n}, v)_{\partial \Omega}\\
\end{aligned}
$$ $H(\mathbf{div}, \mathbb{S})$ 协调的有限元要求
$\boldsymbol{\sigma}$ 在两个相邻单元的边上法向分量连续,在\cite{hu2015family,
hu2015finite}
中 Hu, Zhang 提出了以 $\mathbb{P}_k(\mathbb{S})$ 为形函数空间的
$H(\mathbf{div}, \mathbb{S})$ 协调的有限元: Hu-Zhang 元,其中由于
$P_k(\mathbb{S})$
的光滑性与对称性,使得有限元函数必须在顶点和边上施加额外的连续性。
在\cite{chenhuang2022finitemc}中,Chen, Huang
提出了一种 $H(\mathbf{inc}, \mathbb{S})$ 协调的有限元,与 Hu-Zhang
元,$H^1$ 协调的 Neilan 元,一起组成第一个完整的三维弹性复形。

%\subsection{{$\mathrm{div}\mathbf{div}$ 复形}}
对于 Hessian 复形其最初在\cite{pauly2020divdiv}中被提出用于推导 
$\mathbb{R}^3$ 中双调和问题的 Helmholtz 分解,Hessian 复形如下:
$$
\mathbb{P}_1(\Omega)\stackrel{\subset}{\longrightarrow}
H^2(\Omega;\mathbb{R})\stackrel{\mathrm{hess}}{\longrightarrow}H(\mathrm{curl},\Omega;\mathbb{S})\stackrel{\mathrm{curl}}{\longrightarrow}H(\mathrm{div},\Omega;\mathbb{T})\stackrel{\mathrm{div}}{\longrightarrow}L^2(\Omega;\mathbb{R}^3)\longrightarrow0,
$$
Hu, Liang \cite{huliang2021conforming}
提出了对应的 Hessian 有限元复形,与 Stokes 复形类似,
$H^2$ 协调的有限元在顶点上 4 阶连续,在边上 2
阶连续,因此后续的有限元需要有不同程度的更高连续性要求。

$\mathrm{div}\mathbf{div}$ 复形是最近的研究热点,
其可以用于双调和方程的混合元求解, 线性 Einstein-Bianchi 方程的求解。
$\mathrm{div}\mathbf{div}$ 复形如下: 
$$
RT \xrightarrow{\subset} H^1(\Omega, \mathbb{R}^3) 
\xrightarrow{\mathrm{dev}\ \mathrm{grad}} 
H(\text{sym curl}, \Omega, \mathbb{T}) \xrightarrow{\mathrm{sym\ curl}} 
H(\mathrm{div}\mathbf{div}, \Omega, \mathbb{S}) \xrightarrow{\mathrm{div}\mathbf{div}} L^2(\Omega) \rightarrow 0
$$ 
在双调和方程混合元方法中 未知量的 Hessian 值也被当做未知量,属于
$H(\mathrm{div}\mathbf{div}, \mathbb{S})$ 空间,根据格林公式, 
$$
\begin{aligned}
(\mathrm{div}\mathrm{div}\boldsymbol{\tau},v)_{K} & =
(\boldsymbol{\tau},\nabla^{2}v)_{K} - 
\sum_{F\in \partial K}\sum_{e\in \partial F}
(\boldsymbol{n}_{F,e}^{\intercal}\boldsymbol{\tau}\boldsymbol{n},v)_{e}\\
& - \sum_{F\in \partial K}
[(\boldsymbol{n}^{\intercal}\boldsymbol{\tau}\boldsymbol{n},\partial_{n}v)_{F} 
- (\boldsymbol{n}^{\intercal}\mathrm{div}\boldsymbol{\tau}+\mathrm{div}_{F}
(\boldsymbol{\tau}\boldsymbol{n}),v)_{F}]
\end{aligned}
$$ 
Chen, Huang
\cite{Chenhuangdivdiv2020, chen2022finite} 
提出了 $\boldsymbol{\sigma}$ 在面,边的法平面上连续,
$\mathbf{div}_F(\boldsymbol{\sigma n})+\boldsymbol{n}^T 
\mathbf{div}\boldsymbol{\sigma}$ 在面上连续的 $H(\mathrm{div}\mathbf{div},
\mathbb{S})$ 协调的有限元,并在 \cite{chenhuangsiam2022finite}
中推广到了任意维。需要说明的是,这个有限元在低维的单形上要求了函数值连续性,
基于这个有限元,Hu, Liang, Ma \cite{huliangma2022conforming} 定义了对应的 $H^1$ 
协调的有限元,$H(\text{sym curl}, \mathbb{T})$ 协调的有限元,
共同组成了第一个完整的 $\mathrm{div}\mathbf{div}$ 有限元复形, 并将其应用于线性
Einstein-Bianchi 方程,给出了线性 Einstein-Bianchi
方程对偶公式的有限元离散格式。
根据如下格林公式:
$$
(\mathrm{div}\mathbf{div}\boldsymbol{\sigma}, v) = 
-(\boldsymbol{\sigma}, \mathrm{hess} v) +
(\boldsymbol{\sigma}\cdot \boldsymbol{n}, \mathrm{grad} v)_{\partial \Omega}
+ (\mathrm{div}\boldsymbol{\sigma}\cdot \boldsymbol{n}, v)_{\partial \Omega}
$$ 
Hu, Ma, Zhang \cite{hu2021family}
提出了一种 $H(\mathrm{div}\mathbf{div}, \mathbb{S})$ 协调且 
$H(\mathbf{div}, \mathbb{S})$ 协调的有限元,满足
$\boldsymbol{\sigma}\cdot \boldsymbol{n}$ 以及
$\mathrm{div}\boldsymbol{\sigma}\cdot \boldsymbol{n}$
在两个相邻单元的面上连续。这样的连续性更强,
所以其也包含在顶点上的超光滑自由度。针对超光滑性的问题,最近在
\cite{chen2024new} 中,
作者通过重新分配自由度,将顶点上的自由度分配到边和面上,
从而消去了顶点上的超光滑自由度,构造的 $H(\mathrm{div}\mathbf{div}, \mathbb{S})$
协调有限元可以使用杂交化技术,作者将其应用于了双调和方程的混合元求解,此外,
基于这个有限元,作者还给出了两个新的 divdiv 复形,
其中的 $H^1(\Omega, \mathbb{R}^3)$ 协调的有限元分别是 Lagrange 元和 Hermite 元。

这三个复形中涉及的 $H(\mathbf{inc}, \mathbb{S})$、$H(\mathbf{div}, \mathbb{S})$、
$H(\mathrm{sym\ curl}, \mathbb{T})$、$H(\mathrm{div}\mathbf{div}, \mathbb{S})$
协调有限元都是张量型有限元,自由度相比于向量型的有限元还要更加复杂,
其基函数的构造也更加困难。对于向量型的有限元,尚且可以通过协变或逆变
Piola 变换,
将参考单元上的基函数映射到实际单元上,但是对于张量型的有限元,
由于其复杂的连续性要求,类似的保持连续性的映射构造起来非常困难。 对于 Hu-Zhang 元,
在\cite{christiansen2018nodal},使用了 Lagrange
基函数乘以不同的对称向量的方式,构造了其显式基函数,
而且形式非常简单,实现起来也相对容易,在 \cite{hu2021family} 中作者对于
$H(\mathrm{div}\mathbf{div}, \mathbb{S})\cap H(\mathbf{div}, \mathbb{S})$
协调的有限元, 给出了二维的构造方法。
但是其他的张量型有限元目前还没有显式的基函数构造,这对将这些新型张量有限元
和基于它们设计出来的数值离散格式应用到实际问题中是一个障碍。
从\cite{christiansen2018nodal, Chen2024GeometricDA}中的方法出发,
以\cite{Chenhuangdivdiv2020, chen2022finite}中的 
$H(\mathrm{div}\mathbf{div}, \mathbb{S})$ 协调的有限元为例,
假设 $\boldsymbol{\sigma} = \sigma \boldsymbol{M} \in
\mathbb{P}_k(\mathbb{S})$,其中 $\boldsymbol{M}$ 是一个对称矩阵,那么
$$
\mathrm{div}_F(\boldsymbol{\sigma n}) + \boldsymbol{n}^T 
\mathrm{div}\boldsymbol{\sigma} = 2\sum_{i=0}^{n-2}
\boldsymbol{n}^T \boldsymbol{M} \boldsymbol{t}_{F, i} \frac{\partial
\sigma}{\partial \boldsymbol{t}_{F, i}} + \boldsymbol{n}^T \boldsymbol{M}
\boldsymbol{n} \frac{\partial \sigma}{\partial \boldsymbol{n}}
$$
所以对于
$\mathrm{div}_F(\boldsymbol{\sigma n}) + \boldsymbol{n}^T 
\mathrm{div}\boldsymbol{\sigma}$ 的连续性就可以转化为 $\sigma$
的函数值和法向导数的连续性,因此使用具有特殊连续性的函数,
乘上合适的对称矩阵来构造 $H(\mathrm{div}\mathbf{div}, \mathbb{S})$ 
协调的有限元的基函数是可行的。
这种方法可以推广到其他的张量有限元的基函数构造。

另一方面,根据前面的介绍,由于协调性的要求,张量有限元经常会需要额外的连续性,如
Hu-Zhang 元要求在顶点上连续,$H(\mathrm{div}\mathbf{div}, \mathbb{S})$
协调的有限元要求在顶点和边的法平面上连续,
而且以此构造的有限元复形会有更多的连续性要求,
如 \cite{chenhuang2022finitemc} 中的弹性有限元复形中,
$H(\mathbf{inc}, \mathbb{S})$ 协调的有限元在顶点上 1
阶连续, $H^1$ 协调的有限元在顶点上 2 阶连续。类似的,
\cite{huliangma2022conforming} 中的 $\mathrm{div}\mathbf{div}$
有限元复形中,$H(\mathrm{sym\ curl}, \mathbb{T})$ 协调的有限元在顶点上
1 阶连续,$H^1$ 协调的有限元在顶点上 2 阶连续。
这些额外的连续性对有限元的设计来说是一个挑战,对算法实现来说也是一个挑战,
在最近研究中,基于分布 Hilbert
复形构造分布有限元,可以降低连续性要求,
如对于弹性复形,\cite{christiansen2011linearization}
中作者将 Regge calculus 解释为一种有限元称为 Regge 
有限元,其中微分几何中的度量在 Regge
弹性复形中是一个在边上切向切向连续的分片对称常值矩阵。
Regee 有限元在 \cite{li2018regge} 中被推广到了高次情况,并将其应用于 Einstein 场方程的保结构离散。
对于弹性问题的求解 \cite{pechstein2011tangential, pechstein2018analysis}中 TDNNS
方法使用法向法向连续的对称张量离散应力,切向连续的函数离散位移,
得到了稳定的弹性方程的混合元离散格式。
在\cite{chen2024new}中,提出了一个分布 divdiv 有限元复形,构造了从 0
次到任意次的有限元复形。最近在 \cite{hu2023distributional} 中作者将 Regge
有限元的构造方式推广到了 Hessian 复形和 divdiv 复形,不过仅包含最低次的情况。
在 \cite{chenhuang2023distributional} 中作者研究了 $\mathrm{curl div}$ 复形的分布有限元。
这些分布有限元的构造方式可以降低连续性的要求,构造起来简单,方便应用于实际问题,
具有重要的研究价值。

这些新提出的有限元方法,尤其是张量有限元方法,很多都仅停留在理论研究方面,
除 Hu-Zhang 元外,其他方法在实际问题中应用很少,甚至数值算例都很少见。
当前我们国家大力推进工业软件,科学计算软件等方面的发展,
有限元方法在工程领域有着广泛的应用,
现在又有这些可以用于解决实际问题的新有限元方法,
因此开发一套简单、易用且高效的开源张量有限元模块,
将这些新提出的有限元方法高效的实现,并应用于实际问题中具有重要的意义。
本人在博士期间深度参与了湘潭大学魏华
{\CJKfontspec{SimSun} 祎}
教授主导的开源软件 FEALPy 
的研发,加之现如今各种开源高性能科学计算库的发展,
如 Numpy,Pytorch,Jax 等,其提供了高效的张量计算功能,
且 FEALPy 包含丰富的网格模块,有限元计算模块,
因此基于 FEALPy 开发这样的模块是可行的。

\section{研究目标}
基于上述背景,本研究将围绕分布型的张量有限元理论,
现有的张量型有限元显式基函数的构造,及相关算法的开源程序模块开发及应用展开研究。
具体为以下几个方面:
\begin{enumerate}
    \item[(1)]
        \textbf{分布型张量有限元方法的理论研究}。研究\cite{hu2023distributional} 
        中提出的 Hessian ,divdiv 分布有限元复形,将这些复形推广到任意次,
        构造更简单的有限元算法用于求解线性
        Einstein-Bianchi 方程和双调和方程; 对提出的有限元方法的稳定性,
        收敛性给出理论分析。
    \item[(2)] \textbf{张量有限元显式基函数的构造}。
        以 Bernstein 多项式为基础,研究 \cite{hu2023construction}
        中提出的光滑元的基函数构造方法,并将其应用于
        \cite{chenhuangstokes2024} 中提出的光滑 de Rham 有限元复形,
        \cite{chenhuang2022finitemc, huliangma2022conforming} 中的
        弹性有限元复形,$\mathrm{div}\mathbf{div}$
        复形中的涉及的有限元的显式基函数构造;研究基函数构造的一般原理,
        给出一种一般的有限元基函数构造方法。
    \item[(3)] \textbf{线性 Einstein-Bianchi 方程和双调和方程的数值求解}。
        将提出的有限元方法应用于线性 Einstein-Bianchi
        方程和双调和方程的数值求解,使用典型例子设计数值实验
        来验证算法的有效性,收敛性情况。
    \item[(4)] \textbf{高性能张量有限元软件模块研制}。基于 Numpy,Pytorch 
        Jax 等张量计算库以及开源软件 FEALPy 
        研究如何使用数组化和面向对象的方式实现以上张量有限元方法,
        进而设计高效易用的张量有限元模块,
        充分发挥硬件性能的同时又易于算法工程师修改底层算法。
        最后使用典型的双调和方程验证算法模块的有效性。
\end{enumerate}

\section{时间安排}
\textbf{2025-2026年:} 针对 $H(\mathrm{div}\mathbf{div}, \mathbb{S}), H(\mathrm{sym\ curl},
\mathbb{T})$
协调有限元以及不同光滑性的 de Rham 有限元设计其显式基函数,
给出基函数构造的基本原理,对一般的张量有限元基函数构造方式进行分析。
设计完善张量有限元算法的程序模块,
实现已经提出显式基函数的张量有限元,并使用 Stokes 问题,
双调和方程中的典型问题进行验证。

\textbf{2026-2027年:} 研究\cite{hu2023distributional} 
中提出的 Hessian ,divdiv 分布有限元复形,将这些复形推广到任意次。
设计可以用于求解线性 Einstein-Bianchi 方程的分布有限元方法; 
对提出的有限元方法的稳定性,收敛性给出理论分析。


\section{本人博士期间的研究情况}

本人在博士期间主要的研究内容是虚单元方法和有限元方法。
虚单元方法可以看做是有限元方法的扩展,可以定义在任意多边形,多面体区域上,
但是由于其基函数是某个方程的解,没有显示表达式,因此其计算必须投影到多项式空间,
这样就需要引入稳定化项。

在虚单元方面,本人主要在提高虚单元光滑性,去除虚单元方法中的稳定化项,
以及低正则性解的虚单元方法等方面进行了研究。
本人及合作者给出了任意维任意次任意 $m\in \mathbb{N}$ 的 $H^m$
协调虚单元\cite{chen2022conforming}, 其不含有超光滑的自由度,
且对多项式次数 $k$ 的要求是
$k\geq m$,小于 $H^m$ 协调有限元的要求
$k\geq (m-1)2^d+1$。在\cite{chen2024virtual}中,本人及合作者基于有限元外微分,
构造了多边形多面体上的宏元复形,
将虚单元函数的梯度投影到宏元空间,可以去掉虚单元方法中的稳定化项。
在\cite{chen2023anisotropic}中,最低次的 $H(\mathrm{curl})$
协调虚单元与多边形剖分成的三角网格上最低次棱元同构,
基于这个事实,本人及合作者分析了对于低正则性解,虚单元方法的最优收敛性。
此外,根据虚单元方法对网格单元的要求较低的特点,本人及合作者提出了一种移动界面问题的
虚单元方法,其优点是网格生成简单,且无需对上一个时间层的解在当前时间层的网格上
进行插值。

在有限元方法方面,本人主要对构造有限元基函数进行了研究。对于向量型有限元,
基于对 $(\mathbb{P}_k)^d$ 的几何分解, 本人及合作者使用 Lagrange
有限元基函数乘以不同的向量,对函数施加不同的连续性,
提出了一种不需要映射的 BDM 元和第二类 Nedelec 元基函数的构造方法。
此外,对于最近胡俊教授等人提出的任意维任意次光滑元构造\cite{hu2023construction},
本人及合作者基于\cite{chen2021geometric}中对光滑元的几何分解,
给出了一种任意维任意次任意
$r\in \mathbb{N}$ 的 $C^r$ 光滑有限元基函数的构造方法,
其构造过程仅需要求解一个下三角矩阵的逆,且自由度形式与 Lagrange 元统一,
易于实现。

根据以上情况,本人基本具备了研究张量有限元的基本知识和技能,有能力完成本研究的目标,
并且本人对有限元方法的研究有浓厚的兴趣,希望能够在这个领域做出一些有意义的工作。

%\section{看的论文}
%
%\begin{enumerate}
%\item
%  有限元外微分
%\item
%  complex from complex
%\end{enumerate}
%
%\section{问题}
%
%\begin{enumerate}
%\item
%  有限元外微分为什么能够提供稳定的有限元方法?其和 InfSup
%  条件有什么关系?
%\item
%  为什么 divdiv 复形有这个版本: $$
%   RT \xrightarrow{\subset} H^1(\Omega, \mathbb{R}^3) \xrightarrow{\mathrm{dev}\mathrm{grad}} 
%   H(\text{sym curl}, \Omega, \mathbb{T}) \xrightarrow{\mathrm{sym\ curl}} 
%   H(\mathrm{div}\mathbf{div}, \Omega, \mathbb{S}) \xrightarrow{\mathrm{div}\mathbf{div}} L^2(\Omega) \rightarrow 0
%   $$ 还有这个版本: distributional divdiv complex $$
%   RT \xrightarrow{\subset} H^1(\Omega, \mathbb{R}^3) \xrightarrow{\mathrm{dev}\mathrm{grad}} 
%   H(\text{sym curl}, \Omega, \mathbb{T}) \xrightarrow{\mathrm{sym\ curl}} 
%   L^2(\Omega, \mathbb{S}) \xrightarrow{\mathrm{div}\mathbf{div}} H^{-2}(\Omega) \rightarrow 0
%   $$ 似乎还有这个版本: $$
%   RT \xrightarrow{\subset} H^1(\Omega, \mathbb{R}^3) \xrightarrow{\mathrm{dev}\mathrm{grad}} 
%   H(\text{sym curl}, \Omega, \mathbb{T}) \xrightarrow{\mathrm{sym\ curl}} 
%   H(\mathrm{div}\mathbf{div}^{-1}, \Omega, \mathbb{S}) 
%   \xrightarrow{\mathrm{div}\mathbf{div}} H^{-1}(\Omega) \rightarrow 0
%   $$ 他们有什么关系?
%\item
%  构造 Hilbert
%  复形对应的有限元复形的时候,要求他们之间要有一个上链投影算子,
%  但是看很多文章都没有提到这个上链投影算子,这个是有什么理论保证其存在性么?
%\end{enumerate}

\bibliographystyle{unsrt}
\bibliography{./ref}

\end{document}
