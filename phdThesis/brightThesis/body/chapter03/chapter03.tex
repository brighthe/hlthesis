% !TeX root = ../../brightPhD.tex
\chapter{任意次多单元族拉格朗日有限元拓扑优化比较研究}
\label{chap:lagrange_comparison}

\section{引言}
\label{sec:ch3_intro}

拓扑优化是一类用于在给定设计域内优化材料分布的计算设计方法,旨在满足既定物理与几何约束的同时优化目标函数。其基本流程为 “分析—优化” 闭环:求解相应边值问题获得状态场,并据此更新设计变量以迭代逼近最优。在变密度法框架下,设计域由有限元方法离散,并将每个几何实体(如单元或节点)关联到一个设计变量(密度)$\rho\in[0,1]$。理想的二元拓扑设计满足 $\rho\in\{0,1\}$($\rho=1$ 为实体相,$\rho=0$ 为空洞相),数值实现中则通过惩罚化材料插值、最小特征尺度控制与适当的优化策略,将连续松弛变量逐步逼近上述 0–1 结构(为数值稳定常取 $\rho\in[\rho_{\min},1]$,$\rho_{\min}>0$ 很小)。


在传统拓扑优化中,常以低阶四边形或三角形单元离散设计域,虽实现简便、代价较低,但优化过程中易出现棋盘格与网格依赖性等数值病态,且边界呈现明显锯齿化特征,通常需配合过滤与投影以控制最小特征尺度并获得网格无关解。相较之下,高阶拉格朗日单元在可比自由度规模下具有更高的分析精度与更强的边界刻画能力,有望改善目标函数收敛与几何质量。鉴于此,本章在统一物理长度尺度的设定下(过滤半径 $⁡r_{\min}$ 在物理坐标系中统一设定),系统比较任意阶次 $k$ 与多单元族在拓扑优化中的表现。


就设计变量布置而言,本文对比两类常用方案:其一,单元密度(每个单元一个密度变量,密度场间断),实现简单、装配直接,但更易诱发上述病态且分辨率受平均网格尺寸所限;其二,节点密度(每个节点一个密度变量,密度场连续),该做法在粗网格下可能出现 “岛化” 现象,但当位移分析采用高阶有限元($k\geq2$)时,由于对层状材料分布的刚度评估更为准确,此类不利效应可显著缓解。尽管围绕上述两类表征已有大量工作,在统一最小特征尺度、求解与约束参数的前提下,针对不同网格类型与质量、不同单元族与任意阶 $k$ 的系统比较仍相对不足,且覆盖二维/三维同时并列评估单元密度与节点密度两类表征的综合性研究仍显缺失。本章即在这一统一框架下开展全面对比与量化评估。


\section{线弹性问题的连续模型与变分形式}
\label{sec:linear_elastic}

\subsection{线弹性理论的基本假设}
\label{subsec:elastic_assumption}

结构拓扑优化中的状态方程通常采用经典小变形线弹性理论,其为柔顺度最小化等优化问题提供了明确的物理基础与数学结构。本节简要回顾该理论所依赖的标准假设\cite{bendsoeTopologyOptimization2004}
\begin{enumerate}
	\item 连续介质假设:将由离散原子、分子构成的真实材料在宏观尺度上理想化为连续、致密且可无限分割的介质。
	
	\item 小变形假设:假设结构在载荷作用下产生的位移和转动均是微小的,即位移梯度张量 $\nabla\boldsymbol{u}$ 的范数远小于 1,该假设带来两个关键简化:
	\begin{enumerate}
		\item 几何线性化:应变与位移的关系可由线性的几何方程描述。应变张量 $\boldsymbol{\varepsilon}$ 被定义为位移向量 $\boldsymbol{u}$ 的对称梯度:
		\[
		\boldsymbol{\varepsilon} = \frac{1}{2}(\nabla\boldsymbol{u} + (\nabla\boldsymbol{u})^T).
		\]
		\item 平衡方程简化:可忽略变形前后物体几何构型的差异,直接在未变形的初始构型上建立静力平衡方程。
	\end{enumerate}
	
	\item 线性本构假设:假设材料的力学响应是线性的,即应力张量 $\boldsymbol{\sigma}$ 与应变张量 $\boldsymbol{\varepsilon}$ 之间服从广义胡克定律:
	\[
	\boldsymbol{\sigma} = \mathbb{C}:\boldsymbol{\varepsilon},
	\]
	其中 $\mathbb{C}$ 为四阶弹性刚度张量,包含了描述材料弹性特性的所有信息,且在该假设下被视为不依赖于应变的常数张量。
	
	\item 各向同性假设:假设材料的力学性质在所有方向上都是相同的。在此假设下,$\mathbb{C}$ 仅由两个独立的材料常数来完全决定,以拉梅常数 $\lambda$ 和 $\mu$ 表示:
	\[
	\boldsymbol{\sigma} = 2\mu\boldsymbol{\varepsilon} + \lambda\operatorname{tr}(\boldsymbol{\varepsilon})\boldsymbol{I},
	\]
	其中 $\operatorname{tr}(\boldsymbol{\varepsilon})$ 表示应变张量的迹,$\boldsymbol{I}$ 为二阶单位张量。
\end{enumerate}

工程实践中常用的材料参数体系包括:
\begin{itemize}
	\item 杨氏模量 $E$:描述材料在单轴拉伸或压缩时的刚度。
	\item 泊松比 $\nu$:描述材料在单轴加载时横向应变与纵向应变的比值。
	\item 拉梅第一常数 $\lambda$:主要出现在应力-应变关系的数学表述中,无直接的物理解释。
	\item 剪切模量 $\mu$:描述材料抵抗剪切变形的能力,亦称拉梅第二常数。
	\item 体积模量 $K$:描述材料在各向同性压力下的体积变化特性。
\end{itemize}

在经典三维各向同性线弹性理论中,上述材料参数之间存在如下转换关系:
\[
\lambda = \frac{E\nu}{(1+\nu)(1-2\nu)},\quad \mu = \frac{E}{2(1+\nu)},\quad K = \frac{E}{3(1-2\nu)},
\]
反过来有
\[
E = \frac{\mu(3\lambda + 2\mu)}{\lambda + \mu},\quad \nu = \frac{\lambda}{2(\lambda + \mu)}.
\]

\subsection{线弹性问题的强形式}
\label{subsec:elastic_strong}

在不考虑时间效应的静态情况下,求解域 $\Omega$ 内的任意一点满足本构三类控制方程:

\begin{enumerate}
	\item 静力平衡方程:描述物体内任意点的内力平衡关系
	\[
	\operatorname{div}\boldsymbol{\sigma} + \boldsymbol{b} = \boldsymbol{0}
	\quad\text{in }\Omega.
	\]
	
	\item 几何方程:描述应变场与位移场之间的运动学关系
	\[
	\boldsymbol{\varepsilon}(\boldsymbol{u}) = 
	\frac{1}{2}\bigl(\nabla\boldsymbol{u} + (\nabla\boldsymbol{u})^{\!\top}\bigr)
	\quad\text{in }\Omega.
	\]
	
	\item 本构方程:描述各向同性线弹性材料的应力–应变关系
	\[
	\boldsymbol{\sigma} = \mathbb{C}:\boldsymbol{\varepsilon}
	\quad\text{in }\Omega.
	\]
\end{enumerate}

设 $\partial\Omega=\Gamma_D\cup\Gamma_N$ 且 $\Gamma_D\cap\Gamma_N=\emptyset$,给定边界数据 $\boldsymbol{u}_D$ 与 $\boldsymbol{g}$,求 $\boldsymbol{u}:\Omega\to\mathbb{R}^d$,使得
\begin{equation}
	\begin{cases}
		-\operatorname{div}\boldsymbol{\sigma}(\boldsymbol{u}) = \boldsymbol{b}      &\text{in}\ \Omega,   \\[1pt]
		\boldsymbol{u} = \boldsymbol{u}_D                                            &\text{on}\ \Gamma_D, \\[1pt]
		\boldsymbol{\sigma}\boldsymbol{n} = \boldsymbol{g}                           &\text{on}\ \Gamma_N
	\end{cases}
	\label{eq:strong_form}
\end{equation}
其中 $\boldsymbol{n}$ 为外法向量。

\subsection{线弹性问题的弱形式与变分原理}
\label{subsec:elastic_weak}

为便于建立能量型表述,以下在合适的 Sobolev 空间内给出弱形式,记
\[
\boldsymbol{V} := H^1(\Omega;\mathbb{R}^d),\quad\boldsymbol{V}_0 := \{\boldsymbol{v}\in\boldsymbol{V}:\boldsymbol{v} = \boldsymbol{0}~\text{on}~{\Gamma_D}\}
\]
在上述设置下,取任意 $\boldsymbol{v}\in\boldsymbol{V}_0$ 与强形式 $-\operatorname{div}\boldsymbol{\sigma}(\boldsymbol{u}) = \boldsymbol{b}$ 作内积:
\[
(-\operatorname{div}\boldsymbol{\sigma},\boldsymbol{v}) = (\boldsymbol{b},\boldsymbol{v}),
\]
应用 Green 公式得:
\[
(-\operatorname{div}\boldsymbol{\sigma},\boldsymbol{v}) = -\langle\boldsymbol{\sigma}\boldsymbol{n},\boldsymbol{v}\rangle_{\partial\Omega} + (\boldsymbol{\sigma},\nabla\boldsymbol{v}),
\]
由于测试函数 $\boldsymbol{v}$ 在 $\Gamma_D$ 上为零,且 $\boldsymbol{\sigma}\boldsymbol{n} = \boldsymbol{g}$ 在 $\Gamma_N$ 上成立,于是:
\[
(\boldsymbol{\sigma},\nabla\boldsymbol{v}) = (\boldsymbol{b},\boldsymbol{v}) + \langle\boldsymbol{g},\boldsymbol{v}\rangle_{\Gamma_N},
\]
利用应力张量 $\boldsymbol{\sigma}$ 的对称性,梯度项化为对称梯度项:
\[
(\boldsymbol{\sigma},\boldsymbol{\varepsilon}(\boldsymbol{v})) = (\boldsymbol{b},\boldsymbol{v}) + \langle\boldsymbol{g},\boldsymbol{v}\rangle_{\Gamma_N},\quad\forall\boldsymbol{v}\in\boldsymbol{V}_0
\]
据此定义双线性型与线性泛函
\[
a(\boldsymbol{u},\boldsymbol{v}) := (\boldsymbol{\sigma},\boldsymbol{\varepsilon}(\boldsymbol{v})),\quad\ell(\boldsymbol{v}) := (\boldsymbol{b},\boldsymbol{v}) + \langle\boldsymbol{g},\boldsymbol{v}\rangle_{\Gamma_N},
\]
得到位移法的弱形式:求 $\boldsymbol{u}\in\boldsymbol{V}$,使得
\begin{equation}
	a(\boldsymbol{u},\boldsymbol{v}) = \ell(\boldsymbol{v}),\quad\forall\boldsymbol{v}\in\boldsymbol{V}_0
\end{equation}

在线性各向同性情形,可得两种常用且等价的表达:
\begin{enumerate}
	\item 应变型:
	\[
	a(\boldsymbol{u},\boldsymbol{v}) := 2\mu(\boldsymbol{\varepsilon}(\boldsymbol{u}),\boldsymbol{\varepsilon}(\boldsymbol{v})) + \lambda(\operatorname{div}\boldsymbol{u},\operatorname{div}\boldsymbol{v})
	\]
	
	\item 梯度型:
	\[
	a(\boldsymbol{u},\boldsymbol{v}) := \mu(\nabla\boldsymbol{u},\nabla\boldsymbol{v}) + (\lambda+\mu)(\operatorname{div}\boldsymbol{u},\operatorname{div}\boldsymbol{v})
	\]
\end{enumerate}

由 Korn 与 Poincaré 不等式可知 $a(\cdot,\cdot)$ 在 $\boldsymbol{V}_0$ 上连续且强椭圆,$\ell(\cdot)$ 连续,故由 Lax–Milgram 定理,弱问题存在唯一解\cite{ciarletFiniteElementMethod2002a}。此外,弱形式 $a(\boldsymbol{u},\boldsymbol{v}) = \ell(\boldsymbol{v})$ 正是最小势能原理的欧拉–拉格朗日方程:线弹性体系的总势能由应变能与外力势能构成,求解弱式等价于寻求使总势能极小的稳定平衡态\cite{zienkiewiczFiniteElementMethod2005}。

\section{线弹性问题的任意次多单元族拉格朗日有限元方法}
\label{sec:lagrange_fem}

\subsection{单纯形单元上拉格朗日有限元空间构造}
\label{subsec:simplex_lagrange}

长度为 $d+1$ 的多重指标是一个非负整数数组:
\[
\boldsymbol{\alpha} = (\alpha_0,\alpha_1,\cdots,\alpha_d),\quad\alpha_i\in\mathbb{N},\quad{i}=0,\cdots,d,
\]
其次数与阶乘分别定义为:
\[|\boldsymbol{\alpha}| = \sum_{i=0}^{d}\alpha_i,\qquad\boldsymbol{\alpha}! = \Pi_{i=0}^d(\alpha_i!)
\]
给定多项式阶次 $k\in\mathbb{N}$,所有满足 $|\boldsymbol{\alpha}| = k$ 的多重指标组成单纯形晶格:
\[
\mathbb{T}_k^d = \{\boldsymbol{\alpha}\in\mathbb{N}^{d+1}:|\boldsymbol{\alpha}| = k\}.
\]
在实际实现中,通常对 $\mathbb{T}_k^d$ 进行字典序进行线性编号,一种常用的编号方式为\cite{chenGeometricDecompositionEfficient2024b}:
\[
R_d(\boldsymbol{\alpha}) = \sum_{i=1}^d
\begin{pmatrix}
	\alpha_i+\alpha_{i+1}+\cdots+\alpha_d+d-i\\
	d+1-i
\end{pmatrix}.
\]

设 $\boldsymbol{x}_0,\boldsymbol{x}_1,\cdots,\boldsymbol{x}_d\in\mathbb{R}^d$,由这 $d+1$ 个点的凸包构成一个 $d$- 维单纯形:
\[
T = \text{Convex}(\boldsymbol{x}_0,\boldsymbol{x}_1,\cdots,\boldsymbol{x}_d) = \left\{\sum_{i=0}^d\lambda_i\boldsymbol{x}_i:0\leq\lambda_i\leq1,\sum_{i=0}^d\lambda_i=1\right\},
\]
其中 $\boldsymbol{\lambda} = (\lambda_0,\cdots,\lambda_d)$ 为重心坐标。基于此,定义 $k$ 阶插值点(晶格节点)集合
\[
\mathcal{X}_T = \left\{\boldsymbol{x}_{\boldsymbol{\alpha}} = \frac{1}{k}\sum_{i=0}^{d}\alpha_i\boldsymbol{x}_i:\boldsymbol{\alpha}\in\mathbb{T}_k^d\right\},\qquad\boldsymbol{\lambda}(\boldsymbol{\alpha}) = \frac{1}{k}(\alpha_0,\alpha_1,\cdots,\alpha_d)
\]
$\mathcal{X}_T$ 中插值点的排序 $\mathbb{T}_k^d$ 中多重指标的排序规则相同,都由字典序 $R_d(\boldsymbol{\alpha})$ 给出。

对于任意单纯形 $T$,$k$ 次拉格朗日有限元对应的标量多项式空间
\[
P_k(T) = \text{span}\{\phi_{\alpha}:\alpha\in\mathbb{T}_k^d\}
\]
其节点型拉格朗日基函数 $\phi_{\boldsymbol\alpha}(\boldsymbol{x})$ 可由重心坐标构造\cite{nicolaidesClassFiniteElements1972,chenGeometricDecompositionEfficient2024b}:
\[
\phi_{\boldsymbol{\alpha}}(\boldsymbol{x}) = \frac{1}{\boldsymbol{\alpha}!}\Pi_{i=0}^d\Pi_{j=0}^{\alpha_i-1}(k\lambda_i(\boldsymbol{x}) - j),\quad\boldsymbol{\alpha}\in\mathbb{T}_k^d,
\]
与插值点 $\boldsymbol{x}_{\boldsymbol{\alpha}}\in\mathcal{X}_T$ 对应的自由度取节点值:
\[
N_{\boldsymbol{\alpha}}(u) = u(\boldsymbol{x}_{\boldsymbol{\alpha}}),
\]
并满足对偶关系:
\[
N_{\boldsymbol{\beta}}(\phi_{\boldsymbol{\alpha}}) = \phi_{\boldsymbol{\alpha}}(\boldsymbol{x}_{\boldsymbol{\beta}}) = \delta_{\boldsymbol{\alpha},\boldsymbol{\beta}}
\]
即一个基函数 $\phi_{\boldsymbol{\alpha}}$ 在其对应的插值点 $\boldsymbol{x}_{\boldsymbol{\alpha}}$ 上值为 $1$,而在所有其它插值点 $\boldsymbol{x}_{\boldsymbol{\beta}}$(其中 $\boldsymbol{\beta}\neq\boldsymbol{\alpha}$)上值为 $0$。

为标准化计算流程并便于高效数值积分,物理单元 $T$ 上的量通过仿射映射 $\boldsymbol{F}_T:\hat{T}\to{T}$ 统一转移至参考单纯形 $\hat{T}$(二维取 $(0,0),(1,0),(0,1)$)处理:
\[
\boldsymbol{x} = \boldsymbol{F}_T(\boldsymbol{\xi}) = \boldsymbol{J}\boldsymbol{\xi} + \boldsymbol{x}_0,\qquad\int_T{f}(\boldsymbol{x})~\mathrm{d}\boldsymbol{x} = \int_{\hat{T}}f(\boldsymbol{F}_T(\boldsymbol{\xi}))|\det\boldsymbol{J}|~\mathrm{d}\boldsymbol{\xi},
\]
其中,雅可比矩阵 $\boldsymbol{J}$ 是一个常数矩阵,$\boldsymbol{x}_0$ 是一个常数向量,对直边单纯形单元,$\boldsymbol{J}$ 与 $\vert\det\boldsymbol{J}\vert$ 为常数;若采用曲边等高阶几何映射,则需在各求积点评估 $\boldsymbol{J}(\boldsymbol{\xi})$ 与 $\vert\det\boldsymbol{J}(\boldsymbol{\xi})\vert$。

上述思想可以推广到矢量拉格朗日单元。设 $\boldsymbol{x}_{\boldsymbol{\alpha}}\in\mathcal{X}_T$,在该点引入一组局部基底 $\{\boldsymbol{e}_i^{\boldsymbol{x}_{\boldsymbol{\alpha}}}\}_{i=1}^{d}\subset\mathbb{R}^d$ 以及其对偶基 $\{\hat{\boldsymbol{e}}_i^{\boldsymbol{x}_{\boldsymbol{\alpha}}}\}_{i=1}^{d}$,满足:
\[
(\hat{\boldsymbol{e}}_i^{\boldsymbol{x}_{\boldsymbol{\alpha}}},\boldsymbol{e}_j^{\boldsymbol{x}_{\boldsymbol{\alpha}}})_{\mathbb{R}^d} = \delta_{ij},
\]
则可定义矢量拉格朗日基函数及其自由度:
\[
\boldsymbol{\phi}_{\boldsymbol{\alpha},i}(\boldsymbol{x}) = \phi_{\boldsymbol{\alpha}}(\boldsymbol{x})\hat{\boldsymbol{e}}_i^{\boldsymbol{x}_{\boldsymbol{\alpha}}},\qquad{N}_{\boldsymbol{\alpha,i}}(\boldsymbol{u}) =  \boldsymbol{u}(\boldsymbol{x}_{\boldsymbol{\alpha}})\cdot\boldsymbol{e}_i^{\boldsymbol{x}_{\boldsymbol{\alpha}}},
\]
并有
\[
N_{\boldsymbol{\beta},i}(\boldsymbol{\phi}_{\boldsymbol{\alpha},j}) = \delta_{\boldsymbol{\alpha},\boldsymbol{\beta}}\delta_{ij}.
\]

基于上述构造,标量拉格朗日元在单个 $d$ 维单纯形单元上的局部自由度数为
\[
\text{ldof}^{(s)} = 
\begin{pmatrix}
	k+d\\
	d
\end{pmatrix}.
\]
矢量拉格朗日元(位移场)在每个单元上的局部自由度数为
\[
\text{ldof}^{(v)} = d~\text{ldof}^{(s)} = 
d\begin{pmatrix}
	k+d\\
	d
\end{pmatrix}.
\]
在二维、三维情形下,可得到更直观的表达式:
\begin{itemize}
	\item 二维三角形单元:
	\[
	\text{ldof}^{(v)} = 2\begin{pmatrix}
		k+2\\
		2
	\end{pmatrix} = (k+1)(k+2).
	\]
	\item 三维四面体单元:
	\[
	\text{ldof}^{(v)} = 3\begin{pmatrix}
		k+3\\
		3
	\end{pmatrix} = \frac{(k+1)(k+2)(k+3)}{2}.
	\]
\end{itemize}

在全局层面,记 $\text{NN}$、$\text{NE}$、$\text{NF}$、$\text{NC}$ 分别为网格中的顶点数、边数、三角形面数和单元数,则标量拉格朗日空间的全局自由度数为:
\begin{itemize}
	\item 二维三角形网格:
	\[
	\text{gdof}^{(s)} = 
	\text{NN} + \text{NE}\times(k-1) + \text{NC}\times
	\begin{pmatrix}
		k-1\\
		2
	\end{pmatrix}.
	\]
	\item 三维四面体网格:
	\[
	\text{gdof}^{(s)} = 
	\text{NN} + \text{NE}\times(k-1) + \text{NF}\times\begin{pmatrix}
		k-1\\
		2
	\end{pmatrix} + \text{NC}\times\begin{pmatrix}
		k-1\\
		3
	\end{pmatrix}.
	\]
\end{itemize}
矢量拉格朗日位移空间由标量空间在每个坐标方向复制一份得到,其全局自由度数满足统一关系
\[
\text{gdof}^{(v)} = d~\text{gdof}^{(s)}.
\]
在二维、三维情形下可写为:
\begin{itemize}
	\item 二维三角形网格:
	\[
	\text{gdof}^{(v)} = 
	\text{NN}\times2 + \text{NE}\times2(k-1) + \text{NC}\times2\begin{pmatrix}
		k-1\\
		2
	\end{pmatrix}.
	\]
	\item 三维四面体网格:
	\[
	\text{gdof}^{(v)} = 
	\text{NN}\times3 + \text{NE}\times3(k-1) + \text{NF}\times3\begin{pmatrix}
		k-1\\
		2
	\end{pmatrix} + \text{NC}\times3\begin{pmatrix}
		k-1\\
		3
	\end{pmatrix}.
	\]
\end{itemize}

\subsection{张量积单元上拉格朗日有限元空间构造}
\label{subsec:tensor_lagrange}

与第~\ref{subsec:simplex_lagrange}~节中单纯形单元利用重心坐标和单纯形晶格 $\mathbb{T}_k^d$ 构造拉格朗日有限元不同,四边形、六面体等非单纯形单元缺乏类似的内蕴重心坐标系。对于此类张量积单元,高阶拉格朗日有限元的构造通常基于规整参考单元上的一维插值与张量积结构,再通过等参映射推广到任意物理单元。为保持记号上的统一,本节在张量积晶格 $\mathbb{S}_k^d$ 的框架下,给出四边形、六面体单元上的拉格朗日有限元空间构造,以及局部与全局自由度的计数公式。

考虑 $d$ 维参考张量积单元  $\hat{Q} = [-1,1]^d$,其坐标记为 $\boldsymbol{\xi} = (\xi_1,\cdots,\xi_d)$。在一维参考区间 $I = [-1,1]$ 上,选取 $k+1$ 个互不相同的插值节点 $\{s_i\}_{i=0}^k$。对应于每个节点 $s_j$,一维 $k$ 次拉格朗日基函数定义为
\[
l_j^k(t) = \Pi_{i=0,i\neq{j}}^k\frac{t-s_i}{s_j-s_i},\quad{j}=0,\cdots,k,
\]
并满足插值性质 $l_j^k(s_i) = \delta_{ij}$。为描述张量积结构,引入长度为 $d$ 的多重索引(晶格指标)
\[
\boldsymbol{i} = (i_1,\cdots,i_d),\qquad{i}_{\ell}\in\{0,1,\cdots,k\},\quad\ell=1,\cdots,d,
\]
所有这样的多重索引构成张量积晶格
\[
\mathbb{S}_k^d = \{\boldsymbol{i}\in\mathbb{N}^d~:~0\leq{i}_\ell\leq{k},\ell=1,\cdots,d\}.
\]

基于一维节点 $\{s_i\}_{i=0}^k$,参考单元上的插值点集合为
\[
\mathcal{X}_{\hat{Q}} = \{\boldsymbol{\xi}_{\boldsymbol{i}} = (s_{i_1},\cdots,s_{i_d}):\boldsymbol{i}\in\mathbb{S}_k^d\},
\]
对每个 $\boldsymbol{i} = (i_1,\cdots,i_d)\in\mathbb{S}_k^d$,定义张量积形式的标量拉格朗日基函数
\[
\hat{\phi}_{\boldsymbol{i}}(\boldsymbol{\xi}) = \Pi_{\ell=1}^dl_{i_\ell}^k(\xi_\ell),\quad\boldsymbol{\xi} = (\xi_1,\cdots,\xi_d)\in\hat{Q},
\]
利用一维基函数的 Kronecker $\delta$ 性质,可以得到高维基函数在插值点上的插值特性:
\[
\hat{\phi}_{\boldsymbol{i}}(\boldsymbol{\xi}_{\boldsymbol{p}}) = \Pi_{\ell=1}^dl_{i_\ell}^k(s_{p_\ell}) = \Pi_{\ell=1}^d\delta_{i_{\ell}p_{\ell}} = \delta_{\boldsymbol{i},\boldsymbol{p}},\qquad\boldsymbol{i},\boldsymbol{p}\in\mathbb{S}_k^d,
\]
因此,$\{\hat{\phi}_{\boldsymbol{i}}\}_{\boldsymbol{i}\in\mathbb{S}_k^d}$ 构成了参考单元上张量积多项式空间的一组节点型拉格朗日基函数。用多重指标 $\boldsymbol{\beta} = (\beta_1,\cdots,\beta_d)\in\mathbb{N}^d$ 表示各坐标方向上的幂次,该空间可以写为
\[
P_k(\hat{Q}) = \text{span}\left\{
\xi_1^{\beta_1}\cdots\xi_d^{\beta_d}:\boldsymbol{\beta} = (\beta_1,\cdots,\beta_d)\in\mathbb{N}^d,0\leq\beta_\ell\leq{k}
\right\}.
\]
在实现中,$\mathbb{S}_k^d$ 通常按字典序进行线性编号,与单纯形晶格 $\mathbb{T}_k^d$ 的处理方式完全一致,从而可在统一的数组结构下管理不同单元族的局部与全局自由度编号。

设 $Q$ 为空间中的任一四边形或六面体物理单元,其几何控制点集合记为
\[
\{\boldsymbol{x}_{\boldsymbol{j}}\}_{\boldsymbol{j}\in\mathbb{S}_k^d} \subset \mathbb{R}^d,
\]
借助参考基函数,可以定义从参考单元到物理单元的等参映射
\[
\boldsymbol{F}_Q:\hat{Q}\to{Q},\qquad\boldsymbol{x} = \boldsymbol{F}_Q(\boldsymbol{\xi}) = \sum_{\boldsymbol{j}\in\mathbb{S}_k^d}\boldsymbol{x}_{\boldsymbol{j}}\hat{\phi}_{\boldsymbol{j}}(\boldsymbol{\xi}),
\]
在等参元设定下,几何映射与场变量(如位移场)的插值采用同一组形函数。定义在物理单元 $Q$ 上的标量拉格朗日基函数由参考基函数通过映射 $\boldsymbol{F}_Q$ 的逆得到:
\[
\phi_{\boldsymbol{i}}(\boldsymbol{x}) = \hat{\phi}_{\boldsymbol{i}}(\boldsymbol{F}_Q^{-1}(\boldsymbol{x})),\qquad\boldsymbol{i}\in\mathbb{S}_k^d,
\]
对物理单元上的积分,通过变量代换可统一转移到参考单元上:
\[
\int_Q{f}(\boldsymbol{x})~\mathrm{d}\boldsymbol{x} = \int_{\hat{Q}}f(\boldsymbol{F}_Q(\boldsymbol{\xi}))|\det\boldsymbol{J}(\boldsymbol{\xi})|~\mathrm{d}\boldsymbol{\xi},
\]
其中 $\boldsymbol{J}$ 是映射 $\boldsymbol{F}_Q$ 的雅可比矩阵。与单纯形单元的仿射映射不同,张量积等参元中的 $\boldsymbol{F}_Q$ 一般是非线性的,故 $\boldsymbol{J}$ 及其行列式通常依赖于 $\boldsymbol{\xi}$,需要在每个求积点上重新计算。

类似于单纯形上的矢量拉格朗日元构造,张量积单元上的矢量空间同样可以通过在每个插值点处引入局部基底与对偶基底来刻画。鉴于本文所考虑的线弹性问题均在全局笛卡尔坐标系下描述,且自由度自然对应于各坐标分量上的位移值,为简化记号,在本节中不再区分不同插值点处的局部坐标基,而是统一采用全局标准正交基 $\{\boldsymbol{e}_m\}_{m=1}^d$。

定义矢量基函数
\[
\boldsymbol{\phi}_{\boldsymbol{i},m}(\boldsymbol{x})
= \phi_{\boldsymbol{i}}(\boldsymbol{x})\,\boldsymbol{e}_m,
\qquad \boldsymbol{i}\in\mathbb{S}_k^d,\ m=1,\dots,d,
\]
相应的自由度取为位移分量在插值点处的点值:
\[
N_{\boldsymbol{i},m}(\boldsymbol{u})
= \boldsymbol{u}(\boldsymbol{x}_{\boldsymbol{i}})\cdot \boldsymbol{e}_m,
\]
并满足对偶关系
\[
N_{\boldsymbol{p},m}\big(\boldsymbol{\phi}_{\boldsymbol{i},n}\big)
= \delta_{\boldsymbol{i},\boldsymbol{p}}\,\delta_{mn},
\qquad \boldsymbol{i},\boldsymbol{p}\in\mathbb{S}_k^d,\ m,n=1,\dots,d.
\]

在上述晶格框架下,参考张量积单元上的标量局部自由度数为
\[
\text{ldof}^{(s)} = (k+1)^d.
\]
矢量拉格朗日元(位移场)的局部自由度数为
\[
\text{ldof}^{(v)} = d~\text{ldof}^{(s)} = d(k+1)^d.
\]
在二维四边形单元与三维六面体单元上分别为
\begin{itemize}
	\item 二维四边形单元:
	\[
	\text{ldof}^{(v)} = 2(k+1)^2.
	\]
	\item 三维六面体单元:
	\[
	\text{ldof}^{(v)} = 3(k+1)^3.
	\]
\end{itemize}

标量空间的全局自由度数为:
\begin{itemize}
	\item 二维四边形网格:
	\[
	\text{gdof}^{(s)} = 
	\text{NN} + \text{NE}\times(k-1) + \text{NC}\times
	(k-1)^2
	\]
	\item 三维六面体网格:
	\[
	\text{gdof}^{(s)} = 
	\text{NN} + \text{NE}\times(k-1) + \text{NF}\times(k-1)^2 + \text{NC}\times(k-1)^3
	\]
\end{itemize}
矢量拉格朗日位移空间由标量自由度在每个坐标方向复制得到,其全局自由度数满足统一关系
\[
\text{gdof}^{(v)} = d~\text{gdof}^{(s)}
\]
在二维、三维情形下可写为:
\begin{itemize}
	\item 二维四边形网格:
	\[
	\text{gdof}^{(v)} = 
	\text{NN}\times2 + \text{NE}\times2(k-1) + \text{NC}\times
	2(k-1)^2
	\]
	\item 三维六面体网格:
	\[
	\text{gdof}^{(v)} = 
	\text{NN}\times3 + \text{NE}\times3(k-1) + \text{NF}\times3(k-1)^2 + \text{NC}\times3(k-1)^3
	\]
\end{itemize}

\subsection{线弹性的有限元离散}
\label{subsec:elastic_discrete}

设 $\mathcal{K}_h$ 为区域 $\Omega$ 的一族形状正则的网格剖分,由有限个闭合单元 $K$ 组成,记 $h$ 为网格的特征尺寸参数。基于前文 \ref{subsec:simplex_lagrange} 节和 \ref{subsec:tensor_lagrange} 节中关于单纯形与张量积单元上多项式框架及其拉格朗日基函数的构造,记定义在单元 $K$ 上的局部 $k$ 次向量值多项式空间为 ${P}_k(K;\mathbb{R}^d)$,据此,定义 $\Omega$ 上的连续拉格朗日有限元空间 如下
\[
\boldsymbol{V}_h = \{\boldsymbol{u}_h\in {C}^0(\Omega;\mathbb{R}^d):\boldsymbol{u}_h|_{K}\in{P}_k(K;\mathbb{R}^d),~\forall{K}\in\mathcal{K}_h\},
\]
如此定义的 $\boldsymbol{V}_h$ 构成了原连续问题解空间的有限维子空间,该空间中的任意函数 $\boldsymbol{u}_h$ 可以由其在插值点上的矢量自由度唯一确定,并表示为矢量基函数的线性组合:
\[
\boldsymbol{u}_h(\boldsymbol{x}) = \sum_{I=1}^{N_{\text{dof}}}U_I\boldsymbol{\phi}_I(\boldsymbol{x}),
\]
其中,$N_{\text{dof}}$ 为系统中自由度的总数,$\{\boldsymbol{\phi}_I(\boldsymbol{x})\}_{I=1}^{N_{\text{dof}}}$ 为第 $I$ 个全局矢量基函数,$U_I$ 为对应的自由度值。相应的测试函数空间定义为
\[
\boldsymbol{V}_{h,0} = \{\boldsymbol{v}_h\in\boldsymbol{V}_h :~\boldsymbol{v}_h = \boldsymbol{0}~\text{on}~{\Gamma_D}\}.
\]

有限元近似解 $\boldsymbol{u}_h\in\boldsymbol{V}_h$ 满足离散变分问题
\[
a(\boldsymbol{u}_h,\boldsymbol{v}_h) = l(\boldsymbol{v}_h),\quad\forall\boldsymbol{v}_h\in\boldsymbol{V}_{h,0}
\]
其中
\[
a(\boldsymbol{u}_h,\boldsymbol{v}_h) := (\boldsymbol{\sigma}(\boldsymbol{u}_h),\boldsymbol{\varepsilon}(\boldsymbol{v}_h)),\quad\ell(\boldsymbol{v}_h) := (\boldsymbol{b},\boldsymbol{v}_h) + \langle\boldsymbol{g},\boldsymbol{v}_h\rangle_{\Gamma_N}.
\] 

在小变形假设下,应变张量 $\boldsymbol{\varepsilon}(\boldsymbol{u}_h)$ 由位移梯度给出。将对称应变和对称应力按 Voigt 记号展平,在三维情形中记为
\[
\hat{\boldsymbol{\varepsilon}} = 
\begin{pmatrix}
	\varepsilon_{xx},&\varepsilon_{yy},&\varepsilon_{zz},&2\varepsilon_{xy},&2\varepsilon_{xz},&2\varepsilon_{yz}
\end{pmatrix}^{\top},\quad
\hat{\boldsymbol{\sigma}} = 
\begin{pmatrix}
	\sigma_{xx},&\sigma_{yy},&\sigma_{zz},&2\sigma_{xy},&2\sigma_{xz},&2\sigma_{yz}
\end{pmatrix}^{\top}.
\]
记全局位移和测试函数的自由度列向量分别为
\[
\boldsymbol{U} = \begin{pmatrix}
	U_1, &\cdots, &U_{N_{\text{dof}}}
\end{pmatrix}^{\top},\quad
\boldsymbol{V} = \begin{pmatrix}
	V_1, &\cdots, &V_{N_{\text{dof}}}
\end{pmatrix}^{\top}.
\]
将矢量基函数按分量展开,设每个分量对应的标量拉格朗日形函数为 $\{\phi_j\}_{j=1}^{N_n}$,并记
\[
\boldsymbol{\Phi}(\boldsymbol{x}) = \begin{pmatrix}
	\phi_1(\boldsymbol{x}), &\cdots, &\phi_{N_n}(\boldsymbol{x})
\end{pmatrix},
\]
则矢量有限元函数可写为
\[
\boldsymbol{u}_h(\boldsymbol{x}) = 
\begin{pmatrix}
	\boldsymbol{\Phi}(\boldsymbol{x})\\
	&\ddots\\
	& &\boldsymbol{\Phi}(\boldsymbol{x})
\end{pmatrix}\boldsymbol{U},
\]
其中块对角矩阵由 $\boldsymbol{\Phi}(\boldsymbol{x})$ 在 $d$ 个分量方向上复制而成,由此得到应变–位移关系
\[
\hat{\boldsymbol{\varepsilon}}(\boldsymbol{u}_h(\boldsymbol{x})) = \boldsymbol{B}(\boldsymbol{x})\boldsymbol{U},
\]
其中应变-位移矩阵可写为
\[
\boldsymbol{B} = 
\begin{bmatrix}
	\boldsymbol{\Phi}_{,x} &\boldsymbol{0} &\boldsymbol{0}\\
	\boldsymbol{0} &\boldsymbol{\Phi}_{,y} &\boldsymbol{0}\\
	\boldsymbol{0} &\boldsymbol{0} &\boldsymbol{\Phi}_{,z}\\
	\boldsymbol{\Phi}_{,y} &\boldsymbol{\Phi}_{,x} &\boldsymbol{0}\\
	\boldsymbol{\Phi}_{,z} &\boldsymbol{0} &\boldsymbol{\Phi}_{,x}\\
	\boldsymbol{0} &\boldsymbol{\Phi}_{,z} &\boldsymbol{\Phi}_{,y}\\
\end{bmatrix},
\]
其中 $\boldsymbol{\Phi}_{,x},\boldsymbol{\Phi}_{,y},\boldsymbol{\Phi}_{,z}$ 分别表示标量形函数对相应坐标的导数组成的行向量。

线弹性本构关系  $\boldsymbol{\sigma} = \mathbb{C}:\boldsymbol{\varepsilon}$ 在 Voigt 表示下写为
\[
\hat{\boldsymbol{\sigma}} = \boldsymbol{D}\hat{\boldsymbol{\varepsilon}},
\]
三维各向同性情形下本构矩阵为
\[
\boldsymbol{D} =
\begin{bmatrix}
	2\mu+\lambda &\lambda &\lambda &0 &0 &0\\
	\lambda &2\mu+\lambda &\lambda &0 &0 &0\\
	\lambda &\lambda &2\mu+\lambda &0 &0 &0\\
	0 &0 &0 &\mu &0 &0\\
	0 &0 &0 & 0 &\mu &0\\
	0 &0 &0 &0 &0 &\mu
\end{bmatrix},
\]
二维平面问题下,记
\[
\hat{\boldsymbol{\varepsilon}} = 
\begin{pmatrix}
	\varepsilon_{xx},&\varepsilon_{yy},&2\varepsilon_{xy}
\end{pmatrix}^{\top},\quad\hat{\boldsymbol{\sigma}} = 
\begin{pmatrix}
	\sigma_{xx},&\sigma_{yy},&2\sigma_{xy} 
\end{pmatrix}^{\top},
\]
则在平面应变与平面应力假设下,本构矩阵分别为
\[
\boldsymbol{D}_{\text{pe}} = 
\begin{bmatrix}
	2\mu + \lambda & \lambda & 0 \\
	\lambda & 2\mu + \lambda & 0 \\
	0 & 0 & \mu
\end{bmatrix},\quad
\boldsymbol{D}_{\text{ps}} = \frac{E}{1-\nu^2}
\begin{bmatrix}
	1 &\nu &0\\
	\nu &1 &0\\
	0 &0 &(1-\nu)/2
\end{bmatrix}.
\]

在上述记号下,双线性型可写为
\[
\begin{aligned}
	(\boldsymbol{\sigma}(\boldsymbol{u}_h),\boldsymbol{\varepsilon}(\boldsymbol{v}_h)) 
	&= \int_{\Omega}(\boldsymbol{\sigma}(\boldsymbol{u}_h))^{\top}\boldsymbol{\varepsilon}(\boldsymbol{v}_h)~\mathrm{d}\boldsymbol{x}\\
	&= \int_{\Omega}(\boldsymbol{D}\boldsymbol{\varepsilon}(\boldsymbol{u}_h))^{\top}\boldsymbol{\varepsilon}(\boldsymbol{v}_h)~\mathrm{d}\boldsymbol{x}\\
	&= \int_{\Omega}(\boldsymbol{D}\boldsymbol{B}(\boldsymbol{x})\boldsymbol{U})^{\top}\boldsymbol{B}(\boldsymbol{x})\boldsymbol{V}~\mathrm{d}\boldsymbol{x}\\
	&= \boldsymbol{U}^{\top}\left(\int_{\Omega}\boldsymbol{B}(\boldsymbol{x})^{\top}\boldsymbol{D}\boldsymbol{B}(\boldsymbol{x})~\mathrm{d}\boldsymbol{x}\right)\boldsymbol{V},
\end{aligned}
\]
由此可定义全局刚度矩阵
\[
\boldsymbol{K} = \int_{\Omega}\boldsymbol{B}(\boldsymbol{x})^{\top}\boldsymbol{D}\boldsymbol{B}(\boldsymbol{x})\,\mathrm{d}\boldsymbol{x},
\]
同理,由右端线性泛函得到载荷向量
\[
\boldsymbol{F} = \int_\Omega \boldsymbol{\Phi}^{\top}\boldsymbol{f} ~\mathrm{d}\boldsymbol{x}
+\int_{\Gamma_N} \boldsymbol{\Phi}^{\top}\boldsymbol{g}_N ~ \mathrm{d}\boldsymbol{s},
\]
于是线弹性问题的有限元离散代数系统可写为
$$\boldsymbol{K}\boldsymbol{U} = \boldsymbol{F}.$$

\subsection{算例验证}
\label{subsec:fem_verification}

为验证本文实现的任意次多单元族拉格朗日有限元方法的正确性和有效性,并评估其数值精度与收敛阶次,本节构造一个具有解析解的二维线弹性问题。为聚焦于数值方法的收敛性分析,本算例采用无量纲化的物理参数。

考虑单位正方形求解域 $\Omega = [0,1]^2$,假设材料为均匀各向同性线弹性体,处于平面应变状态,材料拉梅第一常数 $\lambda=1$,剪切模量 $\mu = 0.5$。构造精确位移解如下:
\[
\boldsymbol{u}(x,y) = 
\begin{pmatrix} 
	\sin(\frac{\pi}{2}x)\sin(\pi{y})\\ -2\sin(\pi{x})(\sin(\frac{\pi}{2}{y})-y) 
\end{pmatrix}
\]
根据线弹性控制方程可导得对应的体力密度向量:
\[
\boldsymbol{b}(x,y) = 
\begin{pmatrix}
	\pi^2[\sin(\frac{\pi}{2}x)\sin(\pi{y}) +  \frac{3}{2}\cos(\pi{y})\cos(\frac{\pi}{2}y)] - 2\pi(\lambda+\mu)\cos(\pi{x})\\
	-\frac{3}{4}\pi^2\cos(\frac{\pi}{2}x)\cos({\pi}{y}) - 2\pi^2\sin(\pi{x})\sin(\frac{\pi}{2}y)+2\mu\pi^2y\sin(\pi{x}) 
\end{pmatrix}
\]
在上下边界($y=0, 1$)施加与精确解一致的狄利克雷位移约束,左右边界($x=0,1$)施加诺伊曼面力牵引 $\boldsymbol{g}$,具体表达式为:
\[
\boldsymbol{g}(0,y) = 
\begin{pmatrix}
	-\pi\sin(\pi{y})\\
	\pi\sin(\frac{\pi{y}}{2})-2\mu\pi{y}
\end{pmatrix},\quad
\boldsymbol{g}(1,y) = 
\begin{pmatrix}
	0\\
	\frac{\pi}{2}\cos(\pi{y})+\pi\sin(\frac{\pi{y}}{2})-2\mu\pi{y}
\end{pmatrix}
\]

为了系统评估并对比不同单元族在统一框架下的数值精度与收敛阶,本节在保持多项式阶次 $k$ 固定的前提下,分别针对单纯形(三角形)单元与张量积(四边形)单元进行了 $h$-型网格细化测试。通过计算一系列加密网格上的数值解,提取了位移场的 $L^2$ 误差范数与 $H^1$ 半范数。图 \ref{fig:convergence_comparison} 在双对数坐标系下展示了这两种单元族的误差收敛曲线对比,其中图 \ref{fig:convergence_k1} 与 \ref{fig:convergence_k4} 分别对应于线性单元 ($k=1$) 与高阶单元 ($k=4$) 的情形。

如图所示,无论是单纯形(三角形)单元还是张量积(四边形)单元,在线性单元 ($k=1$) 情形下,位移场 $L^2$ 范数的收敛阶均稳定在 $2.0$ 左右,而在高阶单元 ($k=4$) 情形下,其收敛阶更是都达到了 $5.0$ 的理论最优值。上述结果与有限元先验误差估计理论中 $\mathcal{O}(h^{k+1})$ 与 $\mathcal{O}(h^k)$ 的最优收敛阶次高度吻合,这一结果充分验证了本文实现的任意阶有限元求解器在数值上的正确性与高阶精度。

\begin{figure}[!htbp]
	\centering
	% 强制标题居中
	\captionsetup{justification=centering} 
	
	% 第一个子图:线性单元 (k=1)
	\subfigure[线性单元 ($k=1$)]{
		\label{fig:convergence_k1}
		\includegraphics[width=0.47\textwidth]{fig3-1a.pdf}
	}
	\hfill % 把两张图撑到版心的最左和最右,利用所有空间
	% 第二个子图:高阶单元 (k=4)
	\subfigure[高阶单元 ($k=4$)]{
		\label{fig:convergence_k4}
		\includegraphics[width=0.47\textwidth]{fig3-1b.pdf} 
	}
	\caption{不同阶次下两种单元族的误差收敛曲线对比图。图中带标记的曲线表示数值误差,无标记的直线段表理论示参考斜率。}
	\label{fig:convergence_comparison}
\end{figure}

\clearpage % 避免图片跨章节

\section{不同设计变量表征下的优化问题描述}
\label{sec:diff_design_var_rep}

本节在第 \ref{chap:top_models_numerics} 章中连续模型与变分形式的基础上,给出体积分数约束下柔顺度最小化与柔顺机构设计问题的有限元离散表达。为便于比较,统一采用同一套有限元平衡方程与 SIMP 材料插值,并在此框架下考察不同密度表征对离散优化模型及灵敏度形式的影响。
对给定的设计变量向量 $\boldsymbol{\rho}$,记线弹性问题的有限元平衡方程为
\[
\boldsymbol{K}(\boldsymbol{\rho})\,\boldsymbol{U}(\boldsymbol{\rho})=\boldsymbol{F}, \qquad  c(\boldsymbol{\rho})=\boldsymbol{F}^{T}\boldsymbol{U} =\boldsymbol{U}^{T}\boldsymbol{K}(\boldsymbol{\rho})\,\boldsymbol{U}
\]
其中 $\boldsymbol{K}(\boldsymbol{\rho})$ 为全局刚度矩阵,$\boldsymbol{U}(\boldsymbol{\rho})$ 为位移向量,$c(\boldsymbol{\rho})$ 为柔顺度。为保证一致性,统一采用 SIMP 材料插值,其杨氏模量插值为
\[
E(\rho(\boldsymbol{x})) =  \rho(\boldsymbol{x})^pE_0
\]
泊松比 $\nu$ 取定,据此构造本构矩阵 $\boldsymbol{D}(E(\rho(\boldsymbol{x})),\nu)$。

\subsection{单元密度表征}
\label{subsec:cell_density_rep}

首先考虑最为经典的单元密度表征。设计变量取每个单元的密度 $\rho_e\in[0,1]$($e=1,\cdots,N_c$),记第 $e$ 个单元体积为 $v_e=|\Omega_e|$,则体积分数可写为
\[
V(\boldsymbol{\rho}) = \frac{1}{|\Omega|}\sum_{e=1}^{N_c}\int_{\Omega_e}\rho_e~\mathrm{d}\boldsymbol{x} = \frac{1}{|\Omega|}\sum_{e=1}^{N_c}\rho_ev_e,
\]
据此,柔顺度最小化问题可表述为
\[
\begin{aligned} \min_{\boldsymbol{\rho}}&:c(\boldsymbol{\rho}) = \boldsymbol{F}^T\boldsymbol{U} = \boldsymbol{U}^T\boldsymbol{K}\boldsymbol{U} \\ 
	\mathrm{subject~to}
	&:\begin{cases}
		\boldsymbol{K}(\boldsymbol{\rho})\boldsymbol{U}(\boldsymbol{\rho}) = \boldsymbol{F}\\
		g_V(\boldsymbol{\rho}) = V(\boldsymbol{\rho}) - V_f\leq0\\ 
		\rho_e\in[\rho_{\min},1] 
	\end{cases}
\end{aligned}
\]
材料行为采用 SIMP 插值 $E(\rho_e) = \rho_e^pE_0$,据此,单元刚度矩阵:
\[
\boldsymbol{K}_e(\rho_e) = \int_{\Omega_e}\boldsymbol{B}^T\boldsymbol{D}(\rho_e)\boldsymbol{B}~\mathrm{d}\boldsymbol{x} = \int_{\Omega_e}\boldsymbol{B}^T(\rho_e^p\boldsymbol{D}_0)\boldsymbol{B}~\mathrm{d}\boldsymbol{x} = \rho_e^p\boldsymbol{K}_e^{(0)},
\]
在上述记号下,灵敏度分析可分步给出。首先,体积分数约束函数的对设计变量 $\rho_e$ 的导数为
\[
\frac{\partial{g}_V(\boldsymbol{\rho})}{\partial\rho_e} = \frac{\partial{V}(\boldsymbol{\rho})}{\partial\rho_e} = \frac{v_e}{|\Omega|},
\]
其次,SIMP  模型中杨氏模量的导数为
\[
\frac{\partial{E}(\rho_e)}{\partial\rho_e} = p\rho_e^{p-1}E_0,
\]
从而由链式法则得到单元刚度矩阵的导数
\[
\frac{\partial\boldsymbol{K}_e}{\partial\rho_e} = \int_{\Omega_e}\boldsymbol{B}^T\frac{\partial\boldsymbol{D}(\rho_e)}{\partial\rho_e}\boldsymbol{B}~\mathrm{d}\boldsymbol{x} = \frac{\partial{E}(\rho_e)}{\partial\rho_e}\frac{1}{E_0}\int_{\Omega_e}\boldsymbol{B}^T\boldsymbol{D}_0\boldsymbol{B}~\mathrm{d}\boldsymbol{x} = p\rho_e^{p-1}\boldsymbol{K}_e^{(0)},
\]
最后,柔顺度的导数为
\[
\frac{\partial{c}}{\partial\rho_e} = -\boldsymbol{U}_e^T\frac{\partial\boldsymbol{K}_e}{\partial\rho_e}\boldsymbol{U}_e = -p\rho_e^{p-1}\boldsymbol{U}_e^T\boldsymbol{K}_e^{(0)}\boldsymbol{U}_e.
\]

\subsection{节点密度表征}
\label{subsec:node_density_rep}

在单元密度模型的基础上,进一步考虑节点密度表征。设计变量取每个节点的密度 $\rho_i\in[0,1]$($i = 1,\dots,N_n$ ),采用一阶拉格朗日形函数 $\{\phi_i^{(e)}\}_{i\in\mathcal{N}(e)}$ 对密度场近似:
\[
\rho_h(\boldsymbol{x}) = \sum_{i\in\mathcal{N}(e)}\phi_i^{(e)}(\boldsymbol{x})\rho_i,\quad\phi_i^{(e)}(\boldsymbol{x})\geq0,\quad\sum_{i\in\mathcal{N}(e)}\phi_{i}^{(e)}(\boldsymbol{x}) = 1
\]
因此若 $0\leq\rho_i\leq1$,则
\[
0\leq\rho_h(\boldsymbol{x}) \leq 1,\quad\frac{\partial\rho_h(\boldsymbol{x})}{\partial\rho_i} = \phi_i^{(e)}(\boldsymbol{x}) \geq 0
\]
体积分数可写为
\[
V(\boldsymbol{\rho}) =  \frac{1}{|\Omega|}\int_{\Omega}\rho_h(\boldsymbol{x})~\mathrm{d}\boldsymbol{x} = \frac{1}{|\Omega|}\sum_{e=1}^{N_c}\int_{\Omega_e}\rho_h(\boldsymbol{x})~\mathrm{d}\boldsymbol{x} \approx \frac{1}{|\Omega|}\sum_{e=1}^{N_c}\sum_{g=1}^{N_q}\rho(\boldsymbol{x}_{e,g})\operatorname{det}\boldsymbol{J}_{e}(\boldsymbol{x}_{e,g})\omega_g
\]
由此,柔顺度最小化问题为
\[
\begin{aligned} 
	\min_{\boldsymbol{\rho}}&:c(\boldsymbol{\rho}) = \boldsymbol{F}^T\boldsymbol{U} = \boldsymbol{U}^T\boldsymbol{K}\boldsymbol{U} \\ \mathrm{subject~to}
	&:\begin{cases}
		\boldsymbol{K}(\boldsymbol{\rho})\boldsymbol{U}(\boldsymbol{\rho}) = \boldsymbol{F}\\
		g_V(\boldsymbol{\rho}) = V(\boldsymbol{\rho}) - V_f\leq0\\ 
		\rho(\boldsymbol{x})\in[\rho_{\min},1]
	\end{cases}
\end{aligned}
\]

在经典 SIMP 模型 $E(\rho_h(\boldsymbol{x})) = \rho_h(\boldsymbol{x})^pE_0$ 插值下,单元刚度矩阵:
\[
\boldsymbol{K}_e(\rho_h(\boldsymbol{x})) = \int_{\Omega_e}\boldsymbol{B}^T\boldsymbol{D}(\rho_h(\boldsymbol{x}))\boldsymbol{B}~\mathrm{d}\boldsymbol{x} \approx \sum_{g=1}^{N_q}\boldsymbol{B}_g^T\boldsymbol{D}(\rho(\boldsymbol{x}_g))\boldsymbol{B}_g\operatorname{det}\boldsymbol{J}_e(\boldsymbol{x}_{e,g})\omega_g
\]
体积分数约束函数的导数由链式法则并按单元分片求和:
\[
\frac{\partial{g}_V(\boldsymbol{\rho})}{\partial\rho_i} = \frac{\partial{V}(\boldsymbol{\rho})}{\partial\rho_i} = \frac{1}{|\Omega|}\sum_{e\in{I}_i}\int_{\Omega_e}\phi_i^{(e)}(\boldsymbol{x})~\mathrm{d}\boldsymbol{x} \approx \frac{1}{|\Omega|}\sum_{e\in{I}_i}\sum_{g=1}^{N_q}\phi_i^{(e)}(\boldsymbol{x}_{e,g})\operatorname{det}\boldsymbol{J}_e(\boldsymbol{x}_{e,g})\omega_g
\]
先对杨氏模量求导
\[
\frac{\partial{E}(\rho_h(\boldsymbol{x}))}{\partial\rho_i} = p\rho_h(\boldsymbol{x})^{p-1}E_0\phi_i^{(e)}(\boldsymbol{x})
\]
从而单元刚度矩阵的导数为
\[
\begin{aligned}
	\frac{\partial\boldsymbol{K}_e}{\partial\rho_i} &= \int_{\Omega_e}\boldsymbol{B}^T\left(\frac{\partial{E}(\rho_h(\boldsymbol{x}))}{\partial\rho_i}\frac{1}{E_0}\boldsymbol{D}_0\right)\boldsymbol{B}~\mathrm{d}\boldsymbol{x} = \int_{\Omega_e}\boldsymbol{B}^T(p\rho_h(\boldsymbol{x})^{p-1}\phi_i^{(e)}(\boldsymbol{x})\boldsymbol{D}_0)\boldsymbol{B}~\mathrm{d}\boldsymbol{x}\\
	&\approx \sum_{g=1}^{N_q}p\rho_h(\boldsymbol{x}_{e,g})^{p-1}\phi_i^{(e)}(\boldsymbol{x}_{e,g})\boldsymbol{B}_g^T\boldsymbol{D}_0\boldsymbol{B}_g\operatorname{det}\boldsymbol{J}_e(\boldsymbol{x}_{e,g})\omega_g
\end{aligned}
\]
最后,柔顺度的导数为
\[
\begin{aligned}
	\frac{\partial{c}}{\partial\rho_i} 
	&= -\sum_{e\in{I}_i}\boldsymbol{U}_e^T\frac{\partial\boldsymbol{K}_e}{\partial\rho_i}\boldsymbol{U}_e = -\sum_{e\in{I}_i}\boldsymbol{U}_e^T\left(\int_{\Omega_e}\boldsymbol{B}^T\frac{\partial\boldsymbol{D}(\rho_h(\boldsymbol{x}))}{\partial\rho_i}\boldsymbol{B}\right)\boldsymbol{U}_e\\ &\approx -\sum_{e\in{I}_i}\sum_{g=1}^{N_q}p\rho_h(\boldsymbol{x}_{e,g})^{p-1}\phi_i^{(e)}(\boldsymbol{x}_{e,g})\boldsymbol{B}_g^T\boldsymbol{D}_0\boldsymbol{B}_g\operatorname{det}\boldsymbol{J}_e(\boldsymbol{x}_{e,g})\omega_g
\end{aligned}
\]

值得注意的是,与单元密度表征不同,节点密度表征下的被积函数包含了密度插值函数的高次幂项 $(\rho_h(\boldsymbol{x}))^p$,这意味着为了精确捕捉单元内部的刚度变化并避免数值积分误差,节点密度模型通常需要比同阶位移元更高的数值积分阶次。

\subsection{柔顺机构设计问题}
\label{subsec:compl_mech_design}

在上述两种密度表征下柔顺度最小化问题的基础上,本小节将优化目标从结构柔顺度推广到柔顺机构的输出位移。为突出基本思想,以下给出单元密度表征下的离散模型;节点密度表征的形式可完全类比第 \ref{subsec:node_density_rep} 节推导得到。
设输出端在指定方向上的离散位移定义为
\[
u_{\text{out}} = \boldsymbol{L}^{\top}\boldsymbol{U},
\]
其中 $\boldsymbol{L}$ 为输出选择向量,仅在输出自由度对应分量处取非零值,其余分量为零,用于从全局位移向量 $\boldsymbol{U}$ 提取输出端在给定方向上的位移分量。则柔顺机构设计问题的离散优化模型可写为
\[
\begin{aligned}
	\max_{\boldsymbol{\rho}}&:u_{\text{out}} = \boldsymbol{L}^{\top}\boldsymbol{U}\\
	\mathrm{subject~to}
	&:\begin{cases}
		\boldsymbol{K}_{\text{system}}(\boldsymbol{\rho})\boldsymbol{U} = \boldsymbol{F}_{\text{in}}\\
		g_V(\boldsymbol{\rho}) = V(\boldsymbol{\rho}) - V_f\leq0\\ 
		\rho_e\in[\rho_{\min},1] 
	\end{cases}
\end{aligned}
\]
其中,$\boldsymbol{F}_{\text{in}}$ 为仅在输入自由度非零的物理载荷向量,总刚度矩阵 $\boldsymbol{K}_{\text{system}}$ 由结构刚度和输入/输出弹簧刚度组合而成,结构部分依然采用 SIMP 插值模型,
\[
\boldsymbol{K}_{\text{system}}(\boldsymbol{\rho}) = \sum_{e=1}^{N_c}\rho_e^p\boldsymbol{K}_e^{(0)} + \boldsymbol{K}_{\text{spring}},
\]
其中 $\boldsymbol{K}_{\text{spring}}$ 为由输入端与输出端弹簧刚度构造的刚度矩阵,仅在输入和输出自由度对应的对角线元素上分别叠加弹簧刚度 $k_{\text{in}}$ 和 $k_{\text{out}}$,且与设计变量 $\boldsymbol{\rho}$ 无关。

离散伴随向量 $\boldsymbol{\lambda}$ 满足
\[
\boldsymbol{K}_{\text{system}}\boldsymbol{\lambda} = -\boldsymbol{L},
\]
且在状态方程与伴随方程均满足时,输出位移关于单元密度 $\rho_e$ 的导数为
\[
\frac{\partial u_{\text{out}}}{\partial \rho_e}  = \boldsymbol{\lambda}^{\top} \frac{\partial \boldsymbol{K}_{\text{system}}}{\partial \rho_e} \boldsymbol{U},
\]
由于弹簧刚度矩阵 $\boldsymbol{K}_{\text{spring}}$ 与设计变量无关,有
\[
\frac{\partial\boldsymbol{K}_{\text{system}}}{\partial \rho_e} = \frac{\partial\boldsymbol{K}}{\partial \rho_e},
\]
而第 \ref{subsec:cell_density_rep} 节中单元刚度矩阵的导数为
\[
\frac{\partial\boldsymbol{K}}{\partial \rho_e} = p\rho_e^{p-1}\boldsymbol{K}_e^{(0)},
\]
于是柔顺机构输出位移的灵敏度为:
\[
\frac{\partial u_{\text{out}}}{\partial \rho_e} = \boldsymbol{\lambda}^{\top} \frac{\partial \boldsymbol{K}_{\text{system}}}{\partial \rho_e} \boldsymbol{U} = p\rho_e^{p-1} \boldsymbol{\lambda}_e^{\top} \boldsymbol{K}_e^{(0)} \boldsymbol{U}_e.
\]

\section{不同设计变量表征下的正则化策略}
\label{sec:diff_design_var_regularization}

本节针对拓扑优化中常见的棋盘格现象及网格依赖性等数值不稳定性问题,在不同设计变量表征下,讨论基于过滤技术的正则化策略。

\subsection{单元密度表征}
\label{subsec:cell_density_regularization}

在单元密度表征中,设计变量与有限元网格单元一一对应。定义卷积权重因子:
\[
H_{es} = \max\{0,r_{\min} - \text{dist}(e,s)\},
\]
式中,$r_{\min}$ 为预设的过滤半径,$\text{dist}(e,s)$ 为单元 $e$ 和 $s$ 的中心距离。$H_{es}$ 本质上是 2.7 节中连续卷积核函数 $w(r)$ 在离散网格上的数值实现。具体而言,$H_{es}$ 等价于核函数在单元 $e$ 和 $s$ 几何中心处的取值,即
\[
H_{es} \equiv w(\|\boldsymbol{x}_e - \boldsymbol{x}_s\|),
\]
该表达式通过取最大值自动将过滤邻域外的单元权重置零,确保了权重的非负性,是该算法在数值中的标准实现方式。

灵敏度过滤:该方法主要修正目标函数的灵敏度场,而不直接改变设计变量本身。其修正公式为:
\[
\widetilde{\frac{\partial{c}}{\partial\rho_e}} = \frac{v_e}{\max\{\gamma,\rho_e\}\sum_{s=1}^{N_c}H_{es}v_s}\sum_{s=1}^{N_c}H_{es}\rho_s\frac{1}{v_s}\frac{\partial{c}}{\partial\rho_s},
\]
其中,$v_e$ 和 $v_s$ 分别为单元 $e$ 和 $s$ 的体积,$\gamma$ 为防止奇异的小正数。

密度过滤:通过引入物理密度场 $\tilde{\boldsymbol{\rho}}$ 与设计变量场 $\boldsymbol{\rho}$ 之间的平滑映射来控制最小特征尺寸。物理密度 $\tilde{\rho}_e$ 定义为邻域内设计变量的体积加权平均:
\[
\tilde{\rho}_e = \frac{1}{\sum_{s=1}^{N_c} H_{es}v_s}\sum_{s=1}^{N_c} H_{es}v_s \rho_s,
\]
在此表征下,有限元刚度矩阵的组装及柔顺度的计算均基于物理密度 $\tilde{\boldsymbol{\rho}}$ 进行。因此,对于任意依赖于物理密度的函数 $f(\tilde{\boldsymbol{\rho}})$(如目标函数 $c$ 或体积分数约束 $g_V$),其关于原始设计变量 $\rho_e$ 的灵敏度必须利用链式法则求解:
\[
\frac{\partial{f}}{\partial\rho_e} = \sum_{j=1}^{N_c} \frac{\partial{f}}{\partial \tilde{\rho}_j} \frac{\partial \tilde{\rho}_j}{\partial \rho_e} = v_e\sum_{j=1}^{N_c}H_{je}\left( \frac{1}{\sum_{s=1}^{N_c} H_{js} v_s}\frac{\partial{f}}{\partial \tilde{\rho}_j}\right),
\]
式中 $\frac{\partial f}{\partial \tilde{\rho}_j}$ 为函数关于物理密度的导数。

\subsection{节点密度表征}
\label{subsec:node_density_regularization}

在节点密度表征中,设计变量定义于有限元网格节点。定义节点 $i$ 与节点 $k$ 之间的卷积权重因子:
\[
H_{ik} = \max\{0, r_{\min} - \operatorname{dist}(i,k)\},
\]
式中,$\operatorname{dist}(i,k)$ 为节点 $i$ 和 $k$ 之间的欧几里得距离。

灵敏度过滤:节点表征下的灵敏度过滤修正公式为:
\[
\widetilde{\frac{\partial{c}}{\partial\rho_i}} = \frac{v_i}{\max\{\gamma,\rho_i\}\sum_{k=1}^{N_n}H_{ik}v_k}\sum_{k=1}^{N_n}H_{ik}\rho_k\frac{1}{v_k}\frac{\partial{c}}{\partial\rho_k}.
\]

密度过滤:引入物理节点密度场 $\tilde{\boldsymbol{\rho}}$。物理密度 $\tilde{\rho}_i$ 定义:
\[
\tilde{\rho}_i = \frac{1}{\sum_{k=1}^{N_n} H_{ik}v_k}\sum_{k=1}^{N_n} H_{ik}v_k \rho_k,
\]
其中 $v_k$ 为节点 $k$ 的控制体积。在数值实现中,通常采用集中质量策略计算,即将每个单元的体积均分给其顶点并累加。任意函数 $f(\tilde{\boldsymbol{\rho}})$ 关于原始节点设计变量 $\rho_k$ 的灵敏度为:
\[
\frac{\partial f}{\partial \rho_k} = \sum_{j=1}^{N_n} \frac{\partial{f}}{\partial \tilde{\rho}_j} \frac{\partial \tilde{\rho}_j}{\partial \rho_k} = v_k \sum_{j=1}^{N_n} H_{jk} \left( \frac{1}{\sum_{l=1}^{N_n} H_{jl} v_l} \frac{\partial f}{\partial \tilde{\rho}_j} \right),
\]
值得注意的是,求和项 $\sum_{j=1}^{N_n}$ 表示节点 $k$ 的设计变量变化通过卷积关系影响其邻域内所有节点 $j$ 的物理密度,进而对全局目标函数产生贡献。

\section{数值算例}
\label{sec:ch3_numerical_examples}

\subsection{算例设置与统一框架}
\label{subsec:ch3_case_framework}

为确保本章所有比较研究均在统一且公平的基准下进行,除特别说明外,全部数值算例均采用如下统一设置:
\begin{itemize}
	\item 物理与本构:材料为均匀、各向同性的线弹性体,平面应力假设。

	\item 采用修正的 SIMP 模型,以避免低密度区域的刚度矩阵奇异性。杨氏模量 $E$ 被定义为关于空间位置 $\boldsymbol{x}$ 处相对密度场 $\rho(\boldsymbol{x})$ 的函数:
	\[
	E(\rho(\boldsymbol{x})) = E_{\min} + (\rho(\boldsymbol{x}))^p (E_0 - E_{\min}),
	\]
	其中,$E_0=1$ 为实体材料的杨氏模量,$E_{\min}=10^{-9}$ 为孔洞区域设定的极小非零刚度下限,惩罚因子固定为 $p=3$。
	
	\item 优化问题:主要优化问题为最小化柔顺度,并施加体积分数约束。
	
	\item 最小尺度控制:为获得网格无关且无病态的解,统一采用固定的最小特征尺度控制。具体地,过滤算子采用基于单元邻域的加权平均形式,其过滤半径 $r_{\min}$ 以物理长度为单位度量,除特别说明外统一取 $r_{\min} = 2.4$。该设置确保在不同网格密度下,过滤邻域覆盖的物理范围保持一致,从而实现真正的网格无关性。
	
	\item 优化算法:设计变量更新采用优化准则法,统一设置
	\begin{itemize}
		\item 正向移动限制 $m=0.2$;
		\item 数值阻尼系数 $\eta=0.5$;
		\item 拉格朗日乘子采用二分法搜索,给定上界 $l_{\max}=10^9$ 与容差 $\epsilon_{\text{bisect}}=10^{-3}$;
		\item 设计变量下界 $\rho_{\min}=10^{-9}$。
	\end{itemize}
		
	\item 收敛准则: 当连续两次迭代中设计变量 $\boldsymbol{\rho}$ 的变化量满足 
	\[
	\Vert\boldsymbol{\rho}^{(n+1)} - \boldsymbol{\rho}^{(n)}\Vert_{\infty,\Omega} < 10^{-2},
	\]
	此外,设置最大迭代步数 $n_{\max} = 500$ 作为安全上限。当迭代达到 $n_{\max}$ 而上述条件仍未满足时,算法亦终止。若此时柔顺度与体积分数已在较长迭代区间内保持基本稳定,则本文将该情形记为 “准收敛”。
	
	\item 单位体系:本章所有算例统一采用 $\mathrm{mm}/\mathrm{N}/\mathrm{MPa}$(长度单位 $\mathrm{mm}$,力单位 $\mathrm{N}$,杨氏模量单位 $\mathrm{MPa}$)。

\end{itemize}

值得注意的是,在有限元分析中,刚度矩阵的数值积分精度直接影响优化结果的可靠性。对于单元密度表征,由于设计变量在单元内为常数,刚度阵被积函数的多项式阶次相对较低,采用适当的阶数的标准高斯积分即可满足精度要求。而在节点密度表征下,密度场随空间变化,并通过 SIMP 插值进入材料刚度,刚度矩阵与灵敏度的被积函数中包含高次非线性项,需要提高数值积分精度以避免显著的积分误差。

本章统一采用如下积分设置,对于张量积单元(如二维四边形、三维六面体)上的拉格朗日元,单元密度表征采用 $q=k+1$ 阶高斯积分公式,节点密度表征采用 $q=k+2$ 阶;对于单纯形单元(如二维三角形、三维四面体)上的拉格朗日元,则不再用积分点数表示,而是按代数精度选择积分公式,单元密度表征选用代数精度不低于 $2k$ 阶的高斯积分公式,节点密度表征选用不低于 $2k+2$ 阶的积分公式。

\subsection{基准比较:数值病态与最小尺度控制}
\label{subsec:benchmark_pathology_min_lengthscale}
%算例 3.1:二维 MBB 梁
\noindent{\heiti\zihao{-4} 算例 3.1:二维 MBB 梁}\par\vspace{0.3\baselineskip}

本节首先采用二维 MBB(Messerschmitt-Bolkow-Blohm)梁作为基准算例,以对比有无最小尺度控制下的优化结果,并为后续研究建立参照。

MBB 梁是拓扑优化的经典基准算例 \cite{olhoff1991cad},其完整设计域如图 \ref{fig:mbb_domain_full} 所示,尺寸为 $120~\mathrm{mm}\times20~\mathrm{mm}$。鉴于结构与载荷关于竖向中线对称,为提高计算效率,采用右半域作为计算域,如图 \ref{fig:mbb_domain_half} 所示,半域尺寸为 $60~\mathrm{mm}\times20\mathrm~\mathrm{mm}$,左侧边施加对称约束($u_x = 0$),右下角节点施加滑移支座($u_y = 0$),左上角节点处施加竖直向下的集中载荷 $P = 1~\mathrm{N}$,其余边施加自由边界(即齐次诺伊曼)条件。

材料的杨氏模量 $E=1~\mathrm{MPa}$,泊松比 $\nu=0.3$,优化问题定义为:在满足体积分数约束 $V_f = 0.5$ 的条件下,寻找材料的最优分布以最小化结构柔顺度(即最大化刚度)。

\begin{figure}[!htbp]
	\centering
	% 强制标题居中
	\captionsetup{justification=centering} 

	\subfigure[全设计域]{
		\label{fig:mbb_domain_full}
		\includegraphics[width=0.47\textwidth]{fig3-2a.pdf}
	}
	\hfill % 把两张图撑到版心的最左和最右,利用所有空间
	\subfigure[右半设计域]{
		\label{fig:mbb_domain_half}
		\includegraphics[width=0.47\textwidth]{fig3-2b.pdf} 
	}
	\caption{二维 MBB 梁设计域。}
	\label{fig:mbb_domain}
\end{figure}


为突出数值病态与最小特征尺度控制的影响,本节基准比较统一采用单分辨率的位移型有限元离散,分析单元取 $k=1$ 阶四边形单元,采用 $q=2$ 阶高斯积分公式,设计变量取每个单元上的常值密度 $\rho_e$。在此基础上,考虑三种过滤策略:不施加最小特征尺度控制(无过滤)、灵敏度过滤与密度过滤。为保证物理尺度的一致性,灵敏度过滤与密度过滤的过滤半径均取 $r_{\min}=2.4$,并在三种不同网格剖分下进行对比,三组网格剖分分别为 $60\times20$、$90\times30$ 与 $150\times50$,对应位移自由度数 $u_{\mathrm{dof}}=\{2562,\,5642,\,15402\}$,设计变量自由度数 $\rho_{\mathrm{dof}}=\{1200,\,2700,\,7500\}$。上述对比结果汇总于表 \ref{tab:mbb_mesh_filter_results}。

\begin{table}[!htbp]
	\centering
	\captionsetup{justification=centering}
	\caption{二维 MBB 梁在不同网格剖分与过滤策略下的优化结果}
	\label{tab:mbb_mesh_filter_results}
	
	\setlength{\tabcolsep}{4pt}
	\renewcommand{\arraystretch}{1.15}
	
	\begin{adjustbox}{width=\textwidth}
		\begin{tabular}{
				% 垂直居中
				>{\centering\arraybackslash}m{0.15\textwidth} 
				>{\centering\arraybackslash}m{0.28\textwidth}
				>{\centering\arraybackslash}m{0.28\textwidth}
				>{\centering\arraybackslash}m{0.28\textwidth}
			}
			\toprule
			网格剖分 & 无最小特征尺度 & 灵敏度过滤 & 密度过滤 \\
			\midrule
			
			\makecell{$60\times20$}
			&
			\makecell{\includegraphics[width=\linewidth]{tab3-1_60_20_nolc.pdf}\\[-0.5mm]\footnotesize $n=53$, $c=203.0660$}
			&
			\makecell{\includegraphics[width=\linewidth]{tab3-1_60_20_sens.pdf}\\[-0.5mm]\footnotesize $n=106$, $c=216.8137$}
			&
			\makecell{\includegraphics[width=\linewidth]{tab3-1_60_20_dens.pdf}\\[-0.5mm]\footnotesize $n=144$, $c=233.7146$}
			\\
			\midrule
			
			\makecell{$90\times30$}
			&
			\makecell{\includegraphics[width=\linewidth]{tab3-1_90_30_nolc.pdf}\\[-0.5mm]\footnotesize $n=34$, $c=201.0279$}
			&
			\makecell{\includegraphics[width=\linewidth]{tab3-1_90_30_sens.pdf}\\[-0.5mm]\footnotesize $n=95$, $c=217.9426$}
			&
			\makecell{\includegraphics[width=\linewidth]{tab3-1_90_30_dens.pdf}\\[-0.5mm]\footnotesize $n=265$, $c=234.4671$}
			\\
			\midrule
			
			\makecell{$150\times50$}
			&
			\makecell{\includegraphics[width=\linewidth]{tab3-1_150_50_nolc.pdf}\\[-0.5mm]\footnotesize $n=48$, $c=202.8103$}
			&
			\makecell{\includegraphics[width=\linewidth]{tab3-1_150_50_sens.pdf}\\[-0.5mm]\footnotesize $n=95$, $c=219.7274$}
			&
			\makecell{\includegraphics[width=\linewidth]{tab3-1_150_50_dens.pdf}\\[-0.5mm]\footnotesize $n=362$, $c=235.7332$}
			\\
			\bottomrule
		\end{tabular}
	\end{adjustbox}
\end{table}

表 \ref{tab:mbb_mesh_filter_results} 第 2 列给出了在未施加最小特征尺度控制时,不同网格剖分下的优化结果。以粗网格 $60\times20$ 为例,位移自由度 $u_{\mathrm{dof}}=2562$,设计变量自由度 $\rho_{\mathrm{dof}}=1200$,优化在 $53$ 步即可收敛,柔顺度值约为 $c\approx203$。然而,结构内部尤其在荷载作用区域与右端支承附近出现了大面积规则的棋盘格图案,相邻单元的密度在接近 $0$ 与接近 $1$ 之间交替振荡,整体拓扑呈现出明显的网格振荡,主承载路径难以辨认,属于典型的棋盘格型数值病态。随着网格细化到 $90\times30$  乃至 $150\times50$,位移和设计变量自由度显著增加,但迭代步数和柔顺度值大致保持在同一量级,并未出现性能明显改善的趋势。与此同时,病态特征不仅没有消失,反而在细网格下演化为大量一至两单元宽度的细薄杆件与分枝结构,局部拓扑极其细碎和复杂,不同网格分辨率下腹板走向和承载路径差异显著,粗网格得到的拓扑在细网格上常演化为截然不同的布局,表现出强烈的网格依赖性。

相比之下,施加最小特征尺度控制后的结果(表 \ref{tab:mbb_mesh_filter_results} 第 3、4 列)在几何特征和网格依赖性方面均有显著改善。对三个网格层次,无论采用灵敏度过滤还是密度过滤,优化得到的拓扑结构均基本消除了棋盘格和局部高频振荡,粗、中、细网格之间的主承载路径保持较好的一致性,各主要构件的宽度稳定在由过滤半径 $r_{\min}$ 决定的尺度附近,灰度区主要局限于构件边缘的狭窄过渡带。与未过滤情形相比,过滤操作引入了额外的平滑约束,柔顺度值整体略有上升,迭代步数普遍增加,但换来了物理上更可行、几何上更规整的无病态设计。

需要指出的是,在相同过滤半径下,密度过滤(表 \ref{tab:mbb_mesh_filter_results} 第 4 列)对密度场本身进行平滑,所得结构相对更为保守,边界略显模糊;灵敏度过滤则形成更为清晰的 0–1 型拓扑。在二维 MBB 梁算例所采用的参数配置下,灵敏度过滤在几何清晰度与数值稳定性之间取得了较为满意的折中。基于此,为简化后续比较,本章后续算例如无特别说明均采用灵敏度过滤作为统一的最小特征尺度控制手段,而将密度过滤结果仅作为对照与补充。


\subsection{核心对比:表征方法与单元阶次}
\label{subsec:core_compare_rep_order}

第 \ref{subsec:benchmark_pathology_min_lengthscale} 节通过二维 MBB 梁基准算例说明了在单元密度表征下,若缺乏显式的最小特征尺度控制,则容易产生棋盘格与网格依赖等数值病态;而在施加统一过滤半径的灵敏度过滤与密度过滤后,上述病态现象可以得到显著缓解,并可获得几何清晰、对网格划分不敏感的拓扑结构。基于这一认识,本节在前述 MBB 梁算例的基础上,系统比较不同设计变量表征方式与不同单元阶次对优化结果的影响,物理与优化参数均沿用前文的统一设置。

为突出设计变量表征方式本身的作用,表 \ref{tab:mbb_elem_vs_node_density_unfiltered} 给出了在未施加最小特征尺度控制时,采用四边形单元进行位移分析下,不同网格剖分对应的单元密度与节点密度两种表征方式的优化结果。积分规则按照第 \ref{subsec:ch3_case_framework} 节中的统一设置选取,单元密度表征采用 $q=k+1$ 阶高斯积分公式,而节点密度表征采用 $q=k+2$ 高斯积分公式。在 $60\times20$、$90\times30$ 与 $150\times50$ 三组网格下,单元密度表征的设计变量自由度数分别为 $\rho_{\mathrm{dof}}^{\mathrm{elem}}=\{1200,\,2700,\,7500\}$,而节点密度表征下设计变量自由度数分别为 $\rho_{\mathrm{dof}}^{\mathrm{node}}=\{1281,\,2821,\,7701\}$。




\begin{table}[!htbp]
	\centering
	\captionsetup{justification=centering}
	\caption{二维 MBB 梁在不同网格剖分与密度表征下的优化结果(未过滤)}
	\label{tab:mbb_elem_vs_node_density_unfiltered}
	
	\setlength{\tabcolsep}{4pt}
	\renewcommand{\arraystretch}{1.15}
	
	\begin{adjustbox}{width=\textwidth}
		\begin{tabular}{
				% 垂直居中
				>{\centering\arraybackslash}m{0.17\textwidth}
				>{\centering\arraybackslash}m{0.415\textwidth}
				>{\centering\arraybackslash}m{0.415\textwidth}
			}
			\toprule
			网格剖分 & 单元密度 & 节点密度 \\
			\midrule
			
			\makecell{$60\times20$\\[1mm]}
			&
			\makecell{
				\includegraphics[width=\linewidth]{tab3-1_60_20_nolc.pdf}\\[-0.5mm]
				\footnotesize $n=53$, $c=203.0660$
			}
			&
			\makecell{
				\includegraphics[width=\linewidth]{tab3-2_60_20_node.pdf}\\[-0.5mm]
				\footnotesize $n=302$, $c=204.0256$
			}
			\\
			\midrule
			
			\makecell{$90\times30$\\[1mm]}
			&
			\makecell{
				\includegraphics[width=\linewidth]{tab3-1_90_30_nolc.pdf}\\[-0.5mm]
				\footnotesize $n=34$, $c=201.0279$
			}
			&
			\makecell{
				\includegraphics[width=\linewidth]{tab3-2_90_30_node.pdf}\\[-0.5mm]
				\footnotesize $n=216$, $c=197.3244$
			}
			\\
			\midrule
			
			\makecell{$150\times50$\\[1mm]}
			&
			\makecell{
				\includegraphics[width=\linewidth]{tab3-1_150_50_nolc.pdf}\\[-0.5mm]
				\footnotesize $n=48$, $c=202.8103$
			}
			&
			\makecell{
				\includegraphics[width=\linewidth]{tab3-2_150_50_node.pdf}\\[-0.5mm]
				\footnotesize $n=369$, $c=192.3462$
			}
			\\
			\bottomrule
		\end{tabular}
	\end{adjustbox}
\end{table}

与作为基准的单元密度方案(表 \ref{tab:mbb_elem_vs_node_density_unfiltered} 左列)相比,未过滤的节点密度方案(表 \ref{tab:mbb_elem_vs_node_density_unfiltered} 右列)表现出显著差异。得益于四节点双线性形函数的插值特性,节点密度场天然具备 $C^0$ 连续性,这使得优化结果基本消除了棋盘格等高频振荡病态。沿主承载路径的密度分布呈连续条带状,灰度主要集中在构件边缘的过渡带,视觉上更为平滑清晰,验证了节点表征的 “天然平滑” 机制对数值不稳定性的抑制作用。

然而,节点密度方案并未完全消除网格依赖性:在不同网格分辨率下,腹板数量、走向以及局部细杆件的形态仍存在可见差异,细网格下依然会出现一到两单元宽度的细薄构件与分枝结构。从数值角度看,三组网格中两种表征所得的柔顺度水平大致相当,节点密度并未表现出稳定而显著的目标值优势,但迭代代价却明显增加:单元密度方案通常在约 $30\sim50$ 步内即可满足收敛判据,而节点密度方案则往往需要约 $200\sim400$ 步,其迭代次数约为前者的 $6\sim8$ 倍。这反映了节点设计变量通过形函数在网格上的强耦合,以及连续性约束对快速形成陡峭 0–1 阶跃的阻碍。综合来看,仅将设计变量由单元转移至节点、依赖节点密度的 $C^0$ 连续性,虽然可以在一定程度上抑制棋盘格并改善拓扑,但既不能根除网格依赖,也会显著提高达到给定收敛精度所需的计算成本。

鉴于未过滤情形下两种密度表征方式均存在不同程度的网格依赖性,为在统一且物理可制造的框架下进行公平对比,本节进一步考察施加灵敏度过滤后的优化表现。表 \ref{tab:mbb_elem_vs_node_density_sens} 给出了在统一过滤半径 $r_{\min} = 2.4$ 下,不同网格剖分对应的单元密度与节点密度两种表征方式的优化结果。

\begin{table}[!htbp]
	\centering
	\captionsetup{justification=centering}
	\caption{二维 MBB 梁在不同网格剖分与密度表征下的优化结果(灵敏度过滤)}
	\label{tab:mbb_elem_vs_node_density_sens}
	
	\setlength{\tabcolsep}{4pt}
	\renewcommand{\arraystretch}{1.15}

	\begin{adjustbox}{width=\textwidth}
		\begin{tabular}{
				>{\centering\arraybackslash}m{0.15\textwidth}
				>{\centering\arraybackslash}m{0.425\textwidth}
				>{\centering\arraybackslash}m{0.425\textwidth}
			}
			\toprule
			网格剖分 & 单元密度 & 节点密度 \\
			\midrule
			
			\makecell{$60\times20$}
			&
			\makecell{\includegraphics[width=\linewidth]{tab3-1_60_20_sens}\\[-0.5mm]\footnotesize $n=106$, $c=216.8137$}
			&
			\makecell{\includegraphics[width=\linewidth]{tab3-3_60_20_node}\\[-0.5mm]\footnotesize $n=140$, $c=225.3707$}
			\\
			\midrule
			
			\makecell{$90\times30$}
			&
			\makecell{\includegraphics[width=\linewidth]{tab3-1_90_30_sens.pdf}\\[-0.5mm]\footnotesize $n=95$, $c=217.9426$}
			&
			\makecell{\includegraphics[width=\linewidth]{tab3-3_90_30_node.pdf}\\[-0.5mm]\footnotesize $n=108$, $c=221.4765$}
			\\
			\midrule
			
			\makecell{$150\times50$}
			&
			\makecell{\includegraphics[width=\linewidth]{tab3-1_150_50_sens.pdf}\\[-0.5mm]\footnotesize $n=95$, $c=219.7274$}
			&
			\makecell{\includegraphics[width=\linewidth]{tab3-3_150_50_node.pdf}\\[-0.5mm]\footnotesize $n=112$, $c=220.8803$}
			\\
			
			\bottomrule
		\end{tabular}
	\end{adjustbox}
\end{table}

与作为基准的单元密度方案(表 \ref{tab:mbb_elem_vs_node_density_sens} 左列)相比,施加同样过滤条件的节点密度方案(表 \ref{tab:mbb_elem_vs_node_density_sens} 右列)在最终拓扑形态上仅表现出细微差异。三种网格分辨率下,引入统一半径的灵敏度过滤同样有效地消除了节点密度表征下的棋盘格与网格依赖性。结合第 \ref{subsec:benchmark_pathology_min_lengthscale} 节中对单元密度方案的观察可以看出,一旦采用统一物理尺度的灵敏度过滤,显式的最小特征尺度控制在消除数值病态和缓解网格依赖方面起到了主导作用,而设计变量取单元密度还是节点密度对最终拓扑的影响已经相对次要。从迭代行为的角度看,这一点同样得到了印证。在未施加过滤时,单元密度方案相较于节点密度方案,迭代次数相差接近一个数量级;施加灵敏度过滤后,两种密度表征在三个网格上的迭代次数均集中在约一百步左右,差异已经不再显著。这一前后对比说明:在本算例下,迭代效率和数值稳定性的主导因素并非设计变量取单元还是节点,而是是否施加了统一的最小特征尺度控制。

综上,从网格加密的角度看,本节的结果表明:在缺乏显式最小特征尺度控制时,单元密度方案在各个网格上均出现严重的棋盘格和显著的网格依赖;节点密度表征虽然凭借 $C^0$ 连续性在一定程度上抑制了棋盘格,但仍无法消除内部主承载路径对离散尺度的敏感性,并且迭代代价明显增大。引入统一物理半径的灵敏度过滤后,两种密度表征在不同网格上的拓扑形态、柔顺度水平和收敛步数均趋于一致,数值病态得到根本缓解,说明显式长度尺度正则化在获得网格无关、几何清晰的拓扑结构方面起到了决定性作用,而设计变量取单元密度还是节点密度已退居次要地位。

在前文比较了不同网格剖分下的表现之后,本小节将网格统一固定为 $60\times20$ 的粗网格,首先在不施加过滤的条件下,仅改变单元阶次 $k$,考察单元密度与节点密度两种表征在不同单元阶次下的差异,在保持设计自由度不变的前提下(单元密度方案中 $\rho_{\text{dof}}=1200$,节点密度方案中 $\rho_{\text{dof}}=1281$),随着 $k $ 的提高,位移自由度从 $u_{\text{dof}}=2562$ 增加到 $u_{\text{dof}}=9922$ 以及 $u_{\text{dof}}=39042$,如表 \ref{tab:mbb_order_elem_vs_node_unfiltered} 所示。

\begin{table}[!htbp]
	\centering
	\captionsetup{justification=centering}
	\caption{二维 MBB 梁在不同四边形单元阶次与密度表征下的优化结果(无过滤)}
	\label{tab:mbb_order_elem_vs_node_unfiltered}
	\setlength{\tabcolsep}{4pt}
	\renewcommand{\arraystretch}{1.15}
	
	\begin{adjustbox}{width=\textwidth}
		\begin{tabular}{
				>{\centering\arraybackslash}m{0.16\textwidth}
				>{\centering\arraybackslash}m{0.42\textwidth}
				>{\centering\arraybackslash}m{0.42\textwidth}
			}
			\toprule
			单元阶次
			&
			单元密度
			&
			节点密度
			\\
			\midrule
			
			\makecell{$k=1$}
			&
			\makecell{\includegraphics[width=\linewidth]{tab3-1_60_20_nolc}\\[-0.5mm]\footnotesize $n=53$, $c=203.0660$}
			&
			\makecell{\includegraphics[width=\linewidth]{tab3-2_60_20_node}\\[-0.5mm]\footnotesize $n=302$, $c=204.0256$}
			\\
			\midrule
			
			\makecell{$k=2$}
			&
			\makecell{\includegraphics[width=\linewidth]{tab3-4_k2_elem.pdf}\\[-0.5mm]\footnotesize $n=36$, $c=213.1796$}
			&
			\makecell{\includegraphics[width=\linewidth]{tab3-4_k2_node.pdf}\\[-0.5mm]\footnotesize $n=500^{+}$, $c=212.1727$}
			\\
			\midrule
			
			\makecell{$k=4$}
			&
			\makecell{\includegraphics[width=\linewidth]{tab3-4_k4_elem.pdf}\\[-0.5mm]\footnotesize $n=41$, $c=218.6621$}
			&
			\makecell{\includegraphics[width=\linewidth]{tab3-4_k4_node.pdf}\\[-0.5mm]\footnotesize $n=500^{+}$, $c=216.0374$}
			\\
			
			\bottomrule
		\end{tabular}
	\end{adjustbox}
\end{table}

表 \ref{tab:mbb_order_elem_vs_node_unfiltered} 的第一行($k=1$)对应前述基准算例在未过滤条件下的结果,此处作为参照用以刻画单元阶次提高时的变化趋势。由表可见,提高单元阶次在两种密度表征下都呈现出相似的数值趋势:柔顺度随 $k$ 的增加而单调升高,单元密度方案中由 $c\approx203$ 增加到 $c\approx213$ 和 $c\approx219$,节点密度方案中由 $c\approx204$ 增加到 $c\approx212$ 和 $c\approx216$。这种 “柔顺度变大” 的现象并不意味着结构性能恶化,而是高阶有限元在一定程度上抑制了由离散误差、棋盘格及细微结构引起的 “虚假刚度”,使计算结果更接近物理合理的柔顺度水平,这是两种密度表征在提高阶次时的共性。

在此基础上,两种表征在拓扑形态上的表现有所不同。对单元密度表征而言,$k=1$ 时棋盘格最为显著,大量一单元尺度的细碎条块沿结构内部分布,提高至 $k=2$ 后,棋盘格明显减弱,密度沿构件方向被拉伸成较为连续的条带,局部结构明显平滑,继续提升到 $k=4$ 时,所得拓扑与 $k=2$ 已非常接近,局部振荡难以进一步显著减少,说明在缺乏显式最小特征尺度控制的前提下,高阶形函数对抑制棋盘格的作用在 $k\approx2$ 左右基本趋于饱和。

对节点密度表征而言:由于密度在节点上通过四节点双线性形函数进行 $C^0$ 连续插值,在 $k=1$ 时棋盘格现象已基本消失,三种单元阶次下得到的宏观拓扑框架几乎完全一致,高阶单元仅在主承载杆件和边界附近带来轻微的平滑或细化。此时,提高阶次的主要作用同样是削弱虚假刚度,而非改变拓扑结构。但在迭代行为上,节点密度方案表现出明显的不利影响:迭代步数由 $k=1$ 的 $302$ 步显著增加,到 $k=2$ 与 $k=4$ 均达到最大迭代步数 $n_{\max}=500$ 仍未满足收敛判据。图 3.2 给出了 $k=4$ 情形下两种密度表征的迭代历程对比:单元密度方案(图 \ref{fig:ch3_k4_conv_elem})在约 30 步左右即可使柔顺度和体积分数稳定,并很快满足设计变量变化量的收敛阈值,而节点密度方案(图 \ref{fig:ch3_k4_conv_node})在前若干步内同样迅速降低柔顺度,之后长达数百步仅在极窄区间内小幅震荡,体积分数始终维持在目标值附近,但由于 0–1 过渡带内存在局部微调,设计变量的最大变化量始终略高于阈值,最终在迭代步数达到 $n_{\max}=500$ 时由安全上限触发终止,表现出典型的 “准收敛” 特征。

\begin{figure}[!htbp]
	\centering
	\captionsetup{justification=centering}
	
	\subfigure[单元密度]{
		\label{fig:ch3_k4_conv_elem}
		\includegraphics[width=0.47\textwidth]{fig3-3a.pdf}
	}
	\hfill
	\subfigure[节点密度]{
		\label{fig:ch3_k4_conv_node}
		\includegraphics[width=0.47\textwidth]{fig3-3b.pdf}
	}
	
	\caption{$k=4$ 下两种密度表征的迭代历程对比}
	\label{fig:ch3_k4_conv_compare}
\end{figure}

上述无过滤算例表明,仅通过提高单元阶次并不能从根本上消除数值病态,在节点密度表征下还会诱发明显的 “准收敛” 行为。为进一步考察在最小特征尺度统一控制时单元阶次的作用,本节在表 \ref{tab:mbb_order_elem_vs_node_unfiltered} 的基础上,引入过滤半径为 $r_{\min}=2.4$ 的灵敏度过滤,给出不同单元阶次下单元密度与节点密度两种表征的优化结果,如表 \ref{tab:mbb_order_elem_vs_node_sens} 所示。

\begin{table}[!htbp]
	\centering
	\captionsetup{justification=centering}
	\caption{二维 MBB 梁在不同四边形单元阶次与密度表征下的优化结果(灵敏度过滤)}
	\label{tab:mbb_order_elem_vs_node_sens}
	
	\setlength{\tabcolsep}{4pt}
	\renewcommand{\arraystretch}{1.15}
	
	\begin{adjustbox}{width=\textwidth}
		\begin{tabular}{
				>{\centering\arraybackslash}m{0.16\textwidth}
				>{\centering\arraybackslash}m{0.42\textwidth}
				>{\centering\arraybackslash}m{0.42\textwidth}
			}
			\toprule
			单元阶次 & 单元密度 & 节点密度 \\
			\midrule
			
			\makecell{$k=1$}
			&
			\makecell{\includegraphics[width=\linewidth]{tab3-1_60_20_sens}\\[-0.5mm]\footnotesize $n=106$, $c=216.8137$}
			&
			\makecell{\includegraphics[width=\linewidth]{tab3-3_60_20_node}\\[-0.5mm]\footnotesize $n=140$, $c=225.3707$}
			\\
			\midrule
			
			\makecell{$k=2$}
			&
			\makecell{\includegraphics[width=\linewidth]{tab3-5_k2_elem}\\[-0.5mm]\footnotesize $n=115$, $c=226.2545$}
			&
			\makecell{\includegraphics[width=\linewidth]{tab3-5_k2_node}\\[-0.5mm]\footnotesize $n=144$, $c=229.6092$}
			\\
			\midrule
			
			\makecell{$k=4$}
			&
			\makecell{\includegraphics[width=\linewidth]{tab3-5_k4_elem}\\[-0.5mm]\footnotesize $n=113$, $c=230.8768$}
			&
			\makecell{\includegraphics[width=\linewidth]{tab3-5_k4_node}\\[-0.5mm]\footnotesize $n=144$, $c=233.5225$}
			\\
			
			\bottomrule
		\end{tabular}
	\end{adjustbox}
\end{table}

表 \ref{tab:mbb_order_elem_vs_node_sens} 的第一行($k=1$)对应前述基准算例在灵敏度过滤条件下的结果,此处作为参照用以刻画单元阶次提高时的变化趋势。由表可见,在引入统一半径的灵敏度过滤后,不同单元阶次与两种密度表征下的优化拓扑在形态上已高度一致。无论取单元密度还是节点密度,$k=1,2,4$ 情形均收敛到光滑、无棋盘格的结构框架,主承载路径基本重合,两者主要差异仅在于节点密度方案的边界轮廓和杆端过渡略为平滑一些,而整体构型基本一致,而整体构型几乎相同,说明此时由过滤半径 $r_{\min}$ 决定的最小特征尺度已在很大程度上主导了最终拓扑。

在性能与迭代行为上,随着 $k$ 从 $1$ 提高到 $2$、$4$,两种密度表征下的柔顺度仅出现幅度较小的变化,迭代步数也稳定在约一百步左右,高阶节点密度方案在无过滤时出现的 $500^+$ 步 “准收敛” 现象消失。就本算例和当前参数设置而言,这表明:在设计自由度和过滤半径固定的前提下,提高四边形单元阶次并未带来显著的拓扑改进或目标值收益,高阶形函数的精度优势难以得到充分发挥,单元阶次和密度表征对整体优化效果已退居次要地位。

为检验上述结论在其他单元族上的适用性,进一步考虑等面积剖分的三角形网格,除网格类型与积分阶次外,其余物理参数、优化设置以及灵敏度过滤半径均与前文保持一致。积分规则按照第 \ref{subsec:ch3_case_framework} 节中的统一设置选取:在单元密度表征下,统一采用代数精度不低于 $2k$ 阶的高斯积分公式,在节点密度表征下,不低于 $2k+2$ 阶。表 \ref{tab:mbb_tri_order_elem_vs_node_sens} 给出了在施加灵敏度过滤时,不同单元阶次下单元密度与节点密度两种表征的优化结果。

\begin{table}[!htbp]
	\centering
	\captionsetup{justification=centering}
	\caption{二维 MBB 梁在不同三角形单元阶次与密度表征下的优化结果(灵敏度过滤)}
	\label{tab:mbb_tri_order_elem_vs_node_sens}
	
	\setlength{\tabcolsep}{6pt}
	\renewcommand{\arraystretch}{1.15}
	
	\begin{adjustbox}{width=\textwidth}
		\begin{tabular}{
				>{\centering\arraybackslash}m{0.12\textwidth}
				>{\centering\arraybackslash}m{0.44\textwidth}
				>{\centering\arraybackslash}m{0.44\textwidth}
			}
			\toprule
			单元阶次 & 单元密度 & 节点密度 \\
			\midrule
			
			\makecell{$k=1$}
			&
			\makecell{\includegraphics[width=\linewidth]{tab3-6_tri_k1_elem}\\[-0.5mm]\footnotesize $n=101$, $c=212.2910$}
			&
			\makecell{\includegraphics[width=\linewidth]{tab3-6_tri_k1_node}\\[-0.5mm]\footnotesize $n=112$, $c=221.7030$}
			\\
			\midrule
			
			\makecell{$k=2$}
			&
			\makecell{\includegraphics[width=\linewidth]{tab3-6_tri_k2_elem}\\[-0.5mm]\footnotesize $n=98$, $c=224.2097$}
			&
			\makecell{\includegraphics[width=\linewidth]{tab3-6_tri_k2_node}\\[-0.5mm]\footnotesize $n=109$, $c=228.9346$}
			\\
			\midrule
			
			\makecell{$k=4$}
			&
			\makecell{\includegraphics[width=\linewidth]{tab3-6_tri_k4_elem}\\[-0.5mm]\footnotesize $n=99$, $c=228.8028$}
			&
			\makecell{\includegraphics[width=\linewidth]{tab3-6_tri_k4_node}\\[-0.5mm]\footnotesize $n=105$, $c=233.1645$}
			\\
			
			\bottomrule
		\end{tabular}
	\end{adjustbox}
\end{table}

从表 \ref{tab:mbb_tri_order_elem_vs_node_sens} 可以看到,在三角形单元上,单元密度与节点密度在统一施加灵敏度过滤后,同样收敛到无棋盘格、拓扑结构高度一致的解,不同单元阶次 $k$ 之间的拓扑差异很小,主要表现为杆件略有加粗或局部边界略为平滑,主承载路径和整体构型基本保持不变。相应的迭代步数也都集中在约百步量级,未再无过滤情形下的 “准收敛” 等异常行为。这与前文四边形单元上的观察结论一致,表明在统一的最小特征尺度控制下,“表征方法与单元阶次”的主要规律对不同单元族均具有良好的鲁棒性。

需要注意的是,在节点密度表征下,左侧固定边界附近形成了一条饱和的高密度带,视觉上呈现为一条 “黑色边界”。这可以理解为节点密度离散与体积分数度量方式共同作用的自然结果:边界节点仅被较少单元共享,在体积分数约束中的有效权重较小,在相同体积分数限制下将其取为 $1$ 所 “消耗” 的体积分数更低,因此优化过程倾向于优先填满这些边界节点,再叠加灵敏度过滤的平滑效应,高密度沿边界连续传播,最终形成稳定的边界高密度带。若在实际应用中希望进一步消除这种边界偏好,可通过重新设计体积分数的离散形式来统一各设计变量的体积权重(例如对节点密度采用归一化权重),这类修正可以减弱边界高密度带,但不会改变本节关于拓扑形态与收敛特性的整体结论,本文不再展开。

\subsection{算例推广:不同边界载荷}
\label{subsec:case_boundary_loads}

\subsection{算例推广:三维问题}
\label{subsec:case_3d}

\subsection{算例推广:柔顺机构设计}
\label{subsec:case_compliant_mechanism}
