% !TeX root = ../../brightPhD.tex
\chapter{基于高阶有限元的多分辨率高精度拓扑优化}
\label{key}

\section{引言}
\label{sec:ch4_intro}

\section{设计模型与分析模型}
\label{sec:design_analysis_model}

结构拓扑优化的根本任务是寻找结构的最优材料分布,而有限元分析则是评估该候选结构在给定物理工况下性能(如刚度、强度、应力)的手段。传统单元密度法往往将材料分布的参数化模型(设计模型)与偏微分方程的离散与求解模型(分析模型)混为一谈,使得两者在分辨率与功能上的区别被掩盖。若从数学建模与数值实现的角度加以区分,可以清晰地看到该过程实际上包含两个本质不同但紧密耦合的子模型:其一用于定义并生成可行的密度场,服务于优化变量的表达与正则化;其二用于在给定密度场下求解状态方程并计算目标与约束,服务于结构响应的高精度评估。

\subsection{模型的定义与二分性}
\label{subsec:model_definition_binarity}

在连续层面,变密度拓扑优化的一般形式及其函数空间设定已在第 2.2.2 节中给出。本文在本章中沿用相同的符号约定:以标量密度场 $\rho:\Omega\rightarrow[0,1]$ 描述结构中的材料分布,以位移场 $\boldsymbol{u}:\Omega\rightarrow\mathbb{R}^{d}$ 描述结构在外载与边界条件作用下的力学响应。

需要强调的是,在连续模型中,$\rho$ 与 $\boldsymbol{u}$ 并非并列出现的未知量,而是通过材料插值模型与线弹性状态方程形成一条明确的映射关系:密度场 $\rho$ 决定等效材料刚度,从而确定状态方程的算子形式;位移场 $\boldsymbol{u}$ 则作为该边值问题的解,用于评估结构在给定材料分布下的物理响应。换言之,$\rho$ 通过“定义材料”,而 $\boldsymbol{u}$ 通过 “响应材料”,二者在优化问题中承担着不同的功能角色。
基于上述认识,在变密度拓扑优化语境下,可以将整体计算过程中的两个子模型概括为:


\begin{itemize}
	\item 设计模型:给出可行密度场的参数化表示和优化变量的定义方式,并通过插值、过滤与投影等机制生成用于分析的物理密度场;
	\item 分析模型:在给定密度场(物理密度场)下,对线弹性等状态方程进行离散与求解,以评估柔顺度、应力等目标与约束函数。
\end{itemize}

在离散层面,设计模型通常由一组有限维设计变量向量 $\boldsymbol{d}$ 参数化,例如单元密度、节点密度或子单元密度等。为强调 “设计自由度” 与 “用于分析的密度场” 之间的区别,可将密度生成过程抽象为一条映射链:先由 $\boldsymbol{d}$ 通过插值给出连续密度近似 $\rho_h(\boldsymbol{x};\boldsymbol{d})$,再经由过滤/投影等正则化算子得到物理密度场 $\rho_{\text{phys}}(\boldsymbol{x})$,并最终进入有限元组装形成刚度矩阵 $\boldsymbol{K}(\rho_{\text{phys}})$。相应地,分析模型在给定 $\rho_{\text{phys}}$ 后求解离散平衡方程
\[
\boldsymbol{K}(\rho_{\text{phys}})\boldsymbol{U} = \boldsymbol{F},
\]
并据此计算目标与约束以及灵敏度信息。

从数值分析与工程设计的角度出发,在变密度方法中,设计模型与分析模型在分辨率需求上存在天然的差异:
\begin{itemize}
	\item 为获得边界清晰、细节丰富且具有可控最小特征尺度的结构形态,设计模型倾向于采用更高的空间分辨率,即更密集的设计自由度,以便刻画复杂的材料分布与细小构件。需要指出的是,结构的 “有效最小特征尺度” 通常主要由过滤/投影等正则化参数主导,而设计网格分辨率决定可表达细节的上限;
	\item 为控制计算成本并避免状态方程求解过度精细,分析模型通常在保证足够精度的前提下选取相对较粗的网格,或利用高阶有限元在较粗网格上提升近似能力,从而以更低自由度完成高精度响应评估。
\end{itemize}

理想情况下,设计模型与分析模型应在概念和实现上彼此独立:设计网格用于承载设计变量并通过过滤、投影等机制生成连续密度场;分析网格及其有限元空间则专注于状态方程的高精度求解。二者的解耦不仅可以更充分地发挥高阶有限元的逼近优势,而且为后续构造 “高阶分析、低成本 + 高分辨率设计、细致控制” 的多分辨率拓扑优化框架奠定基础。进一步地,这也自然导向在离散层面引入相互独立的设计网格与分析网格,并在必要时引入密度积分网格以保证刚度组装与约束评估的一致性。

\subsection{单分辨率框架的局限与精度错配}
\label{subsec:single_resolution_mismatch}

在传统 STOP 框架中,普遍采用 “基于单元的设计变量” 策略:每个有限元单元都被赋予一个唯一的设计变量(密度值),即设计模型与分析模型在同一张网格上完全重合。这一做法在实现上简单直接,但也意味着一旦选定分析网格的尺寸与单元类型,设计模型的分辨率便被刚性地锁定,二者之间不存在任何独立调节的自由度。

这种强耦合在概念上模糊了两个模型之间的本质区别,更重要的是,在数值上造成了显著的 “精度错配”。从分析角度看,随着有限元阶次 $k$ 的提高,高阶有限元可以在较粗网格上获得更高的求解精度;但从设计角度看,如果设计变量仍然仅在分析单元上定义(即 “一单元一设计变量”),那么设计模型的表达能力(分辨率)根本不会随 $k$ 的增加而改善。高阶有限元所带来的精细物理场分布,只是对同一幅 “像素化” 几何的更精确响应计算,而对拓扑结构本身的边界清晰度与可表达的细部特征几乎没有收益。从结构形态表达与优化效率的角度看,这种情形往往体现为计算资源的低效利用。

为说明 STOP 框架中的 “精度错配” 现象,本文沿用第 \ref{subsec:benchmark_pathology_min_lengthscale} 节中定义的对称二维 MBB 梁算例,作为对比分析的基准模型。在 STOP 中,设计变量与分析单元一一对应,因此设计变量自由度数为 $\rho_{\mathrm{dof}}=300$,在相同的分析网格($30\times10$ 的四边形网格)下,将分析模型的有限元阶次从 $k=2$ 提升到 $k=4$,总自由度数量由 $u_{\text{dof}}=2562$ 增加到 $u_{\text{dof}}=9922$,计算规模约增加 4 倍。然而,如图 \ref{fig:ch4_stop_order_mismatch} 所示,对比 $k=2$ 与 $k=4$ 的优化结果可以发现:当设计变量仍与分析单元一一对应时,所得拓扑结构在主承载路径、杆件尺度及边界位置等方面几乎一致,差异主要体现在局部灰度过渡带的平滑性上。换言之,单纯提高分析阶次主要是在更精确地分析同一幅“像素化几何”,并未从根本上提升结构形态的表达能力。

\begin{figure}[!htbp]
	\centering
	\captionsetup{justification=centering}
	
	\subfigure[$k=2$($u_{\mathrm{dof}}=2562$)]{
		\label{fig:ch4_stop_order_k2}
		\includegraphics[width=0.47\textwidth]{fig4-1a}
	}
	\hfill
	\subfigure[$k=4$($u_{\mathrm{dof}}=9922$)]{
		\label{fig:ch4_stop_order_k4}
		\includegraphics[width=0.47\textwidth]{fig4-1b}
	}
	
	\caption{STOP 下对称二维 MBB 梁在 $\rho_{\mathrm{dof}}=300$ 时不同有限元阶次的优化结果对比。}
	\label{fig:ch4_stop_order_mismatch}
\end{figure}

进一步地,若将上述错配情形反转,即在保持分析网格不变(甚至分析阶次较低)的同时显著提高设计分辨率,同样会引发新的数值问题。以每个分析单元内采用 $4\times4$ 的子密度划分为例,此时设计自由度增至 $\rho_{\mathrm{dof}}=4800$,而位移自由度仍为 $u_{\text{dof}}=2562$($k=2$)。当分析模型对高频密度变化存在欠解析时,优化迭代更易出现密度振荡、条纹状伪结构或收敛不稳定等现象(见图 \ref{fig:ch4_stop_design_refine})。

\begin{figure}[!htbp]
	\centering
	\captionsetup{justification=centering}
	\includegraphics[width=0.9\textwidth]{fig4-2}
	\caption{分析自由度固定($u_{\mathrm{dof}}=2562$)时提高设计分辨率($4\times4$ 子密度)的优化结构示例。}
	\label{fig:ch4_stop_design_refine}
\end{figure}

上述两类对偶现象表明:单纯追求 “高精度分析” 或 “高分辨率设计” 都不能自动保证获得高质量拓扑结构。为充分发挥高阶有限元在状态求解方面的优势,同时在可控计算代价下获得边界清晰、细节可靠且无伪结构的优化结果,有必要打破设计模型与分析模型的刚性耦合,并引入能够协调 “设计—分析—积分” 三个层面分辨率的多分辨率拓扑优化框架。这也正是后续 MTOP 范式需要建立分辨率匹配机制的根本原因。

\section{多分辨率拓扑优化计算框架}
\label{sec:mtop_framework}

为了在数学上实现设计与分析的解耦,并确保在粗网格高阶分析下的数值计算准确性,本章建立由位移有限元网格、设计变量网格与密度积分网格构成的三层离散系统,并在此基础上构建优化数学模型。

\subsection{多分辨率空间离散策略}
\label{subsec:multires_spatial_discretization}

为了打破传统方法中几何分辨率受限于分析单元尺寸的桎梏,本章引入了三套相互独立但逻辑关联的离散网格系统。首先是位移有限元网格($\mathcal{T}_h$),作为求解线弹性平衡方程 $\boldsymbol{K}\boldsymbol{U}=\boldsymbol{F}$ 的分析模型。在本章研究中,该网格由较粗的高阶拉格朗日单元(阶次 $k \ge 1$)构成,旨在利用高阶形函数优异的逼近能力消除剪切闭锁现象,并在控制整体自由度规模的前提下显著提高位移场与应变能的计算精度。

与分析模型相对应的是用于优化迭代的设计变量网格($\mathcal{T}_d$)。该网格由高密度的节点或中心点组成,每个节点关联一个设计变量 $d_i \in [0, 1]$。设计变量网格的分辨率显著高于位移网格,且二者在空间上相互独立;这种离散策略允许优化器在粗糙的分析网格背景下,探索超越分析单元尺度的精细拓扑结构,从而实现设计空间对分析空间的解耦。

然而,粗位移单元内部由精细设计变量定义的材料分布往往不再均匀,导致传统的全量高斯积分失效。为此,本章进一步引入密度积分网格($\mathcal{T}_\rho$) 作为连接高分辨率设计与低分辨率分析的数值桥梁。具体而言,为了精确捕捉粗位移单元 $\Omega_e$ 内部的非均匀材料分布,每个位移单元被进一步细分为 $N_i$ 个微小的密度子单元。刚度矩阵的数值积分将在这些子单元上进行,以确保在非均匀材料场下刚度评估的准确性。


\subsection{多分辨率框架下的优化问题列式}
\label{subsec:multires_optimization_formulation}

\subsection{基于子单元的刚度矩阵数值积分}
\label{subsec:subcell_stiffness_integration}

\subsection{灵敏度分析}
\label{subsec:sensitivity_analysis}


