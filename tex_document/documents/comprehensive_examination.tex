% !Mode:: "TeX:UTF-8"
\documentclass{beamer}

\usetheme{Darmstadt}
\useinnertheme{rounded}

\usecolortheme{beaver}
%\usecolortheme{albatross}
%\usecolortheme{beetle}
%\usecolortheme{crane}
%\usecolortheme{dolphin}
%\usecolortheme{dove}
%\usecolortheme{fly}
%\usecolortheme{lily}
%\usecolortheme{orchid}
%\usecolortheme{rose}
%\usecolortheme{seagull}
%\usecolortheme{seahorse}
%\usecolortheme{whale}
%\usecolortheme{wolverine}
%\usecolortheme{default}


\setbeamerfont*{frametitle}{size=\normalsize,series=\bfseries}
\setbeamertemplate{navigation symbols}{}

%%%%%%%%------------------------------------------------------------------------
%%%% 必要宏包

%% 数学宏包
\usepackage{amsmath,amssymb}

%% 颜色支持
\usepackage{color,xcolor}

%% 超链接
\usepackage{hyperref}

%% 图形支持
\usepackage{graphicx}
\usepackage{subfig}

%% 基本绘图功能
\usepackage{pgf,tikz}
\usetikzlibrary{shapes,arrows}

%% 演示文稿中使用的基本数学命令
\newcommand{\mbR}{{\mathbb R}}
\newcommand{\mcZ}{{\mathcal Z}}
\newcommand{\mcV}{{\mathcal V}}
\newcommand{\bfx}{{\boldsymbol x}}
\newcommand{\bfz}{{\boldsymbol z}}
\newcommand{\rmd}{\,\mathrm d}

%% 基本颜色命令
\newcommand{\red}{\color{red}}
\newcommand{\blue}{\color{blue}}

%% 为beamer设置参考文献样式
\makeatletter
\@ifclassloaded{beamer}{
	\bibliographystyle{apalike}
}{
	\bibliographystyle{abbrv}
}
\makeatother
%%%%%%%%------------------------------------------------------------------------
%%%%%%%%------------------------------------------------------------------------
%%%% XeCJK必要设置

\usepackage{xltxtra, fontenc, xunicode}
\usepackage[slantfont, boldfont]{xeCJK}

\setlength{\parindent}{1.5em} % 中文段落缩进

%% 中文断行设置
\XeTeXlinebreaklocale "zh"
\XeTeXlinebreakskip = 0pt plus 1pt minus 0.1pt

%% 中文标点样式
\punctstyle{kaiming}

%% 基本中文字体
\setCJKmainfont[BoldFont={Adobe Heiti Std}, ItalicFont={Adobe Kaiti Std}]{Adobe Song Std}

%% 英文字体
\setmainfont{DejaVu Serif}
\setmonofont{DejaVu Sans Mono}
\setsansfont{DejaVu Sans}

%% 常用中文字体系列
\setCJKfamilyfont{song}{Adobe Song Std}
\setCJKfamilyfont{kai}{Adobe Kaiti Std}
\setCJKfamilyfont{hei}{Adobe Heiti Std}

%% 字体命令
\newcommand{\song}{\CJKfamily{song}}
\newcommand{\kai}{\CJKfamily{kai}}
\newcommand{\hei}{\CJKfamily{hei}}

%% 环境的中文名称
\renewcommand{\contentsname}{目录}
\renewcommand{\figurename}{图}
\renewcommand{\tablename}{表}
\renewcommand{\today}{\number\year 年 \number\month 月 \number\day 日}

\makeatletter
\@ifclassloaded{book}{
	\renewcommand{\bibname}{参考文献}
}{
	\renewcommand{\refname}{参考文献}
}
\makeatother
%%%%%%%%------------------------------------------------------------------------

\usepackage{biblatex}
\addbibresource{ref.bib}

\usefonttheme[onlymath]{serif}
\numberwithin{subsection}{section}
%\usefonttheme[onlylarge]{structurebold}
\setbeamercovered{transparent}

\title{基于自动微分的拓扑优化算法、实现与应用的研究} 
\author{何亮}
\institute[XTU]{
学号:202331510117\\
\vspace{5pt}
湘潭大学$\bullet$数学与计算科学学院\\
}
 
\date[XTU]
{
    \today
}


\AtBeginSection[]
{
  \frame<beamer>{ 
    \frametitle{Outline}   
    \tableofcontents[currentsection] 
  }
}

\AtBeginSubsection[]
{
  \frame<beamer>{ 
    \frametitle{Outline}   
    \tableofcontents[currentsubsection] 
  }
}

\begin{document}
\begin{frame}
  \titlepage
\end{frame}

\begin{frame}{Outline}
  \tableofcontents
\end{frame}

\section{引言}

%--------------------------------------------------------------------------------------
\begin{frame}
\frametitle{研究背景}
    \begin{itemize}
	    \item[•]拓扑优化(TO)是一种通过优化材料分布以获得最佳结构性能的技术,在工程设计中具有重要应用价值。
	    \vspace{0.3cm}
	    \item[•]自动微分(AD)技术可以高效、准确地计算数值函数的导数,有潜力在拓扑优化中用于灵敏度分析。
    \end{itemize}
\end{frame}
%--------------------------------------------------------------------------------------

%--------------------------------------------------------------------------------------
\begin{frame}
\frametitle{研究动机}
    \begin{itemize}
	    \item[•]尽管 AD 技术在机器学习和科学计算中广泛应用,但在拓扑优化中的应用仍较少,尤其是将其与传统方法进行对比的研究更少。
	    \vspace{0.3cm}
	    \item[•]现有的自动微分 TO 框架展示了 AD 在拓扑优化中的应用潜力,但其可重用性和扩展性存在局限,需要进一步改进。
    \end{itemize}
\end{frame}
%--------------------------------------------------------------------------------------

%--------------------------------------------------------------------------------------
\begin{frame}
\frametitle{研究目标}
    \begin{itemize}
	    \item[•]本研究旨在基于 FEALPy 框架,应用 AD 技术提高拓扑优化的灵敏度分析效率和准确性。
	    \vspace{0.3cm}
	    \item[•]实现重用性更好的自动微分拓扑优化框架,并将其与传统变密度方法进行对比,验证 AD 对变密度方法的影响。
    \end{itemize}
\end{frame}
%--------------------------------------------------------------------------------------

\section{背景与动机}

%--------------------------------------------------------------------------------------
\begin{frame}
    \frametitle{背景}
    \textbf{TO 已有工作}
    \begin{itemize}
	    \item[$\bullet$] TO 在近年来得到了广泛关注,主要由于制造能力和计算建模的进步。
	    \vspace{0.3cm}
	    \item[$\bullet$] 基于连续体结构的 TO 方法目前分为两大类:变密度方法及其变形(如 SIMP 方法、ESO/BESO 方法)、边界演化方法(如水平集方法、MMC/MMV 方法和相场方法)。
	    \vspace{0.3cm}
	    \item[$\bullet$] 这些方法各有优缺点,但在处理大规模问题时通常面临计算效率和灵敏度分析复杂的问题。
    \end{itemize}
\end{frame}
%--------------------------------------------------------------------------------------

%--------------------------------------------------------------------------------------
\begin{frame}
	\frametitle{背景}
    \textbf{AD 已有工作}
    \begin{itemize}
	    \item[$\bullet$] AD 是一种强大的技术,通过在计算过程中自动生成导数,避免了手动求导的繁琐和易错。
	    \vspace{0.3cm}
	    \item[$\bullet$] AD 已经存在了几十年,被广泛应用于解决各种问题,从分子动力学模拟、流动设计问题到光子晶体的设计等,提高了计算效率和准确性。
    \end{itemize}
\end{frame}
%--------------------------------------------------------------------------------------

%--------------------------------------------------------------------------------------
\begin{frame}

	\frametitle{背景}
    \textbf{AuTO 已有工作}
    \begin{itemize}
	    \item[$\bullet$] AuTO(Automatic differentiation in Topology Optimization)框架结合了拓扑优化和自动微分技术,实现了高效的灵敏度分析。
	    \vspace{0.3cm}
	    \item[$\bullet$] Aaditya Chandrasekhar 基于高性能 Python 库 JAX,自动计算了用户定义的 TO 问题的灵敏度。
	    \vspace{0.3cm}
	    \item[$\bullet$] AuTO 框架展示了在复杂拓扑优化问题中自动微分的应用潜力,但在可重用性和扩展性方面存在局限。
    \end{itemize}
    
\end{frame}

%--------------------------------------------------------------------------------------

\begin{frame}
    \frametitle{动机}    
    \textbf{基于 FEALPy 的原因}
    \begin{itemize}
        \item[$\bullet$]\textbf{模块化设计}:FEALPy 的模块化设计提供了灵活的数值计算环境,便于算法的开发和测试。
        \vspace{0.3cm}
        \item[$\bullet$]\textbf{丰富的网格生成算法}:FEALPy 中实现了大量的网格生成和优化算法,能够处理复杂的几何结构。
        \vspace{0.3cm}
        \item[$\bullet$]\textbf{Torch 模块与自动微分}:FEALPy 中的 Torch 模块借助 PyTorch 实现 AD,提高了灵敏度分析的效率和准确性。
    \end{itemize}
\end{frame}

%--------------------------------------------------------------------------------------

\begin{frame}
    \frametitle{动机}
    \textbf{研究价值}
    \begin{itemize}
        \item[$\bullet$]\textbf{多计算内核支持}:FEALPy 作为一个开源的软件包,可以使用不同的计算内核(如 JAX、PyTorch)来实现 AuTO。
        \vspace{0.3cm}
        \item[$\bullet$]\textbf{结果对比验证}:FEALPy 基于 Numpy 也实现了很多传统 TO 方法,可以很方便地将两者的结果进行对比验证。
        \vspace{0.3cm}
        \item[$\bullet$]\textbf{探索 AD 技术应用}:探讨 AD 技术在复杂非线性和多物理场问题中的应用,填补现有研究的空白。
    \end{itemize}
    
\end{frame}

%--------------------------------------------------------------------------------------

\section{研究目标}

%--------------------------------------------------------------------------------------
\begin{frame}
    \frametitle{研究目标}
    \begin{block}{总体目标}
		本研究旨在 FEALPy 框架下,应用 AD 技术提高拓扑优化的灵敏度分析效率和准确性,并实现重用性更好的 AuTO 框架。
	\end{block}
\end{frame}
%--------------------------------------------------------------------------------------

%--------------------------------------------------------------------------------------
\begin{frame}
	\frametitle{研究目标}
	\textbf{具体目标}
    \begin{itemize}
        \item[$\bullet$]\textbf{利用 FEALPy 的模块化设计}
        \begin{itemize}
            \item[•]通过模块化设计,实现更加灵活和高效的拓扑优化算法流程,便于拓扑优化算法的开发、测试和扩展。
        \end{itemize}
        \vspace{0.3cm}
        \item[$\bullet$]\textbf{利用 FEALPy 的 Torch 模块}
        \begin{itemize}
	        \item[•]实现基于 Torch 的灵敏度分析模块,通过 PyTorch 实现高效的自动微分,简化求导过程,优化拓扑优化流程。
        \end{itemize}
        \vspace{0.3cm}
        \item[$\bullet$]\textbf{探讨 AD 技术在复杂非线性和多物理场问题中的应用}
        \begin{itemize}
            \item[•]扩展 AD 技术在复杂非线性问题中的应用,提高计算稳定性和精度。
            \item[•]探索 AD 技术在多物理场耦合问题中的应用,解决多物理场问题中的灵敏度分析难题。
        \end{itemize}
    \end{itemize}
\end{frame}
%--------------------------------------------------------------------------------------

\section{研究方法}

%--------------------------------------------------------------------------------------
\begin{frame}
	\frametitle{研究方法}
	 \textbf{基于变密度方法的柔顺度最小化问题数学模型}
	 \begin{itemize}
	            \item 描述变密度方法的柔顺度最小化问题。
	            \item[•]目标函数
	            $$\min_{\rho} :C = \int_{\Omega}(E(\rho)\varepsilon(u):\varepsilon(u))~\mathrm{d}x.$$
	            \item[•]约束条件
	            $$\int_\Omega\rho(x)~\mathrm{d}x - V_{\max};\quad0<\rho_{\min}\leq\rho\leq1.$$
	            \item[•]平衡方程
	            $$\int_{\Omega}(\sigma(u):\varepsilon(v))~\mathrm{d}x = \int_{\Omega}f{v}~\mathrm{d}x + \int_{\Gamma_N}F_n{v}~\mathrm{d}s,\quad\mathrm{for~all}~v\in{U}.$$
	 \end{itemize}
\end{frame}
%--------------------------------------------------------------------------------------

%--------------------------------------------------------------------------------------
\begin{frame}
	\frametitle{研究方法}
	 \textbf{基于 FEALPy 中的 Numpy 模块实现传统变密度方法的算法流程}
	 \begin{itemize}
            \item[•]初始化设计变量 $\rho$。
            \item[•]迭代过程:
	        \begin{enumerate}
	            \item[1.]计算当前设计变量下的刚度矩阵 $K(\rho)$。
	            \item[2.]求解平衡方程 $K(\rho) u = f$ 获得位移场 $u$。
	            \item[3.]计算目标函数 $C(\rho, u)$。
	            \item[4.]进行灵敏度分析,计算目标函数对设计变量的导数
	            $\frac{\partial C}{\partial \rho}$。
	            \item[5.]更新设计变量 $\rho$。
	        \end{enumerate}
	        \item[•]收敛判定:检查目标函数变化是否满足精度要求或达到最大迭代次数。
     \end{itemize}
\end{frame}
%--------------------------------------------------------------------------------------

%--------------------------------------------------------------------------------------
\begin{frame}
	\frametitle{研究方法}
	 \textbf{基于 FEALPy 的 Torch 模块实现引入 AD 的变密度方法的算法流程}
	 \begin{itemize}
            \item[•]初始化设计变量 $\rho$。
            \item[•]迭代过程:
	        \begin{enumerate}
	            \item[1.]计算当前设计变量下的刚度矩阵 $K(\rho)$。
	            \item[2.]求解平衡方程 $K(\rho) u = f$ 获得位移场 $u$。
	            \item[3.]计算目标函数 $C(\rho, u)$。
	            \item[4.]使用 PyTorch 自动计算目标函数对设计变量的导数 
	            $\frac{\partial C}{\partial \rho}$。
	            \item[5.]更新设计变量 $\rho$。
	        \end{enumerate}
	        \item[•]收敛判定:检查目标函数变化是否满足精度要求或达到最大迭代次数。
     \end{itemize}
\end{frame}
%--------------------------------------------------------------------------------------

%--------------------------------------------------------------------------------------
\begin{frame}
	\frametitle{研究方法}
	 \textbf{比较两者的结果,验证 AD 对于变密度方法的影响}
	 \vspace{0.3cm}
    \begin{itemize}
        \item[•]比较两种方法在计算时间、收敛速度和计算精度上的表现。
        \vspace{0.3cm}
        \item[•]分析 AD 在复杂问题中的优势和可能的劣势。
        \vspace{0.3cm}
        \item[•]通过数值实验验证 AD 技术在变密度方法中的应用效果。
    \end{itemize}
\end{frame}
%--------------------------------------------------------------------------------------

\section{研究进展}

%--------------------------------------------------------------------------------------
\begin{frame}
    \frametitle{研究进展}
    \textbf{求解基于变密度方法的柔顺度最小化问题}
    \vspace{0.3cm}
    \begin{itemize}
        \item[•]完成工作:基于变密度方法的柔顺度最小化问题的数学模型;使用最优准则方法和移动渐近线方法分别求解基于变密度方法的柔顺度最小化问题的算法流程。
    \end{itemize}
\end{frame}
%--------------------------------------------------------------------------------------

%--------------------------------------------------------------------------------------
\begin{frame}
    \frametitle{研究进展}
    \textbf{基于 FEALPy 中的 Numpy 模块实现传统变密度方法}
    \vspace{0.3cm}
    \begin{itemize}
        \item[•]计划工作:基于 FEALPy 中的 Numpy 模块,按照推导的数学模型实现传统变密度方法的求解流程。
    \end{itemize}
\end{frame}
%--------------------------------------------------------------------------------------

%--------------------------------------------------------------------------------------
\begin{frame}
    \frametitle{研究进展}
    \textbf{基于 FEALPy 的 Torch 模块实现引入 AD 的变密度方法}
    \vspace{0.3cm}
    \begin{itemize}
        \item[•]计划工作:基于 FEALPy 的 Torch 模块,利用 PyTorch 的自动微分功能实现引入 AD 的变密度方法的求解流程。
    \end{itemize}
\end{frame}
%--------------------------------------------------------------------------------------

%--------------------------------------------------------------------------------------
\begin{frame}
    \frametitle{研究进展}
    \textbf{比较两者的结果,验证 AD 对于变密度方法的影响}
    \vspace{0.3cm}
    \begin{itemize}
        \item[•]计划工作:完成基于传统变密度方法和引入 AD 的变密度方法的实现后,比较两者在计算时间、收敛速度和计算精度上的表现,验证自动微分技术的应用效果。
    \end{itemize}
\end{frame}
%--------------------------------------------------------------------------------------

\section{预期结果与贡献}

%--------------------------------------------------------------------------------------
\begin{frame}
    \frametitle{预期结果}
    \textbf{直接将 AD 引入变密度方法的效果}
    \begin{itemize}
        \item[•]\textbf{预期结果}:直接将 AD 技术引入变密度方法的效果可能不好,可能无法超越传统的变密度方法。
        \vspace{0.3cm}
        \item[•]\textbf{原因分析}:传统变密度方法经过多年的发展和优化,已经在很多方面达到了较好的效果,直接引入 AD 可能无法立即表现出优势。
    \end{itemize}
\end{frame}
%--------------------------------------------------------------------------------------

%--------------------------------------------------------------------------------------
\begin{frame}
    \frametitle{预期结果}
    \textbf{额外引入正则化项后的效果}
    \begin{itemize}
        \item[•]\textbf{预期结果}:在引入正则化项之后,应用 AD 的变密度方法可能比传统的变密度方法效果更好。
        \vspace{0.3cm}
        \item[•]\textbf{原因分析}:正则化项可以改善设计变量的分布,使得 AD 的优势在处理复杂问题和提高计算效率方面更加明显。
    \end{itemize}
\end{frame}
%--------------------------------------------------------------------------------------

%--------------------------------------------------------------------------------------
\begin{frame}
    \frametitle{贡献}
    \textbf{引入 AD 技术对拓扑优化研究的贡献}
    \begin{itemize}
        \item[•]目前将 AD 引入拓扑优化的研究工作比较少,特别是将其与传统方法进行比较的工作更少。本研究通过对比分析传统变密度方法和引入 AD 的变密度方法,提供了宝贵的实验数据和理论支持。
        \vspace{0.3cm}
        \item[•]有利于后续应对 AD 技术在非线性问题和多物理场问题中的应用挑战。
    \end{itemize}
\end{frame}
%--------------------------------------------------------------------------------------

%--------------------------------------------------------------------------------------
\begin{frame}
    \frametitle{贡献}
    \textbf{基于 FEALPy 的实现对拓扑优化研究的贡献}
    \begin{itemize}
        \item[•]基于 FEALPy 实现传统的变密度方法和引入 AD 的变密度方法,为后续拓扑优化研究工作的进一步展开提供了坚实的基础。
        \vspace{0.3cm}
        \item[•]FEALPy 的模块化设计和多计算内核支持,使得研究成果具有较高的可重用性和扩展性,利于后续研究人员在此基础上进行优化和创新。
    \end{itemize}
\end{frame}
%--------------------------------------------------------------------------------------

%--------------------------------------------------------------------------------------
\frame{
\begin{center}\color{black} \Huge
\textsl{感谢大家}
\end{center}
}
%--------------------------------------------------------------------------------------

\end{document}
