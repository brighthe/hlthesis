% !Mode:: "TeX:UTF-8"
\documentclass{article}
%%%%%%%%------------------------------------------------------------------------
%%%% 必要宏包

%% 数学宏包
\usepackage{amsmath,amssymb}

%% 颜色支持
\usepackage{color,xcolor}

%% 超链接
\usepackage{hyperref}

%% 图形支持
\usepackage{graphicx}
\usepackage{subfig}

%% 基本绘图功能
\usepackage{pgf,tikz}
\usetikzlibrary{shapes,arrows}

%% 演示文稿中使用的基本数学命令
\newcommand{\mbR}{{\mathbb R}}
\newcommand{\mcZ}{{\mathcal Z}}
\newcommand{\mcV}{{\mathcal V}}
\newcommand{\bfx}{{\boldsymbol x}}
\newcommand{\bfz}{{\boldsymbol z}}
\newcommand{\rmd}{\,\mathrm d}

%% 基本颜色命令
\newcommand{\red}{\color{red}}
\newcommand{\blue}{\color{blue}}

%% 为beamer设置参考文献样式
\makeatletter
\@ifclassloaded{beamer}{
	\bibliographystyle{apalike}
}{
	\bibliographystyle{abbrv}
}
\makeatother
%%%%%%%%------------------------------------------------------------------------
%%%%%%%%------------------------------------------------------------------------
%%%% XeCJK必要设置

\usepackage{xltxtra, fontenc, xunicode}
\usepackage[slantfont, boldfont]{xeCJK}

\setlength{\parindent}{1.5em} % 中文段落缩进

%% 中文断行设置
\XeTeXlinebreaklocale "zh"
\XeTeXlinebreakskip = 0pt plus 1pt minus 0.1pt

%% 中文标点样式
\punctstyle{kaiming}

%% 基本中文字体
\setCJKmainfont[BoldFont={Adobe Heiti Std}, ItalicFont={Adobe Kaiti Std}]{Adobe Song Std}

%% 英文字体
\setmainfont{DejaVu Serif}
\setmonofont{DejaVu Sans Mono}
\setsansfont{DejaVu Sans}

%% 常用中文字体系列
\setCJKfamilyfont{song}{Adobe Song Std}
\setCJKfamilyfont{kai}{Adobe Kaiti Std}
\setCJKfamilyfont{hei}{Adobe Heiti Std}

%% 字体命令
\newcommand{\song}{\CJKfamily{song}}
\newcommand{\kai}{\CJKfamily{kai}}
\newcommand{\hei}{\CJKfamily{hei}}

%% 环境的中文名称
\renewcommand{\contentsname}{目录}
\renewcommand{\figurename}{图}
\renewcommand{\tablename}{表}
\renewcommand{\today}{\number\year 年 \number\month 月 \number\day 日}

\makeatletter
\@ifclassloaded{book}{
	\renewcommand{\bibname}{参考文献}
}{
	\renewcommand{\refname}{参考文献}
}
\makeatother
%%%%%%%%------------------------------------------------------------------------
\begin{document}
\title{建筑结构力学分析计算内核开发项目验收报告}
\author{魏华祎}
\date{\chntoday}
\maketitle

\section{项目计划目标、任务和考核指标}

\subsection{项目计划目标}


\subsection{任务和考核指标}



\section{课题执行情况评价}


\section{研究基础}



\section{主要研究内容}
\subsection{任意次有限元求解线弹性方程}

\subsection{桁架结构求解模块}


\subsection{三维梁单元结构求解模块}

\subsubsection{模型假设}
Timoshenko 梁理论有如下假设:
\begin{itemize}
\item 变形前垂直梁中心线的平剖面,变形后仍然为平面(刚性横截面假定);

\item 梁受力发生变形时,横截面依然为一个平面,但不再垂直于中性轴,即考虑剪切变形和转动惯量的影响。
\end{itemize}

\subsubsection{数学模型}
Timoshenko 梁的应变能分为两部分:一是轴向变形产生的,二是由于剪切变形产生的,
$$\Lambda= \frac{1}{2}\int_z\int_{S(z)}E\varepsilon_z^2\,\mathrm{d}x\mathrm{d}y\mathrm{d}z+\frac{1}{2}
\int_z\int_{S(z)}\frac{G}{k}(\gamma_{xz}^2 + \gamma_{yz}^2)\,\mathrm{d}x\mathrm{d}y\mathrm{d}z$$
其中,$S(z)$表示中轴线上$z$点对应的截面($xy$平面), $k$是剪切修正系数。

\subsubsection{全局到局部的坐标变换}
给定空间中的一个梁单元 $e :=(\bm{x}_0, \bm{x}_1)$,其长度记为 $l$,则其单位切向量为
 $$\bm{t}:=\frac{\bm{x}_1-\bm{x}_0}{l}$$

给定梁的任一截面 $S$,可以指定一个单位法向 $\bm{n}_x$做为 $S$的第一方向(局部的 $x$方向),通过叉乘运算可以确定另外一个方向 $\bm{n}_y = \bm{t}\times \bm{n}_x$,则可以用标架 $(\bm{n}_x, \bm{n}_y, \bm{t})$ 建立梁的局部坐标系,原点选为梁中点 $\bm{c} := \frac{\bm{x}_0 + \bm{x}_1}{2}$。我们用 $|S|$表示截面 $S$的面积。

局部坐标系的坐标向量为 $(\bm{n}_x, \bm{n}_y, \bm{t})$,则全局到局部坐标系的变换矩阵为
$$\bm{T} = 
\begin{bmatrix}
\bm{n}_x^{\rm T} \\
\bm{n}_y^{\rm T} \\
\bm{t}^{\rm T}
\end{bmatrix}$$

设全局的平动位移为 $\bm{d} = \begin{bmatrix}
u & v & w
\end{bmatrix}^{\rm T}$,转动位移为 $\bm{\theta}=\begin{bmatrix}
\theta_x & \theta_y & \theta_z
\end{bmatrix}^T$,则在局部坐标系下的平动位移和转动位移分别为
$$\bar{\bm{d}}=\bm{T}\bm{d},\quad \bar{\bm{\theta}}=\bm{T}\bm{\theta}.$$
同时有
$$\bm{d}=\bm{T}^{\rm T}\bar{\bm{d}},\quad \bm{\theta}=\bm{T}^{\rm T}\bar{\bm{\theta}}.$$

\subsubsection{位移-应变关系}

在局部坐标系下计算梁截面上的任一点 $(x,y,z)$处的位移,由于
\begin{align*}
u(x, y, z) &= \bar{u}(z)-y\bar{\theta}_z(z)\\
v(x,y,z) &= \bar{v}(z)+x\bar{\theta}_z(z)\\
w(x, y, z) &= \bar{w}(z)-x\bar{\theta}_y(z) + y\bar{\theta}_x(z)
\end{align*}
则应变的计算对应的矩阵形式为:
$$\begin{bmatrix}
\varepsilon_z \\
\gamma_{xz} \\
\gamma_{yz}
\end{bmatrix} = 
\begin{bmatrix}
0 & 0 & 
\frac{\mathrm{d}}{\mathrm{d}z} &
y\frac{\mathrm{d}}{\mathrm{d}z} &
-x\frac{\mathrm{d}}{\mathrm{d}z}&
0\\
\frac{\mathrm{d}}{\mathrm{d}z} &
0 & 0 & 0 & -1 &
-y\frac{\mathrm{d}}{\mathrm{d}z}\\
0&\frac{\mathrm{d}}{\mathrm{d}z} &
0 &  1 & 0 &
x\frac{\mathrm{d}}{\mathrm{d}z}
\end{bmatrix}
\begin{bmatrix}
\bar{u}(z) \\
\bar{v}(z) \\
\bar{w}(z) \\
\bar{\theta}_x(z) \\
\bar{\theta}_y(z) \\
\bar{\theta}_z(z) \\
\end{bmatrix} = 
\mathcal{B}\bar{\bm{u}}$$

取线性拉格朗日基
$$\phi_0=\frac{1-\xi}{2},\phi_1=\frac{1+\xi}{2},(-1\leq\xi\leq1)$$
其中,$$\xi=\frac{2z-z_0-z_1}{l}$$

先排 $\bm{x}_0$的 6 个自由度,再排 $\bm{x}_1$上的 6 个自由度,可得相应的向量基函数矩阵

$$\bm{\Phi} = 
\begin{bmatrix}
\bm{\Phi}_0 & \bm{\Phi}_1
\end{bmatrix} 
= 
\begin{bmatrix}
\phi_0\bm{I}_6 & \phi_1\bm{I}_6
\end{bmatrix} 
$$
可得应变矩阵$\bm{B}$:

$$\bm{B} = \mathcal{B}\bm{\Phi} =
\begin{bmatrix}
\bm{B}_0 & \bm{B}_1
\end{bmatrix}$$

\subsubsection{应变-应力关系}

应力-应变关系遵循 Hooke 定律:
$$\bm{\sigma}=\bm{D}\bm{\varepsilon}$$
即:
$$\begin{bmatrix}
\sigma_{z} \\
\tau_{zx} \\
\tau_{zy}
\end{bmatrix}  = \begin{bmatrix}
E & 0 & 0\\
0 & \frac{G}{k} & 0\\
0 & 0 & \frac{G}{k}
\end{bmatrix}\begin{bmatrix}
\varepsilon_z \\
\gamma_{zx} \\
\gamma_{zy}
\end{bmatrix}$$
其中$\bm{D}$是材料的弹性系数常数。

\subsubsection{梁的刚度矩阵}
根据梁的应变能、位移-应变以及应变-应力的关系,可得刚度矩阵计算公式为:
$$\bm{K}=
\int_z\int_{S(z)}
\bm{B}^T\bm{D}\bm{B}\,
\mathrm{d}x\mathrm{d}y\mathrm{d}z$$
由此可得刚度矩阵为:
$$\bm{K} = 
\begin{bmatrix}
\bm{K}_{00} & \bm{K}_{01} \\
\bm{K}_{10} & \bm{K}_{11}
\end{bmatrix}$$
其中,
\begin{align*}
    \bm{K}_{00} &= \int_z\int_{S(z)}
\bm{B}^T_0\bm{D}\bm{B}_0\,\mathrm{d}x\mathrm{d}y\mathrm{d}z \\
    \bm{K}_{11} &= \int_z\int_{S(z)}
\bm{B}^T_1\bm{D}\bm{B}_1\,
\mathrm{d}x\mathrm{d}y\mathrm{d}z \\
    \bm{K}_{01} &= \int_z\int_{S(z)}
\bm{B}^T_0\bm{D}\bm{B}_1\,
\mathrm{d}x\mathrm{d}y\mathrm{d}z \\
\end{align*}
横截面面积为$A=\int_{S}\mathrm{d}x\mathrm{d}y$。

注意:这里假设截面 $S$ 沿中轴线形状不变,因此面积是常数,惯性矩、惯性积和极惯性矩为:

\begin{align*}
	I_{xx} &= \int_Sy^2\,\mathrm{d}x\mathrm{d}y,I_{yy}=\int_Sx^2\,\mathrm{d}x\mathrm{d}y \\
	I_{xy} &= \int_Sxy\,\mathrm{d}x\mathrm{d}y \\
	J &= \int_S(x^2+y^2)\,\mathrm{d}x\mathrm{d}y
\end{align*}

\subsubsection{实验结果}


\section{预研内容}

\subsection{板壳结构求解模块}
\subsubsection{引言}
SHELL63 是 ANSYS 中的一个四节点有限元壳体单元,适用于模拟各种薄厚壳体结构。
\subsubsection{全局到单元局部坐标系推导}

\subsection{断裂模拟的自适应有限元方法}

\subsection{任意次混合有限元求解线弹性方程}

\section{主要成果及工作总结}



\bibliographystyle{abbrv}
\bibliography{Timoshenko3d}
\end{document}
