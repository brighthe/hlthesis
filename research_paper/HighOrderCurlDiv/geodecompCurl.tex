\documentclass[10pt]{amsart}
\usepackage{times,amsmath,amsbsy,amssymb,amscd,mathrsfs}
%\usepackage{slashbox}\usepackage{graphicx,subfigure,epstopdf,wrapfig,chemarrow}

\usepackage{algorithm2e} 
\usepackage{multicol,multirow}
\usepackage{mathtools}
\usepackage[usenames,dvipsnames,svgnames,table]{xcolor}
\usepackage[all]{xy}
\usepackage{wrapfig}
\usepackage{tcolorbox}

%\usepackage[labelformat=simple]{subfig}
%\usepackage[hang,small,bf]{caption}

\usepackage{tikz,tikz-cd}
\usepackage[utf8]{inputenc}
\usepackage{pgfplots} 
\usepackage{pgfgantt}
\usepackage{pdflscape}
\pgfplotsset{compat=newest} 
\pgfplotsset{plot coordinates/math parser=false}
\newlength\fwidth

%\usepackage[notcite,notref]{showkeys}
\usepackage[numbered]{mcode}

\definecolor{myBlue}{rgb}{0.0,0.0,0.55}
%\definecolor{green}{rgb}{0.0,0.7,0.2}
\usepackage[pdftex,colorlinks=true,citecolor=myBlue,linkcolor=myBlue]{hyperref}

\usepackage[hyperpageref]{backref}

%\newcommand{\LC}[1]{\textcolor{cyan}{#1}}
\newcommand{\LC}[1]{\textcolor{red}{#1}}
\newcommand{\XH}[1]{\textcolor{cyan}{#1}}


\usepackage{comment,enumerate,multicol,xspace}

  \newcounter{mnote}
  \setcounter{mnote}{0}
  \newcommand{\mnote}[1]{\addtocounter{mnote}{1}
    \ensuremath{{}^{\bullet\arabic{mnote}}}
    \marginpar{\footnotesize\em\color{red}\ensuremath{\bullet\arabic{mnote}}#1}}
  \let\oldmarginpar\marginpar
    \renewcommand\marginpar[1]{\-\oldmarginpar[\raggedleft\footnotesize #1]%
    {\raggedright\footnotesize #1}}

%\usepackage[pdftex,dvipsnames]{xcolor}

%\usepackage{xargs} % Use more than one optional parameter in a new commands
%\usepackage[colorinlistoftodos,prependcaption,textsize=footnotesize]{todonotes}
%
%\newcounter{mycomment}
%\newcommand{\mycomment}[2][]{%
%% initials of the author (optional) + note in the margin
%\refstepcounter{mycomment}%
%{%
%\todo[linecolor=blue,backgroundcolor=blue!25,bordercolor=blue]{%
%\textbf{Comment [{\sc #1\themycomment}]:}\\#2}%
%}}
%
%\newcommandx{\change}[2][1=]
%{\todo[linecolor=OliveGreen,backgroundcolor=OliveGreen!25,bordercolor=OliveGreen,#1]{%
%{\sc Change}:\\#2}}
%
%\newcommandx{\improvement}[2][1=]
%{\todo[linecolor=Plum,backgroundcolor=Plum!25,bordercolor=Plum,#1]{%
%{\sc Improvement}:\\#2}}
%
%\newcommandx{\unsure}[2][1=]
%{\todo[linecolor=red,backgroundcolor=red!25,bordercolor=red,#1]{%
%{\sc Unsure}:\\ #2}}
%

% \newcommand{\mnote}[1]{}
\newcommand{\breakline}{
\begin{center}
------------------------------------------------------------------------------------------------------------
\end{center}
}
%\usepackage{geometry}
%%\usepackage{graphicx,pst-eps,epstopdf}
%\geometry{letterpaper, margin=1.5in}

\newtheorem{theorem}{Theorem}[section]
\newtheorem{lemma}[theorem]{Lemma}
\newtheorem{corollary}[theorem]{Corollary}
\newtheorem{proposition}[theorem]{Proposition}
\newtheorem{definition}[theorem]{Definition}
\newtheorem{example}[theorem]{Example}
\newtheorem{exercise}[theorem]{Exercise}
\newtheorem{question}[theorem]{Question}
\newtheorem{remark}[theorem]{Remark}
\newtheorem{alg}[theorem]{Algorithm}
%\newtheorem{answer}[proof]{Question}

\newcommand{\dx}{\,{\rm d}x}
\newcommand{\dd}{\,{\rm d}}
\newcommand{\deq}{\stackrel{\rm def}{=}}
\newcommand{\mbb}{\mathbb}
\newcommand{\mbf}{\bs}
\newcommand{\bs}{\boldsymbol}
\newcommand{\mcal}{\mathcal}
\newcommand{\mc}{\mcode}
\newcommand{\lla}{\langle}
\newcommand{\rra}{\rangle}
\newcommand{\supp}{\operatorname{supp}}
\newcommand{\range}{\operatorname{range}}
\newcommand{\PartSize}{\fontsize{0.85cm}{0.85cm}\selectfont} 
\newcommand{\mscr}{\mathscr}
\DeclarePairedDelimiter\ceil{\lceil}{\rceil}
\DeclarePairedDelimiter\floor{\lfloor}{\rfloor}

%\newcommand{\span}{\rm span}
\newcommand{\red}{\color{red}}
\newcommand{\green}{\color{green}}
\newcommand{\blue}{\color{blue}}
\newcommand{\gray}{\color{gray}}

%\DeclareMathOperator*{\curl}{curl}
\DeclareMathOperator*{\img}{img}
\DeclareMathOperator*{\spa}{span}
\DeclareMathOperator*{\diag}{diag}
\newcommand{\curl}{{\rm curl\,}}
\renewcommand{\div}{\operatorname{div}}
%\renewcommand{\grad}{\operatorname{grad}}
\newcommand{\grad}{{\rm grad\,}}
%\DeclareMathOperator*{\tr}{tr}
\DeclareMathOperator*{\rot}{rot}
\DeclareMathOperator*{\var}{Var}
\DeclareMathOperator{\rank}{rank}
\DeclareMathOperator{\dist}{dist}
%\DeclareMathOperator{\span}{span}
\newcommand{\tr}{\operatorname{tr}}
\newcommand{\dev}{\operatorname{dev}}
\newcommand{\sym}{\operatorname{sym}}
\newcommand{\skw}{\operatorname{skw}}
\newcommand{\spn}{\operatorname{spn}}
\newcommand{\mspn}{\operatorname{mspn}}
\newcommand{\vspn}{\operatorname{vspn}}
\newcommand{\mskw}{\operatorname{mskw}}
\newcommand{\vskw}{\operatorname{vskw}}
\newcommand{\defm}{\operatorname{def}}
\newcommand{\hess}{\operatorname{hess}}
\newcommand{\inc}{\operatorname{inc}}
\newcommand{\dex}{\operatorname{dex}}

\newcommand{\step}[1]{\noindent\raisebox{1.5pt}[10pt][0pt]{\tiny\framebox{$#1$}}\xspace}

%\usepackage[margins]{trackchanges}
%\renewcommand{\initialsOne}{chen}

\newcommand{\norm}[1]{\left\Vert#1\right\Vert}
\newcommand{\snorm}[1]{\left\vert#1\right\vert}
\newcommand{\vertiii}[1]{{\left\vert\kern-0.25ex\left\vert\kern-0.25ex\left\vert #1 
    \right\vert\kern-0.25ex\right\vert\kern-0.25ex\right\vert}}

\usepackage{stmaryrd}
\usepackage{scalefnt}
\usepackage{graphicx} 
\newcommand{\Oplus}{\ensuremath{\vcenter{\hbox{\scalebox{1.5}{$\oplus$}}}}}
%\DeclareMathOperator*{\prox}{Prox}
\newcommand{\prox}{\operatorname{Prox}}
%\newcommand{\Lambda}{{\rm Alt}}


\begin{document}
\title{Finite Element Complexes in Three Dimensions}
% \author{Long Chen}%
%  \address{Department of Mathematics, University of California at Irvine, Irvine, CA 92697, USA}%
%  \email{chenlong@math.uci.edu}%
%  \author{Xuehai Huang}%
%  \address{School of Mathematics, Shanghai University of Finance and Economics, Shanghai 200433, China}%
%  \email{huang.xuehai@sufe.edu.cn}%

% \thanks{The first author was supported by NSF DMS-1913080 and DMS-2012465.}
% \thanks{The second author was supported by the National Natural Science Foundation of China Projects 11771338 and 12171300, the Natural Science Foundation of Shanghai 21ZR1480500, and the Fundamental Research Funds for the Central
% Universities 2019110066.}

% \begin{abstract}
% We give a unified construction of div-conforming finite element tensors including vector div element, symmetric div matrix, traceless div matrix, and more. This is the first step towards the finite elements for extended complexes. 
% \end{abstract}
\maketitle

\tableofcontents



\section{Geometric Decompositions}
\subsection{Lagrange elements}
For the polynomial space $\mathbb P_k(T)$ with $k\geq 1$ on an $n$-dimensional simplex $T$, we have the following decomposition of Lagrange element~\cite[(2.6)]{ArnoldFalkWinther2009}
\begin{align}
\label{eq:Prdec}
\mathbb P_r(T) &= \Oplus_{\ell = 0}^n\Oplus_{f\in \Delta_{\ell}(T)} b_f\mathbb P_{r - (\ell +1)} (f).
\end{align}
The function $u\in \mathbb P_r(T)$ is uniquely determined by DoFs
\begin{equation}\label{eq:dofPr}
\int_f u \, p \dd s \quad \forall~p\in \mathbb P_{r - (\ell +1)} (f), f\in \Delta_{\ell}(T), \ell = 0,1,\ldots, n.
\end{equation}
The integral at a vertex is understood as the function value at that vertex. A proof of the unisolvence can be found in~\cite{Chen;Huang:2021Geometric}.

Introduce bubble polynomial spaces on each sub-simplex as
$$
\mathbb B_{r}(\grad; f) = b_f\mathbb P_{r - (\ell +1)} (f), \quad f\in \Delta_{\ell}(T), 0\leq \ell \leq n.
$$
It is called bubble functions as
$$
\tr_f^{\grad} u := u|_{\partial f} = 0, \quad u\in \mathbb B_{r}(\grad; f).
$$
Then \eqref{eq:Prdec} can be written as
\begin{equation}
\mathbb P_r(T) = \mathbb P_1(T)\oplus \Oplus_{\ell = 1}^n\Oplus_{f\in \Delta_{\ell}(T)} \mathbb B_{r}(\grad; f).
\end{equation}
For spaces associated to vertices, it can be changed to
$$
\spa \{\lambda_{i}^{r}, i=0, 1,\ldots, n\}.
$$

The geometric decomposition can be naturally extended to vector Lagrange elements. For $r\geq 1$, define 
\begin{equation*}
\mathbb B_{r}^n(\grad; f) := b_f\mathbb P_{r - (\ell +1)} (f)\otimes \mathbb E^n.
\end{equation*}
Clearly we have
\begin{align}\label{eq:Prvecdec}
\mathbb P_r(T; \mathbb E^n)&= \mathbb P_1(T; \mathbb E^n)\oplus \Oplus_{\ell = 1}^n\Oplus_{f\in \Delta_{\ell}(T)} \mathbb B_{r}^n(\grad; f).
\end{align}

We choose a $t-n$ coordinate to write the decomposition into tangential and normal components. When $r = 0$, define
\begin{equation*}
\mathbb T^f_0 := \left \{\sum_{i=1}^{\ell} c_i \bs t_i^f:  c_i \in\mathbb R\right \},
\quad
\mathbb{N}^f_0 := \left \{\sum_{i=1}^{n-\ell} c_i \bs n_i^f:  c_i \in \mathbb R \right \}.
\end{equation*}
Thus $\mathbb T^f_0$ is the tangent plane of $f$ and $\mathbb N^f_0$ is the normal plane. 
For $r\geq 1$, define
\begin{equation*}%\label{eq:fbubble}
\mathbb T^f_r( T) := \left \{\sum_{i=1}^{\ell} u_i^f \bs t_i^f: u_i^f \in b_f\mathbb P_{r - (\ell +1)} (f) \right \},
\end{equation*}
\begin{equation*}
\mathbb{N}^f_r( T) := \left \{\sum_{i=1}^{n-\ell} u_i^f \bs n_i^f:  u_i^f \in b_f\mathbb P_{r - (\ell +1)} (f) \right \}.
\end{equation*}
We can also write 
$$
\mathbb T^f_r(T) = b_f\mathbb P_{r - (\ell +1)} (f)\otimes \mathbb T_0^f, \quad \mathbb N^f_r( T) = b_f\mathbb P_{r - (\ell +1)} (f)\otimes \mathbb N_0^f.
$$
Note that $\mathbb T^f_r( T), \mathbb N^f_r( T)\subseteq \mathbb P_r(T;\mathbb E^n)$ for $r\geq 1$, while for $r=0$, $\mathbb T_0^f$ and $\mathbb N_0^f$ are independent of $T$. 

When $\ell = 0$, i.e., for vertices, no tangential component and for $\ell = n$, no normal component. We then write the decomposition as
\begin{align}\label{eq:Prvecdec}
\mathbb P_r(T; \mathbb E^n)=\mathbb P_1(T; \mathbb E^n) \,\oplus \Oplus_{\ell = 1}^{n}\Oplus_{f\in \Delta_{\ell}(T)} (\mathbb{T}^f_r(T)\oplus \mathbb{N}^f_r(T)).
\end{align}



%\LC{The trouble of proving the discrete inf-sup condition is from the fact $\int_{f}\phi_e \neq 0$. }

%We prove it is indeed surjective. 
%\begin{lemma}\label{lem:trdivonto}
%For integer $r\geq 1$, the mapping ${\rm tr}^{\div}: \mathbb P_r(T; \mathbb E^n) \to \mathbb P_{r}(\partial T)$ is onto. 
%\end{lemma}

% Indeed we can give a geometric decomposition of $\mathbb B_r(\div, T)$. In the sequel, $\mathbb R^{\ell}(f)$ denotes the tangent plane of $F$ which can be naturally embedded into $\mathbb R^n$ through $F$. The vector function
%$$
%b_f\mathbb P_{r - (\ell +1)} (f)\otimes \mathbb R^{\ell}(f)
%$$
%can be identified with a differential form in $\mathbb P_r\Lambda^{n-1}$ through
%$$
%\prox^{-1} ({\rm Ext}(b_f\mathbb P_{r - (\ell +1)} (f)) \otimes {\rm Ext}(\mathbb R^{\ell}(f))).
%$$
%We will stick to the simpler vector function notation. 

\subsection{Edge elements}
We use $\mathbb T_{r}^f(T), \mathbb N_{r}^f(T)$ to characterize the kernel or range of the trace operator, respectively.
Define the polynomial bubble space
$$
\mathbb B_r(\curl; T) = \ker(\tr^{\curl})\cap \mathbb P_r(T;\mathbb E^n).
$$
For $\bs u \in \mathbb B_r(\curl; T)$, only the tangential component vanishes, i.e., $\bs u\times \bs n|_{\partial T} = 0$. This will imply $\bs u$ vanishes on lower dimensional sub-simplex. 

\begin{lemma}\label{lm:curlbubbleface}
For $\bs u\in \mathbb B_r(\curl; T)$, it holds $\bs u|_f=\bs0$ for all $f\in\Delta_{\ell}(T), 0\leq \ell\leq n-2$.
\end{lemma}
\begin{proof}
It suffices to consider $f\in \Delta_{n-2}(T)$. Let $F_1, F_2\in\Delta_{n-1}(T)$ such that $f=F_1\cap F_2$. By $\tr_{F_i}^{\curl}\bs u=\bs0$ for $i=1,2$, we have
$$
(\bs u\cdot\bs t_i^f)|_f=0, \;\; (\bs u\cdot\bs n_{F_1, f})|_f=(\bs u\cdot\bs n_{F_2, f})|_f=0\quad \textrm{ for } i=1,\ldots, n-2,
$$
where $\bs n_{F_i, f}$ is a normal vector $f$ sitting on $F_i$. By $\spa \{\bs t_1^f, \ldots, \bs t_{n-2}^f, \bs n_{F_1, f}, \bs n_{F_2, f} \} = \mathbb E^n$, we acquire $\bs u|_f=\bs0$. 
\end{proof}

Obviously $\mathbb B_r^n(\grad; T)\subset \mathbb B_r(\curl; T)$. As $\tr^{\curl}$ contains the tangential component only, the normal component is also a bubble. Their sum is precisely all bubble functions.  
\begin{theorem}\label{thm:curlbubbletracespacedecomp}
For $r\geq 1$, it holds that
\begin{equation}\label{eq:curlbubbledecomp}
\mathbb B_r(\curl; T) =  \mathbb B_r^n(\grad; T) \oplus \Oplus_{F\in \Delta_{n-1}(T)} \mathbb B_r(\grad; F)\bs n_F,
\end{equation}
and
\begin{equation}\label{eq:trNrf}
{\rm tr}^{\curl} : \mathbb P_1(T)\oplus \Oplus_{\ell = 1}^{n-2}\Oplus_{f\in \Delta_{\ell}(T)} \mathbb B_r^n(\grad; f)\oplus \Oplus_{F\in \Delta_{n-1}(T)}  \mathbb{T}^F_r( T) \to  {\rm tr}^{\curl} \mathbb P_r(T; \mathbb E^n)
\end{equation} 
% \begin{align}\label{eq:trNrf}
% {\rm tr} : &\Oplus_{i = 0}^{n}\lambda_{i}\mathbb P_{r - 1} (\texttt{v}_i)\otimes \mathbb E^n \oplus \Oplus_{\ell = 1}^{n-1}\Oplus_{f\in \Delta_{\ell}(T)}  \mathbb{T}^f_r( T) \oplus \Oplus_{\ell = 0}^{n-2}\Oplus_{f\in \Delta_{\ell}(T)}  \mathbb{N}^f_r( T) \\
% &\to  {\rm tr} \mathbb P_r(T; \mathbb E^n) \notag
% \end{align}
is a bijection.
\end{theorem}
\begin{proof}
It is obvious that
$$
\mathbb B_r^n(\grad; T) \oplus \Oplus_{F\in \Delta_{n-1}(T)} \mathbb B_r(\grad; F)\bs n_F \subseteq \mathbb B_{r}(\curl,T).
$$
Then apply the trace operator to the decomposition \eqref{eq:Prvecdec} to conclude that 
%$$
%{\rm tr}^{\curl} \left( \mathbb P_1(T)\oplus \Oplus_{\ell = 1}^{n-2}\Oplus_{f\in \Delta_{\ell}(T)} \mathbb B_r^n(\grad; f)\oplus \Oplus_{F\in \Delta_{n-1}(T)}  \mathbb{T}^F_r( T) \right) =  {\rm tr}^{\curl} \mathbb P_r(T; \mathbb E^n).
%$$ 
the map $\tr^{\curl}$ in \eqref{eq:trNrf} is onto. 

Now we prove it is also injective. Take a function $\bs u\in  \mathbb P_1(T)\oplus  \Oplus_{\ell = 1}^{n-2}\Oplus_{f\in \Delta_{\ell}(T)} \mathbb B_r^n(\grad; f)\oplus \Oplus_{F\in \Delta_{n-1}(T)}  \mathbb{T}^F_r( T)$ and $\tr^{\curl} \bs u =\bs 0$. By Lemma~\ref{lm:curlbubbleface}, we can assume $\bs u=\sum\limits_{F\in \Delta_{n-1}(T)}\bs u_r^F$ with $\bs u_r^F \in \mathbb{T}^F_r(T)$.
Take $F\in \Delta_{\ell-1}(T)$. We have $\bs u|_F=\bs u_r^F|_F\in\mathbb{T}^F_r(T)$. Hence $(\bs u_r^F\cdot\bs t)|_F=(\bs u\cdot\bs t)|_F=0$ for any $\bs t\in\mathbb T_0^F$, which results in $\bs u_r^F=\bs0$. Therefore $\bs u = \bs0$.  

% As $\bs u$ contains only normal component, DoF \eqref{eq:dofvectort} will be zero. For each $f\in \Delta_{\ell}(T)$, $\mathbb{N}^f_0 = {\rm span}\{ \bs n_F, f\subset F\}$ and thus $\bs n_F\cdot \bs u = 0$ implies DoF \eqref{eq:dofvectorn} also vanishes. Therefore $\bs u = 0$. 
% Now we prove it is also injective. Take a function $\bs u\in \Oplus_{\ell = 0}^{n-1}\Oplus_{f\in \Delta_{\ell}(T)}  \mathbb{N}^f_r( T)$ and $\tr \bs u = 0$. As $\bs u$ contains only normal component, DoF \eqref{eq:dofvectort} will be zero. For each $f\in \Delta_{\ell}(T)$, $\mathbb{N}^f_0 = {\rm span}\{ \bs n_F, f\subset F\}$ and thus $\bs n_F\cdot \bs u = 0$ implies DoF \eqref{eq:dofvectorn} also vanishes. Therefore $\bs u = 0$. 

Once we have proved the map $\tr$ in \eqref{eq:trNrf} is bijection, we conclude \eqref{eq:curlbubbledecomp} from the decomposition \eqref{eq:Prvecdec}. 
%$ {\rm tr} \mathbb P_r\Lambda^{n-1}(T) = \sum_{\ell = 0}^{n-1}\sum_{f\in \Delta_{\ell}(T)} {\rm tr}  \left (\mathbb{N}^f_r( T)\right )$.
%\eqref{eq:trNrf}. \mnote{How to prove direct sum?}
% To prove ``=", we only need to count the dimensions. 
%Dimension count 
%$$
%\dim \mathbb B_r(\div, T)
%= \dim \mathbb P_r(T;\mathbb R^n) - \dim {\rm tr}^{\div}(\mathbb P_r(T; \mathbb E^n)). 
%$$
%We then do the dimension count which is straight forward in view of Fig. \ref{fig:Hdivdec}. 
\end{proof}

%
%Define $\mathbb B_r(\grad; f) =  b_f\mathbb P_{r - (\ell +1)} (f)$ for $f\in\Delta_{\ell}(T)$ and $\ell=0,\ldots, n$, and
Restricted to a sub-simplex $f\in \Delta_{\ell}(T)$,
$$ 
\mathbb B_{r}^{\ell}(\grad; f) := \mathbb T^f_r(T) = b_f\mathbb P_{r - (\ell +1)} (f)\otimes \mathbb T_0^f
$$ 
is a space of vectors on the tangential space with vanishing trace $\tr^{\grad}$ on $\partial f$. Similar to the previous case, the normal component of the edge $e\in \partial f$ will also contribute to the bubble functions. For  $f\in\Delta_{\ell}(T), \ell=2,\ldots, n-1$, apply Theorem \ref{thm:curlbubbletracespacedecomp} to $f$, we have
$$
\mathbb B_r(\curl; f) = \mathbb B_{r}^{\ell}(\grad; f) \oplus \Oplus_{e\in \partial f}\mathbb B_{r}(\grad; e) \boldsymbol{n}_{f,e}.
$$
For $\ell = 1$, it is no longer a bubble function. For $\texttt{v}\in\Delta_0(T)$, $\boldsymbol{n}_{e,\texttt{v}}$ is $\boldsymbol{t}_{e}$ or $-\boldsymbol{t}_{e}$ where the sign depends on the orientation. Then
$$
\mathbb B_{r}(\grad; e) \bs t_e \oplus \Oplus_{\texttt{v}\in \partial e}\mathbb B_{r}(\grad; \texttt{v}) \boldsymbol{n}_{e,\texttt{v}} = \mathbb P_{r}(e).
$$

\mnote{ When $e\in\Delta_0(T)$, $\mathbb B_{r}(\grad; e)$ is vertex bubble space, which vanish at other vertices.}
\LC{Even $\mathbb B_{r}(\grad; e)$ can't be called the bubble space as no boundary of a vertex. Vanishing at other vertices is not the definition of bubble space.}
\begin{lemma}
For $r\geq 1$, we have
\begin{equation}\label{eq:curlpolynomialdecomp}
\mathbb P_r(T; \mathbb E^n) = \mathbb P_1(T; \mathbb E^n) \oplus \Oplus_{e\in \Delta_1(T)}\mathbb B_r(\grad; e)\bs t_e \oplus \Oplus_{\ell=2}^n\Oplus_{f\in\Delta_{\ell}(T)}\mathbb B_r(\curl; f).
\end{equation}
\begin{equation}\label{eq:curlpolynomialdecomp2}
\mathbb P_r(T; \mathbb E^n) = \Oplus_{e\in \Delta_1(T)}\mathbb P_r(e)\bs t_e \oplus \Oplus_{\ell=2}^n\Oplus_{f\in\Delta_{\ell}(T)}\mathbb B_r(\curl; f).
\end{equation}
\begin{equation}\label{eq:curlpolynomialdecomp3}
\mathbb P_r(T; \mathbb E^n) = \Oplus_{e\in \Delta_1(T)}\mathbb P_0(e)\bs t_e \oplus \Oplus_{e\in \Delta_1(T)}(\mathbb P_r(e)/\mathbb R) \bs t_e \oplus \Oplus_{\ell=2}^n\Oplus_{f\in\Delta_{\ell}(T)}\mathbb B_r(\curl; f).
\end{equation}
\end{lemma}
\begin{proof}
It is a re-arrangement of components in the decomposition \eqref{eq:Prvecdec}. 
For $e\in\Delta_{\ell-1}(T)$, the face normal vectors $\{\bs n_{f,e}: f\in \Delta_{\ell}(T), e\subseteq f\}$ forms a basis of $\mathbb N^e_0$. So we have
$$
\mathbb{N}^e_r(T)=\Oplus_{f\in \Delta_{\ell}(T), e\subseteq f}\mathbb B_r(\grad; e)\boldsymbol{n}_{f,e}.
$$
Then shift the normal component one level up to get the decomposition \eqref{eq:curlpolynomialdecomp}.

We then distribute the $n$-component of vector function value at the vertices to the $n$ edges connected to this vertex. 
$$
\mathbb P_1(T; \mathbb E^n) \oplus \Oplus_{e\in \Delta_1(T)}\mathbb B_r(\grad; e)\bs t_e = \Oplus_{e\in \Delta_1(T)} \left (\mathbb P_1(e)\bs t_e \oplus \mathbb B_r(\grad; e)\bs t_e\right).
$$
It is easy to see that $\mathbb P_r(e)=\mathbb P_1(e)\oplus \mathbb B_r(\grad; e)$. Thus \eqref{eq:curlpolynomialdecomp2} holds.
\end{proof}

\LC{$\Oplus_{e\in \Delta_1(T)}\mathbb P_0(e)\bs t_e$ is single out as this is the Whitney form.}

% \subsection{N\'ed\'elec element}
We shall derive the second kind N\'ed\'elec element~\cite{Nedelec1986} from a special $t-n$ basis. To emphasize the dependence on edges, we shall use $e$ instead $f$. 
% Given an $f\in \Delta_{\ell}(T)$ we choose $\{\bs n_F, f\subseteq F, F\in \Delta_{n-1}(T)\}$ as the basis for its normal plane and an arbitrary basis for the tangent plane. 

\begin{lemma}[Local N\'ed\'elec element]\label{lm:localNedelec}
The shape function space $\mathbb P_r(T; \mathbb E^n)$ is uniquely determined by the DoFs
\begin{align}
\label{eq:vecbdDoF0}
\bs v\cdot \bs t_{e}(\texttt{v}_i), & \quad e\in \Delta_{1}(T), \texttt{v}_i\in e,  i=1,\ldots, n+1, \\
\label{eq:vecbdDoF1}
\int_e (\bs v\cdot \bs t_i^e)\ p \dd s, &\quad  p\in \mathbb P_{r - (\ell +1)} (e), e\in \Delta_{\ell}(T),\\
& \quad  i=1,\ldots, \ell, \ell = 1,\ldots, n-1, \notag\\
\label{eq:vecbdDoF2}
\int_e (\bs v\cdot \bs n_{f,e})\ p \dd s, &\quad  p\in \mathbb P_{r - (\ell +1)} (e),e\in \Delta_{\ell}(T), \\
& \quad f\in \Delta_{\ell+1}(T), e\subseteq f, \ell = 1,\ldots, n-2, \notag\\
\int_T \bs v \cdot \bs p \dx&\quad \bs p\in \mathbb B_r(\curl, T). \label{eq:bubbleDoF}
\end{align}
\end{lemma}
\begin{proof}
First of all, by the geometric decomposition of $ \mathbb P_r(T; \mathbb E^n)$ \eqref{eq:Prvecdec} and Theorem~\ref{thm:curlbubbletracespacedecomp}, %$\mathbb B_r(\div; T)$ \eqref{eq:divbubbledecomp}, 
the number of DoFs is equal to the dimension of the shape function space. The normal component of $F\in \Delta_{n-1}(T)$ is merged into the bubble DoF \eqref{eq:bubbleDoF}. 

%When $\ell = 0$, the DoF \eqref{eq:vecbdDoF2} is $\{(\bs v\cdot \bs t_{e})(\texttt{v}), e\in \Delta_{1}(T), \texttt{v}\in e\}$ for $\texttt{v}\in \Delta_{0}(T)$. 

Assume $\bs v\in\mathbb P_r(T; \mathbb E^n)$ and all the DoFs \eqref{eq:vecbdDoF1}-\eqref{eq:bubbleDoF} vanish. Since $\{\bs t_{e}, e\in \Delta_{1}(T), \texttt{v}\in e\}$ is a basis of $\mathbb E^n$, $\{(\bs v\cdot \bs t_{e})(\texttt{v}), e\in \Delta_{1}(T), \texttt{v}\in e\}$ will determine the vector $\bs v(\texttt{v})$. Thus vanishing \eqref{eq:vecbdDoF0} implies $\bs v$ is zero at vertices. In general, $\{\bs n_{f,e}: f\in \Delta_{\ell+1}(T), e\subseteq f\}$ forms a basis of $\mathbb N^e_0$. DoF \eqref{eq:vecbdDoF2} is equivalent to 
\begin{align*}
\int_e (\bs v\cdot \bs n_i^e)\ p \dd s, &\quad  p\in \mathbb P_{r - (\ell +1)} (e), e\in \Delta_{\ell}(T), i=1,\ldots, n-\ell, \; \ell = 0,\ldots, n-2,
\end{align*}
which together with vanishing DoF \eqref{eq:vecbdDoF1} implies
\begin{align*}
\int_f \bs v\cdot \bs p \dd s = 0, &\quad  \bs p\in \mathbb P_{r - (\ell +1)} (f;\mathbb R^n), f\in \Delta_{\ell}(T),  \ell = 0,\ldots, n-2.
\end{align*}
It follows from the uni-solvence of Lagrange element that $\bs v|_f=\bs0$ for each $f\in\Delta_{n-2}(T)$.
Then by the vanishing DoF \eqref{eq:vecbdDoF1} with $\ell=n-1$, we get $\tr^{\curl}\bs v=\bs0$, i.e. $\bs v\in\mathbb B_r(\curl, T)$. Finally $\bs v=\bs0$ is an immediate result of the vanishing DoF \eqref{eq:bubbleDoF}.
\end{proof}


\begin{lemma}[N\'ed\'elec space]\label{lm:nedelec}
The following {\rm DoFs}
\begin{align}
\label{eq:vecbdDoF0Th}
\bs v\cdot \bs t_{e}(\texttt{v}_i), & \quad e\in \Delta_{1}(T), \texttt{v}_i\in e,  i=1,\ldots, n+1, \\
\label{eq:vecbdDoF1Th}
\int_e (\bs v\cdot \bs t_i^e)\ p \dd s, &\quad  p\in \mathbb P_{r - (\ell +1)} (e), e\in \Delta_{\ell}(\mathcal T_h),\\
& \quad  i=1,\ldots, \ell, \ell = 1,\ldots, n-1, \notag\\
\label{eq:vecbdDoF2Th}
\int_e (\bs v\cdot \bs n_{f,e})\ p \dd s, &\quad  e\in \Delta_{\ell}(\mathcal T_h), f\in \Delta_{\ell+1}(\mathcal T_h), e\subseteq f, \\
& \quad p\in \mathbb P_{r - (\ell +1)} (e), \ell = 0,1,\ldots, n-2, \notag\\
\int_T \bs v \cdot \bs p \dx&\quad T\in\mathcal T_h, \bs p\in \mathbb B_r(\curl, T). \label{eq:bubbleDoFTh}
\end{align}
defines a curl-conforming space $V_h=\{\bs v_h\in H(\curl; \Omega): \bs v_h|_T\in\mathbb P_r(T; \mathbb E^n) \, \forall T\in\mathcal T_h\}$.
%are equivalent to the DoFs
%\begin{equation}\label{eq:vecbdDoFtheorem}
%\int_F \bs v\cdot \bs n_F p \dd s, \quad p\in \mathbb P_r(F), F\in \Delta_{n-1}(\mathcal T_h),
%\end{equation}
%and 
%\begin{equation}\label{eq:intDoF}
%\int_T \bs v\cdot \bs p \dx\quad \bs p\in \grad \mathbb P_{k-1}(T) \oplus \mathbb P_{k-2}(T;\mathbb K)\boldsymbol x,  T\in \mathcal T_h.
%\end{equation}
%Therefore the corresponding $V_h$ is the BDM space and the discrete inf-sup condition holds.
\end{lemma}
\begin{proof}
On each element $T$, DoFs \eqref{eq:vecbdDoF1Th}-\eqref{eq:bubbleDoFTh} will determine a function in $\mathbb P_r(T; \mathbb E^n)$ by Lemma \ref{lm:localNedelec}. DoFs \eqref{eq:vecbdDoF1Th}-\eqref{eq:vecbdDoF2Th} will determine the trace $\tr^{\curl}\bs v$ on $F$ independent of the element containing $F$ and thus the function is $H(\curl;\Omega)$-conforming. 
%In view of \eqref{eq:BDMface}, the obtained space $V_h$ is the BDM space. 
\end{proof}

We write $C^{0}(\Delta_0(T))$ to denote function space which is single valued at vertices. Based on the decomposition \eqref{eq:curlpolynomialdecomp}, we can define an $H(\curl;\Omega)$-conforming finite element space with vertex continuity \cite{ChristiansenHuHu2018}. 
\LC{Formulate as a lemma and prove it.}
\begin{align*}
\bs v(\texttt{v}), &\quad  \texttt{v}\in \Delta_{0}(\mathcal T_h),\\
\int_e (\bs v\cdot \bs t_i^e)\ p \dd s, &\quad  p\in \mathbb P_{r - (\ell +1)} (e), e\in \Delta_{\ell}(\mathcal T_h),\\
& \quad  i=1,\ldots, \ell, \ell = 1,\ldots, n-1, \notag\\
\int_e (\bs v\cdot \bs n_{f,e})\ p \dd s, &\quad  e\in \Delta_{\ell}(\mathcal T_h), f\in \Delta_{\ell+1}(\mathcal T_h), e\subseteq f, \\
& \quad p\in \mathbb P_{r - (\ell +1)} (e), \ell = 1,\ldots, n-2, \notag\\
\int_T \bs v \cdot \bs p \dx&\quad T\in\mathcal T_h, \bs p\in \mathbb B_r(\curl, T)
\end{align*}
defines a curl-conforming finite element space being continuous at vertex in $\Delta_{0}(\mathcal T_h)$.

\LC{More on the motivation of nodal edge element and comment on our decomposition of shape function and DoFs for the original Nedelec element enable the use of Lagrange basis.}
\subsection{Face element}
%\LC{Decomposition will stop on $F$ as the trace of $n-1$-form is in $L^2(F)$. No further trace.}

Define the polynomial bubble space
$$
\mathbb B_r(\div; T) = \ker(\tr^{\div})\cap \mathbb P_r(T;\mathbb E^n).
$$
By Theorem~3.2 in \cite{Chen;Huang:2021Geometric}, it holds
$$
\mathbb B_r(\div; T) = \Oplus_{\ell=1}^n\Oplus_{f\in\Delta_{\ell}(T)}\mathbb B_{r}^{\ell}(\grad; f).
$$
This implies $\bs u(\texttt{v})=\bs0$ for $\texttt{v}\in\Delta_0(T)$ for $\bs u\in\mathbb B_r(\div; T)$.

%For $F\in\Delta_{n-1}(T)$, \Oplus_{F\in\Delta_{n-1}(T)}
%$$
%\mathbb B_r(\div; F) = \mathbb P_1(F)\bs n_F\oplus \Oplus_{\ell=1}^{n-1}\Oplus_{f\in\Delta_{\ell}(T), f\subseteq F}\mathbb B_{r}(\grad; f)\bs n_F.
%$$
\begin{lemma}
For $r\geq 1$, we have
\begin{equation}\label{eq:divpolynomialdecomp}
\mathbb P_r(T; \mathbb E^n) = \mathbb P_1(T; \mathbb E^n) \oplus \Oplus_{\ell=1}^{n-1}\Oplus_{f\in\Delta_{\ell}(T)}\mathbb B_{r}(\grad; f)\bs n_F\oplus \mathbb B_r(\div; T),
\end{equation}
\begin{equation}\label{eq:divpolynomialdecomp2}
\mathbb P_r(T; \mathbb E^n) = \Oplus_{F\in\Delta_{n-1}(T)} \Oplus_{\ell = 0}^{n-1} \Oplus_{f\in \Delta_{\ell}(F)}b_f\mathbb P_{r-(\ell + 1)}(f)\bs n_F\oplus \mathbb B_r(\div; T).
\end{equation}
\end{lemma}
\begin{proof}
The first decomposition \eqref{eq:divpolynomialdecomp} is a rearrangement of \eqref{eq:Prvecdec} by merging the tangential component into the bubble space. 

For $f\in\Delta_{\ell}(T)$, by the face $\{\bs n_{F}: F\in \Delta_{n-1}(T), f\subseteq F\}$ forms a basis of $\mathbb N^f_0$,  we have
$$
\mathbb{N}^f_r(T)=\Oplus_{F\in \Delta_{n-1}(T), f\subseteq F}\mathbb B_{r}(\grad; f)\bs n_F.
$$
Then \eqref{eq:divpolynomialdecomp} follows from \eqref{eq:Prvecdec}.

For $i=0,\ldots, n$ and $F\in\Delta_{n-1}(T)$, apparently $\lambda_i\bs n_F\in P_1(F)\bs n_F$ if $i\in F$ and  $\lambda_i\bs n_F\in \sum_{f\in\Delta_{n-1}(T), f\neq F}P_1(f)\bs n_f$ if $i\not\in F$. Hence $\mathbb P_1(T; \mathbb E^n)=\sum_{F\in\Delta_{n-1}(T)}\mathbb P_1(F)\bs n_F$, which together with the matching dimensions yields $\mathbb P_1(T; \mathbb E^n)=\Oplus_{F\in\Delta_{n-1}(T)}\mathbb P_1(F)\bs n_F$.
Thus \eqref{eq:divpolynomialdecomp2} holds from \eqref{eq:divpolynomialdecomp}.
\end{proof}


%\LC{Write out BDM and Sternberg element based on similar decomposition.}

\begin{lemma}[BDM space]\label{lm:bdm}
The following {\rm DoFs}
\begin{align}
\label{eq:divbdDoF0}
(\bs v\cdot \bs n_F)(\texttt{v}), &\quad  \texttt{v}\in \Delta_{0}(\mathcal T_h), F\in \Delta_{n-1}(T),\\
\label{eq:divbdDoF}
\int_f (\bs v\cdot \bs n_F)\ p \dd s, &\quad  f\in \Delta_{\ell}(T), F\in \Delta_{n-1}(T), f\subseteq F, \\
& \quad p\in \mathbb P_{r - (\ell +1)} (f), \ell = 1,\ldots, n-1, \notag\\
\int_T \bs v \cdot \bs p \dx&\quad \bs p\in \mathbb B_r(\div, T). \label{eq:divbubbleDoF}
\end{align}
defines a div-conforming space $V_h=\{\bs v_h\in H(\div; \Omega): \bs v_h|_T\in\mathbb P_r(T; \mathbb E^n) \, \forall T\in\mathcal T_h\}$.
\end{lemma}

Based on the decomposition \eqref{eq:divpolynomialdecomp}, the DoFs
\begin{align*}
\bs v(\texttt{v}), &\quad  \texttt{v}\in \Delta_{0}(\mathcal T_h),\\
\int_f (\bs v\cdot \bs n_F)\ p \dd s, &\quad  f\in \Delta_{\ell}(T), F\in \Delta_{n-1}(T), f\subseteq F, \\
& \quad p\in \mathbb P_{r - (\ell +1)} (f), \ell = 1,\ldots, n-1, \notag\\
\int_T \bs v \cdot \bs p \dx&\quad \bs p\in \mathbb B_r(\div, T). 
\end{align*}
defines a div-conforming finite element space being continuous at vertex in $\Delta_{0}(\mathcal T_h)$, i.e. Stenberg element \cite{Stenberg2010}. Another variant is: for $f\in \Delta_{\ell}(\mathcal T_h), \ell =0,\ldots, n-1$, choose a fixed normal basis. Then the obtained element is normal continuous. See  \cite{Chen;Huang:2021Geometric} for details.


\section{Finite Element de Rham Complexes}
We have the following decomposition in terms of Whitney form and  bubble functions.
\begin{align*}
\mathbb V^{\grad}_r(\mathcal T_h) &= \mathbb P_1(\mathcal T_h) \oplus \Oplus_{\ell = 1}^3\Oplus_{f\in \Delta_{\ell}(\mathcal T_h)} \mathbb B_{r}(\grad; f),\\
\mathbb V^{\curl}_r(\mathcal T_h) &= {\rm ND}_0 \oplus \Oplus_{e\in \Delta_1(\mathcal T_h)}(\mathbb P_r(e)/\mathbb R) \bs t_e \oplus \Oplus_{\ell = 2}^3\Oplus_{f\in \Delta_{\ell}(\mathcal T_h)} \mathbb B_{r}(\curl; f),\\
\mathbb V^{\div}_r(\mathcal T_h) &= {\rm RT}_0\oplus \Oplus_{F\in \Delta_2(\mathcal T_h)}\left((\mathbb P_{r}(F)/\mathbb R)\bs n_F\right) \oplus \Oplus_{T\in \Delta_{3}(\mathcal T_h)} \mathbb B_{r}(\div; T),\\
\mathbb V^{L_2}_r(\mathcal T_h) &= \mathbb P_0(\mathcal T_h) \oplus \Oplus_{T\in \Delta_{\ell}(\mathcal T_h)} \mathbb P_r(T)/\mathbb R.
\end{align*}
%\LC{Write as a table to see the pattern more clearly.  See my iPad notes.} \mnote{ Only tangential parts are involved in the sub-simplex complex, no normal derivative part}
We first collect two trace complexes for de Rham complex
\begin{equation}\label{eq:tracecomplex1}
\begin{array}{c}
\xymatrix{
   \phi \ar[d]^{id} \ar[r]^-{\nabla}
& \bs u \ar[d]^{\bs n \times}   \ar[r]^-{\nabla \times} 
& \bs v \ar[d]^{\bs n\cdot} \ar[r]^{\nabla\cdot} 
& p \\
 \phi \ar[r]^{\nabla_F^{\bot}}
& \bs n \times \bs u   \ar[r]^{\nabla_F\cdot} 
& \bs n\cdot \bs v \ar[r]^{}& 0    }
\end{array},
\end{equation}
and its rotation
\begin{equation}\label{eq:tracecomplex1}
\begin{array}{c}
\xymatrix{
   \phi \ar[d]^{id} \ar[r]^-{\nabla}
& \bs u \ar[d]^{\Pi_F}   \ar[r]^-{\nabla \times} 
& \bs v \ar[d]^{\bs n\cdot} \ar[r]^{\nabla\cdot} 
& p \\
 \phi \ar[r]^{\nabla_F}
& \Pi_F \bs u   \ar[r]^{\nabla_F^{\bot}\cdot} 
& \bs n\cdot \bs v \ar[r]^{}& 0    }
\end{array},
\end{equation}
Those two diagrams are commutative. The trace complex from 2D to 1D is straight forward. For each trace complex, we look at the bubble function spaces which forms an exact sequence. 

Each column forms an exact sequence.
\begin{figure}[htbp]
\begin{center}
\includegraphics[width=3.2in]{figures/bubbledec.png}
\caption{Exact sequences of bubble spaces.}
\label{default}
\end{center}
\end{figure}

% Let
% $$
% \mathbb B_r(\curl; f) =  \mathbb T^f_r(T) \oplus \Oplus_{e\in \partial f}b_e\mathbb P_{r - \ell} (e)\boldsymbol{n}_{f,e}\quad \textrm{ for } f\in\Delta_{\ell}(T), \ell=1,\ldots, n-1.
% $$
% \begin{equation*}%\label{eq:divbubbledecomp}
% \mathbb B_r(\div; T) =  \left(b_T\mathbb P_{r - 4} (T)\otimes \mathbb E^3\right) \oplus \Oplus_{f\in \Delta_{2}(T)} \mathbb T^f_r(T)\oplus \Oplus_{e\in \Delta_{1}(T)} \mathbb T^e_r(T),
% \end{equation*}
% \begin{equation*}%\label{eq:divbubbledecomp}
% \mathbb B_r(\div; f) =  \mathbb N^f_r(T)\oplus \Oplus_{e\in \Delta_{1}(f)} b_e\mathbb P_{r - 2} (e)\boldsymbol{n}_{f}\oplus \Oplus_{i=0}^3\textrm{span}\{ \lambda_i^r\boldsymbol{n}_{f}\}.
% \end{equation*}


\begin{lemma}
The bubble complex
\begin{align*}
% \resizebox{1.0\hsize}{!}{$
0\xrightarrow{\subset} \mathbb B_{r+1}(\grad; T)\xrightarrow{\grad}\mathbb B_r(\curl; T)\xrightarrow{\curl} \mathbb B_{r-1}(\div; T) \xrightarrow{\div} \mathbb P_{r-2}(T)/\mathbb R\xrightarrow{}0
% $}
\end{align*}
is exact. On each face $F\in \Delta_2(\mathcal T_h)$, 
\begin{align*}
% \resizebox{1.0\hsize}{!}{$
0\xrightarrow{\subset} \mathbb B_{r+1}(\grad; F)\xrightarrow{\grad_F}\mathbb B_r(\curl; F)\xrightarrow{\curl_F} \mathbb P_{r-1}(F)/\mathbb R\xrightarrow{}0
% $}
\end{align*}
is exact. On each edge $e\in \Delta_1(\mathcal T_h)$
\begin{align*}
% \resizebox{1.0\hsize}{!}{$
0\xrightarrow{\subset} \mathbb B_{r+1}(\grad; e)\xrightarrow{\grad_e} \mathbb P_{r}(e)/\mathbb R\xrightarrow{}0
% $}
\end{align*}
Finally we have the exact sequence of Whitney forms
\begin{align*}
% \resizebox{1.0\hsize}{!}{$
\mathbb R\xrightarrow{\subset} \mathbb P_1(\grad; \mathcal T_h)\xrightarrow{\grad} {\rm ND}_0\xrightarrow{\curl}{\rm RT}_0\xrightarrow{\div} \mathbb P_{r-2} (\mathcal T_h)\xrightarrow{}0
% $}
\end{align*}
\end{lemma}

\begin{theorem}
\begin{align*}
% \resizebox{1.0\hsize}{!}{$
\mathbb R\xrightarrow{\subset} \mathbb V^{\grad}_{r+1}(\mathcal T_h)\xrightarrow{\grad}\mathbb V^{\curl}_r(\mathcal T_h) \xrightarrow{\curl} \mathbb V^{\div}_{r-1}(\mathcal T_h) \xrightarrow{\div}\mathbb V^{L_2}_{r-2}(\mathcal T_h)\xrightarrow{}0
% $}
\end{align*} 
\end{theorem}
\begin{proof}
As all some finite element spaces are conforming, it forms a complex. The dimension will match as in the decomposition the trace forms a complex. 
\LC{Check Finite Element System.}
\end{proof}

\section{Smooth finite element de Rham complex in three dimensions}
\subsection{Nodal finite element complex}
Recall the finite element complex in \cite[Section 2.3]{ChristiansenHuHu2018}.
\begin{align*}
% \resizebox{1.0\hsize}{!}{$
\mathbb R\xrightarrow{\subset} \mathbb V^{\grad}_{{\rm H},r+1}(\mathcal T_h)\xrightarrow{\grad}\mathbb V^{\curl}_{{\rm CHH}, r}(\mathcal T_h) \xrightarrow{\curl} \mathbb V^{\div}_{r-1}(\mathcal T_h) \xrightarrow{\div}\mathbb V^{L_2}_{r-2}(\mathcal T_h)\xrightarrow{}0
% $}
\end{align*} 
Here $\mathbb V^{\grad}_{{\rm H},r+1}(\mathcal T_h)$ is the Hermite finite element in 3D with polynomial degree $r+1$, and $\mathbb V^{\curl}_{{\rm CHH}, r}(\mathcal T_h)$ is the CHH $H$ (curl) space \cite{ChristiansenHuHu2018}, which is tangentially continuous on edges and faces and $C^{0}$ continuous at the vertices. And the next is the standard BDM element; see also \S 1.2. 

%\LC{Recall the proof.}

We present a proof based on the decomposition. 
\begin{align*}
\mathbb V^{\grad}_{{\rm H}, r}(\mathcal T_h) &= \{ Du(\texttt{v})\}^* \oplus \mathbb P_1\oplus \Oplus_e b_e^2 \mathbb P_{r-4}(e)\Oplus_{\ell = 2}^3\Oplus_{f\in \Delta_{\ell}(\mathcal T_h)} \mathbb B_{r}(\grad; f),\\
\mathbb V^{\curl}_r(\mathcal T_h) &= \{\bs u(\texttt{v})\}^* \oplus {\rm ND}_0 \oplus \Oplus_{e\in \Delta_1(\mathcal T_h)}(b_e\mathbb P_{r-2}(e)/\mathbb R) \bs t_e \oplus \Oplus_{\ell = 2}^3\Oplus_{f\in \Delta_{\ell}(\mathcal T_h)} \mathbb B_{r}(\curl; f),\\
\mathbb V^{\div}_r(\mathcal T_h) &= {\rm RT}_0\oplus \Oplus_{F\in \Delta_2(\mathcal T_h)}\left((\mathbb P_{r}(F)/\mathbb R)\bs n_F\right) \oplus \Oplus_{T\in \Delta_{3}(\mathcal T_h)} \mathbb B_{r}(\div; T),\\
\mathbb V^{L_2}_r(\mathcal T_h) &= \mathbb P_0(\mathcal T_h) \oplus \Oplus_{T\in \Delta_{\ell}(\mathcal T_h)} \mathbb P_r(T)/\mathbb R.
\end{align*}
So the element bubble complex and the face bubble complex remain the same. 
We use $\{u(\texttt{v}), Du(\texttt{v})\}^*$ to denote the dual space of the DoFs. 
Indeed $\mathbb P_3 = \{u(\texttt{v}), Du(\texttt{v})\}^*$ and $\mathbb P_1 = \{u(\texttt{v})\}^*$ and thus $\{Du(\texttt{v})\}^* = \mathbb P_3\backslash \mathbb P_1$, i.e. function in Hermite finite element space but vanishing at vertices. 

One more exact sequence is on the vertex $ \{ Du(\texttt{v})\}^* \to  \{ \bs u(\texttt{v})\}^*$ which is the jet used in \cite{ChristiansenHuHu2018}. The identity mapping is given by $\partial_i u$ to $\bs u_i$. 

\subsection{Modified nodal finite element complex}
Assume $r\geq 3$.
\subsubsection{A div conforming element}
We shall revise the Stenberg element in \cite{Stenberg2010} using the idea proposed in \cite{Chen;Huang:2020Discrete}. The motivation is have the inf-sup condition for $r\geq 2$
Take the space of shape functions as $\mathbb{P}_{r}(T;\mathbb R^3)$. The degrees of freedom are
\begin{align}
\label{eq:divdof1}
\bs u(\texttt{v}_i),& \quad i=0,\ldots, 3, \\
\label{eq:divdof2}
\int_F (\bs u\cdot \bs n_F)\ q \dd s, &\quad q\in\mathbb P_{1}(F)\oplus\mathbb P_{r,2}^{\perp}(F), F\in\Delta_{2}(T),\\
\label{eq:divdof3}
\int_T \bs u\cdot \bs q \dd x, &\quad \bs q\in\nabla\mathbb P_{r-1}(T)\oplus\bs x\times\mathbb P_{r-2}(T;\mathbb R^3), \\
& \LC{\text{ we can still change to bubble functions}}
\end{align}
where $\mathbb P_{r-1}^0(F)$ is a subspace of $\mathbb P_{r-1}(F)$ with vanishing function values at vertices.

The degree of freedom \eqref{eq:divdof2} is inspired by \cite{Chen;Huang:2020Discrete}, which is different from those in \cite{Stenberg2010}.
\LC{include the unisovlence}.

Define
\begin{align*}
V_r^{\div}(\mathcal T_h):=\{\bs v_h\in \bs H(\div, \Omega):&\, \bs v_h|_T\in\mathbb P_{r-1}(T;\mathbb R^3) \textrm{ for each } T\in\mathcal T_h, \\
&\textrm{ all the DoFs \eqref{eq:divdof1}-\eqref{eq:divdof2} are single-valued}\},
\end{align*}

\LC{inf-sup condition.}

\LC{Dimension change compare to BDM space. $3N_v -3 N_f$}

\subsubsection{A curl conforming element}
Take the space of shape functions as $\mathbb{P}_{r}(T;\mathbb R^3)$. The degrees of freedom are
\begin{align}
\label{eq:curldof1}
\bs u(\texttt{v}_i), \curl\bs u(\texttt{v}_i),& \quad i=0,\ldots, 3, \\
\label{eq:curldof2}
\int_e (\bs u\cdot \bs t)\ q \dd s, &\quad q\in\mathbb P_{r-2}(e), e\in\Delta_{1}(T),\\
\label{eq:curldof3}
\int_F (\bs u\times \bs n)\cdot\ \bs q \dd s, &\quad q\in \mathbb P_{0}(F;\mathbb R^2)\oplus\nabla_F\mathbb P_{r-1,2}^{\perp}(F)\oplus \bs x^{\perp}\mathbb P_{r-2}(F), F\in\Delta_{2}(T),\\
\label{eq:curldof4}
\int_T \bs u\cdot \bs q \dd x, &\quad \bs q\in\curl\mathbb P_{r-2}(T;\mathbb R^3)\oplus\bs x\mathbb P_{r-3}(T).
\end{align}
\LC{Change to bubble function point of view for the last one.}
For the original edge element, the face moment is
$$
\int_F (\bs u\cdot \bs t_i) p\quad p\in \mathbb P_{r-3}(F) \iff \int_F (\bs u\times \bs n) \cdot \bs p\quad \bs p\in \mathbb P_{r-3}(F; \mathbb R^2).
$$
\LC{More explanation on the face moment change.} 
%It is a different DoF for the bubble space $\mathbb B_r(\curl, F)$. The space decomposition still holds. 

\LC{Dimension change compare to CHH element. Still $3N_v -3 N_f$. Right?}

So if the first complex is exact, the constraint will reduce the matched dimension and thus no need to count the dimension. 

\begin{lemma}
The degrees of freedom \eqref{eq:curldof1}-\eqref{eq:curldof4} are uni-solvent for $\mathbb{P}_{r}(T;\mathbb R^3)$.
\end{lemma}
\begin{proof}
The number of DoFs \eqref{eq:curldof1}-\eqref{eq:curldof4} is 
$$
24+6(r-1)+4\left({r+1\choose2}-4+{r\choose2}\right)+\frac{1}{6}(r-2)(r-1)(2r+3)+{r\choose3}=3{r+3\choose3},
$$
which equals to $\dim\mathbb{P}_{r}(T;\mathbb R^3)$.

Assume $\bs u\in\mathbb{P}_{r}(T;\mathbb R^3)$ and all the DoFs \eqref{eq:curldof1}-\eqref{eq:curldof4} vanish. By the vanishing \eqref{eq:curldof1}-\eqref{eq:curldof2}, we have $\bs u\cdot\bs t|_e=0$ for each $e\in\Delta_1(T)$.
Then it follows from the vanishing \eqref{eq:curldof3} that 
$$
((\curl\bs u)\cdot\bs n, q)_F=(\div_F(\bs u\times\bs n), \bs q)_F=-(\bs u\times\bs n, \nabla_F q)_F=0
$$
for all $q\in\mathbb P_{1}(F)\oplus\mathbb P_{r-1,2}^{\perp}(F), F\in\Delta_{2}(T)$. Hence $\div_F(\bs u\times\bs n)|_F=0$. This combined with the last part of the vanishing \eqref{eq:curldof3} and the uni-solvence of BDM element in two dimensions implies $(\bs u\times\bs n)|_F=0$. Finally we conclude $\bs u=\bs0$ from the vanishing \eqref{eq:curldof4} and the uni-solvence of the second kind N\'ed\'elec element.
\end{proof}

Let
\begin{align*}
V_r^{\curl}(\mathcal T_h):=\{\bs v_h\in \bs H(\curl, \Omega):&\, \bs v_h|_T\in\mathbb P_{r}(T;\mathbb R^3) \textrm{ for each } T\in\mathcal T_h, \\
&\textrm{ all the DoFs \eqref{eq:curldof1}-\eqref{eq:curldof3} are single-valued}\},
\end{align*}

\subsubsection{Finite element complex}
Introduce Hermite element space
\begin{align*}
V_{{\rm H}, r}^{\grad}(\mathcal T_h):=\{v_h\in H^1(\Omega):&\, v_h|_T\in\mathbb P_{r}(T) \textrm{ for each } T\in\mathcal T_h, \\
&\nabla v_h \textrm{ is single-valued at each vertex in } \mathcal V_h \}.
\end{align*}
$$
\mathbb P_{r-2}(\mathcal T_h):=\{q_h\in L^2(\Omega): q_h|_T\in\mathbb P_{r-2}(T) \textrm{ for each } T\in\mathcal T_h\}.
$$
\begin{lemma}
The finite element complex
\begin{equation}\label{eq:C0Vfemderhamcomplex}
\mathbb R\xrightarrow{\subset} V_{r+1}^{\grad}(\mathcal T_h) \xrightarrow{\grad}V_r^{\curl}(\mathcal T_h) \xrightarrow{\curl} V_{r-1}^{\div}(\mathcal T_h)\xrightarrow{\div} \mathbb P_{r-2}(\mathcal T_h)\xrightarrow{}0
\end{equation}
is exact.
\end{lemma}
\begin{proof}
It is obvious that \eqref{eq:C0Vfemderhamcomplex} is a complex. Next we prove the exactness of the complex~\eqref{eq:C0Vfemderhamcomplex}.

By Theorem~1 in \cite{Stenberg2010}, \mnote{ different. cite ours or give a proof.} we get $\div V_h^{\div}=\mathbb P_{r-2}(\mathcal T_h)$, and
\begin{align*}
\dim V_h^{\div}\cap\ker(\div)&=\dim V_h^{\div}-\dim \mathbb P_{r-2}(\mathcal T_h)\\
&=3\#\mathcal V_h+\left(\frac{1}{2}r(r+1)-3\right)\#\mathcal F_h+\frac{1}{6}r(r+1)(2r-5)\#\mathcal T_h.
\end{align*}
For $\bs v_h\in V_h^{\curl}\cap\ker(\curl)$, there exists $q_h\in H^1(\Omega)\cap \mathbb P_{r+1}(\mathcal T_h)$ such that $\bs v_h=\grad q_h$.
Since $\bs v_h$ is single-valued at each vertex in $\mathcal V_h$, we have $q_h\in V_h^{\grad}$. Hence $V_h^{\curl}\cap\ker(\curl)=\grad V_h^{\grad}$, and
\begin{align*}
&\quad\dim\curl V_h^{\curl}=\dim V_h^{\curl}-\dim\grad V_h^{\grad}\\
&=2\#\mathcal V_h+\#\mathcal E_h+\left(\frac{1}{2}r(r+1)-4\right)\#\mathcal F_h+\frac{1}{6}(r-2)(r-1)(2r+3)\#\mathcal T_h+1.
\end{align*}
Then
$$
\dim V_h^{\div}\cap\ker(\div)-\dim\curl V_h^{\curl}=\#\mathcal V_h-\#\mathcal E_h+\#\mathcal F_h-\#\mathcal T_h-1.
$$
Finally $V_h^{\div}\cap\ker(\div)=\curl V_h^{\curl}$ follows from the Euler's formula.
\end{proof}


\bibliographystyle{abbrv}
\bibliography{geodecomp}
\end{document}
