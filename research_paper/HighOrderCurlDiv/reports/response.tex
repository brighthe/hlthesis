% ----------------------------------------------------------------
% AMS-LaTeX Paper ************************************************
% **** -----------------------------------------------------------
\documentclass[10pt]{amsart}
%\textwidth 14.5cm
%\textheight 22cm
%\hoffset -1.5cm
%\voffset -2.2cm
\usepackage{graphicx}
\usepackage{latexsym}
\usepackage{amsfonts}
\usepackage{amsthm}
\usepackage{amssymb}
\usepackage{amsmath}
\usepackage{enumerate}
\usepackage{color}
\usepackage{stmaryrd}
\usepackage{chemarrow}
\usepackage[all]{xy}
\usepackage[pdftex,bookmarksnumbered,bookmarksopen,colorlinks,linkcolor=blue,anchorcolor=black,citecolor=blue,urlcolor=blue]{hyperref}

%\usepackage{mathabx}
% ----------------------------------------------------------------
\vfuzz2pt % Don't report over-full v-boxes if over-edge is small
\hfuzz2pt % Don't report over-full h-boxes if over-edge is small
% THEOREMS -------------------------------------------------------
\newtheorem{thm}{Theorem}[section]
\newtheorem{cor}[thm]{Corollary}
\newtheorem{lem}[thm]{Lemma}
\newtheorem{prop}[thm]{Proposition}
\theoremstyle{definition}
\newtheorem{defn}[thm]{Definition}
\theoremstyle{remark}
\newtheorem{rem}[thm]{Remark}
%\numberwithin{equation}{section}
% MATH -----------------------------------------------------------
\newcommand{\norm}[1]{\left\Vert#1\right\Vert}
\newcommand{\abs}[1]{\left\vert#1\right\vert}
\newcommand{\set}[1]{\left\{#1\right\}}
\newcommand{\Real}{\mathbb R}
\newcommand{\eps}{\varepsilon}
\newcommand{\To}{\longrightarrow}
\newcommand{\BX}{\mathbf{B}(X)}
\newcommand{\A}{\mathcal{A}}

\newcommand{\dx}{\,{\rm d}x}
\newcommand{\dd}{\,{\rm d}}
\newcommand{\bs}{\boldsymbol}
\newcommand{\mcal}{\mathcal}

\DeclareMathOperator*{\img}{img}
%\DeclareMathOperator*{\span}{span}
\newcommand{\sign}{\operatorname{sign}}
\newcommand{\curl}{\operatorname{curl}}
\renewcommand{\div}{\operatorname{div}}
%\renewcommand{\grad}{\operatorname{grad}}
\newcommand{\grad}{\operatorname{grad}}
\newcommand{\tr}{\operatorname{tr}}
% \DeclareMathOperator*{\tr}{tr}
\DeclareMathOperator*{\rot}{rot}
\DeclareMathOperator*{\var}{Var}
\newcommand{\dev}{\operatorname{dev}}
\newcommand{\sym}{\operatorname{sym}}
\newcommand{\skw}{\operatorname{skw}}
\newcommand{\spn}{\operatorname{spn}}
\newcommand{\mspn}{\operatorname{mspn}}
\newcommand{\mskw}{\operatorname{mskw}}
\newcommand{\vskw}{\operatorname{vskw}}
\newcommand{\vspn}{\operatorname{vspn}}
\newcommand{\defm}{\operatorname{def}}
\newcommand{\hess}{\operatorname{hess}}
% ----------------------------------------------------------------
\begin{document}

\title{\large Detailed Response to Referees}%

\date{}%
%\dedicatory{}%
%\commby{}%
% ----------------------------------------------------------------

\maketitle

We express our gratitude to the Referee for their insightful comments, which have significantly contributed to the enhancement of our manuscript. To facilitate easy identification of the modifications, we have highlighted the substantial changes in colored text in this revised version. These adjustments aim to refine our arguments, clarify our methodologies, and strengthen the overall contribution of our work.% We greatly appreciate the questions and concerns the reviewers have raised in the report, as well as the careful reading that has fixed our grammatical errors. Other than some single word typos, 





% \tableofcontents

%We thank the Referee for the valuable comments that helped us to improve the manuscript. In
%the revised version of the manuscript, all non-minor modifications are highlighted in red colour.
%Below, we address the points raised in the report.

% \vskip0.5cm
% \section{Response to report$\_$X}
\begin{enumerate}[1.]


\item \textsf{Can the authors clarify how the ``Geometric Decompositions of Lagrange Elements'' are related to the construction of hierarchical bases presented in \cite{DevlooBravoRylo2009}?}

\smallskip \noindent \textcolor[rgb]{1.00,0.00,0.00}{Reply.}
The constructed hierarchical bases in \cite{DevlooBravoRylo2009} also form a geometric decomposition of Lagrange element only in two and three dimensions, in which orthogonal Chebyshev polynomials are employed. In our manuscript, we use the nodal basis to form the geometric decomposition of Lagrange element in arbitrary dimension.
%  following the way in \cite{ArnoldFalkWinther2009}.

Further, we integrate the nodal basis functions of the Lagrange element with the tangential-normal decomposition. This integration enables the development of the geometric decomposition of $H(\div)$ and $H(\curl)$ elements in arbitrary dimensions and orders. The extension to edge elements presents a significant challenge, necessitating a detailed characterization of the $\curl$ operator and its associated polynomial bubble space.

To maintain focus and brevity in our manuscript, we have chosen to include only the detailed basis for edge elements in the revised section. This choice is aimed at highlighting our novel contributions while efficiently managing the manuscript's length.




\medskip

\item \textsf{It appears that the image resolution should be improved. If this is the case, please make the necessary adjustments.}

\smallskip \noindent \textcolor[rgb]{1.00,0.00,0.00}{Reply.}
Thank you for pointing out this. We have updated Figure 2 and Figure~3.


\end{enumerate}







\bibliographystyle{abbrv}
\bibliography{../paper, ../HighorderFEM}

% ----------------------------------------------------------------
\end{document}
% ----------------------------------------------------------------
