To emphasize the dependence on edges, in the following DoFs, we shall use $e$ instead $f$ for a generic sub-simplex. 
% Given an $f\in \Delta_{\ell}(T)$ we choose $\{\bs n_F, f\subseteq F, F\in \Delta_{n-1}(T)\}$ as the basis for its normal plane and an arbitrary basis for the tangent plane. 

\begin{lemma}[Local N\'ed\'elec element]\label{lm:localNedelec}
For $e\in \Delta_{\ell}(T)$, let $\{\bs t^e_i, i=1,\ldots, \ell\}$ be a basis of the tangent plane of $e$ and choose $\{\bs n_{f,e}: f\in \Delta_{\ell+1}(T), e\subseteq f\}$ as the basis of $\mathscr{N}^e$. 
The shape function space $\mathbb P_k^n(T)$ is uniquely determined by the DoFs
\begin{align}
\label{eq:vecbdDoF0}
\bs v\cdot \bs t_{e}({\bs x}), & \quad e\in \Delta_{1}(T), {\bs x}\in \partial e, \\
\label{eq:vecbdDoF1}
\int_e (\bs v\cdot \bs t_i^e)\ p \dd s, &\quad  p\in \mathbb P_{k - (\ell +1)} (e), e\in \Delta_{\ell}(T),\\
& \quad  i=1,\ldots, \ell, \ell = 1,\ldots, n-1, \notag\\
\label{eq:vecbdDoF2}
\int_e (\bs v\cdot \bs n_{f,e})\ p \dd s, &\quad  p\in \mathbb P_{k - (\ell +1)} (e),e\in \Delta_{\ell}(T), \\
& \quad f\in \Delta_{\ell+1}(T), e\subseteq f, \ell = 1,\ldots, n-1, \notag\\
\int_T \bs v \cdot \bs p \dx&\quad \bs p\in \mathbb P_{k-(n+1)}^n(T). \label{eq:bubbleDoF}
\end{align}
\end{lemma}
\begin{proof}
First of all, by the geometric decomposition \eqref{eq:Pkvecdec} of $\mathbb P_k^n(T)$ and Theorem~\ref{thm:curlbubbletracespacedecomp}, %$\mathbb B_k(\div; T)$ \eqref{eq:divbubbledecomp}, 
the number of DoFs is equal to the dimension of the shape function space. 
% The normal component of $F\in \Delta_{n-1}(T)$ is merged into the bubble DoF \eqref{eq:bubbleDoF}. 

%When $\ell = 0$, the DoF \eqref{eq:vecbdDoF2} is $\{(\bs v\cdot \bs t_{e})({\bs x}), e\in \Delta_{1}(T), {\bs x}\in e\}$ for ${\bs x}\in \Delta_{0}(T)$. 

Assume $\bs v\in\mathbb P_k^n(T)$ and all the DoFs \eqref{eq:vecbdDoF0}-\eqref{eq:bubbleDoF} vanish. Since $\{\bs t_{e}, e\in \Delta_{1}(T), {\bs x}\in\partial e\}$ is a basis of $\mathbb R^n$, $\{(\bs v\cdot \bs t_{e})({\bs x}), e\in \Delta_{1}(T), {\bs x}\in\partial e\}$ will determine the vector $\bs v({\bs x})$. Thus vanishing \eqref{eq:vecbdDoF0} implies $\bs v$ is zero at vertices. In general, $\{\bs n_{f,e}: f\in \Delta_{\ell+1}(T), e\subseteq f\}$ forms a basis of $\mathscr{N}^e$. DoF \eqref{eq:vecbdDoF2} is equivalent to 
\begin{align*}
\int_e (\bs v\cdot \bs n_i^e)\ p \dd s, &\quad  p\in \mathbb P_{k - (\ell +1)} (e), e\in \Delta_{\ell}(T), i=1,\ldots, n-\ell, \; \ell = 1,\ldots, n-1,
\end{align*}
which together with vanishing DoF \eqref{eq:vecbdDoF1} implies
\begin{equation*}
\int_f \bs v\cdot \bs p \dd s = 0, \quad  \bs p\in \mathbb P_{k - (\ell +1)} (f;\mathbb R^n), f\in \Delta_{\ell}(T),  \ell = 1,\ldots, n-1.
\end{equation*}
It follows from the uni-solvence of Lagrange element that $\boldsymbol{v}|_{\partial T}=\boldsymbol{0}$, i.e. $\bs v\in\mathbb B_k^n(T)$.
% $\bs v|_f=\bs0$ for each $f\in\Delta_{n-2}(T)$.
% Then by the vanishing DoF \eqref{eq:vecbdDoF1} with $\ell=n-1$, we get $\tr^{\curl}\bs v=\bs0$, i.e. $\bs v\in\mathbb B_k(\curl, T)$. 
Finally $\bs v=\bs0$ is an immediate result of the vanishing DoF \eqref{eq:bubbleDoF}.
\end{proof}


\begin{lemma}[N\'ed\'elec space]\label{lm:nedelec}
\LC{Choice of $\bs t_e$ and $\bs n_{f,e}$.}
The following {\rm DoFs}
\begin{align}
\label{eq:vecbdDoF0Th}
\bs v\cdot \bs t_{e}({\bs x}), & \quad e\in \Delta_{1}(\mathcal T_h), {\bs x}\in \partial e, \\
\label{eq:vecbdDoF1Th}
\int_e (\bs v\cdot \bs t_i^e)\ p \dd s, &\quad  p\in \mathbb P_{k - (\ell +1)} (e), e\in \Delta_{\ell}(\mathcal T_h),\\
& \quad  i=1,\ldots, \ell, \ell = 1,\ldots, n-1, \notag\\
\label{eq:vecbdDoF2Th}
\int_e (\bs v\cdot \bs n_{f,e})\ p \dd s, &\quad  p\in \mathbb P_{k - (\ell +1)} (e), , e\in \Delta_{\ell}(\mathcal T_h), \\
& \quad f\in \Delta_{\ell+1}(\mathcal T_h), e\subseteq f, \ell = 1,\ldots, n-1, \notag\\
\int_T \bs v \cdot \bs p \dx&\quad \bs p\in \mathbb P_{k-(n+1)}^n(T), T\in \mathcal T_h \label{eq:bubbleDoFTh}
\end{align}
define a $H(\curl)$-conforming space $V_h=\{\bs v_h\in H(\curl; \Omega): \bs v_h|_T\in\mathbb P_k^n(T) \;\; \forall~T\in\mathcal T_h\}$.
%are equivalent to the DoFs
%\begin{equation}\label{eq:vecbdDoFtheorem}
%\int_F \bs v\cdot \bs n_F p \dd s, \quad p\in \mathbb P_k(F), F\in \Delta_{n-1}(\mathcal T_h),
%\end{equation}
%and 
%\begin{equation}\label{eq:intDoF}
%\int_T \bs v\cdot \bs p \dx\quad \bs p\in \grad \mathbb P_{k-1}(T) \oplus \mathbb P_{k-2}(T;\mathbb K)\boldsymbol x,  T\in \mathcal T_h.
%\end{equation}
%Therefore the corresponding $V_h$ is the BDM space and the discrete inf-sup condition holds.
\end{lemma}
\begin{proof}
On each element $T$, DoFs \eqref{eq:vecbdDoF0Th}-\eqref{eq:bubbleDoFTh} will determine a function in $\mathbb P_k^n(T)$ by Lemma~\ref{lm:localNedelec}. For $F\in\Delta_{n-1}(\mathcal T_h)$, DoFs \eqref{eq:vecbdDoF0Th}-\eqref{eq:vecbdDoF2Th} restricted to $F$ are
\begin{align*}
\bs v\cdot \bs t_{e}({\bs x}), & \quad e\in \Delta_{1}(F), {\bs x}\in \partial e, \\
\int_e (\bs v\cdot \bs t_i^e)\ p \dd s, &\quad  p\in \mathbb P_{k - (\ell +1)} (e), e\in \Delta_{\ell}(F),\\
& \quad  i=1,\ldots, \ell, \ell = 1,\ldots, n-1, \notag\\
\int_e (\bs v\cdot \bs n_{f,e})\ p \dd s, &\quad  e\in \Delta_{\ell}(F), f\in \Delta_{\ell+1}(F), e\subseteq f, \\
& \quad p\in \mathbb P_{k - (\ell +1)} (e), \ell = 1,\ldots, n-2.
\end{align*}
Since $\{\bs t_1^e, \ldots, \bs t_{\ell}^e, \bs n_{f,e}, f\in \Delta_{\ell+1}(F), e\subseteq f\}$ spans the tangent plane $\mathscr T^F$, by the unisolvence of the Lagrange element, these DoFs will determine the trace $\tr^{\curl}_F\bs v$ on $F$ independent of the elements containing $F$ and thus the function is $H(\curl;\Omega)$-conforming. 
%In view of \eqref{eq:BDMface}, the obtained space $V_h$ is the BDM space. 
\end{proof}

The vertex DoF can be merged into the edge DoF and result in the classical DoF
$$
\int_e \bs v\cdot \bs t^e\, p \dd s \quad p\in \mathbb P_{k} (e), e\in \Delta_1(\mathcal T_h).
$$
It can be also used to define an $H(\curl;\Omega)$-conforming finite element space with vertex continuity \cite{Christiansen;Hu;Hu:2018finite}. 

%which can be further split into 
%\begin{align}
%\label{eq:ND0DoF}
%\int_e \bs v\cdot \bs t^e \dd s, &\\
%\label{eq:PrRDoF}
%\int_e \bs v\cdot \bs t^e\, p \dd s &\quad p\in \mathbb P_{k} (e)/\mathbb R.
%\end{align}
%Then we have the decomposition
%\begin{equation}\label{eq:curlpolynomialdecomp3}
%\mathbb P_k^n(T) = {\rm ND}_0 \oplus \Oplus_{e\in \Delta_1(T)}(\mathbb B_k(\curl; e) + \nabla b_e) \oplus \Oplus_{\ell=2}^n\Oplus_{f\in\Delta_{\ell}(T)}\mathbb B_k(\curl_f; f).
%\end{equation}
%The component . From the DoF, locally it is isomorphism to $(\mathbb P_k(e)/\mathbb R) \bs t^e$.

%\begin{lemma}
%Let $((\mathbb P_k(e)/\mathbb R) \bs t^e)^*\subset V_h$ be the subspace dual to the DoF \eqref{eq:PrRDoF}. Then
% $$
% ((\mathbb P_k(e)/\mathbb R) \bs t^e)^* \cong \mathbb B_k(\curl; e) \oplus \nabla b_e.
% $$
%\end{lemma}
%\begin{proof}
% It is obvious from DoF point of view.
%\end{proof}
%\LC{$\Oplus_{e\in \Delta_1(T)}\mathbb P_0(e)\bs t^e$ is single out as this is the Whitney form.}
%
%\begin{lemma}\label{lm:nodalcurlfem}
%The following {\rm DoFs}
%\begin{align*}
%\bs v({\bs x}), &\quad  {\bs x}\in \Delta_{0}(\mathcal T_h),\\
%\int_e (\bs v\cdot \bs t_i^e)\ p \dd s, &\quad  p\in \mathbb P_{k - (\ell +1)} (e), e\in \Delta_{\ell}(\mathcal T_h),\\
%& \quad  i=1,\ldots, \ell, \ell = 1,\ldots, n-1, \notag\\
%\int_e (\bs v\cdot \bs n_{f,e})\ p \dd s, &\quad  p\in \mathbb P_{k - (\ell +1)} (e), , e\in \Delta_{\ell}(\mathcal T_h), \\
%& \quad f\in \Delta_{\ell+1}(\mathcal T_h), e\subseteq f, \ell = 1,\ldots, n-1, \notag\\
%\int_T \bs v \cdot \bs p \dx, &\quad \bs p\in \mathbb B_k^n(T),T\in\mathcal T_h \end{align*}
%defines a curl-conforming space $V_h=\{\bs v_h\in H(\curl; \Omega)\cap C^{0}(\Delta_0(\mathcal T_h)): \bs v_h|_T\in\mathbb P_k^n(T) \, \forall T\in\mathcal T_h\}$.
%\end{lemma}
%\begin{proof}
%Since $\{(\bs v\cdot \bs t_{e})({\bs x}), e\in \Delta_{1}(T), {\bs x}\in e\}$ is determined by the vector $\bs v({\bs x})$, we conclude the result from Lemma~\ref{lm:nedelec}.   
%\end{proof}
%

In general, given an integer $-1\leq k \leq n-1$, we can split the DoFs: for $\ell \leq k$, it is Lagrange and for $\ell > k$, it is N\'ed\'elec. It returns to the vector Lagrange element when $k = n-1$, and N\'ed\'elec element for $k=-1$.

\begin{lemma}\label{lm:nodalcurlfem}
\LC{Choice of $\bs t_e$ and $\bs n_{f,e}$. Some are local and some are global.}
The following {\rm DoFs}
\begin{align}
\int_{e}\bs v\cdot \bs p \dd s, &\quad \bs p\in  \mathbb P_{k - (\ell +1)}^n(e), e\in \Delta_{\ell}(\mathcal T_h), \ell = 0,\ldots, k, \label{eq:edgefemThDoF1} \\
\int_e (\bs v\cdot \bs t_i^e)\ p \dd s, &\quad  p\in \mathbb P_{k - (\ell +1)} (e), e\in \Delta_{\ell}(\mathcal T_h),\label{eq:edgefemThDoF2}\\
& \quad  i=1,\ldots, \ell, \ell = k+1,\ldots, n-1, \notag\\
\int_e (\bs v\cdot \bs n_{f,e})\ p \dd s, &\quad  p\in \mathbb P_{k - (\ell +1)} (e), e\in \Delta_{\ell}(\mathcal T_h), \label{eq:edgefemThDoF3}\\
& \quad f\in \Delta_{\ell+1}(\mathcal T_h), e\subseteq f, \ell = k+1,\ldots, n-1, \notag\\
\int_T \bs v \cdot \bs p \dx&\quad \bs p\in \mathbb P_{k-(n+1)}^n(T), T\in\mathcal T_h, \label{eq:edgefemThDoF4}
\end{align}
define a $H(\curl)$-conforming space $V_h=\{\bs v_h\in H(\curl; \Omega)\cap C^{0}(\Delta_k(\mathcal T_h)): \bs v_h|_T\in\mathbb P_k^n(T) \;\; \forall~T\in\mathcal T_h\}$, where $H(\curl; \Omega)\cap C^{0}(\Delta_{-1}(\mathcal T_h))=H(\curl; \Omega)$.
\end{lemma}
\begin{proof}
Clearly the number of DoFs \eqref{eq:edgefemThDoF1}-\eqref{eq:edgefemThDoF4} equals to the number of DoFs \eqref{eq:vecbdDoF0Th}-\eqref{eq:bubbleDoFTh}. DoF \eqref{eq:edgefemThDoF1} determines DoFs \eqref{eq:vecbdDoF0Th}-\eqref{eq:vecbdDoF2Th} for $\ell = 0,\ldots, k$. Then we conclude the result from Lemma~\ref{lm:nedelec}.
\end{proof}

With vertex DoFs, we can choose the basis of the edge element based on those of
the Lagrange element, which is related to the geometric decomposition
\eqref{eq:curlpolynomialdecomp}. The $H(\curl)$-conforming space $V_h$ defined
in Lemma~\ref{lm:nedelec} is same as the second kind N\'ed\'elec element space,
$V_h$ in Lemma~\ref{lm:nodalcurlfem} with $k=0$ same as the $H(\curl)$-conforming space in \cite{Christiansen;Hu;Hu:2018finite} , while DoFs \eqref{eq:vecbdDoF0Th}-\eqref{eq:bubbleDoFTh} are different from those in \cite{Nedelec1986,Christiansen;Hu;Hu:2018finite}. And the corresponding geometric decomposition \eqref{eq:curlpolynomialdecomp} is also different from those in \cite{ArnoldFalkWinther2009,ArnoldFalkWinther2006}. 
The geometric decomposition \eqref{eq:curlpolynomialdecomp} enable the use of Lagrange basis.

% \LC{Treat those lemmas as a special example of the general case.}
% \begin{lemma}[N\'ed\'elec element on a triangle]
%   Any function $\bs u \in \mathbb P_k^2(T)$ 
%   can be uniquely determined by the DoFs:
%   $$
%  N^i_{\bs \alpha }(\bs u):=  \bs u(\bs x_{\bs \alpha}) \cdot \bs e_{\bs x_{\bs \alpha}}^i, \quad \bs x_{\bs \alpha} \in \mathcal X_{T}, 
%   i = 0, 1.
%   $$
%   The basis function of $k$th order second type  N\'ed\'elec element space on $T$ is:
%   $$
%   \bs{\phi}_{\bs \alpha}^i(\bs x) = \phi_{\bs\alpha}(\bs x) 
%   \hat{\bs e}_{\bs x_{\bs \alpha}}^i, \quad \boldsymbol \alpha \in 
%   \mathbb T^2_k, i = 0, 1.
%   $$
% \end{lemma}
% \begin{proof}
% It is straightforward to verify the duality
% $$
% N^j_{\bs \beta }(\bs{\phi}_{\bs \alpha}^i) =   \bs{\phi}_{\bs \alpha}^i(\bs x_{\beta}) \cdot \bs e_{\bs x_{\beta}}^j = \delta_{i}^j \delta_{\bs \alpha}^{\bs \beta}\quad \bs \alpha, \bs \beta\in  \mathbb T^2_k, i,j = 0, 1. 
% $$ 
% \end{proof}


 \begin{lemma}[N\'ed\'elec space on a triangulation]
 Given a conforming triangulation $\mathcal T_h$, for each edge $e\in \Delta_1(\mathcal T_h)$, we choose a global tangential vector $\bs t_e$ and the local outwards normal vector $\bs n_e(T)$. The following {\rm DoFs}
   \begin{align}
     \bs u(\bs x) \cdot \bs t_{e}, \quad & 
     \bs x \in \mathcal X_{e}, e \in \Delta_1(\mathcal T_h) \label{eq:nedelec2d1}\\ 
     \bs u(\bs x)|_{T} \cdot \bs n_e(T), \quad & 
     \bs x \in \mathcal X_{\mathring{e}}, \ e \in \Delta_1(T), T 
     \in \mathcal T_h\label{eq:nedelec2d2}\\
     \bs u(\bs x) \cdot (1, 0), \quad 
     \bs u(\bs x) \cdot (0, 1), \quad & 
     \bs x \in \mathcal X_{\mathring{T}},  
     T \in \mathcal T_h\label{eq:nedelec2d3}
   \end{align}
   define an $H(\curl)$-conforming space 
   $V_h=\{\bs v_h\in H(\curl; \Omega): \bs v_h|_T\in\mathbb P_k^2(T) \, 
   \forall T\in\mathcal T_h\}$.
 \end{lemma}
 \begin{proof}
  Similar to that of Lemma \ref{lm:BDMTh}.
 \end{proof}
 We call \eqref{eq:nedelec2d1} as DoF on edge, which is single valued, \eqref{eq:nedelec2d2} and 
 \eqref{eq:nedelec2d3} as DoF in cell.



 \begin{lemma}[N\'ed\'elec element on a tetrahedron]\label{lemma:nedelec3d}
   A polynomial function $\bs u \in \mathbb P_k^3(T)$ 
   can be uniquely determined by the DoFs:
   $$
  N^i_{\bs \alpha }(\bs u):=   \bs u(\bs x_{\bs \alpha}) \cdot \bs e_{\bs x_{\bs \alpha}}^i, \quad \bs x_{\bs \alpha} \in \mathcal X_{T}, 
   i = 0, 1, 2.
   $$
   The basis function of $k$th second type N\'ed\'elec element space on $T$ is:
   $$
   \label{eq:nedelecbasis3d}
   \bs{\phi}_{\bs \alpha}^i(\bs x) = 
   \phi_{\bs\alpha}(\bs x) \hat{\bs e}_{\bs x_{\bs \alpha}}^i, \quad 
   \boldsymbol \alpha \in \mathbb T^3_k, i = 0, 1, 2.
   $$
 \end{lemma}

 \begin{lemma}[N\'ed\'elec space on tetrahedron mesh]
 \LC{Please indicate which is global and which is local.}
 For edge $e \in \Delta_1(\mathcal T_h)$, let $\bs t_e$ be its unit tangent vector.
 For face $f \in \Delta_2(\mathcal T_h)$, let $\bs n_f$ be its unit normal vector, and $\bs t_f^0$ be the unit tangent vector of 
 the first edge of $f$. Set $\bs t_f^1 = \bs t_f^0 \times \bs n_f$. The following {\rm DoFs}
 \begin{align}
   \bs u \cdot \bs t_e(\bs x), \quad & 
   \bs x \in \mathcal X_{e}, e \in \Delta_1(\mathcal T_h) \label{eq:nedelec3d1}\\
   \bs u(\bs x)|_f \cdot (\bs n_f \times \bs t_e), \quad & 
   \bs x \in \mathcal X_{\mathring{e}}, \ e \in \Delta_1(f), f \in 
   \Delta_2(\mathcal T_h) \label{eq:nedelec3d2}\\
   \bs u(\bs x) \cdot \bs t_{f}^0, \quad 
   \bs u(\bs x) \cdot \bs t_f^1, \quad &  
   \bs x \in \mathcal X_{\mathring{f}}, f \in \Delta_2(\mathcal T_h)\label{eq:nedelec3d3}\\
   \bs u(\bs x)|_{T} \cdot \bs n_f, \quad & 
   \bs x \in \mathcal X_{\mathring{f}}, \ f \in \Delta_2(T), 
   T \in \Delta_3(\mathcal T_h)\label{eq:nedelec3d4}\\
   \bs u(\bs x) \cdot \bs e_{\bs x}^i \quad & 
   \bs x \in \mathcal X_{\mathring{T}}, i=0,1,2, T \in \Delta_3(\mathcal T_h)\label{eq:nedelec3d5}
 \end{align}
 define an $H(\curl)$-conforming space 
 $V_h=\{\bs v_h\in H(\curl, \Omega): \bs v_h|_{T}\in\mathbb P_k^3(T),\ 
 \forall T\in\mathcal T_h\}$.
 \end{lemma}
 \begin{proof}
 %  Proof it is $H(\curl)$-conforming
 Due to Lemma~\ref{lemma:nedelec3d},
 DoFs \eqref{eq:nedelec3d1}-\eqref{eq:nedelec3d5} on each tetrahedron $T$ will determine a 
 function in $\mathbb P_k(T, \mathbb R^3)$. 

 For $f \in \Delta_{2}(\mathcal T_h)$, DoFs \eqref{eq:nedelec3d1}-\eqref{eq:nedelec3d2} restricted to $f$ will determine the trace $\boldsymbol{u}\times\boldsymbol{n}$ on $\partial f$, which together with DoF \eqref{eq:nedelec3d3} restricted to $f$ will determine the trace $\boldsymbol{u}\times\boldsymbol{n}$. Thus the function in $V_h$ is 
 $H(\mathrm{curl})$-conforming.
 \end{proof}
 We call \eqref{eq:nedelec3d1} as DoF on edge,
 \eqref{eq:nedelec3d2} and \eqref{eq:nedelec3d3} as DoF in face,
 \eqref{eq:nedelec3d4} and \eqref{eq:nedelec3d5} as DoF in cell.
\LC{Correct equation number}
 



