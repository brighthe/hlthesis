\documentclass[12pt, a4paper]{article}
\usepackage[utf8]{inputenc}
\usepackage{geometry}
\usepackage{setspace}
\usepackage{authblk} % 用于处理多作者和多单位
\usepackage{amsmath}

% 页面边距设置
\geometry{left=2.5cm, right=2.5cm, top=2.5cm, bottom=2.5cm}

% --- SOPTX 专家修改点 1:调整单位字体大小和样式 ---
% 将单位字体设为小号(\small)并倾斜(\itshape),视觉上更轻盈
\renewcommand\Affilfont{\itshape\small} 

\title{\textbf{SOPTX: A Modular and Extensible Framework for Topology Optimization with Multi-Backend Support}}

% 作者信息设置
\author[1]{Liang He}
\author[1,2,*]{Huayi Wei}
\author[3]{Tian Tian}

% 单位信息
\affil[1]{School of Mathematics and Computational Science, Xiangtan University, Xiangtan 411105, China}
\affil[2]{National Center of Applied Mathematics in Hunan, Hunan Key Laboratory for Computation and Simulation in Science and Engineering, Xiangtan 411105, China}
\affil[3]{School of Mathematics and Computational Science, Xiangtan University, Xiangtan 411105, China}

% 去掉默认日期
\date{}

\begin{document}
	
	\maketitle
	
	% 通讯作者脚注
	\thispagestyle{empty} % 首页不显示页码
	\let\thefootnote\relax\footnotetext{\textbf{* Corresponding author.}}
	\footnotetext{Email address: \texttt{weihuayi@xtu.edu.cn} (Huayi Wei)}
	
	% --- SOPTX 专家修改点 2:微调元数据间距 ---
	\vspace{0.5cm} % 原来的 1cm 可能略大,改为 0.5cm 更紧凑
	\noindent \textbf{Running Head:} SOPTX: A Framework for Topology Optimization
	
	\vspace{0.3cm}
	\noindent \textbf{AMS Subject Classifications:} 74P15, 68N30, 65N30
	
	\vspace{0.3cm}
	\noindent \textbf{Keywords:} Topology Optimization, Multiple Computational Backends, Automatic Differentiation, Modular Framework
	
	% 致谢部分
	% 使用 section* 也可以,但有时直接用粗体标题更像 Title Page 风格,这里保留你的 section* 没问题
	\section*{Acknowledgments}
	This work was supported by the National Natural Science Foundation of China (NSFC) (Grant Nos. 12371410, 12261131501) and the Construction of Innovative Provinces in Hunan Province (Grant No. 2021GK1010).
	
\end{document}