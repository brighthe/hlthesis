%% 
%% Copyright 2019-2024 Elsevier Ltd
%% 
%% This file is part of the 'CAS Bundle'.
%% --------------------------------------
%% 
%% It may be distributed under the conditions of the LaTeX Project Public
%% License, either version 1.3c of this license or (at your option) any
%% later version.  The latest version of this license is in
%%    http://www.latex-project.org/lppl.txt
%% and version 1.3c or later is part of all distributions of LaTeX
%% version 1999/12/01 or later.
%% 
%% The list of all files belonging to the 'CAS Bundle' is
%% given in the file `manifest.txt'.
%% 
%% Template article for cas-sc documentclass for 
%% double column output.

\documentclass[a4paper,fleqn]{cas-sc}

% If the frontmatter runs over more than one page
% use the longmktitle option.

%\documentclass[a4paper,fleqn,longmktitle]{cas-sc}

%\usepackage[numbers]{natbib}
%\usepackage[authoryear]{natbib}
\usepackage[authoryear,longnamesfirst]{natbib}

%%%Author macros
\def\tsc#1{\csdef{#1}{\textsc{\lowercase{#1}}\xspace}}
\tsc{WGM}
\tsc{QE}
%%%

% Uncomment and use as if needed
%\newtheorem{theorem}{Theorem}
%\newtheorem{lemma}[theorem]{Lemma}
%\newdefinition{rmk}{Remark}
%\newproof{pf}{Proof}
%\newproof{pot}{Proof of Theorem \ref{thm}}

\begin{document}
\let\WriteBookmarks\relax
\def\floatpagepagefraction{1}
\def\textpagefraction{.001}

% Short title
\shorttitle{}    

% Short author
\shortauthors{}  

% Main title of the paper
\title [mode = title]{SOPTX: A High-Performance Multi-Backend Framework for Topology Optimization}  

% Title footnote mark
% eg: \tnotemark[1]
\tnotemark[1] 

% Title footnote 1.
% eg: \tnotetext[1]{Title footnote text}
\tnotetext[1]{} 

% First author
%
% Options: Use if required
% eg: \author[1,3]{Author Name}[type=editor,
%       style=chinese,
%       auid=000,
%       bioid=1,
%       prefix=Sir,
%       orcid=0000-0000-0000-0000,
%       facebook=<facebook id>,
%       twitter=<twitter id>,
%       linkedin=<linkedin id>,
%       gplus=<gplus id>]

\author[1]{Liang He}%[<options>]

% Corresponding author indication
\cormark[1]

% Footnote of the first author
\fnmark[1]

% Email id of the first author
\ead{}

% URL of the first author
\ead[url]{}

% Credit authorship
% eg: \credit{Conceptualization of this study, Methodology, Software}
\credit{}

% Address/affiliation
\affiliation[1]{organization={},
            addressline={}, 
            city={},
%          citysep={}, % Uncomment if no comma needed between city and postcode
            postcode={}, 
            state={},
            country={}}

\author[2]{}%[]

% Footnote of the second author
\fnmark[2]

% Email id of the second author
\ead{}

% URL of the second author
\ead[url]{}

% Credit authorship
\credit{}

% Address/affiliation
\affiliation[2]{organization={},
            addressline={}, 
            city={},
%          citysep={}, % Uncomment if no comma needed between city and postcode
            postcode={}, 
            state={},
            country={}}

% Corresponding author text
\cortext[1]{Corresponding author}

% Footnote text
\fntext[1]{}

% For a title note without a number/mark
%\nonumnote{}

% Here goes the abstract
\begin{abstract}
In recent years, topology optimization (TO) has gained widespread attention in both industry and academia as an ideal structural design method. However, its application has high barriers to entry due to the deep expertise and extensive development work typically required. Traditional numerical methods for TO are tightly coupled with computational mechanics methods such as finite element analysis (FEA), making the algorithms intrusive and requiring comprehensive understanding of the entire system. This paper presents SOPTX, a TO package based on FEALPy, which implements a modular architecture that decouples analysis from optimization, supports multiple computational backends (NumPy/PyTorch/JAX), and achieves a non-intrusive design paradigm.

The main innovations of SOPTX include: (1) A cross-platform, multi-backend support system compatible with various computational backends such as NumPy, PyTorch, and JAX, enabling efficient algorithm execution on CPUs and flexible acceleration using GPUs, as well as efficient sensitivity computation for objective and constraint functions via automatic differentiation (AD); (2) A fast matrix assembly technique, overcoming the performance bottleneck of traditional numerical integration methods and significantly enhancing computational efficiency; (3) A modular framework designed to support TO problems for arbitrary dimensions and meshes, complemented by a rich library of composable components, including diverse filters and optimization methods, enabling users to flexibly configure and extend optimization workflows according to specific needs.

Taking the density-based method as an example, this paper elaborates the architecture, computational workflow, and usage of SOPTX through the classical compliance minimization problem with volume constraints. Numerical examples demonstrate that SOPTX significantly outperforms existing open-source packages in terms of computational efficiency and memory usage, especially exhibiting superior performance in large-scale problems. The modular design of the software not only enhances the flexibility and extensibility of the code but also provides opportunities for exploring novel TO problems, offering robust support for education, research, and engineering applications in TO.
\end{abstract}

% Use if graphical abstract is present
%\begin{graphicalabstract}
%\includegraphics{}
%\end{graphicalabstract}

% Research highlights
\begin{highlights}
\item 
\item 
\item 
\end{highlights}


% Keywords
% Each keyword is seperated by \sep
\begin{keywords}
 \sep \sep \sep
\end{keywords}

\maketitle

% Main text
\section{Introduction}
Topology optimization (TO) is an essential class of structural optimization techniques aimed at improving structural performance by optimizing material distribution within a design domain. In fields such as aerospace, automotive, and civil engineering, TO addresses critical design challenges through efficient material utilization and mechanical performance optimization. Among various approaches, density-based methods are particularly popular due to their intuitiveness and practicality. In these methods, the distribution of material and void within the design domain is optimized, and the relative density of each finite element is treated as a design variable. The most widely adopted density-based method is the Solid Isotropic Material with Penalization (SIMP) approach. This approach promotes binary ($0-1$) solutions by penalizing intermediate densities, and due to its simplicity and seamless integration with finite element analysis (FEA), it has been extensively employed since its introduction by \citet{Bendsøe2004}.

However, TO problems inherently involve large-scale computations. This is due to the tight coupling between structural analysis and element-level optimization of design variables: at each iteration, it is necessary not only to solve boundary value problems to obtain structural responses but also to calculate derivatives of the objective and constraint functions with respect to design variables, supporting gradient-based optimization algorithms. For large-scale problems, this implies solving large linear systems and performing sensitivity analysis at every iteration, placing significant demands on computing resources and performance. Therefore, improving computational efficiency and scalability while ensuring accuracy has become a major challenge in TO research and applications.

To lower the entry barrier and promote widespread adoption of TO, researchers have published numerous educational studies and literature. A pioneering example is the 99-line MATLAB code by \citet{sigmund200199}, which demonstrated the fundamentals of a two-dimensional SIMP algorithm in a concise and self-contained manner, profoundly impacting both TO education and research.  Subsequently, improved versions of this educational code emerged continuously, including the more efficient 88-line version proposed by \citet{andreassen2011efficient} and the extended 3D implementation by \citet{liu2014efficient}. These educational codes convey practical knowledge of TO in the simplest possible form and provide self-contained examples of basic numerical algorithms.

Meanwhile, efforts have also been made to leverage the advantages of open-source software development for addressing TO problems. For example, \citet{chung2019topology} proposed a modular TO approach based on OpenMDAO \citep{gray2019openmdao}, an open-source multidisciplinary design optimization framework. They decomposed the TO problem into multiple components, where users provide forward computations and analytic partial derivatives, while OpenMDAO automatically assembles and computes total derivatives, enhancing code flexibility and extensibility. Subsequently, \citet{gupta202055} developed a parallel-enabled TO implementation based on the open-source finite element software FEniCS\citep{alnaes2015fenics}, demonstrating its potential for handling large-scale problems. More recently, \citet{ferro2023simple} advanced this direction by introducing a concise and efficient TO implementation in just 51 lines of code. Their work utilized FEniCS for modeling and FEA, Dolfin Adjoint for automatic sensitivity analysis, and Interior Point OPTimizer for optimization, greatly simplifying the implementation process. These projects provided standardized interfaces enabling convenient integration with existing FEA tools, thereby lowering implementation complexity. However, despite improvements in modularity achieved by these open-source software packages, challenges in terms of functionality extension and flexibility remain when dealing with complex engineering applications.

To accelerate the development of TO, automating sensitivity analysis has become a critical step. This involves automatically computing derivatives of objectives, constraints, material models, projections, filters, and other components with respect to the design variables. Currently, the common practice involves manually calculating sensitivities, which, despite not being theoretically complex, can be tedious and error-prone, often becoming a bottleneck in the development of new TO modules and exploratory research. Automatic differentiation (AD) provides an efficient and accurate approach for evaluating derivatives of numerical functions \citep{griewank2008evaluating}. By decomposing complex functions into a series of elementary operations (such as addition and multiplication), AD accurately computes derivatives of arbitrary differentiable functions. In TO, the Jacobian matrices represent sensitivities of objective functions and constraints with respect to design variables, and software can automate this process, relieving developers from manually deriving and implementing sensitivity calculations. With its capability of easily obtaining accurate derivative information, AD offers significant advantages in design optimization, particularly for highly nonlinear problems.

In recent years, the use of AD in TO has gradually increased. For instance, \citet{norgaard2017applications} employed the AD tools CoDiPack and Tapenade to achieve automatic sensitivity analysis in unsteady flow TO, significantly enhancing computational efficiency. \citet{chandrasekhar2021auto} leveraged JAX \citep{bradbury2018jax}, a high-performance Python library, to implement AD within density-based TO, efficiently solving classical topology optimization problems such as compliance minimization.

Although the programs described above—including early educational codes, open-source software implementations, and initial frameworks incorporating AD—have significantly promoted the adoption and development of TO methods, they typically employ a procedural programming paradigm. This approach divides the numerical computation process into multiple interdependent subroutines, leading to a tightly coupled relationship between analysis modules (e.g., FEA) and optimization modules. Such tightly coupled architectures limit code extensibility and reusability: on one hand, the strong interdependencies between subroutines mean that even adding a new objective function or constraint often requires invasive modifications across multiple modules, increasing development time and potentially introducing new errors; on the other hand, this coupled architectural pattern makes it challenging to integrate topology optimization programs as standalone modules within multidisciplinary design optimization (MDO) frameworks or system-level engineering processes. For instance, in aerospace applications, TO modules need seamless integration with other disciplines (such as fluid mechanics or thermodynamics), yet tightly coupled architectures typically result in complicated and inefficient integration procedures, hindering the application of TO in more complex scenarios. Consequently, designing an architecture that decouples analysis from optimization, thereby enhancing extensibility and reusability without sacrificing algorithmic accuracy, has become an important challenge that urgently needs addressing in the TO community.

\section{}\label{}

% Numbered list
% Use the style of numbering in square brackets.
% If nothing is used, default style will be taken.
%\begin{enumerate}[a)]
%\item 
%\item 
%\item 
%\end{enumerate}  

% Unnumbered list
%\begin{itemize}
%\item 
%\item 
%\item 
%\end{itemize}  

% Description list
%\begin{description}
%\item[]
%\item[] 
%\item[] 
%\end{description}  

\clearpage %%Remove this from your manuscript


% Figure
\begin{figure}%[]
  \centering
%    \includegraphics{}
    \caption{}\label{fig1}
\end{figure}


\begin{table}%[]
\caption{}\label{tbl1}
\begin{tabular*}{\tblwidth}{@{}LL@{}}
\toprule
  &  \\ % Table header row
\midrule
 & \\
 & \\
 & \\
 & \\
\bottomrule
\end{tabular*}
\end{table}

% Uncomment and use as the case may be
%\begin{theorem} 
%\end{theorem}

% Uncomment and use as the case may be
%\begin{lemma} 
%\end{lemma}

%% The Appendices part is started with the command \appendix;
%% appendix sections are then done as normal sections
%% \appendix

\section{}\label{}

% To print the credit authorship contribution details
\printcredits

%% Loading bibliography style file
%\bibliographystyle{model1-num-names}
\bibliographystyle{cas-model2-names}

% Loading bibliography database
\bibliography{cas-refs}

% Biography
%\bio{}
% Here goes the biography details.
%\endbio

%\bio{pic1}
% Here goes the biography details.
%\endbio

\end{document}

