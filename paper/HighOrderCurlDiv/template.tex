\documentclass[mathpazo]{cicp}


%%%%% author macros %%%%%%%%%
% place your own macros HERE
%%%%% end %%%%%%%%%

\begin{document}
%%%%% title : short title may not be used but TITLE is required.
% \title{TITLE}
% \title[short title]{TITLE}
\title{Here is the Title}

%%%%% author(s) :
% single author:
% \author[name in running head]{AUTHOR\corrauth}
% [name in running head] is NOT OPTIONAL, it is a MUST.
% Use \corrauth to indicate the corresponding author.
% Use \email to provide email address of author.
% \footnote and \thanks are not used in the heading section.
% Another acknowlegments/support of grants, state in Acknowledgments section
% \section*{Acknowledgments}
\author[O.~Author]{Only Author\corrauth}
\address{School of Mathematical Sciences, Beijing Normal University,
Beijing 100875, P.R. China}
\email{{\tt author@email} (O.~Author)}

% multiple authors:
% Note the use of \affil and \affilnum to link names and addresses.
% The author for correspondence is marked by \corrauth.
% use \emails to provide email addresses of authors
% e.g. below example has 3 authors, first author is also the corresponding
%      author, author 1 and 3 having the same address.
% \author[Zhang Z R et.~al.]{Zhengru Zhang\affil{1}\comma\corrauth,
%       Author Chan\affil{2}, and Author Zhao\affil{1}}
% \address{\affilnum{1}\ School of Mathematical Sciences,
%          Beijing Normal University,
%          Beijing 100875, P.R. China. \\
%           \affilnum{2}\ Department of Mathematics,
%           Hong Kong Baptist University, Hong Kong SAR}
% \emails{{\tt zhang@email} (Z.~Zhang), {\tt chan@email} (A.~Chan),
%          {\tt zhao@email} (A.~Zhao)}
% \footnote and \thanks are not used in the heading section.
% Another acknowlegments/support of grants, state in Acknowledgments section
% \section*{Acknowledgments}


%%%%% Begin Abstract %%%%%%%%%%%
\begin{abstract}
The abstract should provide a brief summary of the main findings of the paper.
\end{abstract}
%%%%% end %%%%%%%%%%%

%%%%% AMS/PACs/Keywords %%%%%%%%%%%
%\pac{}
\ams{52B10, 65D18, 68U05, 68U07}
\keywords{moving mesh method, conservative interpolation, iterative method, $l^2$ projection.}

%%%% maketitle %%%%%
\maketitle


%%%% Start %%%%%%
\section{Introduction}
\label{sec1}
In the past two decades, there has been important progress in developing adaptive mesh methods for PDEs.
Mesh adaptivity is usually of two types in form: local mesh refinement and moving mesh method.  ....

\section{Preparation of Manuscript}
\label{sec2}
The Title Page should contain the article title, authors' names and complete affiliations,
and email addresses of all authors. The Abstract should provide a brief summary of the main findings of the paper.

References should be cited in the text by a number in square brackets.
Literature cited should appear on a separate page at the end of the article
and should be styled and punctuated using standard abbreviations for journals
(see Chemical Abstracts Service Source Index, 1989). For unpublished lectures of symposia,
include title of paper, name of sponsoring society in full, and date.
Give titles of unpublished reports with "(unpublished)" following the reference.
Only articles that have been published or are in press should be included in the references.
Unpublished results or personal communications should be cited as such in the text.
Please note the sample at the end of this paper.

Equations should be typewritten whenever possible and the number placed in parentheses at the right margin.
Reference to equations should use the form "Eq. (2.1)" or simply "(2.1)." Superscripts and subscripts should
be typed or handwritten clearly above and below the line, respectively.

Figures should be in a finished form suitable for publication. Number figures consecutively with Arabic numerals.
Lettering on drawings should be of professional quality or generated by high-resolution computer graphics and must be
large enough to withstand appropriate reduction for publication.
For example, if you use {\sf MATLAB} to do figure plots,
axis labels should be at least point 18. Title should be 24 points or above. Tick marks labels
better have 14 points or above. Line width should be 2 (or above).

Illustrations in color in most cases can be accepted
only if the authors defray the cost. At the Editor's discretion a limited number of color figures each year of special
interest will be published at no cost to the author.


%%%% Acknowledgments %%%%%%%%
\section*{Acknowledgments}
The author would like to thank  ....

%%%% Bibliography  %%%%%%%%%%
\begin{thebibliography}{99}
\bibitem{Berger}M. J. Berger and P. Collela, Local adaptive mesh refinement
for shock hydrodynamics,
J. Comput. Phys., 82 (1989), 62-84.
\bibitem{deBoor}C. de Boor,  Good Approximation By Splines With Variable Knots II, in Springer Lecture
 Notes Series 363, Springer-Verlag, Berlin, 1973.
\bibitem{TanTZ} Z. J. Tan, T. Tang and Z. R. Zhang, A simple moving mesh method for one- and
two-dimensional phase-field equations, J. Comput. Appl. Math., to appear.
\bibitem{Toro}E. F. Toro, Riemann Solvers and Numerical Methods for Fluid Dynamics,
Springer-Verlag Berlin Heidelbert, 1999.
\end{thebibliography}

\end{document}
